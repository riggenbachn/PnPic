\documentclass{article}
\usepackage[utf8]{inputenc}
%Packages Used------------------------------------------
% the following is to get qoppa and Qoppa
\DeclareFontFamily{T1}{cbgreek}{}
\DeclareFontShape{T1}{cbgreek}{m}{n}{<-6>  grmn0500 <6-7> grmn0600 <7-8> grmn0700 <8-9> grmn0800 <9-10> grmn0900 <10-12> grmn1000 <12-17> grmn1200 <17-> grmn1728}{}
\DeclareSymbolFont{quadratics}{T1}{cbgreek}{m}{n}
\DeclareMathSymbol{\qoppa}{\mathord}{quadratics}{19}
\DeclareMathSymbol{\Qoppa}{\mathord}{quadratics}{21}

\usepackage{amsmath, amssymb, amsthm}
\usepackage{longtable}
%\usepackage{amsfonts}
%\usepackage{mathtools}
%\usepackage{wasysym}
%\usepackage{MnSymbol}
%\usepackage{thmtools}
%\usepackage{stmaryrd}
\usepackage[letterpaper,margin=1in]{geometry}   
%\usepackage{slashed}
%\usepackage[english]{babel}				
%\usepackage[pdfencoding=auto, psdextra, draft=false]{hyperref}
\usepackage{bookmark}
\usepackage{url}		
\usepackage{lmodern}			
\usepackage[T1]{fontenc}
%\usepackage{xspace}		
%\usepackage{fancyhdr}
\usepackage{enumerate}
%\usepackage{mathrsfs}
\usepackage{graphicx}
\usepackage{soul,color}
\usepackage{tikz-cd}
\usepackage[maxbibnames=99,style=alphabetic]{biblatex}
\usepackage{csquotes}
\usepackage{chngcntr}
\usepackage[bbgreekl]{mathbbol}
\counterwithin{equation}{section}
\addbibresource{biblio.bib}
\usepackage{todonotes}

%hyperlink setup
\definecolor{darkred}{RGB}{128,0,0}
\definecolor{darkgreen}{RGB}{0,128,0}
\definecolor{darkblue}{RGB}{0,0,128}

\hypersetup{linktocpage,
	pdfborder = {0 0 0},
	colorlinks,
	citecolor=darkgreen,
	filecolor=darkred,
	linkcolor=darkblue,
	urlcolor=cyan!50!black!90}

%Greek and Latin black board bold-----------------------
\DeclareSymbolFontAlphabet{\mathbb}{AMSb}
\DeclareSymbolFontAlphabet{\mathbbl}{bbold}

%shortcut commands-------------------------------------------
\DeclareMathOperator{\Br}{Br} % Brauer functor
\DeclareMathOperator{\Brp}{Br^p} % Poincare Brauer functor
\DeclareMathOperator{\CAlg}{CAlg} % Commutative Algebra objects
\DeclareMathOperator{\CAlgp}{CAlg^p} % Poincare ring spectra
\DeclareMathOperator{\Cat}{Cat} % Categories
\DeclareMathOperator{\Catex}{\Cat_\infty^{ex}} % stable categories with exact functors
\DeclareMathOperator{\Cath}{Cat^h_\infty} % Hermitian Categories
\DeclareMathOperator{\Catp}{Cat^p_\infty} % Poincare Categories
\DeclareMathOperator{\Catpidem}{Cat^p_{\infty, idem}} % idempotent complete Poincare Categories
\DeclareMathOperator{\Einfty}{\mathbf{E}_\infty} % E-infinity 
\DeclareMathOperator{\ex}{ex} % exact 
\DeclareMathOperator{\Fun}{Fun} % Functors
\DeclareMathOperator{\gp}{gp} % grouplike
\DeclareMathOperator{\id}{id} % identity
\DeclareMathOperator{\idem}{idem} % idempotent
\DeclareMathOperator{\Mod}{Mod} % Modules
%\DeclareMathOperator{\op}{op} % opposite functor
\DeclareMathOperator{\Pic}{Pic} % Picard functor
\DeclareMathOperator{\Picp}{Pic^p} % Poincare Picard functor
\DeclareMathOperator{\Pn}{Pn} % Poincare space functor
\DeclareMathOperator{\Spectra}{Sp} % Spectra
\DeclareMathOperator{\Spaces}{\mathcal{S}} % Spaces
\DeclareMathOperator{\Mon}{Mon} % monoids

\newcommand{\pf}{{\bf Proof. \ }}
\renewcommand{\epsilon}{\varepsilon}
\renewcommand{\rho}{\varrho}
\renewcommand{\phi}{\varphi}
\newcommand{\NN}{\ensuremath{\mathbb{N}}\xspace}
\newcommand{\ZZ}{\ensuremath{\mathbb{Z}}\xspace}
\newcommand{\QQ}{\ensuremath{\mathbb{Q}}\xspace}
\newcommand{\RR}{\ensuremath{\mathbb{R}}\xspace}
\newcommand{\CC}{\ensuremath{\mathbb{C}}\xspace}
\newcommand{\FF}{\ensuremath{\mathbb{F}}\xspace}
\newcommand{\EE}{\mathbb{E}}
\newcommand{\TT}{\ensuremath{\mathbb{T}}\xspace}
\newcommand{\RP}{\ensuremath{\mathbb{RP}}\xspace}
\newcommand{\DD}{\ensuremath{\mathbbl{\Delta}}\xspace}
\newcommand{\tc}{\ensuremath{\mathrm{TC}}}
\newcommand{\thh}{\ensuremath{\mathrm{THH}}}
\newcommand{\tp}{\ensuremath{\mathrm{TP}}}
\newcommand{\tr}{\ensuremath{\mathrm{TR}}}
\newcommand{\pnpic}{\ensuremath{\mathrm{PnPic}}}
\newcommand{\pnbr}{\ensuremath{\mathrm{PnBr}}}
\newcommand{\pic}{\ensuremath{\mathrm{Pic}}}
\newcommand{\br}{\ensuremath{\mathrm{Br}}}
\DeclareMathOperator*{\colim}{\ensuremath{\operatorname{colim}}}
\newcommand{\aps}{\mathrm{APS}}
\newcommand{\psch}{\mathrm{PSch}}
\newcommand{\perf}{\mathrm{Perf}}
\newcommand{\perfpn}{\mathrm{Perf}^{\mathrm{Pn}}}
\newcommand{\op}{\mathrm{op}}

%Theorem Environments ----------------------------------------------------------------
\newtheorem{theorem}{Theorem}[section]
\newtheorem{proposition}[theorem]{Proposition}
\newtheorem{lemma}[theorem]{Lemma}
\newtheorem{corollary}[theorem]{Corollary}

\theoremstyle{definition}
\newtheorem{definition}[theorem]{Definition}
\newtheorem{construction}[theorem]{Construction}
\newtheorem{remark}[theorem]{Remark}
\newtheorem{observation}[theorem]{Observation}
\newtheorem{notation}[theorem]{Notation}
\newtheorem{example}[theorem]{Example}
\newtheorem{question}[theorem]{Question}

\newcommand{\Viktor}[1]{\todo{V: #1}}
\newcommand{\Noah}[1]{\todo[color=red]{N: #1}}
\newcommand{\Lucy}[1]{\todo[color=cyan]{L: #1}}

\title{Et cetera}
\author{Ben Antieau, Viktor Burghardt, Noah Riggenbach, Lucy Yang}
\date{}
\addbibresource{biblio.bib}

\begin{document}

\maketitle
\begin{abstract}
   Dumping ground for other stuff: Notes, one-off observations, stuff that we can collectively use when preparing talks, etc. \Lucy{I make no promises re: organization but I will do my best to keep it reasonably readable} 
\end{abstract}
\tableofcontents

\section{Talk prep}

\section{References}
\begin{itemize}
    \item \href{https://ems.press/journals/dm/articles/8965687}{Involutions of Azumaya algebras} by First and Williams (2020 \emph{Documenta})
    \item \href{https://arxiv.org/abs/2405.15260}{Counterexamples in involutions of Azumaya algebras} by First and Williams; much more readable than the 2020 Documenta paper
\end{itemize}
\section{Questions and directions}
\begin{question}
    [Morita theory for $ \Catp $]
    Let $ R $ be a Poincaré ring. 
    Suppose given two $ R $-algebras (suitably interpreted so their module categories are canonically endowed with $ R $-linear Poincaré structures--perhaps $ \mathbb{E}_\sigma $) $ A $, $ B $. 
    Can we characterize
    \begin{equation*}
        \hom_{\Catp_R}\left(\left(\Mod_A^\omega,\Qoppa_A\right),\left(\Mod_B^\omega,\Qoppa_B\right)\right)
     \end{equation*} 
     in terms of something bimodule-like? 
\end{question}
\begin{question}
    On page 2 of the \emph{Counterexamples} paper, First and Williams write that `` existence of an extraordinary involution means classificaiton of Azumaya algebras with involution...\emph{cannot} be reduced to questions about projective modules and hermitian forms on them.'' 

    What if we replaced projective modules by perfect complexes thereof? 
\end{question}
\begin{question}
    First--Williams show (see discussion in \S4 of the \emph{Counterexamples} paper) that coarse type classify many (most?) Azumaya algebras up to (étale-local) \emph{isomorphism}. 

    What is a suitable derived version of ``coarse type''?
\end{question}

\section{Thoughts \& observations}
\begin{question}
   When $ R $ has the Tate Poincaré structure and $ (\Mod_A^\omega, M_A, N_A, N_A \to M_A^{tC_2}) $ is invertible, then by invertibility have an equivalence $ \hom_R(A, R)\simeq N_A\otimes_R N_{A^\op} $ of $ A \otimes_R A^\op $-modules. 
   Restricting the left-hand side along the unit map $ R \to A $ gives a map $ N_A \otimes_R N_{A^\op} \to \hom_R(R,R) \simeq R $. 
   Is this a perfect ($R$-linear) pairing? 

   I \emph{think} using that $ R^{\varphi C_2} \simeq R $ and combining the linear and bilinear part conditions, we get something like
   \begin{equation*}
       M_A \otimes_R M_{A^\op} \simeq (N_A \otimes_R N_{A^\op})^{\otimes_R 2} \qquad \text{ as $A \otimes_R A^\op$-bimodules. }
   \end{equation*}
   Is this useful?
\end{question}

\paragraph{Brauer-Severi schemes} 
We know there is a correspondence between Azumaya algebras $ A $ over $ X $ and Brauer-Severi schemes. 
What does a Poincaré structure on $ \Mod_A^\omega $ mean `geometrically' for $ D^b_{\mathrm{coh}} $ of the corresponding Brauer-Severi scheme? 
(Lucy: I didn't get very far here, but just typing up what I had)
\begin{itemize}
    \item $ \Mod_A^\omega $ corresponds to $ \alpha $-twisted sheaves on $ X $ (see Proposition 3.2.2.1 of Max Lieblich's thesis)
    \item The bounded derived category of $ \alpha $-twisted sheaves on $ X $ includes as one `piece' of a semiorthogonal decomposition on $ D^b_{\mathrm{coh}} $ of the corresponding Brauer-Severi scheme (see Theorem 5.1 \href{https://arxiv.org/abs/math/0511497}{here})
\end{itemize}

\section{Desperate Flailing}

This section is a cronical of my thoughts about $\mathbb{G}_m^\Qoppa$.
\paragraph{Goal} The goal is to build a Poincar{\'e} ring $\mathbb{G}_{m}^\Qoppa:=(\mathrm{Mod}_R, \Qoppa_R)$  such that $B\mathbb{G}_m^\Qoppa(\underline{S}) = \Picp(\underline{S})$ for any Poincar{\'e} ring $\underline{S}$.
\begin{lemma}
Let $\underline{S}$ be a Poincar{\'e} ring. Then $\pi_0(\mathrm{Aut}_{\mathrm{Pn}(\mathrm{Mod}_S)}(S,u))=\{s\in \pi_0(S)^\times | s=1 \textrm{ in }\pi_0(S^{C_2})\}$.
\end{lemma}
\begin{proof}
Since the functor $\mathrm{Pn}(\mathrm{Mod}_S)\to \mathrm{Mod}_S$ is conservative it follows that an element of $\pi_0(\mathrm{Aut}_{\mathrm{Pn}(\mathrm{Mod}_S)}(S,u))$ must have underlying map an element of $\pi_0\mathrm{Aut}(S)=\pi_0(S)^\times$. Then in order for $s\in \pi_0(S)^\times$ to induce a map $(S,u)\to (S,u)$, the induced map $s^*:S^{C_2}\to S^{C_2}$ must satisfy $s^*(u)=u$. The pullback is given by multiplication by $s$, so this requirement translates into $s$ being the unit, as desired.
\end{proof}

The problem I thought existed maybe doesn't. Here is a candidate construction:

\begin{construction}
Define $R$ to be the $\mathbb{E}_\infty$ ring given by $\mathbb{S}\{x^{\pm 1}, y^{\pm 1}\}\otimes_{\mathbb{S}\{z\}}\mathbb{S}$ where the map $\mathbb{S}\{z\}\to \mathbb{S}\{x^{\pm 1}, y^{\pm 1}\}$ is induced by the map $z\mapsto xy$, and the map $\mathbb{S}\{z\}\to \mathbb{S}$ is induced by $z\mapsto 1$. We can give $R$ an $\mathbb{E}_\infty$ ring structure in $\mathrm{Sp}^{BC_2}$ by taking the trivial action on $\mathbb{S}\{z\}$ and $\mathbb{S}$, and taking the action induced by $x\mapsto y$ and $y\mapsto x$ on $\mathbb{S}\{x^{\pm 1}, y^{\pm 1}\}$. Thus in $\mathrm{CAlg}(\mathrm{Sp}^{BC_2})$ the ring $R$ corepresents the functor $S\mapsto \{s\in \pi_0(S)^\times| s\sigma(s)=1\}$.

Now take $\underline{R}$ to be the Poincar{\'e} ring with underlying Borel $C_2$ structure as described in the previous paragraph and geometric fixed points $R^{\phi C_2}=\mathbb{S}$ and the map $R^{\phi C_2}\to R^{tC_2}$ given by the unit map. Endowing $R^{\phi C_2}$ with the $R$-module structre given by $x,y\mapsto 1$, it remains to show that the unit map $R^{\phi C_2}\to R^{tC_2}$ factors the Tate valued Frobenius $R\to R^{tC_2}$ in order to promote $\underline{R}$ to a Poincar{\'e} ring. By construction of $R$ it is then enough to show that on $\pi_0$ the Tate valued Frobenius sends $x,y\mapsto 1$ in $\pi_0(R^{tC_2})$. This map sends both $x$ and $y$ to $xy\in \pi_0(R^{tC_2})$. These areequal to $1$ in $\pi_0(R^{tC_2})$ since the functor $(-)^{tC_2}$ is lax-monoidal so $R^{tC_2}$ is a modules over $\mathbb{S}\{x^{\pm 1}, y^{\pm 1}\}^{tC_2}\otimes_{\mathbb{S}\{z\}^{tC_2}}\mathbb{S}^{tC_2}$ which has the image of $xy$ equal to $1$. 
\end{construction}

Now consider another Poincar{\'e} ring $\underline{S}$. We then have that maps $\pi_0(\mathrm{Maps}(\underline{R},\underline{S}))$ is the data of a unit $s\in \pi_0(S)^\times$, a path $s\sigma(s)\to 1$ in $\Omega^\infty S$, and paths $x,y\to 1$ in $\Omega^\infty S^{\phi C_2}$.  This then agrees with $\mathbb{G}_m^\Qoppa$ by the following lemma.

\begin{lemma}
Let $S\in \mathrm{CAlg}(\mathrm{Sp}^{BC_2})$ and $s\in \pi_0(S)^\times$. Then $s\sigma(s)=1$ in $\pi_0(S)$ if and only if $(s\otimes s)^*$ acts by $1$ on $\pi_0(S^{hC_2})=\pi_0(\mathrm{Hom}_{S\otimes S}(S\otimes S, S)^{hC_2})$.
\end{lemma}
\begin{proof}
The 'only if' direction follows from the fact that the map $S^{hC_2}\to S$ is an $S$-bimodule map. Now suppose that $s\sigma(s)=1$ in  $S$. Then before taking homotopy fixed points the induced map $s^*=id$ because $S$ is $\mathbb{E}_\infty$.\footnote{Or just $\mathbb{E}_2$.} 
\end{proof}

\printbibliography
\end{document}
