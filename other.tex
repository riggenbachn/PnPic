\documentclass{article}
\usepackage[utf8]{inputenc}
%Packages Used------------------------------------------
% the following is to get qoppa and Qoppa
\DeclareFontFamily{T1}{cbgreek}{}
\DeclareFontShape{T1}{cbgreek}{m}{n}{<-6>  grmn0500 <6-7> grmn0600 <7-8> grmn0700 <8-9> grmn0800 <9-10> grmn0900 <10-12> grmn1000 <12-17> grmn1200 <17-> grmn1728}{}
\DeclareSymbolFont{quadratics}{T1}{cbgreek}{m}{n}
\DeclareMathSymbol{\qoppa}{\mathord}{quadratics}{19}
\DeclareMathSymbol{\Qoppa}{\mathord}{quadratics}{21}

\usepackage{amsmath, amssymb, amsthm}
\usepackage{longtable}
%\usepackage{amsfonts}
%\usepackage{mathtools}
%\usepackage{wasysym}
%\usepackage{MnSymbol}
%\usepackage{thmtools}
%\usepackage{stmaryrd}
\usepackage[letterpaper,margin=1in]{geometry}   
%\usepackage{slashed}
%\usepackage[english]{babel}				
%\usepackage[pdfencoding=auto, psdextra, draft=false]{hyperref}
\usepackage{bookmark}
\usepackage{url}		
\usepackage{lmodern}			
\usepackage[T1]{fontenc}
%\usepackage{xspace}		
%\usepackage{fancyhdr}
\usepackage{enumerate}
\usepackage{enumitem}
%\usepackage{mathrsfs}
\usepackage{graphicx}
\usepackage{soul,color}
\usepackage{tikz-cd}
\usepackage[maxbibnames=99,style=alphabetic]{biblatex}
\usepackage{csquotes}
\usepackage{chngcntr}
\usepackage[bbgreekl]{mathbbol}
\counterwithin{equation}{section}
\addbibresource{biblio.bib}
\usepackage{todonotes}

%hyperlink setup
\definecolor{darkred}{RGB}{128,0,0}
\definecolor{darkgreen}{RGB}{0,128,0}
\definecolor{darkblue}{RGB}{0,0,128}

\hypersetup{linktocpage,
	pdfborder = {0 0 0},
	colorlinks,
	citecolor=darkgreen,
	filecolor=darkred,
	linkcolor=darkblue,
	urlcolor=cyan!50!black!90}

%Greek and Latin black board bold-----------------------
\DeclareSymbolFontAlphabet{\mathbb}{AMSb}
\DeclareSymbolFontAlphabet{\mathbbl}{bbold}

%shortcut commands-------------------------------------------
\DeclareMathOperator{\Br}{Br} % Brauer functor
\DeclareMathOperator{\Brp}{Br^p} % Poincare Brauer functor
\DeclareMathOperator{\CAlg}{CAlg} % Commutative Algebra objects
\DeclareMathOperator{\Alg}{Alg}
\DeclareMathOperator{\CAlgp}{CAlg^p} % Poincare ring spectra
\DeclareMathOperator{\Cat}{Cat} % Categories
\DeclareMathOperator{\Catex}{\Cat_\infty^{ex}} % stable categories with exact functors
\DeclareMathOperator{\Cath}{Cat^h_\infty} % Hermitian Categories
\DeclareMathOperator{\Catp}{Cat^p_\infty} % Poincare Categories
\DeclareMathOperator{\Catpidem}{Cat^p_{\infty, idem}} % idempotent complete Poincare Categories
\DeclareMathOperator{\Einfty}{\mathbf{E}_\infty} % E-infinity 
\DeclareMathOperator{\ex}{ex} % exact 
\DeclareMathOperator{\Fun}{Fun} % Functors
\DeclareMathOperator{\gp}{gp} % grouplike
\DeclareMathOperator{\id}{id} % identity
\DeclareMathOperator{\idem}{idem} % idempotent
\DeclareMathOperator{\Mod}{Mod} % Modules
\DeclareMathOperator{\Modh}{{}^{\sigma}Mod} % Hermitian modules
%\DeclareMathOperator{\op}{op} % opposite functor
\DeclareMathOperator{\Pic}{Pic} % Picard functor
\DeclareMathOperator{\Picp}{Pic^p} % Poincare Picard functor
\DeclareMathOperator{\Pn}{Pn} % Poincare space functor
\DeclareMathOperator{\Spectra}{Sp} % Spectra
\DeclareMathOperator{\Spaces}{\mathcal{S}} % Spaces
\DeclareMathOperator{\Mon}{Mon} % monoids

\newcommand{\pf}{{\bf Proof. \ }}
\renewcommand{\epsilon}{\varepsilon}
\renewcommand{\rho}{\varrho}
\renewcommand{\phi}{\varphi}
\newcommand{\NN}{\ensuremath{\mathbb{N}}\xspace}
\newcommand{\ZZ}{\ensuremath{\mathbb{Z}}\xspace}
\newcommand{\QQ}{\ensuremath{\mathbb{Q}}\xspace}
\newcommand{\RR}{\ensuremath{\mathbb{R}}\xspace}
\newcommand{\CC}{\ensuremath{\mathbb{C}}\xspace}
\newcommand{\FF}{\ensuremath{\mathbb{F}}\xspace}
\newcommand{\EE}{\mathbb{E}}
\newcommand{\TT}{\ensuremath{\mathbb{T}}\xspace}
\newcommand{\RP}{\ensuremath{\mathbb{RP}}\xspace}
\newcommand{\DD}{\ensuremath{\mathbbl{\Delta}}\xspace}
\newcommand{\tc}{\ensuremath{\mathrm{TC}}}
\newcommand{\thh}{\ensuremath{\mathrm{THH}}}
\newcommand{\tp}{\ensuremath{\mathrm{TP}}}
\newcommand{\tr}{\ensuremath{\mathrm{TR}}}
\newcommand{\pnpic}{\ensuremath{\mathrm{PnPic}}}
\newcommand{\pnbr}{\ensuremath{\mathrm{PnBr}}}
\newcommand{\pic}{\ensuremath{\mathrm{Pic}}}
\newcommand{\br}{\ensuremath{\mathrm{Br}}}
\DeclareMathOperator*{\colim}{\ensuremath{\operatorname{colim}}}
\newcommand{\aps}{\mathrm{APS}}
\newcommand{\psch}{\mathrm{PSch}}
\newcommand{\perf}{\mathrm{Perf}}
\newcommand{\perfpn}{\mathrm{Perf}^{\mathrm{Pn}}}
\newcommand{\op}{\mathrm{op}}
\newcommand{\Associnv}{\mathrm{Assoc}_\sigma}
\newcommand{\BMinv}{\mathrm{BM}_\sigma} % for `hermitian' bimodules; made a macro so we can change this later easily. 

%Theorem Environments ----------------------------------------------------------------
\newtheorem{theorem}{Theorem}[section]
\newtheorem{proposition}[theorem]{Proposition}
\newtheorem{lemma}[theorem]{Lemma}
\newtheorem{corollary}[theorem]{Corollary}

\theoremstyle{definition}
\newtheorem{definition}[theorem]{Definition}
\newtheorem{construction}[theorem]{Construction}
\newtheorem{remark}[theorem]{Remark}
\newtheorem{observation}[theorem]{Observation}
\newtheorem{notation}[theorem]{Notation}
\newtheorem{example}[theorem]{Example}
\newtheorem{question}[theorem]{Question}

\newcommand{\Viktor}[1]{\todo{V: #1}}
\newcommand{\Noah}[1]{\todo[color=red]{N: #1}}
\newcommand{\Lucy}[1]{\todo[color=cyan!30]{L: #1}}
\newcommand{\Lucyil}[1]{\todo[inline,color=cyan!30]{L: #1}}

\title{Et cetera}
\author{Ben Antieau, Viktor Burghardt, Noah Riggenbach, Lucy Yang}
\date{}
\addbibresource{biblio.bib}

\begin{document}

\maketitle
\begin{abstract}
   Dumping ground for other stuff: Notes, one-off observations, stuff that we can collectively use when preparing talks, etc. \Lucy{I make no promises re: organization but I will do my best to keep it reasonably readable} 
\end{abstract}
\tableofcontents

\section{Talk prep}

\section{References}
\begin{itemize}
    \item \href{https://ems.press/journals/dm/articles/8965687}{Involutions of Azumaya algebras} by First and Williams (2020 \emph{Documenta})
    \item \href{https://arxiv.org/abs/2405.15260}{Counterexamples in involutions of Azumaya algebras} by First and Williams; much more readable than the 2020 Documenta paper
\end{itemize}
\section{Questions and directions}
\begin{question}
    [Morita theory for $ \Catp $]
    Let $ R $ be a Poincaré ring. 
    Suppose given two $ R $-algebras (suitably interpreted so their module categories are canonically endowed with $ R $-linear Poincaré structures--perhaps $ \mathbb{E}_\sigma $) $ A $, $ B $. 
    Can we characterize
    \begin{equation*}
        \hom_{\Catp_R}\left(\left(\Mod_A^\omega,\Qoppa_A\right),\left(\Mod_B^\omega,\Qoppa_B\right)\right)
     \end{equation*} 
     in terms of something bimodule-like? 
\end{question}
\begin{question}
    On page 2 of the \emph{Counterexamples} paper, First and Williams write that `` existence of an extraordinary involution means classificaiton of Azumaya algebras with involution...\emph{cannot} be reduced to questions about projective modules and hermitian forms on them.'' 

    What if we replaced projective modules by perfect complexes? 
\end{question}
\begin{question}
    First--Williams show (see discussion in \S4 of the \emph{Counterexamples} paper) that coarse type classify many (most?) Azumaya algebras up to (étale-local) \emph{isomorphism}. 

    What is a suitable derived version of ``coarse type''?
\end{question}

\section{Thoughts \& observations}
\begin{question}
   When $ R $ has the Tate Poincaré structure and $ (\Mod_A^\omega, M_A, N_A, N_A \to M_A^{tC_2}) $ is invertible, then by invertibility have an equivalence $ \hom_R(A, R)\simeq N_A\otimes_R N_{A^\op} $ of $ A \otimes_R A^\op $-modules. 
   Restricting the left-hand side along the unit map $ R \to A $ gives a map $ N_A \otimes_R N_{A^\op} \to \hom_R(R,R) \simeq R $. 
   Is this a perfect ($R$-linear) pairing? 

   I \emph{think} using that $ R^{\varphi C_2} \simeq R $ and combining the linear and bilinear part conditions, we get something like
   \begin{equation*}
       M_A \otimes_R M_{A^\op} \simeq (N_A \otimes_R N_{A^\op})^{\otimes_R 2} \qquad \text{ as $A \otimes_R A^\op$-bimodules. }
   \end{equation*}
   Is this useful?
\end{question}

\paragraph{Brauer-Severi schemes} 
We know there is a correspondence between Azumaya algebras $ A $ over $ X $ and Brauer-Severi schemes. 
What does a Poincaré structure on $ \Mod_A^\omega $ mean `geometrically' for $ D^b_{\mathrm{coh}} $ of the corresponding Brauer-Severi scheme? 
(Lucy: I didn't get very far here, but just typing up what I had)
\begin{itemize}
    \item $ \Mod_A^\omega $ corresponds to $ \alpha $-twisted sheaves on $ X $ (see Proposition 3.2.2.1 of Max Lieblich's thesis)
    \item The bounded derived category of $ \alpha $-twisted sheaves on $ X $ includes as one `piece' of a semiorthogonal decomposition on $ D^b_{\mathrm{coh}} $ of the corresponding Brauer-Severi scheme (see Theorem 5.1 \href{https://arxiv.org/abs/math/0511497}{here})
\end{itemize}

\section{Desperate Flailing}

This section is a cronical of my thoughts about $\mathbb{G}_m^\Qoppa$.
\paragraph{Goal} The goal is to build a Poincar{\'e} ring $\mathbb{G}_{m}^\Qoppa:=(\mathrm{Mod}_R, \Qoppa_R)$  such that $B\mathbb{G}_m^\Qoppa(\underline{S}) = \Picp(\underline{S})$ for any Poincar{\'e} ring $\underline{S}$.
\begin{lemma}
Let $\underline{S}$ be a Poincar{\'e} ring. Then $\pi_0(\mathrm{Aut}_{\mathrm{Pn}(\mathrm{Mod}_S)}(S,u))=\{s\in \pi_0(S)^\times | s=1 \textrm{ in }\pi_0(S^{C_2})\}$.
\end{lemma}
\begin{proof}
Since the functor $\mathrm{Pn}(\mathrm{Mod}_S)\to \mathrm{Mod}_S$ is conservative it follows that an element of $\pi_0(\mathrm{Aut}_{\mathrm{Pn}(\mathrm{Mod}_S)}(S,u))$ must have underlying map an element of $\pi_0\mathrm{Aut}(S)=\pi_0(S)^\times$. Then in order for $s\in \pi_0(S)^\times$ to induce a map $(S,u)\to (S,u)$, the induced map $s^*:S^{C_2}\to S^{C_2}$ must satisfy $s^*(u)=u$. The pullback is given by multiplication by $s$, so this requirement translates into $s$ being the unit, as desired.
\end{proof}

The problem I thought existed maybe doesn't. Here is a candidate construction:

\begin{construction}
Define $R$ to be the $\mathbb{E}_\infty$ ring given by $\mathbb{S}\{x^{\pm 1}, y^{\pm 1}\}\otimes_{\mathbb{S}\{z\}}\mathbb{S}$ where the map $\mathbb{S}\{z\}\to \mathbb{S}\{x^{\pm 1}, y^{\pm 1}\}$ is induced by the map $z\mapsto xy$, and the map $\mathbb{S}\{z\}\to \mathbb{S}$ is induced by $z\mapsto 1$. We can give $R$ an $\mathbb{E}_\infty$ ring structure in $\mathrm{Sp}^{BC_2}$ by taking the trivial action on $\mathbb{S}\{z\}$ and $\mathbb{S}$, and taking the action induced by $x\mapsto y$ and $y\mapsto x$ on $\mathbb{S}\{x^{\pm 1}, y^{\pm 1}\}$. Thus in $\mathrm{CAlg}(\mathrm{Sp}^{BC_2})$ the ring $R$ corepresents the functor $S\mapsto \{s\in \pi_0(S)^\times| s\sigma(s)=1\}$.

Now take $\underline{R}$ to be the Poincar{\'e} ring with underlying Borel $C_2$ structure as described in the previous paragraph and geometric fixed points $R^{\phi C_2}=\mathbb{S}$ and the map $R^{\phi C_2}\to R^{tC_2}$ given by the unit map. Endowing $R^{\phi C_2}$ with the $R$-module structre given by $x,y\mapsto 1$, it remains to show that the unit map $R^{\phi C_2}\to R^{tC_2}$ factors the Tate valued Frobenius $R\to R^{tC_2}$ in order to promote $\underline{R}$ to a Poincar{\'e} ring. By construction of $R$ it is then enough to show that on $\pi_0$ the Tate valued Frobenius sends $x,y\mapsto 1$ in $\pi_0(R^{tC_2})$. This map sends both $x$ and $y$ to $xy\in \pi_0(R^{tC_2})$. These areequal to $1$ in $\pi_0(R^{tC_2})$ since the functor $(-)^{tC_2}$ is lax-monoidal so $R^{tC_2}$ is a modules over $\mathbb{S}\{x^{\pm 1}, y^{\pm 1}\}^{tC_2}\otimes_{\mathbb{S}\{z\}^{tC_2}}\mathbb{S}^{tC_2}$ which has the image of $xy$ equal to $1$. 
\end{construction}

Now consider another Poincar{\'e} ring $\underline{S}$. We then have that maps $\pi_0(\mathrm{Maps}(\underline{R},\underline{S}))$ is the data of a unit $s\in \pi_0(S)^\times$, a path $s\sigma(s)\to 1$ in $\Omega^\infty S$, and paths $x,y\to 1$ in $\Omega^\infty S^{\phi C_2}$.  This then agrees with $\mathbb{G}_m^\Qoppa$ by the following lemma.

\begin{lemma}
Let $S\in \mathrm{CAlg}(\mathrm{Sp}^{BC_2})$ and $s\in \pi_0(S)^\times$. Then $s\sigma(s)=1$ in $\pi_0(S)$ if and only if $(s\otimes s)^*$ acts by $1$ on $\pi_0(S^{hC_2})=\pi_0(\mathrm{Hom}_{S\otimes S}(S\otimes S, S)^{hC_2})$.
\end{lemma}
\begin{proof}
The 'only if' direction follows from the fact that the map $S^{hC_2}\to S$ is an $S$-bimodule map. Now suppose that $s\sigma(s)=1$ in  $S$. Then before taking homotopy fixed points the induced map $s^*=id$ because $S$ is $\mathbb{E}_\infty$.\footnote{Or just $\mathbb{E}_2$.} 
\end{proof}

\section{Modules with genuine involution} 
\begin{remark}
    [Lucy] I'm just going to put drafts of stuff pertaining to hermitian modules\Lucy{or whatever we want to keep calling these} here. 
    Eventually when it gets to be more complete, I will hopefully move this entire section over to the main file. 
\end{remark}
\paragraph{Meta-commentary} There are (at least) three things we want to do: 
\begin{enumerate}[label=(\alph*)]
    \item Define a category of `bimodules with involution over algebras with anti-involution' equipped with a forgetful functor $ \Theta \colon \mathrm{BMod}_{\mathrm{inv}}(-) \to \EE_1\Alg(-)^{hC_2} $. 
    \item Show that $ \Theta $ is a coCartesian fibration. 
    For this, it suffices to show that it is a \emph{Cartesian} fibration and that it satisfies the hypotheses of \cite[Corollary 5.2.2.5]{HTT}
    \begin{itemize}
        \item I used to think that we could obtain this by `bootstrapping' a result from Higher Algebra, plus some facts about assembly. 
        This doesn't seem to be working, so I'm just going to try to do this directly (imitating certain aspects of Chapter 4 of higher algebra.) 
    \end{itemize}
    \item Define a relative tensor product for hermitian bimodules 
    \item Show that the formula for the cocartesian pushforward along a map $ A \to B $ in $ \EE_1\Alg(-)^{hC_2} $ is something like $ - \otimes_{A \otimes A^\op} \left(B \otimes B^\op \right) \otimes_{B\otimes B^\op} B $. 
    \begin{itemize}
        \item In Higher Algebra, the formula for the cocartesian pushforward is proven in \cite[\S4.6]{LurHA}; in particular, this is in the section on duality. 
        In particular, see Proposition 4.6.2.17 and the paragraph immediately preceding this.  
        \item I don't know how to do this yet--while (a) and (b) are not useful if I can't show (c), I can't suss out the feasibility of (c) without (a) and (b) already in place. 
    \end{itemize}
\end{enumerate}

\begin{definition}\label{defn:colored_operad_monoid_with_involution}
\Lucy{This is just an imitation of \cite[Definition 4.1.1.1]{LurHA}, modified in accordance with ideas from \S5.4.2. }
    Define a colored operad $ \Associnv $ as follows:
    \begin{enumerate}[label=(\roman*)]
        \item The colored operad has a single object, which we denote by $ \mathfrak{a} $. 
        \item For every finite set $ I $, the set of operations $ \mathrm{Mul}_{\Associnv}\left(\left\{\mathfrak{a}_i\right\}_{i \in I}, \mathfrak{a}\right) \simeq \mathcal{L}I \times \{\pm 1\}^I $, where $ \mathcal{L}I $ is the set of linear orderings on $ I $ and an element of $ \{\pm 1\}^I $ is a function $ I \to \{\pm 1 \} $. 
        \item Suppose given a map of finite sets $ \alpha \colon I \to J $, together with operations $ (\preceq_j, f_j \colon I_j \to \{\pm 1\}) \in \mathrm{Mul}_{\Associnv}\left(\left\{\mathfrak{a}_i\right\}_{\alpha(i)=j}, \mathfrak{a}\right) $ and $ (\preceq_J, g \colon J \to \{\pm 1\}) \in \mathrm{Mul}_{\Associnv}\left(\left\{\mathfrak{a}_j\right\}_{j\in J}, \mathfrak{a}\right) $. 
        Define a linear ordering on the set $ I $ as follows: $ i \leq i' $ if $ \alpha(i) \preceq_J \alpha(i') $ or $ \alpha(i) = \alpha(i')= j $ and $ i \preceq_j i' $ and $ g(j)= +1 $ or $ \alpha(i) = \alpha(i')= j $ and $ i \succeq_j i' $ and $ g(j)= -1 $. 
        Finally, define a function 
        \begin{align*}
            I &\to \{\pm 1\} \\
            i &\mapsto f_{\alpha(i)} (i) \cdot  g(\alpha(i))\,,
        \end{align*}  
        where the multiplication on $ \{\pm 1\} $ is the usual one. 
    \end{enumerate} 
\end{definition}
\begin{definition}\label{defn:infty_operad_monoid_with_involution}
    Let $ \Associnv^\otimes $ denote the associated $ \infty $-operad (via Construction 2.1.1.7 and Example 2.1.1.21 of \cite{LurHA}). 
\end{definition} 
\begin{remark}
    Unwinding definitions
    \begin{itemize}
        \item Objects $ \Associnv^\otimes $ are finite pointed sets $ \langle n \rangle \in \mathrm{Fin}_* $ 
        \item Morphisms $ \langle m \rangle \to \langle n \rangle $ consist of 
        \begin{itemize}
            \item $ \alpha \colon \langle m \rangle \to \langle n \rangle $ a map of finite pointed sets
            \item for each $ i \in \langle n \rangle^\circ $, a linear ordering $ \preceq_i $ on the inverse image $ \alpha^{-1}(\{i\}) $ 
            \item a map of sets $ s \colon  \alpha^{-1}\left(\langle m \rangle^\circ\right) \to \{\pm 1\} $
        \end{itemize}
        \item For each pair of morphisms
        \begin{equation*}
             \left(\beta \colon \langle \ell \rangle \to \langle m \rangle, \preceq_j, s\right) \qquad \left(\alpha \colon \langle m \rangle \to \langle n \rangle, \preceq_i, t\right) \,,
        \end{equation*}
        the composite is the triple $ \left(\alpha \circ \beta, \preceq_j'', u \right) $ where $ \preceq_j'' $ is the ordering on $ (\alpha \circ \beta)^{-1}(\{i\}) $ so that if $ a,b \in \langle \ell \rangle $ so that $ \alpha (\beta(a)) = \alpha(\beta(b)) $, then $ a \preceq_j'' b $ if $ \beta(a) \preceq_i \beta(b) $ or $ \beta(a) =_i \beta(b) = i $ and $ a \preceq_i b $ if $ s(i) = 1 $ or $ a \succeq_i b $ if $ s(i) = -1 $. 
        Finally $ u (l) = s(l) \cdot t(\beta(l))  $. 
        \Lucy{Note that when $ s, t$ are identically one, the resulting order $ \preceq_j''$ agrees with the lexicographic order defined in \cite[Remark 4.1.1.4]{LurHA}.}  
    \end{itemize}
\end{remark}
\begin{definition}
    Define a category $ \Delta_\sigma $ 
    \begin{itemize}
        \item objects are pairs $ ([n], s \colon \{1, \ldots, n\} \to \{\pm 1\}) $ \Lucy{maybe better to write $ s $ as a function defined on the set of \emph{morphisms} $ i < i+1 $ in $[n]$.}
        \item a morphism from $ ([n], s \colon \{1, \ldots, n\} \to \{\pm 1\}) $ to $ ([m], t \colon \{0, 1, \ldots, m\} \to \{\pm 1\}) $ is an order-preserving map $ [n] \to [m] $ in $ \Delta $. 
    \end{itemize}
\end{definition}
\begin{construction}\label{cons:involutive_cut}
    Define a functor $ \mathrm{Cut} \colon \Delta_\sigma^\op \to \Associnv^\otimes $: 
    \begin{itemize}
        \item For each $ ([n],s) $, we have $ \mathrm{Cut}([n],s) = \langle n \rangle $. 
        \item Given a morphism $ \alpha\colon ([n],s) \to ([m], t) $, the associated morphism $ \mathrm{Cut} ([n],s) \to \mathrm{Cut}([m], t) $ consists of
        \begin{itemize}
            \item On underlying finite pointed sets $ \langle m\rangle \to \langle n \rangle $, $ \mathrm{Cut} $ agrees with that appearing in \cite[Construction 4.1.2.9]{LurHA}
            \item Identifying the cut $ \{k \mid k< j\} \sqcup \{k \mid k \geq j \} $ with the morphism $ j-1 < j $, we may regard $ s \colon \langle n\rangle^\circ \to \{\pm 1\} $ and likewise $ t \colon \langle m\rangle^\circ \to \{\pm 1\} $. 
            Define $ u\colon \mathrm{Cut}(\alpha)^{-1}\left(\langle n\rangle^\circ\right) \to \{\pm 1\} $ to be the unique function so that $ u(j)t(j) = s(\mathrm{Cut}(\alpha)(j)) $. 
            % This means that $ \alpha^{-1}(\{k \mid k< j\}) $ and $ \alpha^{-1}(\{k \mid k \geq j\}) $ 
        \end{itemize}
    \end{itemize}
\end{construction}
\begin{lemma}
    The functor $ \mathrm{Cut} \colon \Delta^\op_\sigma \to \Associnv^\otimes $ exhibits $ \Delta^\op_\sigma $ as an approximation to the $ \infty $-operad $ \Associnv^\otimes $. 
    \Lucyil{I think the proof of this lemma is not too different from the proof of Proposition 4.1.2.11 of \cite{LurHA}; the point here is just to unravel the definitions of locally coCartesian and Cartesian; the morphisms in $ \Delta^\op_\sigma $ are a little more complicated than $ \Delta^\op $, but not by much. } 
\end{lemma}

\begin{definition}
    Define a colored operad $ \mathbf{BM}_\mathrm{inv} $
    \begin{enumerate}[label=(\roman*)]
        \item The set of objects of $ \mathbf{BM}_\mathrm{inv} $ has two elements, which we denote by $ \mathfrak{a}, \mathfrak{m} $. 
        \item Let $ \{X_i\}_{i \in I} $ be a finite collection of objects of $ \mathbf{BM}_{\mathrm{inv}} $ and let $ Y $ be another object of $ \mathbf{BM}_{\mathrm{inv}} $. 
        If $ Y = \mathfrak{a} $, then $ \mathrm{Mul}_{\mathbf{BM}_\mathrm{inv}} \left(\{X_i\}_{i \in I}, Y\right) $ is the set of pairs consisting of a linear ordering on $ I $ and a function $ I \to \{\pm 1\} $ if $ X_i = \mathfrak{a} $ for all $ i $, and empty otherwise. 
        If $ Y = \mathfrak{m} $, then $ \mathrm{Mul}_{\mathbf{BM}_\mathrm{inv}} \left(\{X_i\}_{i \in I}, Y\right) $ is the set of pairs consisting of a linear ordering $ \{i_1 < i_2 < \cdots < i_n\} $ on $ I $ and a function $ I \to \{\pm 1 \} $ IF $ X_{i_1} = \mathfrak{m} $ and $ X_j = \mathfrak{a} $ for all $ j \neq i_1 $, and $ \mathrm{Mul}_{\mathbf{BM}_\mathrm{inv}} \left(\{X_i\}_{i \in I}, Y\right) $ is empty otherwise. 
        \item The composition law on $ \mathbf{BM}_{\mathrm{inv}} $ is determined by the composition of linear orderings, with reversal of linear orderings according to Definition \ref{defn:colored_operad_monoid_with_involution} 
    \end{enumerate}
\end{definition} 
\begin{remark} % compare Remark 4.2.1.2
    Restricting to the object $ \mathfrak{a} \in \mathbf{BM}_\mathrm{inv} $, we see that $ \mathbf{BM}_\mathrm{inv} $ has a sub-colored operad which is canonically identified with $ \mathbf{Assoc}_\mathrm{inv} $ of Definition \ref{defn:colored_operad_monoid_with_involution}. 
\end{remark}
\begin{definition}
    Let $ \mathcal{BM}_\mathrm{inv}^\otimes $ denote the associated $ \infty $-operad (via Construction 2.1.1.7 and Example 2.1.1.21 of \cite{LurHA}).
\end{definition}
\begin{remark}
    \Lucy{compare Higher Algebra Notation 4.2.1.6} We can describe the category $ \mathcal{BM}_\mathrm{inv}^\otimes $ as follows: 
    \begin{enumerate}[label=(\arabic*)]
        \item An object of $ \mathcal{BM}_\mathrm{inv}^\otimes $ is a pair $ (\langle n\rangle,S) $ where $ S $ is a subset of $ \langle n \rangle^\circ $. 
        \item Morphisms $ (\langle m \rangle, T) \to (\langle n \rangle,S) $ consist of a map $ \alpha \colon \langle m \rangle \to \langle n \rangle $ in $ \Associnv^\otimes $ satisfying: 
        \begin{itemize}
            \item The map $ \alpha $ takes $ T \cup \{*\} $ to $ S \cup \{*\} $
            \item For each $ s \in S $, then $ \alpha^{-1}(\{s\}) $ contains exactly one element of $ t $, and that element is minimal with respect to the linear ordering on $ \alpha^{-1}(\{s\}) $. 
            \Lucyil{I've changed things a little so that $ S $ (in the notation of Higher Algebra) has been replaced by $ S^c $--this way,we can regard $ [n] $ as representing the ordered set $\{-n <-n+1 < \cdots -1<0<1< \cdots < n-1 < n\} $ where $ C_2 $ acts by $ \cdot (-1) $ (or something along these lines). This is really a generalization of Notation 4.2.1.7 but for $ RM^\otimes $.} 
        \end{itemize}
    \end{enumerate}
\end{remark}
\begin{remark} %compare Remark 4.2.1.5 in Higher algebra
    Each morphism $ \phi \in \mathrm{Mul}_{\mathbf{BM}_\mathrm{inv}} \left(\{X_i\}_{i \in I}, Y\right) $ determines a linear ordering $\ell $ on the set $ I $ and a function $ s \colon I \to \{\pm 1\} $. 
    Passing from $ \phi $ to the pair $ (\ell, s) $ determines a map of colored operads $ j \colon \mathbf{BM}_\mathrm{inv} \to \mathbf{Assoc}_\mathrm{inv} $. 
    For any monoidal\Lucy{more general?} $ \infty $-category $ \mathcal{C} $, restriction along $ j $ sends an $ \EE_\sigma $-algebra $ A \colon \mathbf{Assoc}_\mathrm{inv} \to \mathcal{C}^{\otimes} $ to the pair $ (A, A) $ where $ A $ is regarded as an involutive\Lucy{hermitian} bimodule over itself. 
\end{remark}
\begin{construction}
    Define a functor $ \mathrm{MCut} \colon \Delta_\sigma^\op \to BM^\otimes_{\mathrm{inv}} $: \Lucy{maybe this overloaded notation is not good. I'm running out of ideas.}
    \begin{itemize}
        \item For each $ ([n],s) $, we have $ \mathrm{MCut}([n],s) = \langle n + 1 \rangle \simeq \mathrm{RCut}_0([n]) $ where $ \mathrm{RCut} $ is from \cite[Construction 4.8.4.4]{LurHA}. 
        \item Given a morphism $ \alpha\colon ([n],s) \to ([m], t) $, the associated morphism $ \mathrm{MCut}([m], t) \to \mathrm{MCut} ([n],s) $ consists of
        \begin{itemize}
            \item On underlying finite pointed sets $ \langle m+1\rangle \to \langle n+1 \rangle $, $ \mathrm{MCut} $ agrees with (the reverse of) that appearing in \cite[Construction 4.2.2.6]{LurHA}
            \item Identifying the cut $ \{k \mid k< j\} \sqcup \{k \mid k \geq j \} $ with the morphism $ j-1 < j $, we may regard $ s \colon \langle n+1\rangle^\circ \to \{\pm 1\} $ and likewise $ t \colon \langle m+1\rangle^\circ \to \{\pm 1\} $. \Lucy{check later}
            Define $ u\colon \mathrm{MCut}(\alpha)^{-1}\left(\langle n+1\rangle^\circ\right) \to \{\pm 1\} $ to be the unique function so that $ u(j)t(j) = s(\mathrm{MCut}(\alpha)(j)) $. 
            % This means that $ \alpha^{-1}(\{k \mid k< j\}) $ and $ \alpha^{-1}(\{k \mid k \geq j\}) $ 
        \end{itemize}
    \end{itemize}
\end{construction}
\begin{remark}\label{rmk:nat_transformation_of_cut_functors} % compare remark 4.2.2.8
    We can identify $ \Associnv^\otimes $ with the full subcategory of $ \mathcal{BM}^\otimes_{\mathrm{inv}} $ spanned by objects of the form $ (\langle n\rangle, \langle n\rangle^\circ) $. 
    We can regard Construction \ref{cons:involutive_cut} as defining a functor $ \Delta^\op_\sigma \to \mathcal{BM}^\otimes_{\mathrm{inv}} $. 
    For each $ ([n], s) \in \Delta^\op_\sigma $, there is a map of sets $ \theta \colon \mathrm{MCut} ([n], s) \to \mathrm{Cut}([n], s) $ defined as in \cite[Remark 4.2.2.8]{LurHA}. 
    Concretely, on underlying pointed sets, $ \theta $ takes the form \Lucy{check that the signs $s$ work out!}
    \begin{equation*}
    \begin{split}
        \theta \colon \langle n+1 \rangle &\to \langle n \rangle \\
                            k &\mapsto \begin{cases}
                                k-1 & \text{ if }k>0 \\
                                * & \text{ if }k = 0, * \,.
                            \end{cases}
    \end{split}    
    \end{equation*} 
    This construction determines a morphism $ \gamma $ in the $ \infty $-category $ \Fun\left(\Delta^\op_\sigma, \mathcal{BM}^\otimes_{\mathrm{inv}}\right) $, or equivalently a map $ \Delta^\op_\sigma \times \Delta^1 \to \mathcal{BM}^\otimes_{\mathrm{inv}} $. 
\end{remark}
\begin{lemma}
    The morphism $ \gamma \colon \Delta^\op_\sigma \times \Delta^1 \to \mathcal{BM}^\otimes_{\mathrm{inv}} $ defined in Remark \ref{rmk:nat_transformation_of_cut_functors} exhibits $\Delta^\op_\sigma \times \Delta^1 $ as an approximation to the $ \infty $-operad $ \mathcal{BM}^\otimes_{\mathrm{inv}} $. 
\end{lemma}
\begin{definition}
    Let $ \mathcal{C}^\otimes \to \Associnv^\otimes $ be a fibration of $ \infty $-operads and let $ \mathcal{M} $ be an $ \infty $-category. 
    Suppose given a fibration of $ \infty $-operads $ q \colon \mathcal{O}^\otimes \to \mathcal{BM}^\otimes_{\mathrm{inv}} $ together with equivalences $ \mathcal{O}^\otimes_{\mathfrak{a}} \simeq \mathcal{C}^\otimes $ and $ \mathcal{O}^\otimes_{\mathfrak{m}} \simeq \mathcal{M} $. \Lucy{Lurie gives this a name (Definition 4.2.1.12 \emph{weakly enriched})--not sure what to call this. something \emph{bi-enriched}?}
    Let $ \Modh\left(\mathcal{M}\right) $ denote the $ \infty $-category $ \Alg_{/\mathcal{BM}}\left(\mathcal{O}\right) $. 
    We will refer to $ \Modh\left(\mathcal{M}\right) $ as the \emph{$\infty$-category of hermitian module objects of $ \mathcal{M}$}. 
    Composition with the inclusion $ \Associnv^\otimes \to \mathcal{BM}^\otimes_{\mathrm{inv}} $ induces a categorical fibration
    \begin{equation*}
        \Modh\left(\mathcal{M}\right) = \Alg_{/\mathcal{BM}}\left(\mathcal{O}\right) \to \Alg_{\Associnv}(\mathcal{C}) \,.
    \end{equation*}
    If $ A $ is an $ \Associnv $-algebra object of $ \mathcal{C} $, we let $ \Modh_A\left(\mathcal{M}\right) $ denote the fiber $ \Modh\left(\mathcal{M}\right) \times_{\Alg_{\Associnv}(\mathcal{C})} \{A\} $. 
    We will refer to $ \Modh_A\left(\mathcal{M}\right) $ as the \emph{$\infty$-category of hermitian $A$-module objects of $ \mathcal{M} $}. 
\end{definition}
\begin{example}
    \Lucy{see Example 4.2.1.17 of higher algebra}
\end{example}

\printbibliography

\end{document}
