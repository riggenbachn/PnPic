\documentclass{article}
\usepackage[utf8]{inputenc}
%Packages Used------------------------------------------
% the following is to get qoppa and Qoppa
\DeclareFontFamily{T1}{cbgreek}{}
\DeclareFontShape{T1}{cbgreek}{m}{n}{<-6>  grmn0500 <6-7> grmn0600 <7-8> grmn0700 <8-9> grmn0800 <9-10> grmn0900 <10-12> grmn1000 <12-17> grmn1200 <17-> grmn1728}{}
\DeclareSymbolFont{quadratics}{T1}{cbgreek}{m}{n}
\DeclareMathSymbol{\qoppa}{\mathord}{quadratics}{19}
\DeclareMathSymbol{\Qoppa}{\mathord}{quadratics}{21}

\usepackage{amsmath, amssymb, amsthm}
\usepackage{longtable}
%\usepackage{amsfonts}
%\usepackage{mathtools}
%\usepackage{wasysym}
%\usepackage{MnSymbol}
%\usepackage{thmtools}
\usepackage{stmaryrd} % wanted to use this for \varoast
\usepackage[letterpaper,margin=1in]{geometry}   
%\usepackage{slashed}
%\usepackage[english]{babel}				
%\usepackage[pdfencoding=auto, psdextra, draft=false]{hyperref}
\usepackage{bookmark}
\usepackage{url}		
\usepackage{lmodern}			
\usepackage[T1]{fontenc}
%\usepackage{xspace}		
%\usepackage{fancyhdr}
\usepackage{enumerate}
\usepackage{enumitem}
%\usepackage{mathrsfs}
\usepackage{graphicx}
\usepackage{soul,color}
\usepackage{tikz-cd}
\usepackage[maxbibnames=99,style=alphabetic]{biblatex}
\usepackage{csquotes}
\usepackage{chngcntr}
\usepackage[bbgreekl]{mathbbol}
\counterwithin{equation}{section}
\addbibresource{biblio.bib}
\usepackage{todonotes}

%hyperlink setup
\definecolor{darkred}{RGB}{128,0,0}
\definecolor{darkgreen}{RGB}{0,128,0}
\definecolor{darkblue}{RGB}{0,0,128}

\hypersetup{linktocpage,
	pdfborder = {0 0 0},
	colorlinks,
	citecolor=darkgreen,
	filecolor=darkred,
	linkcolor=darkblue,
	urlcolor=cyan!50!black!90}

%Greek and Latin black board bold-----------------------
\DeclareSymbolFontAlphabet{\mathbb}{AMSb}
\DeclareSymbolFontAlphabet{\mathbbl}{bbold}

%shortcut commands-------------------------------------------
\DeclareMathOperator{\Br}{Br} % Brauer functor
\DeclareMathOperator{\Brp}{Br^p} % Poincare Brauer functor
\DeclareMathOperator{\Spec}{Spec}
\DeclareMathOperator{\CAlg}{CAlg} % Commutative Algebra objects
\DeclareMathOperator{\Alg}{Alg}
\DeclareMathOperator{\CAlgp}{CAlg^p} % Poincare ring spectra
\DeclareMathOperator{\Cat}{Cat} % Categories
\DeclareMathOperator{\Catex}{\Cat_\infty^{ex}} % stable categories with exact functors
\DeclareMathOperator{\Cath}{Cat^h_\infty} % Hermitian Categories
\DeclareMathOperator{\Catp}{Cat^p_\infty} % Poincare Categories
\DeclareMathOperator{\Catpidem}{Cat^p_{\infty, idem}} % idempotent complete Poincare Categories
\DeclareMathOperator{\Einfty}{\mathbf{E}_\infty} % E-infinity 
\DeclareMathOperator{\ex}{ex} % exact 
\DeclareMathOperator{\Fun}{Fun} % Functors
\DeclareMathOperator{\gp}{gp} % grouplike
\DeclareMathOperator{\id}{id} % identity
\DeclareMathOperator{\idem}{idem} % idempotent
\DeclareMathOperator{\Mod}{Mod} % Modules
\DeclareMathOperator{\LMod}{LMod} % Left modules
\DeclareMathOperator{\BiMod}{BiMod} % Bimodules
\DeclareMathOperator{\Modh}{{}^{\sigma}Mod} % Hermitian modules
%\DeclareMathOperator{\op}{op} % opposite functor
\DeclareMathOperator{\Pic}{Pic} % Picard functor
\DeclareMathOperator{\Picp}{Pic^p} % Poincare Picard functor
\DeclareMathOperator{\Pn}{Pn} % Poincare space functor
\DeclareMathOperator{\Spectra}{Sp} % Spectra
\DeclareMathOperator{\Spaces}{\mathcal{S}} % Spaces
\DeclareMathOperator{\sSet}{sSet}
\DeclareMathOperator{\Mon}{Mon} % monoids

\newcommand{\pf}{{\bf Proof. \ }}
\renewcommand{\epsilon}{\varepsilon}
\renewcommand{\rho}{\varrho}
\renewcommand{\phi}{\varphi}
\newcommand{\NN}{\ensuremath{\mathbb{N}}\xspace}
\newcommand{\ZZ}{\ensuremath{\mathbb{Z}}\xspace}
\newcommand{\QQ}{\ensuremath{\mathbb{Q}}\xspace}
\newcommand{\RR}{\mathbb{R}}
\newcommand{\CC}{\ensuremath{\mathbb{C}}\xspace}
\newcommand{\FF}{\ensuremath{\mathbb{F}}\xspace}
\newcommand{\EE}{\mathbb{E}}
\renewcommand{\AA}{\mathbb{A}}
\newcommand{\TT}{\ensuremath{\mathbb{T}}\xspace}
\newcommand{\RP}{\ensuremath{\mathbb{RP}}\xspace}
\newcommand{\DD}{\ensuremath{\mathbbl{\Delta}}\xspace}
\newcommand{\tc}{\ensuremath{\mathrm{TC}}}
\newcommand{\thh}{\ensuremath{\mathrm{THH}}}
\newcommand{\tp}{\ensuremath{\mathrm{TP}}}
\newcommand{\tr}{\ensuremath{\mathrm{TR}}}
\newcommand{\pnpic}{\ensuremath{\mathrm{PnPic}}}
\newcommand{\pnbr}{\ensuremath{\mathrm{PnBr}}}
\newcommand{\pic}{\ensuremath{\mathrm{Pic}}}
\newcommand{\br}{\ensuremath{\mathrm{Br}}}
\DeclareMathOperator*{\colim}{\ensuremath{\operatorname{colim}}}
\newcommand{\aps}{\mathrm{APS}}
\newcommand{\psch}{\mathrm{PSch}}
\newcommand{\perf}{\mathrm{Perf}}
\newcommand{\perfpn}{\mathrm{Perf}^{\mathrm{Pn}}}
\newcommand{\op}{\mathrm{op}}
\newcommand{\Assoc}{\mathrm{Assoc}}
\newcommand{\Associnv}{\mathrm{Assoc}_\sigma}
\newcommand{\BMinv}{\mathrm{BM}_\sigma} % for `hermitian' bimodules; made a macro so we can change this later easily. 
\newcommand{\Fin}{\mathrm{Fin}} % finite sets 
\newcommand{\epoly}{\mathrm{epoly}}

%Theorem Environments ----------------------------------------------------------------
\newtheorem{theorem}[equation]{Theorem}
\newtheorem{proposition}[equation]{Proposition}
\newtheorem{lemma}[equation]{Lemma}
\newtheorem{corollary}[equation]{Corollary}

\theoremstyle{definition}
\newtheorem{definition}[equation]{Definition}
\newtheorem{construction}[equation]{Construction}
\newtheorem{remark}[equation]{Remark}
\newtheorem{remarks}[equation]{Remarks}
\newtheorem{observation}[equation]{Observation}
\newtheorem{notation}[equation]{Notation}
\newtheorem{example}[equation]{Example}
\newtheorem{question}[equation]{Question}
\newtheorem{recollection}[equation]{Recollection}
\newtheorem{variant}[equation]{Variant}

\newcommand{\Viktor}[1]{\todo{V: #1}}
\newcommand{\Noah}[1]{\todo[color=red]{N: #1}}
\newcommand{\Lucy}[1]{\todo[color=cyan!30]{\footnotesize L: #1}}
\newcommand{\Lucyil}[1]{\todo[inline,color=cyan!30]{\footnotesize L: #1}}

\title{Et cetera}
\author{Viktor Burghardt, Noah Riggenbach, Lucy Yang}
\date{}
\addbibresource{biblio.bib}

\begin{document}

\maketitle
\begin{abstract}
   Dumping ground for other stuff: Notes, one-off observations, stuff that we can collectively use when preparing talks, etc. \Lucy{I make no promises re: organization but I will do my best to keep it reasonably readable} 
\end{abstract}
\tableofcontents

\section{Talk prep}

\section{References}
\begin{itemize}
    \item \href{https://ems.press/journals/dm/articles/8965687}{Involutions of Azumaya algebras} by First and Williams (2020 \emph{Documenta})
    \item \href{https://arxiv.org/abs/2405.15260}{Counterexamples in involutions of Azumaya algebras} by First and Williams; much more readable than the 2020 Documenta paper 
    \item \href{https://arxiv.org/abs/1510.06133}{Azumaya algebras without involution} by Auel, First, and Williams: the introduction of this provides a very helpful historical overview of the connection between involutions on Azumaya algebras and 2-torsion/kernel of coRes
\end{itemize}
\section{Questions and directions}
\begin{question}
    [Literature]
\begin{itemize}
    \item In \cite{MR1162189} Parimala--Srinivas assume that 2 is invertible in the ring of functions. 
    Has anyone been able to extend their results to the 2 not necessarily invertible case in the meantime? 
\end{itemize}
\end{question}
\begin{question}
    [Morita theory for $ \Catp $]
    Let $ R $ be a Poincaré ring. 
    Suppose given two $ R $-algebras (suitably interpreted so their module categories are canonically endowed with $ R $-linear Poincaré structures--perhaps $ \mathbb{E}_\sigma $) $ A $, $ B $. 
    Can we characterize
    \begin{equation*}
        \hom_{\Catp_R}\left(\left(\Mod_A^\omega,\Qoppa_A\right),\left(\Mod_B^\omega,\Qoppa_B\right)\right)
     \end{equation*} 
     in terms of something bimodule-like? 
\end{question}
\begin{question}
    On page 2 of the \emph{Counterexamples} paper, First and Williams write that `` existence of an extraordinary involution means classificaiton of Azumaya algebras with involution...\emph{cannot} be reduced to questions about projective modules and hermitian forms on them.'' 

    What if we replaced projective modules by perfect complexes? 
\end{question}
\begin{question}
    First--Williams show (see discussion in \S4 of the \emph{Counterexamples} paper) that coarse type classify many (most?) Azumaya algebras up to (étale-local) \emph{isomorphism}. 

    What is a suitable derived version of ``coarse type''?
\end{question}

\begin{question}
    [asked by Andrew Nov 2, 2024] 
    C. Schlichtkrull shows in \href{https://arxiv.org/pdf/math/0405079}{this paper} that a map $ BGL_1(R) \to K(R) \to THH(R) \to R $ in terms of the Hopf map $ \eta $. 

    Is there a ``Poincaré'' version of this result? 
\end{question}

\begin{question}
    Are there some general conditions for a ring with involution $ R $ so that the inclusion $ R^{C_2} \to R $ is `nice'?

    There's some stuff in section 4 \href{https://www.math.nagoya-u.ac.jp/~hasimoto/paper/knop2.pdf}{here}, idk. 
    Also see M. Hochster and J. L. Roberts, Rings of invariants of reductive groups acting on regular rings are Cohen–Macaulay. 
\end{question}

\paragraph{Applications to Hodge theory?} 
\begin{itemize}
    \item \href{https://arxiv.org/abs/1501.05294}{A Deligne pairing for Hermitian Azumaya modules}
    \item \href{https://www.math.ucla.edu/~totaro/papers/public_html/structure.pdf}{Burt Totaro's paper} on Hodge structures of type $ (n,0,\ldots, 0, n) $
    \item There's some stuff about endomorphisms of Hodge structures \href{https://webusers.imj-prg.fr/~claire.voisin/Articlesweb/verbanianotes.pdf}{here}
\end{itemize}

\section{Thoughts \& observations}
\begin{question}
   When $ R $ has the Tate Poincaré structure and $ (\Mod_A^\omega, M_A, N_A, N_A \to M_A^{tC_2}) $ is invertible, then by invertibility have an equivalence $ \hom_R(A, R)\simeq N_A\otimes_R N_{A^\op} $ of $ A \otimes_R A^\op $-modules. 
   Restricting the left-hand side along the unit map $ R \to A $ gives a map $ N_A \otimes_R N_{A^\op} \to \hom_R(R,R) \simeq R $. 
   Is this a perfect ($R$-linear) pairing? 

   I \emph{think} using that $ R^{\varphi C_2} \simeq R $ and combining the linear and bilinear part conditions, we get something like
   \begin{equation*}
       M_A \otimes_R M_{A^\op} \simeq (N_A \otimes_R N_{A^\op})^{\otimes_R 2} \qquad \text{ as $A \otimes_R A^\op$-bimodules. }
   \end{equation*}
   Is this useful?
\end{question}

\paragraph{Brauer-Severi schemes} 
We know there is a correspondence between Azumaya algebras $ A $ over $ X $ and Brauer-Severi schemes. 
What does a Poincaré structure on $ \Mod_A^\omega $ mean `geometrically' for $ D^b_{\mathrm{coh}} $ of the corresponding Brauer-Severi scheme? 
(Lucy: I didn't get very far here, but just typing up what I had)
\begin{itemize}
    \item $ \Mod_A^\omega $ corresponds to $ \alpha $-twisted sheaves on $ X $ (see Proposition 3.2.2.1 of Max Lieblich's thesis)
    \item The bounded derived category of $ \alpha $-twisted sheaves on $ X $ includes as one `piece' of a semiorthogonal decomposition on $ D^b_{\mathrm{coh}} $ of the corresponding Brauer-Severi scheme (see Theorem 5.1 \href{https://arxiv.org/abs/math/0511497}{here})
\end{itemize}

\section{Desperate Flailing}

This section is a cronical of my thoughts about $\mathbb{G}_m^\Qoppa$.
\paragraph{Goal} The goal is to build a Poincar{\'e} ring $\mathbb{G}_{m}^\Qoppa:=(\mathrm{Mod}_R, \Qoppa_R)$  such that $B\mathbb{G}_m^\Qoppa(\underline{S}) = \Picp(\underline{S})$ for any Poincar{\'e} ring $\underline{S}$.
\begin{lemma}
Let $\underline{S}$ be a Poincar{\'e} ring. Then $\pi_0(\mathrm{Aut}_{\mathrm{Pn}(\mathrm{Mod}_S)}(S,u))=\{s\in \pi_0(S)^\times | s=1 \textrm{ in }\pi_0(S^{C_2})\}$.
\end{lemma}
\begin{proof}
Since the functor $\mathrm{Pn}(\mathrm{Mod}_S)\to \mathrm{Mod}_S$ is conservative it follows that an element of $\pi_0(\mathrm{Aut}_{\mathrm{Pn}(\mathrm{Mod}_S)}(S,u))$ must have underlying map an element of $\pi_0\mathrm{Aut}(S)=\pi_0(S)^\times$. Then in order for $s\in \pi_0(S)^\times$ to induce a map $(S,u)\to (S,u)$, the induced map $s^*:S^{C_2}\to S^{C_2}$ must satisfy $s^*(u)=u$. The pullback is given by multiplication by $s$, so this requirement translates into $s$ being the unit, as desired.
\end{proof}

The problem I thought existed maybe doesn't. Here is a candidate construction:

\begin{construction}
Define $R$ to be the $\mathbb{E}_\infty$ ring given by $\mathbb{S}\{x^{\pm 1}, y^{\pm 1}\}\otimes_{\mathbb{S}\{z\}}\mathbb{S}$ where the map $\mathbb{S}\{z\}\to \mathbb{S}\{x^{\pm 1}, y^{\pm 1}\}$ is induced by the map $z\mapsto xy$, and the map $\mathbb{S}\{z\}\to \mathbb{S}$ is induced by $z\mapsto 1$. We can give $R$ an $\mathbb{E}_\infty$ ring structure in $\mathrm{Sp}^{BC_2}$ by taking the trivial action on $\mathbb{S}\{z\}$ and $\mathbb{S}$, and taking the action induced by $x\mapsto y$ and $y\mapsto x$ on $\mathbb{S}\{x^{\pm 1}, y^{\pm 1}\}$. Thus in $\mathrm{CAlg}(\mathrm{Sp}^{BC_2})$ the ring $R$ corepresents the functor $S\mapsto \{s\in \pi_0(S)^\times| s\sigma(s)=1\}$.

Now take $\underline{R}$ to be the Poincar{\'e} ring with underlying Borel $C_2$ structure as described in the previous paragraph and geometric fixed points $R^{\phi C_2}=\mathbb{S}$ and the map $R^{\phi C_2}\to R^{tC_2}$ given by the unit map. Endowing $R^{\phi C_2}$ with the $R$-module structre given by $x,y\mapsto 1$, it remains to show that the unit map $R^{\phi C_2}\to R^{tC_2}$ factors the Tate valued Frobenius $R\to R^{tC_2}$ in order to promote $\underline{R}$ to a Poincar{\'e} ring. By construction of $R$ it is then enough to show that on $\pi_0$ the Tate valued Frobenius sends $x,y\mapsto 1$ in $\pi_0(R^{tC_2})$. This map sends both $x$ and $y$ to $xy\in \pi_0(R^{tC_2})$. These areequal to $1$ in $\pi_0(R^{tC_2})$ since the functor $(-)^{tC_2}$ is lax-monoidal so $R^{tC_2}$ is a modules over $\mathbb{S}\{x^{\pm 1}, y^{\pm 1}\}^{tC_2}\otimes_{\mathbb{S}\{z\}^{tC_2}}\mathbb{S}^{tC_2}$ which has the image of $xy$ equal to $1$. 
\end{construction}

Now consider another Poincar{\'e} ring $\underline{S}$. We then have that maps $\pi_0(\mathrm{Maps}(\underline{R},\underline{S}))$ is the data of a unit $s\in \pi_0(S)^\times$, a path $s\sigma(s)\to 1$ in $\Omega^\infty S$, and paths $x,y\to 1$ in $\Omega^\infty S^{\phi C_2}$.  This then agrees with $\mathbb{G}_m^\Qoppa$ by the following lemma.

\begin{lemma}
Let $S\in \mathrm{CAlg}(\mathrm{Sp}^{BC_2})$ and $s\in \pi_0(S)^\times$. Then $s\sigma(s)=1$ in $\pi_0(S)$ if and only if $(s\otimes s)^*$ acts by $1$ on $\pi_0(S^{hC_2})=\pi_0(\mathrm{Hom}_{S\otimes S}(S\otimes S, S)^{hC_2})$.
\end{lemma}
\begin{proof}
The 'only if' direction follows from the fact that the map $S^{hC_2}\to S$ is an $S$-bimodule map. Now suppose that $s\sigma(s)=1$ in  $S$. Then before taking homotopy fixed points the induced map $s^*=id$ because $S$ is $\mathbb{E}_\infty$.\footnote{Or just $\mathbb{E}_2$.} 
\end{proof}

\section{Modules with genuine involution} 
\begin{remark}
    [Lucy] I'm just going to put drafts of stuff pertaining to hermitian modules\Lucy{or whatever we want to keep calling these} here. 
    Eventually when it gets to be more complete, I will hopefully move this entire section over to the main file. 
\end{remark}
\paragraph{Meta-commentary} There are (at least) three things we want to do: 
\begin{enumerate}[label=(\alph*)]
    \item Define a category of `bimodules with involution over algebras with anti-involution' equipped with a forgetful functor $ \Theta \colon \mathrm{BMod}_{\mathrm{inv}}(-) \to \EE_1\Alg(-)^{hC_2} $. 
    \item Show that $ \Theta $ is a coCartesian fibration. 
    For this, it suffices to show that it is a \emph{Cartesian} fibration and that it satisfies the hypotheses of \cite[Corollary 5.2.2.5]{HTT}
    \begin{itemize}
        \item I used to think that we could obtain this by `bootstrapping' a result from Higher Algebra, plus some facts about assembly. 
        This doesn't seem to be working, so I'm just going to try to do this directly (imitating certain aspects of Chapter 4 of higher algebra.) 
    \end{itemize}
    \item Define a relative tensor product for hermitian bimodules 
    \item Show that the formula for the cocartesian pushforward along a map $ A \to B $ in $ \EE_1\Alg(-)^{hC_2} $ is something like $ - \otimes_{A \otimes A^\op} \left(B \otimes B^\op \right) \otimes_{B\otimes B^\op} B $. 
    \begin{itemize}
        \item In Higher Algebra, the formula for the cocartesian pushforward is proven in \cite[\S4.6]{LurHA}; in particular, this is in the section on duality. 
        In particular, see Proposition 4.6.2.17 and the paragraph immediately preceding this.  
        \item I don't know how to do this yet--while (a) and (b) are not useful if I can't show (c), I can't suss out the feasibility of (c) without (a) and (b) already in place. 
    \end{itemize} 
    \item Towards an adjunction between $ \EE_\sigma $-algebras and categories with additional structure.  
    \begin{itemize}
        \item Involutive version of statement that, for a monoidal $ \infty $-category $ \mathcal{C} $ and an $ \EE_1 $-algebra $ A $, $ \mathrm{LMod}_A(\mathcal{C}) $ is right-tensored over $ \mathcal{C} $? 
        \item Involutive version of endomorphism categories? \cite[\S4.7.1]{LurHA}
    \end{itemize}
\end{enumerate} 
I think that the equivalence of part (b) of the definition of an Azumaya algebra with genuine involution follows from the property of being Azumaya; see Lemma 1(b) (and p.216 for the `type 2' case) of \cite{MR1162189}. 
\begin{lemma}
    Let $ R $ be an $ \EE_\infty $-ring with an involution $ \sigma \colon R \xrightarrow{\sim} R$ and suppose $ A $ is an $ \EE_1 $-$ R $-algebra with an anti-involution $ \lambda \colon A \xrightarrow{\sim} \sigma^* A^\op $. 
    Suppose $ A $ is further Azumaya in the sense of \Lucy{reference}. 
    Then the bilinear pairing 
    \begin{equation*}
    \begin{split}
        A \otimes_R \sigma^* A \xrightarrow{\id \otimes \sigma^*\lambda} A \otimes_R A^\op \simeq \mathrm{End}_R(A) \xrightarrow{\mathrm{tr}} R     
    \end{split}    
    \end{equation*}
    is perfect, i.e. its adjoint $ A \to (\sigma^* A)^\vee $ is an equivalence. 
\end{lemma}
\begin{question}
    Does the map in part (e) of the definition of an Azumaya algebra with genuine involution follow from property of being Azumaya? 
\end{question}

\subsection{Step (a)}
\begin{definition}\label{defn:colored_operad_monoid_with_involution}
\Lucy{This is just an imitation of \cite[Definition 4.1.1.1]{LurHA}, modified in accordance with ideas from \S5.4.2. }
    Define a colored operad $ \Associnv $ as follows:
    \begin{enumerate}[label=(\roman*)]
        \item The colored operad has a single object, which we denote by $ \mathfrak{a} $. 
        \item For every finite set $ I $, the set of operations $ \mathrm{Mul}_{\Associnv}\left(\left\{\mathfrak{a}_i\right\}_{i \in I}, \mathfrak{a}\right) \simeq \mathcal{L}I \times \{\pm 1\}^I $, where $ \mathcal{L}I $ is the set of linear orderings on $ I $ and an element of $ \{\pm 1\}^I $ is a function $ I \to \{\pm 1 \} $. 
        \item Suppose given a map of finite sets $ \alpha \colon I \to J $, together with operations $ (\preceq_j, f_j \colon I_j \to \{\pm 1\}) \in \mathrm{Mul}_{\Associnv}\left(\left\{\mathfrak{a}_i\right\}_{\alpha(i)=j}, \mathfrak{a}\right) $ and $ (\preceq_J, g \colon J \to \{\pm 1\}) \in \mathrm{Mul}_{\Associnv}\left(\left\{\mathfrak{a}_j\right\}_{j\in J}, \mathfrak{a}\right) $. 
        Define a linear ordering on the set $ I $ as follows: $ i \leq i' $ if $ \alpha(i) \preceq_J \alpha(i') $ or $ \alpha(i) = \alpha(i')= j $ and $ i \preceq_j i' $ and $ g(j)= +1 $ or $ \alpha(i) = \alpha(i')= j $ and $ i \succeq_j i' $ and $ g(j)= -1 $. 
        Finally, define a function 
        \begin{align*}
            I &\to \{\pm 1\} \\
            i &\mapsto f_{\alpha(i)} (i) \cdot  g(\alpha(i))\,,
        \end{align*}  
        where the multiplication on $ \{\pm 1\} $ is the usual one. 
    \end{enumerate} 
\end{definition}
\begin{remark}\label{rmk:assoc_opd_to_assoc_inv_opd}
    There is a map of colored operads $ \iota \colon \mathrm{Assoc} \to \Associnv $ which is the identity on objects and on operations $ \mathrm{Mul}_{\mathrm{Assoc}}\left(\left\{\mathfrak{a}_i\right\}_{i \in I}, \mathfrak{a}\right) \simeq  \mathcal{L}I \to \mathrm{Mul}_{\Associnv}\left(\left\{\mathfrak{a}_i\right\}_{i \in I}, \mathfrak{a}\right) \simeq  \mathcal{L}I \times \{\pm 1\}^I $ is $ \mathrm{id}_{\mathcal{L}I} \times \{ c_1\} $ where $ c_1 $ is the constant function on $ I $ with value $ 1 $. 

    There is another map of colored operads $ \iota^{\mathrm{rev}} \colon \mathrm{Assoc} \to \Associnv $ which is the identity on objects and on operations $ \mathrm{Mul}_{\mathrm{Assoc}}\left(\left\{\mathfrak{a}_i\right\}_{i \in I}, \mathfrak{a}\right) \simeq  \mathcal{L}I \to \mathrm{Mul}_{\Associnv}\left(\left\{\mathfrak{a}_i\right\}_{i \in I}, \mathfrak{a}\right) \simeq  \mathcal{L}I \times \{\pm 1\}^I $ sends a linear ordering $ \ell $ to $ (\ell^{\mathrm{rev}}, c_{-1})$ where $ c_{-1} $ is the constant function on $ I $ with value $ 1 $. 
\end{remark}
\begin{definition}\label{defn:infty_operad_monoid_with_involution}
    Let $ \Associnv^\otimes $ denote the associated $ \infty $-operad (via Construction 2.1.1.7 and Example 2.1.1.21 of \cite{LurHA}). 
\end{definition} 
\begin{remark}
    Unwinding definitions
    \begin{itemize}
        \item Objects $ \Associnv^\otimes $ are finite pointed sets $ \langle n \rangle \in \mathrm{Fin}_* $ 
        \item Morphisms $ \langle m \rangle \to \langle n \rangle $ consist of 
        \begin{itemize}
            \item $ \alpha \colon \langle m \rangle \to \langle n \rangle $ a map of finite pointed sets
            \item for each $ i \in \langle n \rangle^\circ $, a linear ordering $ \preceq_i $ on the inverse image $ \alpha^{-1}(\{i\}) $ 
            \item a map of sets $ s \colon  \alpha^{-1}\left(\langle m \rangle^\circ\right) \to \{\pm 1\} $
        \end{itemize}
        \item For each pair of morphisms
        \begin{equation*}
             \left(\beta \colon \langle \ell \rangle \to \langle m \rangle, \preceq_j, s\right) \qquad \left(\alpha \colon \langle m \rangle \to \langle n \rangle, \preceq_i, t\right) \,,
        \end{equation*}
        the composite is the triple $ \left(\alpha \circ \beta, \preceq_j'', u \right) $ where $ \preceq_j'' $ is the ordering on $ (\alpha \circ \beta)^{-1}(\{i\}) $ so that if $ a,b \in \langle \ell \rangle $ so that $ \alpha (\beta(a)) = \alpha(\beta(b)) $, then $ a \preceq_j'' b $ if $ \beta(a) \preceq_i \beta(b) $ or $ \beta(a) =_i \beta(b) = i $ and $ a \preceq_i b $ if $ s(i) = 1 $ or $ a \succeq_i b $ if $ s(i) = -1 $. 
        Finally $ u (l) = s(l) \cdot t(\beta(l))  $. 
        \Lucy{Note that when $ s, t$ are identically one, the resulting order $ \preceq_j''$ agrees with the lexicographic order defined in \cite[Remark 4.1.1.4]{LurHA}.}  
    \end{itemize}
\end{remark}
\begin{remark}\label{rmk:assoc_infty_opd_to_assoc_inv_infty_opd}
    The maps $ \iota, \iota^{\mathrm{rev}} $ of Remark \ref{rmk:assoc_opd_to_assoc_inv_opd} induce maps of $ \infty $-operads $ \mathrm{Assoc}^\otimes \to \Associnv^\otimes $. 
    There is a canonical identification $ \iota^{\mathrm{rev}} = \sigma \circ \iota $, where $ \sigma $ is the automorphism of the associative operad considered in \cite[Remark 4.1.1.7]{LurHA}. 

    Note that each object $ \langle n \rangle \in \Associnv^\otimes $ has a distinguished automorphism $ \mathrm{rev}_{\langle n \rangle} $ of order two given by the identity map on $ \langle n \rangle $ and the constant map $ c_{-1} \colon \langle n \rangle^\circ \to \{\pm 1\} $ at $ -1 $. 
    There is a canonical natural equivalence $ \iota \overset{\sim}{\implies} \iota^{\mathrm{rev}} $ whose component at $ \langle n \rangle $ is $ \mathrm{rev}_{\langle n \rangle} $. 
\end{remark}
\begin{definition}\label{defn:naive_involutive_algebras}
    Let $ \mathcal{C}^\otimes  $ be a $ \infty $-operad equipped with the data of a fibration\Lucy{do we need weaker than cocartesian fibration?} $ p \colon \mathcal{C}^\otimes \to \Associnv^\otimes $. 
    Let $ \Alg^\sigma (\mathcal{C}) $ denote the $ \infty $-category $ \Alg_{/\Associnv}(\mathcal{C}) $ of $ \infty $-operad sections of $ p $. 
    We will refer to $ \Alg^\sigma(\mathcal{C}) $ as the $ \infty $-category of \emph{involutive algebra objects of $ \mathcal{C}$}. 

    An \emph{involutive monoidal $ \infty $-category} is the data of a cocartesian fibration $ \mathcal{C}^\otimes \to \Associnv^\otimes $. 
\end{definition}
\begin{remark}\label{rmk:assoc_inv_mon_cat_as_mon_cat_with_autoequiv}
    Suppose given a cocartesian fibration $ f \colon \mathcal{D}^\otimes \to \Associnv^\otimes $ of $ \infty $-operads. 
    Write $ \mathcal{C}^\otimes := \mathcal{D}^\otimes \times_{\Associnv^\otimes, \iota} \mathrm{Assoc}^\otimes $; $ \mathcal{C}^\otimes $ is a monoidal $ \infty $-category in the sense of \cite[Definition 4.1.1.10]{LurHA}. 
    Furthermore, $ \mathcal{C}^\otimes_{\mathrm{rev}}:= \mathcal{D}^\otimes \times_{\Associnv^\otimes, \iota^{\mathrm{rev}}} \mathrm{Assoc}^\otimes $ is a monoidal $ \infty $-category. 
    By Remark \ref{rmk:assoc_infty_opd_to_assoc_inv_infty_opd}, this notation is consistent with that of \cite[Remark 4.1.1.7]{LurHA}. 
    In particular, a $ \Associnv $-monoidal $ \infty $-category $ \mathcal{D}^\otimes $ determines a monoidal $ \infty $-category $ \mathcal{C}^\otimes $ equipped with a monoidal equivalence $ \sigma_{\mathcal{C}} \colon \mathcal{C}^\otimes \xrightarrow{\sim} \mathcal{C}^\otimes_{\mathrm{rev}} $. 
    Pullback along the involution of $ \mathrm{Assoc}^\otimes $ determines another monoidal equivalence $ \sigma_{\mathcal{C}}^{\mathrm{rev}} \colon \mathcal{C}^\otimes_{\mathrm{rev}} \xrightarrow{\sim} \mathcal{C}^\otimes $, and our assumptions imply that $ \sigma_{\mathcal{C}}^{\mathrm{rev}} \circ \sigma_{\mathcal{C}} $ is equivalent to the identity on $ \mathcal{C}^\otimes $. 

    Now suppose that $ A $ is an involutive algebra object of $ \mathcal{D} $. 
    With the same notation as before, pullback along $ \iota $ (resp. $ \iota^{\mathrm{rev}} $) determines associative algebra objects $ u(A) $, $ u^{\mathrm{rev}}(A) $ of $ \mathcal{C} $ and $ \mathcal{C}_{\mathrm{rev}} $, respectively. 
    Note that $ \sigma_{\mathcal{C}}(u(A)) $ is an algebra object of $ \mathcal{C}_{\mathrm{rev}} $, which we may regard as an algebra object of $ \mathcal{C} $ by precomposing with the autoequivalence $ \sigma \colon \mathrm{Assoc}^\otimes \xrightarrow{\sim} \mathrm{Assoc}^\otimes $. 
    It follows from Remark \ref{rmk:assoc_infty_opd_to_assoc_inv_infty_opd} that $ A $ determines an equivalence $ \sigma_A \colon u(A) \xrightarrow{\sim} \sigma_{\mathcal{C}}(u(A))^{\mathrm{rev}} $ of algebra objects in $ \mathcal{C} $. 

    Now suppose furthermore that $ \mathcal{D}^\otimes $ is of the form $ \mathcal{E}^\otimes \times_{\mathrm{Fin}_*} \Associnv^\otimes $ for some symmetric monoidal $ \infty $-category $ \mathcal{E} $. 
    Then the associated involution $ \sigma_{\mathcal{C}} $ is the identity, and for any involutive algebra object $ A $ of $ \mathcal{D} $, $ \sigma_A $ is an equivalence $ u(A) \simeq u(A)^{\mathrm{rev}} $ satisfying $ \sigma_A^{\mathrm{rev}} \circ \sigma_A \simeq \mathrm{id}_A $. 
\end{remark}
\begin{remark}\label{rmk:naive_inv_alg_as_assoc_alg}
    Suppose given a cocartesian fibration $ f \colon \mathcal{D}^\otimes \to \Associnv^\otimes $ of $ \infty $-operads, and write $ \mathcal{C} $ for its underlying monoidal $ \infty $-category with equivalence $ \sigma_{\mathcal{C}} $ as in Remark \ref{rmk:assoc_inv_mon_cat_as_mon_cat_with_autoequiv}. 
    The functors of Remark \ref{rmk:assoc_infty_opd_to_assoc_inv_infty_opd} induce a functor 
    \begin{equation*}
        \Alg^\sigma(\mathcal{D}) \to \Alg(\mathcal{C}) \times \Alg(\mathcal{C}_{\mathrm{rev}}) \,;
    \end{equation*}
    moreover, the aforementioned functor factors through the fixed points of the $ C _2 $-action on the right-hand side given by $ \Alg(\mathcal{C}) \times \Alg(\mathcal{C}_{\mathrm{rev}}) \xrightarrow{\mathrm{swap}} \Alg(\mathcal{C}_{\mathrm{rev}}) \times \Alg(\mathcal{C}) \xrightarrow{\sigma_{\mathcal{C}} \times \sigma_{\mathcal{C}}} \Alg(\mathcal{C}) \times \Alg(\mathcal{C}_{\mathrm{rev}}) $. 
\end{remark}
\begin{definition}
    Define a category $ \Delta_\sigma $ 
    \begin{itemize}
        \item objects are pairs $ ([n], s \colon \{1, \ldots, n\} \to \{\pm 1\}) $ \Lucy{maybe better to write $ s $ as a function defined on the set of \emph{morphisms} $ i < i+1 $ in $[n]$.}
        \item a morphism from $ ([n], s \colon \{1, \ldots, n\} \to \{\pm 1\}) $ to $ ([m], t \colon \{0, 1, \ldots, m\} \to \{\pm 1\}) $ is an order-preserving map $ [n] \to [m] $ in $ \Delta $. 
    \end{itemize}
\end{definition}
\begin{construction}\label{cons:involutive_cut}
    Define a functor $ \mathrm{Cut} \colon \Delta_\sigma^\op \to \Associnv^\otimes $: 
    \begin{itemize}
        \item For each $ ([n],s) $, we have $ \mathrm{Cut}([n],s) = \langle n \rangle $. 
        \item Given a morphism $ \alpha\colon ([n],s) \to ([m], t) $, the associated morphism $ \mathrm{Cut} ([n],s) \to \mathrm{Cut}([m], t) $ consists of
        \begin{itemize}
            \item On underlying finite pointed sets $ \langle m\rangle \to \langle n \rangle $, $ \mathrm{Cut} $ agrees with that appearing in \cite[Construction 4.1.2.9]{LurHA}
            \item Identifying the cut $ \{k \mid k< j\} \sqcup \{k \mid k \geq j \} $ with the morphism $ j-1 < j $, we may regard $ s \colon \langle n\rangle^\circ \to \{\pm 1\} $ and likewise $ t \colon \langle m\rangle^\circ \to \{\pm 1\} $. 
            Define $ u\colon \mathrm{Cut}(\alpha)^{-1}\left(\langle n\rangle^\circ\right) \to \{\pm 1\} $ to be the unique function so that $ u(j)t(j) = s(\mathrm{Cut}(\alpha)(j)) $. 
            % This means that $ \alpha^{-1}(\{k \mid k< j\}) $ and $ \alpha^{-1}(\{k \mid k \geq j\}) $ 
        \end{itemize}
    \end{itemize}
\end{construction}
\begin{lemma}\label{lemma:involutive_cut_is_approximation}
    The functor $ \mathrm{Cut} \colon \Delta^\op_\sigma \to \Associnv^\otimes $ exhibits $ \Delta^\op_\sigma $ as an approximation to the $ \infty $-operad $ \Associnv^\otimes $. 
    \Lucyil{I think the proof of this lemma is not too different from the proof of Proposition 4.1.2.11 of \cite{LurHA}; the point here is just to unravel the definitions of locally coCartesian and Cartesian; the morphisms in $ \Delta^\op_\sigma $ are a little more complicated than $ \Delta^\op $, but not by much. } 
\end{lemma}
\begin{notation}\label{ntn:E_sigma_monoidal_cat_variant}
    Let $ \mathcal{C}^\otimes \to \Associnv^\otimes $ exhibit $ \mathcal{C} $ as $ \EE_\sigma $-monoidal. 
    Let $ \mathcal{C}^\varoast $ denote the fiber product $ \mathcal{C}^\otimes \times_{\Associnv^\otimes} \Delta^\op_{\sigma} $. 
\end{notation}
\begin{definition}\label{defn:inert_map_in_involutive_simplicial_cat}
    Say that a morphism $ ([n], s) \to ([m], t) $ is \emph{inert} if the induced map $ \mathrm{Cut}([m],t) \to \mathrm{Cut}([n],s) $ is an inert morphism in $ \Associnv^\otimes $. 
\end{definition}
\begin{definition}
    A \emph{$ \RR^\sigma $-planar operad} is an $ \infty $-category $ \mathcal{O}^\varoast $ equipped with a functor $ q \colon \mathcal{O}^\varoast \to \Delta^\op_\sigma $ so that 
    \begin{enumerate}
        \item For every object $ X \in \mathcal{O}^\varoast $ and every inert morphism $ \alpha\colon ([n],s) \to q(X) $ in $ \Delta_\sigma $, there is a $ q $-cocartesian morphism $ \overline{\alpha} \colon X \to Y $ satisfying $ q (\overline{\alpha}) = \alpha $
        \item Let $ X $ be an object satisfying $ q (X) = ([n],s) $, and choose $ q $-cocartesian morphisms $ \overline{\alpha}_i \colon X \to X_i $ corresponding to the morphism $ ([i-1< i], s_i) \to ([n], s) $ which is the inclusion on underlying sets and satisfies $ s_i(i) = s(i) $. 
        Then the morphisms $ \overline{\alpha}_i $ exhibit $ X $ as the $ q $-product of the $ X_i $. 
        \item For each $ n \geq 0 $, the construction $ C \mapsto \{C_i \}_{1 \leq i \leq n} $ induces an equivalence of $ \infty $-categories 
        \begin{equation*}
            \mathcal{O}^\varoast \times_{\Delta^\op_\sigma} \{([n], s)\} \xrightarrow{\sim} \left(\mathcal{O}^\varoast \times_{\Delta^\op_\sigma} \{([1], s|_{\{i\}})\}\right)^{\times n}
        \end{equation*}
    \end{enumerate}
    We say that a morphism $ \alpha $ in $ \RR^\sigma $-planar operad is \emph{inert} if it is $ q $-cocartesian and $ q (\alpha) $ is inert in $ \Delta^\op_\sigma $ in the sense of Definition \ref{defn:inert_map_in_involutive_simplicial_cat}. 
\end{definition}
\begin{definition}\label{defn:inv_planar_alg}
    Let $ q \colon \mathcal{O}^\varoast \to \Delta^\op_\sigma $ be a $ \RR^\sigma $-planar operad. 
    An $ \AA^\sigma_\infty $-algebra object of $ \mathcal{O}^{\varoast} $ is a section of $ q $ which carries inert morphisms to inert morphisms. 
    Write $ \Alg_{\AA^\sigma_\infty}(\mathcal{O}) $ for the full subcategory of $ \Fun_{\Delta^\op_{\sigma}}\left(\Delta^\op_\sigma, \mathcal{O}^\varoast\right) $ on $ \AA^\sigma_\infty $-algebra objects.  
\end{definition}
\begin{proposition}
    Let $ \mathcal{O}^\otimes \to \Associnv^\otimes $ be a fibration of $ \infty $-operads. 
    Then precomposition with the functor $ \mathrm{Cut} $ of Construction \ref{cons:involutive_cut} induces an equivalence of $ \infty $-categories
    \begin{equation*}
        \Alg_{\Associnv}(\mathcal{O}) \xrightarrow{\sim} \Alg_{\AA_\infty^\sigma}\left(\mathcal{O}\right)\,.
    \end{equation*}
\end{proposition}
\begin{proof}
    Combine Lemma \ref{lemma:involutive_cut_is_approximation} with \cite[Theorem 2.3.3.23]{LurHA}. 
\end{proof}
\begin{definition}\label{defn:inv_leftmod_operad}
    Define a colored operad $ \mathbf{LM}_\mathrm{inv} $
    \begin{enumerate}[label=(\roman*)]
        \item The set of objects of $ \mathbf{LM}_\mathrm{inv} $ has two elements, which we denote by $ \mathfrak{a}, \mathfrak{m} $. 
        \item Let $ \{X_i\}_{i \in I} $ be a finite collection of objects of $ \mathbf{LM}_{\mathrm{inv}} $ and let $ Y $ be another object of $ \mathbf{LM}_{\mathrm{inv}} $. 
        If $ Y = \mathfrak{a} $, then $ \mathrm{Mul}_{\mathbf{LM}_\mathrm{inv}} \left(\{X_i\}_{i \in I}, Y\right) $ is the set of pairs consisting of a linear ordering on $ I $ and a function $ I \to \{\pm 1\} $ if $ X_i = \mathfrak{a} $ for all $ i $, and empty otherwise. 
        If $ Y = \mathfrak{m} $, then $ \mathrm{Mul}_{\mathbf{LM}_\mathrm{inv}} \left(\{X_i\}_{i \in I}, Y\right) $ is a subset of the set of pairs $ (\lambda, c) $ consisting of a linear ordering $ \lambda = \{i_1 < i_2 < \cdots < i_n\} $ on $ I $ and a function $ c\colon I \to \{\pm 1 \} $ satisfying either 
        \begin{itemize}
            \item  $ X_{i_n} = \mathfrak{m} $ and $ c(i_n) = 1 $ and $ X_{j} = \mathfrak{a} $ otherwise
            \item  $ X_{i_1} = \mathfrak{m} $ and $ c(i_n) = -1 $ and $ X_{j} = \mathfrak{a} $ otherwise
            % \item if there is exactly one index $ i_k $ so that $ X_{i_k} = \mathfrak{m} $ and $ X_j = \mathfrak{a} $ for all $ j \neq i_k $, and $ \mathrm{Mul}_{\mathbf{LM}_\mathrm{inv}} \left(\{X_i\}_{i \in I}, Y\right) $ is empty otherwise. 
        \end{itemize}
        \item The composition law on $ \mathbf{LM}_{\mathrm{inv}} $ is determined by the composition of linear orderings, with reversal of linear orderings according to Definition \ref{defn:colored_operad_monoid_with_involution} 
    \end{enumerate}
\end{definition} 
\begin{remark}
    There is a colored operad $ \mathbf{RM}_{\mathrm{inv}} $ defined exactly in the same way as $ \mathbf{LM}_{\mathrm{inv}} $ in Definition \ref{defn:inv_leftmod_operad}. 
    In the interest of precision: $ \mathbf{RM}_{\mathrm{inv}} $ has the same objects $ \mathfrak{a}, \mathfrak{m} $. 
    Let $ \{X_i\}_{i \in I} $ be a finite collection of objects of $ \mathbf{RM}_{\mathrm{inv}} $ and let $ Y $ be another object of $ \mathbf{RM}_{\mathrm{inv}} $. 
    If $ Y = \mathfrak{m} $, then $ \mathrm{Mul}_{\mathbf{RM}_\mathrm{inv}} \left(\{X_i\}_{i \in I}, Y\right) $ is a subset of the set of pairs $ (\lambda, c) $ consisting of a linear ordering $ \lambda = \{i_1 < i_2 < \cdots < i_n\} $ on $ I $ and a function $ c\colon I \to \{\pm 1 \} $ satisfying either 
    \begin{itemize}
        \item  $ X_{i_n} = \mathfrak{m} $ and $ c(i_n) = -1 $ and $ X_{j} = \mathfrak{a} $ otherwise
        \item  $ X_{i_1} = \mathfrak{m} $ and $ c(i_n) = 1 $ and $ X_{j} = \mathfrak{a} $ otherwise
        % \item if there is exactly one index $ i_k $ so that $ X_{i_k} = \mathfrak{m} $ and $ X_j = \mathfrak{a} $ for all $ j \neq i_k $, and $ \mathrm{Mul}_{\mathbf{LM}_\mathrm{inv}} \left(\{X_i\}_{i \in I}, Y\right) $ is empty otherwise. 
    \end{itemize}
\end{remark}
\begin{remark} % compare Remark 4.2.1.2
    Restricting to the objects which are both called $ \mathfrak{a} $, we see that both $ \mathbf{LM}_\mathrm{inv} $ and $ \mathbf{RM}_{\mathrm{inv}} $ have a sub-colored operad which is canonically identified with $ \mathbf{Assoc}_\mathrm{inv} $ of Definition \ref{defn:colored_operad_monoid_with_involution}. 
\end{remark}
\begin{remark}\label{rmk:lmod_opd_to_inv_lmod_opd}
    There is a map of colored operads $ \iota \colon \mathrm{LM} \to \mathrm{LM}_{\sigma} $ which sends $ \mathfrak{m} $ to $ \mathfrak{m} $ and sends $ \mathfrak{a} $ to $ \mathfrak{a} $. 
    On $ \mathrm{Mul}_{\mathrm{LM}}\left(\left\{(\mathfrak{a}_{\pm})_i\right\}_{i \in I}, \mathfrak{a}\right) \simeq  \mathcal{L}I \to \mathrm{Mul}_{\mathrm{LM}_\sigma}\left(\left\{\mathfrak{a}_i\right\}_{i \in I}, \mathfrak{a}\right) \simeq  \mathcal{L}I \times \{\pm 1\}^I $  is $ \mathrm{id}_{\mathcal{L}I} \times \{ c_1\} $, this map agrees with $ \iota $ of Remark \ref{rmk:assoc_opd_to_assoc_inv_opd}. 
    On $ \mathrm{Mul}_{\mathrm{BM}}\left(\left\{(\mathfrak{a}_{\pm})_i\right\}_{i \in I} \sqcup\{\mathfrak{m}\}, \mathfrak{m}\right) \subseteq  \mathcal{L}(I\sqcup \{j\}) \to \mathrm{Mul}_{\mathrm{BM}_\sigma}\left(\left\{\mathfrak{a}_i\right\}_{i \in I} \sqcup \{\mathfrak{m}\}, \mathfrak{m}\right) \simeq  \mathcal{L}I \times \{\pm 1\}^I $ is the restriction of the map $ \mathrm{id}_{\mathcal{L}(I\sqcup \{j\})} \times \{ c_1\} $ where $ c_1 $ is the constant function on $ I \sqcup \{j\} $ with value $ 1 $. 

    There is a map of colored operads $ \iota^{\mathrm{rev}} \colon \mathrm{RM} \to \mathrm{LM}_{\sigma} $ which sends $ \mathfrak{m} $ to $ \mathfrak{m} $ and sends $ \mathfrak{a} $ to $ \mathfrak{a} $. 
    On $ \mathrm{Mul}_{\mathrm{RM}}\left(\left\{(\mathfrak{a}_{\pm})_i\right\}_{i \in I}, \mathfrak{a}\right) \simeq  \mathcal{L}I \to \mathrm{Mul}_{\mathrm{LM}_\sigma}\left(\left\{\mathfrak{a}_i\right\}_{i \in I}, \mathfrak{a}\right) \simeq  \mathcal{L}I \times \{\pm 1\}^I $  is $ \mathrm{rev}_{\mathcal{L}I} \times \{ c_{1}\} $, this map agrees with $ \iota^{\mathrm{rev}} $ of Remark \ref{rmk:assoc_opd_to_assoc_inv_opd}. 
    On $ \mathrm{Mul}_{\mathrm{BM}}\left(\left\{(\mathfrak{a}_{\pm})_i\right\}_{i \in I} \sqcup\{\mathfrak{m}\}, \mathfrak{m}\right) \subseteq  \mathcal{L}(I\sqcup \{j\}) \to \mathrm{Mul}_{\mathrm{BM}_\sigma}\left(\left\{\mathfrak{a}_i\right\}_{i \in I} \sqcup \{\mathfrak{m}\}, \mathfrak{m}\right) \simeq  \mathcal{L}I \times \{\pm 1\}^I $ is the restriction of the map $ \mathrm{rev}_{\mathcal{L}(I\sqcup \{j\})} \times \{ c_1\} $ where $ c_1 $ is the constant function on $ I \sqcup \{j\} $ with value $ 1 $. 
\end{remark}
\begin{definition}\label{defn:inv_bimod_operad}
    Define colored operads $ \mathbf{BM}_\mathrm{inv} $ and $ \mathbf{BM}_\sigma $
    \begin{enumerate}[label=(\roman*)]
        \item The set of objects of $ \mathbf{BM}_\mathrm{inv} $ has three elements, which we denote by $ \mathfrak{a}_\ell, \mathfrak{a}_r, \mathfrak{m} $. 
        $ \mathbf{BM}_\sigma $ has the same objects
        \item Let $ \{X_i\}_{i \in I} $ be a finite collection of objects of $ \mathbf{BM}_{\mathrm{inv}} $ and let $ Y $ be another object of $ \mathbf{BM}_{\mathrm{inv}} $. 
        If $ Y = \mathfrak{a}_\ell $ (resp. $ Y = \mathfrak{a}_r $), then $ \mathrm{Mul}_{\mathbf{BM}_\mathrm{inv}} \left(\{X_i\}_{i \in I}, Y\right) $ is the set of pairs consisting of a linear ordering on $ I $ and a function $ I \to \{\pm 1\} $ if $ X_i = \mathfrak{a}_\ell $ (resp. $ X_i = \mathfrak{a}_r $) for all $ i $, and empty otherwise. 
        If $ Y = \mathfrak{m} $, then $ \mathrm{Mul}_{\mathbf{BM}_\mathrm{inv}} \left(\{X_i\}_{i \in I}, Y\right) $ is the subset of pairs $ (\lambda, c) $ consisting of a linear ordering $ \lambda = \{i_1 < i_2 < \cdots < i_n\} $ on $ I $ and a function $ c \colon I \to \{\pm 1 \} $ satisfying: if  there is exactly one index $ i_k $ so that $ X_{i_k} = \mathfrak{m} $, either
        \begin{itemize}
            \item $ c(i_k) = 1 $, $ X_{j} = \mathfrak{a}_\ell $ for $ j < i_k $ and $ X_j = \mathfrak{a}_{r} $ for $ j > i_k $; or
            \item $ c(i_k) = - 1 $, $ X_{j} = \mathfrak{a}_\ell $ for $ j > i_k $ and $ X_j = \mathfrak{a}_{r} $ for $ j < i_k $ 
        \end{itemize}
        % if there is exactly one index $ i_k $ so that $ X_{i_k} = \mathfrak{m} $ and $ X_j = \mathfrak{a} $ for all $ j \neq i_k $, and $ \mathrm{Mul}_{\mathbf{BM}_\mathrm{inv}} \left(\{X_i\}_{i \in I}, Y\right) $ is empty otherwise. 
        $ \mathrm{Mul}_{\mathbf{BM}_\sigma} \left(\{X_i\}_{i \in I}, Y\right) $ is the subset of pairs $ (\lambda, c) $ consisting of a linear ordering $ \lambda = \{i_1 < i_2 < \cdots < i_n\} $ on $ I $ and a function $ c \colon I \to \{\pm 1 \} $ satisfying: if  there is exactly one index $ i_k $ so that $ X_{i_k} = \mathfrak{m} $, $ c(i_k) = 1 $, $ X_{j} = \mathfrak{a}_\ell $ for $ j < i_k $ and $ X_j = \mathfrak{a}_{r} $ for $ j > i_k $. 
        \item The composition law on $ \mathbf{BM}_{\mathrm{inv}} $ is determined by the composition of linear orderings, with reversal of linear orderings according to Definition \ref{defn:colored_operad_monoid_with_involution}. 
        The composition law on $ \mathbf{BM}_{\sigma} $ is determined by the composition of linear orderings 
    \end{enumerate}
\end{definition} 
\begin{remark}\label{rmk:bimod_opd_involution}
    The colored operad $ \mathbf{BM}_{\mathrm{inv}} $ has a canonical involution $ \sigma $ which fixes $ \mathfrak{m} $, exchanges $ \mathfrak{a}_\ell $ and $ \mathfrak{a}_r $, and sends a morphism $ (\lambda, c) $ to $ (\lambda^{\mathrm{rev}}, I \xrightarrow{c} \{\pm 1\} \xrightarrow{\cdot (-1)} \{\pm 1 \}) $. 
\end{remark}
\begin{remark}\label{rmk:bimod_opd_to_inv_bimod_opd}
    There is a map of colored operads $ \iota \colon \mathbf{BM} \to \mathbf{BM}_{\mathrm{inv}} $ which sends $ \mathfrak{m} $ to $ \mathfrak{m} $ and sends $ \mathfrak{a}_{-} $ to $ \mathfrak{a}_\ell $ and $ \mathfrak{a}_+ $ to $ \mathfrak{a}_r $. 
    On $ \mathrm{Mul}_{\mathbf{BM}}\left(\left\{(\mathfrak{a}_{\pm})_i\right\}_{i \in I}, \mathfrak{a}_{\pm}\right) \simeq  \mathcal{L}I \to \mathrm{Mul}_{\mathbf{BM}_\mathrm{inv}}\left(\left\{\mathfrak{a}_i\right\}_{i \in I}, \mathfrak{a}\right) \simeq  \mathcal{L}I \times \{\pm 1\}^I $  is $ \mathrm{id}_{\mathcal{L}I} \times \{ c_1\} $, this map agrees with $ \iota $ of Remark \ref{rmk:assoc_opd_to_assoc_inv_opd}. 
    On $ \mathrm{Mul}_{\mathbf{BM}}\left(\left\{(\mathfrak{a}_{\pm})_i\right\}_{i \in I} \sqcup\{\mathfrak{m}\}, \mathfrak{m}\right) \subseteq  \mathcal{L}(I\sqcup \{j\}) \to \mathrm{Mul}_{\mathbf{BM}_{\mathrm{inv}}}\left(\left\{\mathfrak{a}_i\right\}_{i \in I} \sqcup \{\mathfrak{m}\}, \mathfrak{m}\right) \simeq  \mathcal{L}I \times \{\pm 1\}^I $ is the restriction of the map $ \mathrm{id}_{\mathcal{L}(I\sqcup \{j\})} \times \{ c_1\} $ where $ c_1 $ is the constant function on $ I \sqcup \{j\} $ with value $ 1 $. 

    There is \emph{also} a map of colored operads $ \iota^{\mathrm{rev}} \colon \mathbf{BM} \to \mathbf{BM}_{\mathrm{inv}} $ which sends $ \mathfrak{m} $ to $ \mathfrak{m} $ and  and sends $ \mathfrak{a}_{-} $ to $ \mathfrak{a}_r $ and $ \mathfrak{a}_+ $ to $ \mathfrak{a}_\ell $. 
    On $ \mathrm{Mul}_{\mathbf{BM}}\left(\left\{(\mathfrak{a}_{\pm})_i\right\}_{i \in I}, \mathfrak{a}_{\pm}\right) \simeq  \mathcal{L}I \to \mathrm{Mul}_{\mathbf{BM}_{\mathrm{inv}}}\left(\left\{\mathfrak{a}_i\right\}_{i \in I}, \mathfrak{a}\right) \simeq  \mathcal{L}I \times \{\pm 1\}^I $  is $ \mathrm{id}_{\mathcal{L}I} \times \{ c_1\} $, this map agrees with $ \iota^{\mathrm{rev}} $ of Remark \ref{rmk:assoc_opd_to_assoc_inv_opd}. 
    On $ \mathrm{Mul}_{\mathbf{BM}}\left(\left\{(\mathfrak{a}_{\pm})_i\right\}_{i \in I} \sqcup\{\mathfrak{m}\}, \mathfrak{m}\right) \subseteq  \mathcal{L}(I\sqcup \{j\}) \to \mathrm{Mul}_{\mathbf{BM}_\sigma}\left(\left\{\mathfrak{a}_i\right\}_{i \in I} \sqcup \{\mathfrak{m}\}, \mathfrak{m}\right) \simeq  \mathcal{L}I \times \{\pm 1\}^I $ is the restriction of the map $ \mathrm{rev}_{\mathcal{L}(I\sqcup \{j\})} \times \{ c_{-1}\} $ where $ c_{-1} $ is the constant function on $ I \sqcup \{j\} $ with value $ -1 $. 
\end{remark}
\begin{definition}\label{defn:inv_leftrightbimod_infty_operad}
    Let $ \mathcal{LM}_\mathrm{inv}^\otimes $, $ \mathcal{RM}_\mathrm{inv}^\otimes $, $ \mathcal{BM}_\mathrm{inv}^\otimes $, and $ \mathcal{BM}_\sigma^\otimes $ denote the $ \infty $-operads associated to the colored operads of Definitions \ref{defn:inv_leftmod_operad} and \ref{defn:inv_bimod_operad} (via Construction 2.1.1.7 and Example 2.1.1.21 of \cite{LurHA}).
\end{definition}
\begin{remark}
    We can describe the category $ \mathcal{LM}_\mathrm{inv}^\otimes $ as follows: 
    \begin{enumerate}[label=(\arabic*)]
        \item An object of $ \mathcal{LM}_\mathrm{inv}^\otimes $ is a pair $ (\langle n\rangle,S) $ where $ S $ is a subset of $ \langle n \rangle^\circ $. 
        \item Morphisms $ (\langle m \rangle, T) \to (\langle n \rangle,S) $ consist of a map $ (\alpha \colon \langle m \rangle \to \langle n \rangle, \lambda \colon \langle m \rangle^\circ \to \{\pm 1\}) $ in $ \Associnv^\otimes $ satisfying: 
        \begin{itemize}
            \item The map $ \alpha $ takes $ T \cup \{*\} $ to $ S \cup \{*\} $
            \item For each $ s \in S $, then $ \alpha^{-1}(\{s\}) $ contains exactly one element $ t_s $ of $ T $, and it is maximal (resp. minimal) with respect to the linear ordering on $ \alpha^{-1}(\{s\}) $ if $ \lambda (t_s) = 1 $ (resp. $ \lambda(t_s) = -1 $). 
        \end{itemize}
    \end{enumerate}
\end{remark}
\begin{remark}
    We can describe the category $ \mathcal{BM}_\mathrm{inv}^\otimes $ as follows: 
    \begin{enumerate}[label=(\arabic*)]
        \item An object of $ \mathcal{BM}_\mathrm{inv}^\otimes $ is a triple $ (\langle n\rangle,c_+, c_{-}) $ where $ c_{\pm} $ are functions $ \langle n \rangle^\circ \to \{0,1\} $ and $ c_{-}(i) \leq c_{+}(i) $ for all $ i \in \langle n \rangle^\circ $. 
        \item Morphisms $ (\langle m \rangle, c_+, c_{-}) \to (\langle n \rangle, c_{+}', c_{-}') $ consist of a map $ (\alpha \colon \langle m \rangle \to \langle n \rangle, \lambda \colon \langle m \rangle^\circ \to \{\pm 1\}) $ in $ \Associnv^\otimes $ satisfying: if $ j \in \langle n \rangle^\circ $ and $ \alpha^{-1}(j) = \{i_1 < i_2 < \cdots < i_\ell\} $,
        \begin{itemize}
             \item If $ c_{-} (j) = c_+(j) $, then 
                \begin{equation*}
                    c'_{-}(j) = c_{-}(i_1) \leq c_{+}(i_1) = c_{-}(i_2) \leq c_{+}(i_2) \cdots \cdots c_{-}(i_{m-1}) \leq c_{+}(i_m) = c'_{+}(j)
                \end{equation*}
            \item If $ c_{-} (j) < c_+(j) $, then there exists a unique $ k $ so that $ c_{-}(i_k) < c_{+}(i_k) $ and 
                \begin{equation*}
                \begin{split}
                    \lambda(i_k) \cdot c'_{-}(j) = \lambda(i_k) \cdot c_{-}(i_1) \leq \lambda(i_k) \cdot c_{+}(i_1) = \lambda(i_k) \cdot c_{-}(i_2) \leq \lambda(i_k) \cdot c_{+}(i_2)  \cdots \\ \lambda(i_k) \cdot c_{-}(i_{m-1}) \leq \lambda(i_k) \cdot c_{+}(i_m) = \lambda(i_k) \cdot c'_{+}(j)           
                \end{split}
                \end{equation*}
         \end{itemize} 
        % \begin{itemize}
        %     \item The map $ \alpha $ takes $ T \cup \{*\} $ to $ S \cup \{*\} $
        %     \item For each $ s \in S $, then $ \alpha^{-1}(\{s\}) $ contains exactly one element of $ t $. 
        % \end{itemize}
    \end{enumerate}
\end{remark}
\begin{remark}\label{rmk:inv_bimod_opd_to_inv_alg_opd} %compare Remark 4.2.1.5 in Higher algebra
    Each morphism $ \phi \in \mathrm{Mul}_{\mathbf{BM}_\mathrm{inv}} \left(\{X_i\}_{i \in I}, Y\right) $ determines a linear ordering $\ell $ on the set $ I $ and a function $ s \colon I \to \{\pm 1\} $. 
    Passing from $ \phi $ to the pair $ (\ell, s) $ determines a map of colored operads $ j \colon \mathbf{BM}_\mathrm{inv} \to \mathbf{Assoc}_\mathrm{inv} $. 
    The map $ j $ induces a morphism of $ \infty $-operads $  \mathcal{BM}_{\mathrm{inv}}^\otimes \to \Associnv^\otimes $ which we will also denote by $ j $. 
\end{remark} 
\begin{remark}\label{rmk:lmod_infty_opd_to_inv_lmod_infty_opd}
    The maps $ \iota, \iota^{\mathrm{rev}} $ of Remark \ref{rmk:lmod_opd_to_inv_lmod_opd} induce maps of $ \infty $-operads $ \iota \colon \mathcal{LM}^\otimes \to \mathcal{LM}^\otimes_{\mathrm{inv}} $ and $ \iota^{\mathrm{rev}} \colon \mathcal{RM}^\otimes \to \mathcal{LM}^\otimes_{\mathrm{inv}} $. 
\end{remark}
\begin{remark}\label{rmk:bimod_infty_opd_to_inv_bimod_infty_opd}
    The maps $ \iota, \iota^{\mathrm{rev}} $ of Remark \ref{rmk:bimod_opd_to_inv_bimod_opd} induce maps of $ \infty $-operads $ \iota, \iota^{\mathrm{rev}} \colon \mathcal{BM}^\otimes \to \BMinv^\otimes $. 
    There are canonical identifications $ \iota \circ \mathrm{rev} \simeq \sigma \circ \iota^{\mathrm{rev}} $ where $ \sigma $ is the involution on $ \BMinv^\otimes $ induced by Remark \ref{rmk:bimod_opd_involution} and $ \mathrm{rev} $ is the involution on $ \mathcal{BM}^\otimes $ of \cite[Construction 4.6.3.1]{LurHA}. 
\end{remark}
\begin{remark}\label{rmk:rmod_lmod_opd_to_inv_bimod_opd}
    There are canonical maps of $ \infty $-operads $ \mathcal{LM}^\otimes_{\mathrm{inv}} \to \mathcal{BM}^\otimes_{\mathrm{inv}} $ and $ \mathcal{RM}^\otimes_{\mathrm{inv}} \to \mathcal{BM}^\otimes_{\mathrm{inv}} $ sending $ \mathfrak{a} $ to $ \mathfrak{a}_\ell $, resp. $ \mathfrak{a}_r $ and making the diagram 
    \begin{equation*}
    \begin{tikzcd}[row sep=tiny]
        \mathrm{Assoc}^\otimes \ar[dd,"\sigma"] \ar[r] & \mathcal{LM}^\otimes_{\mathrm{inv}} \ar[dd,"{\mathrm{rev}}"] \ar[rd] & \\
        & & \mathcal{BM}^\otimes_{\mathrm{inv}}  \\ 
        \mathrm{Assoc}^\otimes  \ar[r] & \mathcal{RM}^\otimes_{\mathrm{inv}} \ar[ru] & 
    \end{tikzcd}   
    \end{equation*}
    commute, where $ \mathrm{rev} $ is (an involutive version of) the reversal involution of \cite[Remark 4.6.3.2]{LurHA}. 
\end{remark}
\begin{definition}\label{defn:leftrightbimodules_with_involution}
    Let $ \mathcal{C}^\otimes \to \Associnv^\otimes $ and $ \mathcal{D}^\otimes \to \Associnv^\otimes $ be fibrations of $ \infty $-operads and let $ \mathcal{M} $ be an $ \infty $-category. 
    Suppose given a fibration of $ \infty $-operads $ q \colon \mathcal{O}^\otimes \to \mathcal{LM}^\otimes_{\mathrm{inv}} $ together with equivalences $ \mathcal{O}^\otimes_{\mathfrak{a}} \simeq \mathcal{C}^\otimes $ and $ \mathcal{O}^\otimes_{\mathfrak{m}} \simeq \mathcal{M} $. 
    Let $ L\Modh\left(\mathcal{M}\right) $ denote the $ \infty $-category $ \Alg_{/\mathcal{LM}_{\mathrm{inv}}}\left(\mathcal{O}\right) $. 
    We will refer to $ L\Modh\left(\mathcal{M}\right) $ as the \emph{$\infty$-category of left hermitian module objects of $ \mathcal{M}$}. 

    Suppose given a fibration of $ \infty $-operads $ q \colon \mathcal{O}^\otimes \to \mathcal{BM}^\otimes_{\mathrm{inv}} $ together with equivalences $ \mathcal{O}^\otimes_{\mathfrak{a}_\ell} \simeq \mathcal{C}^\otimes $, $ \mathcal{O}^\otimes_{\mathfrak{a}_R} \simeq \mathcal{D}^\otimes $ and $ \mathcal{O}^\otimes_{\mathfrak{m}} \simeq \mathcal{M} $. \Lucy{Lurie gives this a name (Definition 4.2.1.12 \emph{weakly enriched})--not sure what to call this. something \emph{bi-enriched}?}
    Let $ \Modh\left(\mathcal{M}\right) $ denote the $ \infty $-category $ \Alg_{/\mathcal{BM}_{\mathrm{inv}}}\left(\mathcal{O}\right) $. 
    We will refer to $ \Modh\left(\mathcal{M}\right) $ as the \emph{$\infty$-category of hermitian bimodule objects of $ \mathcal{M}$}. 
    Composition with the inclusions $ \Associnv^\otimes \to \mathcal{BM}^\otimes_{\mathrm{inv}} $ induces a categorical fibration
    \begin{equation*}
        \Modh\left(\mathcal{M}\right) = \Alg_{/\mathcal{BM}_{\mathrm{inv}}}\left(\mathcal{O}\right) \to \Alg_{\Associnv}(\mathcal{C}) \times \Alg_{\Associnv}(\mathcal{D}) \,.
    \end{equation*}
    If $ A $ is an $ \Associnv $-algebra object of $ \mathcal{C} $, we let $ \Modh_A\left(\mathcal{M}\right) $ denote the fiber $ \Modh\left(\mathcal{M}\right) \times_{\Alg_{\Associnv}(\mathcal{C})} \{A\} $. 
    We will refer to $ \Modh_A\left(\mathcal{M}\right) $ as the \emph{$\infty$-category of hermitian $A$-bimodule objects of $ \mathcal{M} $}. 
\end{definition}
\begin{definition}
    Let $ q \colon \mathcal{O}^\otimes \to \mathcal{BM}^\otimes_{\mathrm{inv}} $ be a fibration of $ \infty $-operads. 
    We say that $ q $ exhibits $ \mathcal{O}_{\mathfrak{m}} $ as \emph{$ \EE_\sigma $-bitensored over $ \mathcal{O}_{\mathfrak{a}_\ell} $ and $ \mathcal{O}_{\mathfrak{a}_r} $} if $ q $ is a cocartesian fibration. 
\end{definition}
\begin{remark}\label{rmk:bitensored_cat_informal_description} % Compare Remark 4.2.1.21 in higher algebra
    Let $ q \colon \mathcal{O}^\otimes \to \mathcal{BM}^\otimes_{\mathrm{inv}} $ be a cocartesian fibration of $ \infty $-operads. 
    Then $ q $ is classified by a map $ \chi \colon \mathcal{BM}^\otimes_{\mathrm{inv}} \to \mathrm{Cat}_\infty $. 
    By Remark \ref{rmk:bimod_infty_opd_to_inv_bimod_infty_opd}, we can think of $ q $ as giving two $ \EE_\sigma $ algebras $ \mathcal{C} $, $ \mathcal{D} $ in $ \mathrm{Cat}_\infty $ with an $ \infty $-category $ \mathcal{M} $ equipped with both the structure of a $ \mathcal{C} $-$ \mathcal{D} $-bimodule (equivalently, the structure of a left $ \mathcal{C} \times \mathcal{D}_{\mathrm{rev}} $-module) and of a $ \mathcal{D} $-$ \mathcal{C} $-bimodule, and an autoequivalence $ \sigma_{\mathcal{M}} \colon \mathcal{M} \simeq \mathcal{M} $ of order two which is linear with respect to the autoequivalence $ \mathcal{C} \times \mathcal{D}_{\mathrm{rev}} \xrightarrow{\mathrm{flip}} \mathcal{D}_{\mathrm{rev}} \times \mathcal{C} \xrightarrow{\sigma_{\mathcal{D}}^{-1} \times \sigma_{\mathcal{C}}}  \mathcal{D} \times \mathcal{C}_{\mathrm{rev}} $. 
\end{remark}
\begin{remark} 
    Let $ q \colon \mathcal{O}^\otimes \to \mathcal{LM}^\otimes_{\mathrm{inv}} $ be a cocartesian fibration of $ \infty $-operads. 
    Consider a left hermitian module object $ F \colon \mathcal{LM}^\otimes_{\mathrm{inv}} \to \mathcal{O}^\otimes $. 
    By Remark \ref{rmk:rmod_lmod_opd_to_inv_bimod_opd}, $ F $ determines an associative algebra $ A $ of $ \mathcal{C} $ with an equivalence of algebras $ \sigma_A \colon A \simeq \sigma_{\mathcal{C}} (A)^{\mathrm{rev}} $, an object $ M \in \mathcal{M} $ so that $ M $ (resp. $ \sigma_{\mathcal{M}}(M) $) is equipped with the structure of a left $ A $-module (resp. right $ \sigma_{\mathcal{C}}(A) $-module). %and the structure of a right $ \sigma_{\mathcal{C}} (A) $-module on $ \sigma_{\mathcal{M}}(M) $. \Lucy{Want this to be a bimodule structure} 
    % By \cite[\S4.6.3]{LurHA}, equivalently $ M $ and $ \sigma_{\mathcal{M}}(M) $ both have the structure of a left $ A \otimes \sigma_{\mathcal{C}}(A)^{\mathrm{rev}} $-module.  
    Furthermore, we have an equivalence $ \sigma_{M} \colon M \simeq \sigma_{\mathcal{M}} (M) $ which is linear with respect to the equivalence $ A \xrightarrow{\sigma_A} \sigma_{\mathcal{C}}(A)^{\mathrm{rev}} $. 
    \Lucyil{is this related to ``modules with involution'' from \cite[\S3.1]{CDHHLMNNSI}?}
\end{remark}
\begin{remark}\label{rmk:inv_bimod_informal_description}
    Let $ q \colon \mathcal{O}^\otimes \to \mathcal{BM}^\otimes_{\mathrm{inv}} $ be a cocartesian fibration of $ \infty $-operads. 
    Consider a hermitian module object $ F \colon \mathcal{BM}^\otimes_{\mathrm{inv}} \to \mathcal{O}^\otimes $ and keep the notation of Remark \ref{rmk:bitensored_cat_informal_description}. 
    By Remark \ref{rmk:rmod_lmod_opd_to_inv_bimod_opd}, $ F $ determines an associative algebra $ A $ of $ \mathcal{C} $ with an equivalence of algebras $ \sigma_A \colon A \simeq \sigma_{\mathcal{C}} (A)^{\mathrm{rev}} $ and an associative algebra $ B $ of $ \mathcal{D} $ with an equivalence of algebras $ \sigma_B \colon B \simeq \sigma_{\mathcal{D}} (B)^{\mathrm{rev}} $, an object $ M \in \mathcal{M} $ so that $ M $ (resp. $ \sigma_{\mathcal{M}}(M) $) is equipped with the structure of a $ A $-$ B $-bimodule (resp. $ \sigma_{\mathcal{D}}(B) $-$ \sigma_{\mathcal{C}}(A) $-bimodule). %and the structure of a right $ \sigma_{\mathcal{C}} (A) $-module on $ \sigma_{\mathcal{M}}(M) $. \Lucy{Want this to be a bimodule structure} 
    % By \cite[\S4.6.3]{LurHA}, equivalently $ M $ and $ \sigma_{\mathcal{M}}(M) $ both have the structure of a left $ A \otimes \sigma_{\mathcal{C}}(A)^{\mathrm{rev}} $-module.  
    Furthermore, we have an equivalence $ \sigma_{M} \colon M \simeq \sigma_{\mathcal{M}} (M) $ which is linear with respect to the equivalence $ A \otimes B  \xrightarrow{\mathrm{flip}} B \otimes A \xrightarrow{\sigma_B^{-1} \otimes \sigma_A} \sigma_{\mathcal{D}}(B)^{\mathrm{rev}} \otimes \sigma_{\mathcal{C}}(A)^{\mathrm{rev}} $. 
    \Lucyil{when $ \mathcal{C} = \mathcal{D} $ and $ \sigma_{\mathcal{M}} $ and $ \sigma_{\mathcal{C}} $ are both the identity and $ A = B $, I think this recovers the ``module with involution'' from \cite[\S3.1]{CDHHLMNNSI}.}
\end{remark}
\begin{remark}\label{rmk:inv_alg_as_bimodule_over_itself}
    Let $ q \colon \mathcal{O}^\otimes \to \Associnv $ be a cocartesian fibration of $ \infty $-operads exhibiting $ \mathcal{C} $ as a monoidal $ \infty $-category with an involution $ \sigma_{\mathcal{C}} $. 
    Let us write $ \mathcal{C}^{\otimes}_{\pm} \to \mathcal{BM}^{\otimes} $ for the pullback of $ q $ the functors of Remark \ref{rmk:bimod_opd_to_inv_bimod_opd} and \ref{rmk:inv_bimod_opd_to_inv_alg_opd}. 
    Restriction along the functors of Remark \ref{rmk:bimod_opd_to_inv_bimod_opd} and \ref{rmk:inv_bimod_opd_to_inv_alg_opd} defines functors $ \Alg^\sigma(\mathcal{D}) \to \Modh(\mathcal{D}) \to \mathrm{BMod}(\mathcal{C}_{\pm}^\otimes) $ which is equivariant with respect to the action of $ \sigma_{\mathcal{C}} $ on the target. \Lucy{explain the action on $ \mathcal{C}_\pm $ or delete the last half of the sentence.}
    Moreover, there is a commutative diagram
    \begin{equation*}
    \begin{tikzcd}[column sep=large]
        \Alg^\sigma(\mathcal{D}) \ar[r] \ar[d,"{\Delta_\tau}"] & \Modh(\mathcal{D}) \ar[r] \ar[ld] & \mathrm{BMod}\left(\mathcal{C}^\otimes_{\pm}\right) \ar[d] \\
           \Alg^\sigma(\mathcal{D}) \times \Alg^\sigma(\mathcal{D}) \ar[r,"{(A,B) \mapsto A}"'] & \Alg^\sigma(\mathcal{D}) \ar[r] & \Alg(\mathcal{C}) \times \Alg(\mathcal{C}_{\mathrm{rev}})  
    \end{tikzcd}
    \end{equation*}
    where the right vertical arrow is the categorical fibration of \cite[Definition 4.3.1.12]{LurHA}, the central diagonal arrow is the categorical fibration of Definition \ref{defn:leftrightbimodules_with_involution}, and the left vertical arrow $ \Delta_\tau $ is the identity on one component and precomposition with the involution of Remark \ref{rmk:bimod_opd_involution}. 
    The total composite from the upper left to the lower right agrees with the forgetful functor of Remark \ref{rmk:naive_inv_alg_as_assoc_alg}. 
    If $ A \colon \Associnv \to \mathcal{O}^\otimes $ is an involutive algebra object (Definition \ref{defn:naive_involutive_algebras}), informally the functor sends $ A $ to $ A $, regarded as an $ A $-$ \sigma_{\mathcal{C}}(A) $-bimodule in the notation of Remark \ref{rmk:inv_bimod_informal_description} (compare \cite[Remark 4.2.1.17]{LurHA}). 
\end{remark}
\begin{construction}\label{cons:involutive_cut_module_operad}
    Define a functor $ \mathrm{MCut} \colon \Delta_\sigma^\op \to \mathcal{RM}^\otimes_{\mathrm{inv}} $: \Lucy{maybe this overloaded notation is not good. I'm running out of ideas.}
    \begin{itemize}
        \item For each $ ([n],s) $, we have $ \mathrm{MCut}([n],s) = \langle n + 1 \rangle \simeq \mathrm{RCut}_0([n]) $ where $ \mathrm{RCut} $ is from \cite[Construction 4.8.4.4]{LurHA}. 
        \item Given a morphism $ \alpha\colon ([n],s) \to ([m], t) $, the associated morphism $ \mathrm{MCut}([m], t) \to \mathrm{MCut} ([n],s) $ consists of
        \begin{itemize}
            \item On underlying finite pointed sets $ \langle m+1\rangle \to \langle n+1 \rangle $, $ \mathrm{MCut} $ agrees with (the reverse of) that appearing in \cite[Construction 4.2.2.6]{LurHA}
            \item Identifying the cut $ \{k \mid k< j\} \sqcup \{k \mid k \geq j \} $ with the morphism $ j-1 < j $, we may regard $ s \colon \langle n+1\rangle^\circ \to \{\pm 1\} $ and likewise $ t \colon \langle m+1\rangle^\circ \to \{\pm 1\} $. \Lucy{check later}
            Define $ u\colon \mathrm{MCut}(\alpha)^{-1}\left(\langle n+1\rangle^\circ\right) \to \{\pm 1\} $ to be the unique function so that $ u(j)t(j) = s(\mathrm{MCut}(\alpha)(j)) $. 
            % This means that $ \alpha^{-1}(\{k \mid k< j\}) $ and $ \alpha^{-1}(\{k \mid k \geq j\}) $ 
        \end{itemize}
    \end{itemize}
\end{construction}
\begin{remark}\label{rmk:nat_transformation_of_cut_functors} % compare remark 4.2.2.8
    We can identify $ \Associnv^\otimes $ with the full subcategory of $ \mathcal{RM}^\otimes_{\mathrm{inv}} $ spanned by objects of the form $ (\langle n\rangle, \langle n\rangle^\circ) $. 
    We can regard Construction \ref{cons:involutive_cut} as defining a functor $ \Delta^\op_\sigma \to \mathcal{RM}^\otimes_{\mathrm{inv}} $. 
    For each $ ([n], s) \in \Delta^\op_\sigma $, there is a map of sets $ \theta \colon \mathrm{MCut} ([n], s) \to \mathrm{Cut}([n], s) $ defined as in \cite[Remark 4.2.2.8]{LurHA}. 
    Concretely, on underlying pointed sets, $ \theta $ takes the form \Lucy{check that the signs $s$ work out!}
    \begin{equation*}
    \begin{split}
        \theta \colon \langle n+1 \rangle &\to \langle n \rangle \\
                            k &\mapsto \begin{cases}
                                k-1 & \text{ if }k>0 \\
                                * & \text{ if }k = 0, * \,.
                            \end{cases}
    \end{split}    
    \end{equation*} 
    This construction determines a morphism $ \gamma $ in the $ \infty $-category $ \Fun\left(\Delta^\op_\sigma, \mathcal{RM}^\otimes_{\mathrm{inv}}\right) $, or equivalently a map $ \gamma \colon \Delta^\op_\sigma \times \Delta^1 \to \mathcal{RM}^\otimes_{\mathrm{inv}} $. 
\end{remark}
\begin{lemma}\label{lemma:involutive_cut_is_approx_on_module_operad}
    The morphism $ \gamma \colon \Delta^\op_\sigma \times \Delta^1 \to \mathcal{RM}^\otimes_{\mathrm{inv}} $ defined in Remark \ref{rmk:nat_transformation_of_cut_functors} exhibits $\Delta^\op_\sigma \times \Delta^1 $ as an approximation to the $ \infty $-operad $ \mathcal{RM}^\otimes_{\mathrm{inv}} $. 
\end{lemma}
\begin{definition}\label{defn:involutive_planar_module}
    Let $ q \colon \mathcal{O}^\otimes \to \mathcal{RM}^\otimes_{\mathrm{inv}} $ be a fibration of $ \infty $-operads, so $ q $ exhibits $ \mathcal{M}:= \mathcal{O}^\otimes_{\mathfrak{m}} $ as weakly bi-enriched over $ \mathcal{O}^\otimes_{\mathfrak{a}} $. 
    Let $ \gamma $ be as in Remark \ref{rmk:nat_transformation_of_cut_functors}. 
    Let $ R\Modh^{\AA^\sigma_\infty}(\mathcal{M}) $ denote the full subcategory of $ \Fun_{\mathcal{RM}^\otimes_{\mathrm{inv}}}\left(\Delta^\op_\sigma \times \Delta^1,\mathcal{O}^\otimes\right) $ spanned by those maps $ f \colon \Delta^\op_\sigma \times \Delta^1 \to \mathcal{O}^\otimes $ satisfying 
    \begin{enumerate}
        \item The restriction of $ f $ to $ \Delta^\op_\sigma \times \{1\} $ belongs to $ \Alg_{\AA^\sigma_\infty}(\mathcal{O}) $ of Definition \ref{defn:inv_planar_alg}
        \item If $ \alpha \colon ([m], s)\to ([n], t) $ so that $ \alpha(0) = 0 $, then the induced map $ f ([m], s, 0)\to f([n], t, 0) $ is an inert map in $ \mathcal{O}^\otimes $
        \item for each object $ ([n], s) $ in $ \Delta^\op_\sigma $, the induced map $ f ([n], s, 0) \to f([n], s, 1) $ is an inert map in $ \mathcal{O}^\otimes $
    \end{enumerate}
\end{definition}
\begin{example}
    \Lucy{see Example 4.2.1.17 of higher algebra} 
    Let $ \mathcal{C}^\otimes \to \mathcal{RM}^\otimes $ be a fibration of $ \infty $-operads. 
    Restriction along the map\Lucy{fibration?} of $ \infty $-operads $ \mathcal{RM}^\otimes_{\mathrm{inv}} \to \Associnv^\otimes $ induced by Remark \ref{rmk:inv_bimod_opd_to_inv_alg_opd} induces a map $ \EE_\sigma\Alg\left(\mathcal{C}\right) \to R\Modh\left(\mathcal{C}\right) $ which is a section of the projection map $ R\Modh\left(\mathcal{C}\right) \to \EE_\sigma\Alg\left(\mathcal{C}\right) $. 
\end{example}
\begin{notation}\label{ntn:E_sigma_tensored_cat_variant} % basically lifted straight from Higher algebra 4.2.2.17
    Let $ q \colon \mathcal{O}^\otimes \to \mathcal{RM}^\otimes_{\mathrm{inv}} $ be a fibration of $ \infty $-operads, so $ q $ exhibits $ \mathcal{M}:= \mathcal{O}^\otimes_{\mathfrak{m}} $ as weakly bi-enriched over $ \mathcal{O}^\otimes_{\mathfrak{a}} $. 
    Define a new simplicial set $ \overline{\mathcal{M}}^\varoast $ by the following universal property
    \begin{equation*}
        \hom_{\sSet_{/\Delta^{\op}_\sigma}} \left(K, \overline{\mathcal{M}}^\varoast \right) \simeq \hom_{\sSet_{/\mathcal{RM}^\otimes_{\mathrm{inv}}}}\left(K \times \Delta^1 , \mathcal{O}^\otimes\right) \,.
    \end{equation*}
    Here we regard $ K \times \Delta^1 $ as a simplicial set over $ \mathcal{RM}^\otimes_{\mathrm{inv}} $ via the composite $ K \times \Delta^1 \to \Delta^\op_{\sigma} \times \Delta^1 \xrightarrow{\gamma} \mathcal{RM}^\otimes_{\mathrm{inv}} $ where $ \gamma $ is from Remark \ref{rmk:nat_transformation_of_cut_functors}. 

    Unwinding definitions, we see that a vertex in $ \overline{\mathcal{M}}^\varoast $ lying over an object $ ([n], s \colon \{1, \ldots, n\} \to \{\pm 1\}) \in \Delta^\op_{\sigma} $ corresponds to a morphism $ \alpha $ in $ \mathcal{O}^\otimes $ whose image in $ \mathcal{RM}^\otimes_{\mathrm{inv}} $ is the map $ (\langle n+1\rangle, \{0\}) \to (\langle n\rangle, \varnothing) $\Lucy{this might be off--revisit later!}.   
    Now let $ \mathcal{M}^\varoast $ denote the full simplicial subset of $ \overline{\mathcal{M}}^\varoast  $ spanned by those vertices for which $ \alpha $ is inert. 
\end{notation}
\begin{remark}
    Let $ q \colon \mathcal{O}^\otimes \to \mathcal{RM}^\otimes_{\mathrm{inv}} $ be a fibration of $ \infty $-operads, so $ q $ exhibits $ \mathcal{M}:= \mathcal{O}^\otimes_{\mathfrak{m}} $ as weakly enriched over $ \mathcal{O}^\otimes_{\mathfrak{a}} $. 
    By \cite[Example 4.3.1.4 \& Proposition 4.3.2.15]{HTT}, composition with the inclusion $ \{0\} \to \Delta^1 $ induces a trivial Kan fibration $ \mathcal{M}^\varoast \xrightarrow{\sim} \mathcal{O}^\otimes \times_{\mathcal{RM}^\otimes_{\mathrm{inv}}} \Delta^\op_\sigma $. \Lucy{Jacob explains this in a really terse way--just by citing Prop 4.3.2.15 of HTT. It does just follow from definitions/observations, but there are many (for instance, definition of inert edge).}
    In particular, the fiber of $ \mathcal{M}^\varoast $ over an object $ ([n], s) \in \Delta^\op_\sigma $ is canonically equivalent to $ \mathcal{M} \times \mathcal{C}^{\times n} $. 

    Finally, since $ q $ is a categorical fibration and categorical fibrations are closed under pullback and composition with trivial fibrations, $ q $ induces categorical fibrations $ \mathcal{M}^\varoast \to \mathcal{C}^\varoast \to \Delta^\op_\sigma $.     
\end{remark}
\begin{lemma} % Compare Lemma 4.2.2.20 of Higher algebra
    Let $ q \colon \mathcal{O}^\otimes \to \mathcal{RM}^\otimes_{\mathrm{inv}} $ be a cocartesian fibration of $ \infty $-operads, so $ q $ exhibits $ \mathcal{M}:= \mathcal{O}^\otimes_{\mathfrak{m}} $ as tensored over $ \mathcal{O}^\otimes_{\mathfrak{a}} $. 
    Then the associated functor $ \mathcal{M}^\varoast  \to \mathcal{C}^\varoast $ (Notation \ref{ntn:E_sigma_monoidal_cat_variant}) is a locally coCartesian fibration.     
\end{lemma}
\begin{proposition}
    Let $ q \colon \mathcal{O}^\otimes \to \mathcal{RM}^\otimes_{\mathrm{inv}} $ be a cocartesian fibration of $ \infty $-operads, so $ q $ exhibits $ \mathcal{M}:= \mathcal{O}^\otimes_{\mathfrak{m}} $ as tensored over $ \mathcal{O}^\otimes_{\mathfrak{a}} $. 
    Then precomposition with the functor $ \mathrm{MCut} $ of Construction \ref{cons:involutive_cut_module_operad} induces an equivalence of $ \infty $-categories
    \begin{equation*}
        R\Modh(\mathcal{M}) \simeq \Alg_{/\mathcal{RM}_{\mathrm{inv}}}(\mathcal{O}) \xrightarrow{\sim} R\Modh^{\AA_\infty^\sigma}\left(\mathcal{M}\right)\,.
    \end{equation*}
\end{proposition}
\begin{proof}
    Combine Lemma \ref{lemma:involutive_cut_is_approx_on_module_operad} with \cite[Theorem 2.3.3.23]{LurHA}. 
\end{proof} 
\begin{definition}
    \href{https://arxiv.org/abs/2504.02143}{Introduction to section 3.4 here} gives a recipe for defining $ \mathbb{E}_\sigma^\otimes $ as a genuine/parametrized $ \infty $-operad. 
    Relate the `underlying' $ \infty $-operad with $ C_2 $-action to $ \Associnv $? 
\end{definition} 
\Lucyil{Unravel what it means to be $ \mathbb{E}_\sigma $-monoidal in a genuine way!}
\begin{lemma}
    Let $ R $ be a Poincaré ring. 
    Then $ \mathrm{He}(\Mod_R^\omega,\Qoppa_R) \to \mathrm{He}\left(\mathrm{Hyp}(\Mod_R^\omega)\right) $ is canonically endowed with the structure of an $ \mathbb{E}_\sigma $-monoidal $ C_2 $-$ \infty $-category. 
    Recall that $ \mathrm{He}\left(\mathrm{Hyp}(\Mod_R^\omega)\right) \simeq \mathrm{TwAr}(\Mod_R^\omega) $ by \cite[\S2.2]{CDHHLMNNSI}. 
\end{lemma}
\Lucyil{What does it mean for a $ C_2 $-$ \infty $-category to be tensored over a $ \mathbb{E}_\sigma $-monoidal $ C_2 $-$ \infty $-category? Show that if $ (\mathcal{C},\Qoppa) $ is an $ R $-linear Poincaré $ \infty $-category, then $ \mathrm{He}(\mathcal{C},\Qoppa) \to \mathrm{He}(\mathcal{C}^e) $ is an $ \mathbb{E}_\sigma $-linear/-tensored over $ \underline{He}\left(\Mod_R^\omega\right) $?}

\subsection{Part (b)} 
\begin{proposition} \Lucy{This statement is \cite[Proposition 4.2.3.1]{LurHA} with some words changed; no claim of originality here.}
    Let $ \mathcal{C} $ be an involutive monoidal $ \infty $-category and let $ \mathcal{M} $ be an $ \infty $-category which is bitensored over $ \mathcal{C} $. 
    Let $ K $ be a simplicial set so that $ \mathcal{M} $ admits $ K $-indexed limits, and let $ \theta \colon R\Modh(\mathcal{M}) \to \Alg^\sigma(\mathcal{C}) $ be the forgetful functor. 
    Then
    \begin{enumerate}[label=(\arabic*)]
        \item For every commutative square 
        \begin{equation*}
        \begin{tikzcd}
            K \ar[r] \ar[d] & R\Modh(\mathcal{M})  \ar[d,"\theta"] \\
            K^{\triangleleft} \ar[r] \ar[ru,dashed] & \Alg^\sigma(\mathcal{C}) \,,
        \end{tikzcd}
        \end{equation*}
        there exists a dashed arrow which is a $ \theta $-limit diagram. 
        \item An arbitrary map $ \overline{g} \colon K^{\triangleleft} \to R\Modh(\mathcal{M}) $ is a $ \theta $-limit diagram if and only if the induced map $ K^{\triangleleft} \to \mathcal{M} $ is a limit diagram.   
    \end{enumerate}
\end{proposition}
\begin{proof}
    \Lucy{todo}
\end{proof}
\begin{corollary}
    $ \theta $ is a cartesian fibration, and a morphism $ f \colon \Delta^1 \to R\Modh(\mathcal{M}) $ is $ \theta $-cartesian if and only if the image of $ f $ in $ \mathcal{M} $ is an equivalence.     
\end{corollary}
\begin{corollary}
    Let $ \mathcal{C} $ be an involutive monoidal $ \infty $-category and let $ \mathcal{M} $ be an $ \infty $-category which is bitensored over $ \mathcal{C} $. 
    Let $ K $ be a simplicial set so that $ \mathcal{M} $ admits $ K $-indexed limits, and let $ \theta \colon R\Modh(\mathcal{M}) \to \Alg^\sigma(\mathcal{C}) $ be the forgetful functor. 
    Let $ A $ be an involutive algebra object of $ \mathcal{C} $. 
    Then
    \begin{enumerate}[label=(\arabic*)]
        \item $ R\Modh_A(\mathcal{M}) $ admits $ K $-indexed limits. 
        \item A diagram $ K^{\triangleleft} \to R\Modh_A(\mathcal{M}) $ is a limit diagram if and only if the induced diagram $ K^{\triangleleft} \to \mathcal{M} $ is a limit diagram. 
        \item Given a morphism $ A \to B $ of involutive algebra objects of $ \mathcal{C} $, the induced functor $ R\Modh_B(\mathcal{M}) \to R\Modh_A(\mathcal{M}) $ preserves $ K $-indexed limits. 
    \end{enumerate}
\end{corollary}
\subsection{Towards (e)}
\begin{construction}
    Define a functor $ \mathrm{Pr}\colon \mathbf{LM}_{\mathrm{inv}}^\otimes \times \mathbf{RM}_{\mathrm{inv}}^\otimes \to \mathbf{BM}_{\mathrm{inv}}^\otimes $. 
\begin{enumerate}[label=(\arabic*)]
    \item Let $ (\langle m\rangle, S) $ be an object of $ \mathbf{LM}_{\mathrm{inv}}^\otimes $ and let $ ( \langle n \rangle, T) $ be an object of $ \mathbf{RM}_{\mathrm{inv}}^\otimes $. 
    Let $ \mathrm{Pr}((\langle m\rangle, S), (\langle n\rangle, T)) = (X_*, c_{-}, c_{+}) $ where $ X_*, c_{-}, c_{+} $ are described in \cite[Construction 4.3.2.1(1)]{LurHA}. 
    \item Let $ (\alpha, \lambda) \colon (\langle m\rangle, S) \to (\langle m'\rangle, S') $ be a morphism in $ \mathbf{LM}_{\mathrm{inv}}^\otimes $ and let $ (\beta, \mu) \colon (\langle n\rangle, T) \to (\langle n'\rangle, T') $ be a morphism in $ \mathbf{RM}_{\mathrm{inv}}^\otimes $. 
    Write $ \mathrm{Pr}((\langle m'\rangle, S'), (\langle n'\rangle, T')) = (X_*', c_{-}', c_{+}') $. 
    Then $ \mathrm{Pr}\left((\alpha, \lambda),(\beta, \mu)\right) $ is the unique morphism in $ \mathbf{BM}_{\mathrm{inv}}^\otimes $ lying over the map $ \gamma \colon X_* \to X_*' $ described by 
    \begin{enumerate}[label=(\roman*)]
        \item $ \displaystyle \gamma(i,j) = \begin{cases}
            (\alpha(i), \beta(j)) & \text{ if } \alpha(i) \in \langle m' \rangle^\circ, \beta(j) \in \langle n' \rangle^\circ \\
            * & \text{ otherwise. }
        \end{cases} $
        \item Let $ i' \in \langle m' \rangle^\circ \setminus S' $ and $ j' \in T' $ so $ j' = \beta(j) $ for a unique $ j \in T $. 
        Then the linear ordering on $ \gamma^{-1}(i', j') = \alpha^{-1}(i') \times \{j\} $ is (a) determined by the map $ \alpha $ if $ \mu(j) = 1 $, and (b) it is the reverse of the linear ordering determined by $ \alpha $ if $ \mu(j) = -1 $. 
        The map $ \gamma^{-1}(i', j') = \alpha^{-1}(i') \times \{j\} \to \{\pm 1\} $ is determined by $ \lambda $ if $ \mu(j) = 1 $ and it is $ -\lambda $ if $ \mu(j ) = -1 $. 
        \item Likewise if $ i' \in S' $ and $ j' \in \langle n' \rangle^\circ \setminus T' $ 
        \item Let $ i' \in S' $ and $ j' \in T' $ so $ i' = \alpha(i) $ for a unique $ i \in S $ and $ j' = \beta(j) $ for a unique $ j \in T $. 
        Then $ \gamma^{-1} \{(i',j')\} = \{i\} \times \beta^{-1}\{(j')\} \sqcup_{\{(i,j)\}} \alpha^{-1}\{(i')\} \times \{j\} $. 
        Define $ \gamma^{-1} \{(i',j')\} \to \{\pm 1\} $ by $ \lambda \times \mu $. 
        Endow $ \gamma^{-1} \{(i',j')\} $ with the linear ordering from \cite[Construction 4.3.2.1(2)(iv)]{LurHA} if $ \lambda(i) = \mu(j) $ and endow $ \gamma^{-1} \{(i',j')\} $ with the opposite ordering if $ \lambda(i) \neq \mu(j) $ (or equivalently, if $\lambda(i) = -\mu(j) $). 
    \end{enumerate}
\end{enumerate}
    Write $ \mathrm{Pr} $ for the induced map $ \mathcal{LM}^\otimes_\sigma \times \mathcal{RM}^\otimes_\sigma \to \mathcal{BM}^\otimes_\sigma $ of $ \infty $-categories. 
\end{construction}
\begin{construction}
    Let $ q \colon \mathcal{C}^\otimes \to \mathcal{BM}^\otimes_\sigma $ be a fibration of $ \infty $-operads. 
    We define a map of simplicial sets $ \overline{L\Modh}(\mathcal{C}_{\mathfrak{m}})^\otimes \to \mathcal{RM}^\otimes_\sigma $ by the universal property: For any simplicial set $ K \to \mathcal{RM}^\otimes_\sigma $, there is a bijection
    \begin{equation*}
        \mathrm{Hom}_{\mathrm{sSet}_{/\mathcal{RM}^\otimes_\sigma}}\left(K,\overline{L\Modh}(\mathcal{C}_{\mathfrak{m}})^\otimes \right) \simeq  \mathrm{Hom}_{\mathrm{sSet}_{/\mathcal{BM}^\otimes_\sigma}}\left(\mathcal{LM}^\otimes_\sigma \times K, \mathcal{C}^\otimes \right)\,.
    \end{equation*}
    Let $ L\Modh(\mathcal{C}_{\mathfrak{m}})^\otimes $ denote the full simplicial subset of $ \overline{L\Modh}(\mathcal{C}_{\mathfrak{m}})^\otimes $ spanned by those vertices which correspond to a vertex $ X \in \mathcal{RM}^\otimes_\sigma $ and a functor $ F \colon \mathcal{LM}^\otimes_\sigma \{X\} \to \mathcal{BM}^\otimes_\sigma $ which takes inert morphisms in $ \mathcal{LM}^\otimes_\sigma $ to inert morphisms in $ \mathcal{BM}^\otimes_\sigma $. 
\end{construction}
\begin{remark}
   The composite $ \mathcal{LM}^\otimes_\sigma \times \{\mathfrak{m}\} \hookrightarrow \mathcal{LM}^\otimes_\sigma \times \mathcal{RM}^\otimes_\sigma \xrightarrow{\mathrm{Pr}} \mathcal{BM}^\otimes_\sigma $ agrees with the inclusion of Remark \ref{rmk:rmod_lmod_opd_to_inv_bimod_opd}. 
   Taking $ K \to \mathcal{RM}^\otimes_\sigma $ to be the inclusion $ \{\mathfrak{m}\} \hookrightarrow \mathcal{RM}^\otimes_\sigma $, we have an isomorphism of simplicial sets $ L\Modh(\mathcal{C}_{\mathfrak{m}})^\otimes \times_{\mathcal{RM}^\otimes_\sigma} \{\mathfrak{m}\} \simeq L\Modh(\mathcal{C}_{\mathfrak{m}}) $ where $ L\Modh(\mathcal{C}_{\mathfrak{m}}) $ is the $ \infty $-category of left modules associated to the fibration of $ \infty $-operads $ \mathcal{C}^\otimes \times_{\mathcal{BM}^\otimes_{\sigma}} \mathcal{LM}^\otimes_\sigma \to \mathcal{LM}^\otimes_\sigma $. 
\end{remark} 
\begin{proposition}
    Let $ q \colon \mathcal{C}^\otimes \to \mathcal{BM}^\otimes_\sigma $ be a fibration of $ \infty $-operads. Then
    \begin{enumerate}[label=(\arabic*)]
        \item the induced map $ p \colon L\Modh(\mathcal{C}_{\mathfrak{m}})^\otimes \to \mathcal{RM}^\otimes_\sigma $ is a fibration of $ \infty $-operads 
        \item a morphism $ \alpha $ in $ L\Modh(\mathcal{C}_{\mathfrak{m}})^\otimes $ is inert if and only if $ p (\alpha) $ is inert in $ \mathcal{RM}^\otimes_\sigma $ and for all $ X \in \mathcal{LM}_\sigma $, $ \alpha(X) $ is an inert morphism in $ \mathcal{C}^\otimes $. 
        \item if $ q $ is a cocartesian fibration of $ \infty $-operads, then so is $ p $
        \item if $ q $ is a cocartesian fibration of $ \infty $-operads, a morphism $ \alpha $ in $ L\Modh(\mathcal{C}_{\mathfrak{m}})^\otimes $ is $ p $-cocartesian if and only if, for all $ X \in \mathcal{LM}^\otimes_\sigma $, $ \alpha(X) $ is $ q $-cocartesian in $ \mathcal{C}^\otimes $. 
    \end{enumerate}
\end{proposition}
\begin{proof}
    Similar to \cite[Proposition 4.3.2.5]{LurHA}.    
\end{proof}
\begin{theorem}
    Let $ \mathcal{C} $ be an $ \EE_\sigma $-monoidal $ \infty $-category, and let $ A $ be an $ \EE_\sigma $-algebra in $ \mathcal{C} $. 
    Then $ L\Modh_A(\mathcal{C}) $ is right $ \EE_\sigma $-tensored over $ \mathcal{C} $. 
\end{theorem}

\subsection{Endomorphisms} 
Let $ \mathcal{C} $ be an $ \EE_\sigma $-monoidal $ \infty $-category, and write $ \sigma_{\mathcal{C}} \colon \mathcal{C} \xrightarrow{\sim} \mathcal{C} $ for its involution. 
Suppose $ M \in \mathcal{C} $ is an object equipped with an equivalence $ \sigma_M \colon M \simeq \sigma_{\mathcal{C}}(M) $. 
By \cite[\S4.7.1]{LurHA}, endomorphisms of $ M $ can be regarded as an $ \EE_1 $-algebra in $ u\left(\mathcal{C}\right)^{\otimes} $, where $ u $ is from Remark \ref{rmk:assoc_inv_mon_cat_as_mon_cat_with_autoequiv}. 
Now $ \sigma_{M} $ induces an equivalence $ \mathrm{End}_{\mathcal{C}}(M) \simeq \mathrm{End}_{\mathcal{C}}(\sigma_{\mathcal{C}} (M)) $
On the other hand, $ \sigma_{\mathcal{C}} $ induces an equivalence $ \mathrm{End}_{\mathcal{C}}(\sigma_{\mathcal{C}} (M)) \simeq \mathrm{End}_{\mathcal{C}}(M)^{\mathrm{rev}} $. 
In particular, for any $ \infty $-category $ \mathcal{M} $ left $ \EE_\sigma $-tensored over $ \mathcal{C} $ and any object $ M \in \mathcal{M} $ which is fixed by the involution on $ \mathcal{M} $, we expect the endomorphisms of $ M $ to admit the structure of an $ \EE_\sigma $-algebra in $ \mathcal{C} $. 

To this end, we will define an $ \infty $-category of objects acting on $ M $, show that it has an $ \EE_\sigma $-monoidal structure, and locate endomorphisms of $ M $ as the final object in this $ \infty $-category. 
Informally, we may define a category $ \mathcal{C}[M] $ whose objects consist of either
\begin{itemize}
    \item pairs $ (C, \eta) $ where $ C \in \mathcal{C} $ and $ \eta \colon C \otimes M \to M $ is a morphism in $ \mathcal{M} $; or
    \item pairs $ (C', \eta') $ where $ C'\in \mathcal{C} $ and $ \eta' \colon \sigma_{\mathcal{M}}(M) \otimes C' \to \sigma_{\mathcal{M}}(M) $. 
    \item pairs $ (D, \xi) $ where $ D \in \mathcal{C} $ and $ \xi \colon M \otimes D \to \sigma_{\mathcal{M}}(M) $ \Lucy{added July 26th}
    \item pairs $ (D, \xi') $ where $ D \in \mathcal{C} $ and $ \xi' \colon D \otimes \sigma_{\mathcal{M}}(M) \to M $ 
\end{itemize} 
The monoidal structure is as described in \cite[\S4.7.1]{LurHA}. 
Note that given an object $ (C, \eta) $, the involution $ \sigma_\mathcal{M} $ on $ \mathcal{M} $ sends $ \eta $ to the map $ \sigma_{\mathcal{M}}\left(C \otimes M\right)\simeq \sigma_{\mathcal{M}}(M) \otimes \sigma_{\mathcal{C}}(C) \to \sigma_{\mathcal{M}}(M) $. 
Given an object $ (D, \xi) $, the involution $ \sigma_\mathcal{M} $ on $ \mathcal{M} $ we may consider the map $ \xi' \colon \sigma_{\mathcal{C}}(D) \otimes M \xrightarrow{\mathrm{id} \otimes {\sigma_M}} \sigma_{\mathcal{C}}(D) \otimes \sigma_{\mathcal{M}}^2(M) \simeq \sigma_{\mathcal{M}}\left(M \otimes D\right)\xrightarrow{\sigma_{\mathcal{M}}(\xi)} \sigma_{\mathcal{M}}(M) $. 
The assignment $ (D,\xi) \mapsto (\sigma_{\mathcal{C}}(D), \xi') $... \Lucy{is an involution?} 
This is the involution on $ \mathcal{C}[M] $. 
\begin{definition}
    Let $ p \colon \mathcal{M}^\varoast  \to \Delta^1\times \Delta^\op_\sigma $ exhibit $ \mathcal{M}^\varoast $ as weakly enriched over $ \mathcal{C}^\varoast $. 
    An \emph{enriched morphism} of $ \mathcal{M} $ is a diagram 
    \begin{equation*}
        M \xleftarrow{\alpha} X \xrightarrow{\beta} N
    \end{equation*}
    satisfying either 
    \begin{itemize}
        \item $ p (\alpha) $ is the morphism $ (0, [1], c_{1}) \to (0,[0]) $ in $ \Delta^\op_\sigma $ determined by the embedding $ [0] \simeq \{0\} \hookrightarrow [1] $ and $ c_1 \colon \{1\} \to \{\pm 1\} $ is the constant function at $ +1 $, and
        \item the map $ \beta $ is inert, and $ p(\beta) $ is the morphism $ (0,[1], c_{1}) \to (0, [0]) $ in $ \Delta^1 \times \Delta^\op_\sigma $ determined by the embedding $ [0]\simeq \{1\} \hookrightarrow [1] $
    \end{itemize}
    or
    \begin{itemize}
        \item $ p (\alpha) $ is the morphism $ (0, [1], c_{-1}) \to (0,[0]) $ in $ \Delta^\op_\sigma $ determined by the embedding $ [0] \simeq \{0\} \hookrightarrow [1] $ and $ c_{-1} \colon \{1\} \to \{\pm 1\} $ is the constant function at $ -1 $. 
        \item the map $ \beta $ is inert, and $ p(\beta) $ is the morphism $ (0,[1], c_{-1}) \to (0, [0]) $ in $ \Delta^1 \times \Delta^\op_\sigma $ determined by the embedding $ [0]\simeq \{1\} \hookrightarrow [1] $
    \end{itemize}
    Let $ \mathrm{Str}\, \mathcal{M}^{\mathrm{en}}_{[1]} $ denote the full subcategory of $ \Fun_{\Delta^1 \times \Delta^\op_\sigma }\left(\Lambda^2_0, \mathcal{M}^{\varoast} \right) $ spanned by the enriched morphisms of $ \mathcal{M} $. 

    Note that there are two evaluation functors $ \mathrm{Str}\, \mathcal{M}^{\mathrm{en}}_{[1]} \to \mathcal{M} $. 
    Given $ M \in \mathcal{M} $, write $ \mathcal{C}[M] := \{M\} \times_{\mathcal{M}} \mathrm{Str}\, \mathcal{M}^{\mathrm{en}}_{[1]} \times_{\mathcal{M}} \{M\} $ and refer to it as the endomorphism $ \infty $-category of $ M $. 
\end{definition} 
\begin{definition}
    \emph{enriched $ n $-string} 
\end{definition}
\begin{proposition}
    [Segal condition] 
\end{proposition}
In the course of thinking about the `involutive' generalization of the statement that given an $ \EE_1 $-algebra, its category of modules is $ \EE_0 $ (and conversely, that given an object in a stable $\infty$-category, that its endomorphism spectrum is an $ \EE_1 $-algebra), I have run up against some questions. 
\begin{question}
\begin{itemize}
    \item Can we sidestep an involutive version of the construction of endomorphism categories of \cite[\S4.7.1]{LurHA}? 
    \item Suppose $ \mathcal{C} $ is a monoidal $ \infty $-category and $ \mathcal{M} $ is an $ \infty $-category which is enriched over $ \mathcal{C} $ in the sense of \cite[\S4.2.1]{LurHA}. 
    The opposite category $ \mathcal{M}^\op $ is enriched over $ \mathcal{C} $ by \cite[\S10]{Heine_2023}.   
\end{itemize}
\end{question}

\section{Categorification and structure--new leaf}
Use \href{https://arxiv.org/pdf/1904.01465}{genuine operadic nerve here} to go from simplicial genuine operads to parametrized operads. 
\begin{notation}\label{ntn:t_order_equivariant_map}
    Let $ S, T $ be finite $ C_2 $-sets, and let $ f \colon S\to T $ be a $ C_2 $-equivariant map. 
    A \emph{t-ordering} on $ f $ is the data of, for all free orbits $ V \subset S $, $ V \simeq C_2 $ so that $ f(V) = \{*\} $, a choice of an ordering $ \le $ on $ V $. 
    A \emph{t-ordered map $ S \to T $} is the data of $ f $ and a t-ordering on $ f $.  

    Note that if $ f_i \colon S_i \to T_i $ are all t-ordered, there is a canonical t-ordering on $ \displaystyle \bigsqcup_i f_i $. 
    If $ f \colon S \to T $ is t-ordered and $ g \colon T \to U $ is t-ordered, there is a canonical t-ordering on $ g \circ f $. 
    Let us spell this out: Suppose given an orbit $ V \subset S $ on which $ C_2 $ acts freely so that $ g\circ f(V) $ is trivial. 
    There are two cases
    \begin{itemize}
        \item Suppose $ f $ restricts to an isomorphism $ f|_V \colon V \xrightarrow{\sim} f(V) $ and $ g $ sends $ f(V) $ to $ * $. 
        Then the t-ordering on $ g $ means we have an ordering on $ f(V) $, which we transport to an ordering on $ V $ via $ f|_V $. 
        \item Suppose $ f(V) = * $. 
        Then the t-ordering on $ f $ endows $ V $ with a canonical ordering.  
    \end{itemize} 
    Finally, we define a $ t $-ordering on pullbacks as follows: Suppose given a pullback square of $ C_2 $-sets. 
    \begin{equation*}
    \begin{tikzcd}
        Z \times_X W \ar[r,"{g^*f}"] \ar[d,"{\pi_1}"] &  W \ar[d,"{g}"] \\
        Z \ar[r,"f"] & X        
    \end{tikzcd}     
    \end{equation*} 
    and a t-ordering on $ f $. 
    Suppose given a free orbit $ C_2 \simeq U \subseteq Z \times_X W $ so that $ g^*(f)(U) $ is a singleton (with trivial $ C_2 $-action). 
    Because the square above is a pullback, $ \pi_1(U) \subset Z $ is acted on by $ C_2 $ freely, and $ \pi_1|_U $ defines an isomorphism $ U \simeq \pi_1(U) $. 
    By $ C_2 $-equivariance of $ g $, $ g(g^*(f)(U)) = f(\pi_1(U)) $ is a singleton. 
    The given t-ordering on $ f $ means $ \pi_1(U) $ has a given ordering, which we use to endow $ U $ with an ordering using the isomorphism $ \pi_1|_U $. 
\end{notation}
\begin{remark}
    In the following diagram, both t-orderings on $ f $ induce the same t-ordering on $ g^*(f) $ 
    \begin{equation*}
    \begin{tikzcd}
        C_2 \times C_2 \ar[r] \ar[d] & C_2 \ar[d,"{g}"] \\
        C_2 \ar[r,"{f}"] & C_2/C_2        
    \end{tikzcd}     
    \end{equation*}     
    (in fact the set of t-orderings on $ g^*(f) $ is a singleton.)
\end{remark}
In what follows, we will refer to the following diagram
\begin{equation}\label{diagram:C2_operad_generic_span}
\begin{tikzcd}
    U \ar[d] & \ar[l,"g"] Z \ar[r,"f"]  \ar[d] & X \ar[d] \\
    V &  \ar[l] Y \ar[r,equals] & Y
\end{tikzcd}
\end{equation} 
repeatedly; in particular, the induced map $ Z \to U \times_V Y $ is always assumed to be a summand inclusion. 
\begin{definition}\label{defn:Poincare_ring_param_operad}
    Define a $ C_2 $-$ \infty $-operad $ \mathbb{E}_{\mathrm{p}}^\otimes \to \underline{\mathrm{Fin}}_{C_2, *} $ as having\Lucy{I think this is the $ C_2 $-$ \infty $-operad whose algebras give Poincaré rings.} 
    \begin{itemize}
        \item objects consist of those arrows $ U \to V $ so that $ C_2 $ acts freely on $ U $ and transitively on $ V $ 
        \item morphisms consist of spans (\ref{diagram:C2_operad_generic_span}) in $ \underline{\mathrm{Fin}}_{C_2, *} $, plus the data of a t-ordering on $ f $ in the sense of Notation \ref{ntn:t_order_equivariant_map}.  
        \item composition is defined to agree with that in $ \underline{\mathrm{Fin}}_{C_2, *} $. 
    \end{itemize}
    there is a canonical map to $ \underline{\mathrm{Fin}}_{C_2, *} $ which is an inclusion on objects and on morphisms forgets the t-ordering. 
\end{definition} 
\begin{definition}\label{defn:naive_assoc_param_operad}
    Define a $ C_2 $-$ \infty $-operad $ \underline{\mathrm{Assoc}}_{\mathrm{naive}}^\otimes \to \underline{\mathrm{Fin}}_{C_2, *} $ as having 
    \begin{itemize}
        \item objects consist of those arrows $ U \to V $ so that $ C_2 $ acts freely on $ U $ and transitively on $ V $ 
        \item morphisms consist of spans (\ref{diagram:C2_operad_generic_span}) in $ \underline{\mathrm{Fin}}_{C_2, *} $ plus the data of, for all orbits $ T \subset X $ an ordering on the orbits in the preimage $ f^{-1}(T) \subseteq Z $. 
        \item composition is defined to agree with that in $ \underline{\mathrm{Assoc}}^\otimes $. %\Lucy{Define a sub $ C_2 $-operad of $ \underline{\mathrm{Fin}}_{C_2, *} $ on free $ C_2 $-sets, then use Lemma 4.1.13 of Barkan--Haugseng--Steinebrunner to define via a fiber product?}
    \end{itemize}
    there is a canonical map to $ \underline{\mathrm{Fin}}_{C_2, *} $ which is an inclusion on objects and on morphisms forgets the ordering. 
\end{definition}
\begin{remark}\label{rmk:naive_assoc_alg_description}
    Let $ \underline{\mathrm{Free}}_{C_2, *} $ be the full subcategory of $ \underline{\mathrm{Fin}}_{C_2, *} $ on those arrows $ U \to V $ where $ U $ is a free $ C_2 $-set. 
    It is evident that $ \underline{\mathrm{Free}}_{C_2, *} $ is a sub $ C_2 $-$ \infty $-operad of $ \underline{\mathrm{Fin}}_{C_2, *} $ in the sense of \cite[Definition 2.2.7]{NS22}. 
    By \cite[Lemma 4.1.13]{BHS2022envelopes}, there is an equivalence of $ C_2 $-$ \infty $-operads $ \underline{\mathrm{Free}}_{C_2, *} \times_{\underline{\mathrm{Fin}}_{C_2, *}} \underline{\mathrm{Assoc}}^\otimes \simeq \underline{\mathrm{Assoc}}_{\mathrm{naive}}^\otimes $. 
    Now for any $ C_2 $-$ \infty $-operad $ \mathcal{C}^\otimes $, there is an equivalence 
    \begin{equation*}
        \hom_{\underline{\mathrm{Fin}}_{C_2, *}}\left(\underline{\mathrm{Assoc}}_{\mathrm{naive}}^\otimes, \mathcal{C}^\otimes\right) \simeq \hom_{\underline{\mathrm{Free}}_{C_2, *}}\left(\underline{\mathrm{Assoc}}_{\mathrm{naive}}^\otimes, \mathcal{C}^\otimes \times_{\underline{\mathrm{Fin}}_{C_2, *}} \underline{\mathrm{Free}}_{C_2, *}\right) \,. 
    \end{equation*}
    \Lucy{Now pass to spans of finite free $ C_2 $-sets (not spans of arrows!), and write $ \mathcal{C}^\otimes_{\mathrm{naive}} \to \mathrm{Free}_{C_2,*} $ and $ \mathrm{Assoc}_{\mathrm{naive}}^\otimes $. Use Higher algebra Remark 2.3.3.4 and (Thm 2.3.3.23 or Cor 2.3.2.13).} 
    The functor $ \mathrm{Free}_{C_2,*} \to \mathrm{Fin}_{*} $ which sends a (pointed, finite) free $ C_2 $-set to its set of orbits exhibits $ \mathrm{Free}_{C_2,*} $ as a generalized $ \infty $-operad in the sense of \cite[Definition 2.3.2.1]{LurHA}. \Lucy{(1) and (2) are clear, think about (3).} 
    Now choose a free $ C_2 $-set with transitive action; this determines a functor $ BC_2 \to \mathrm{Free}_{C_2,*} $. 
    This induces a functor $ p\colon BC_2 \times \mathrm{Fin}_{*} \to \mathrm{Free}_{C_2,*} $. 
    By \cite[Corollary 2.3.2.13]{LurHA}, there is an equivalence
    \begin{equation*}
        \hom_{\mathrm{Free}_{C_2, *}}\left(\mathrm{Assoc}_{\mathrm{naive}}^\otimes, \mathcal{C}^\otimes_{\mathrm{naive}} \right) \simeq \hom_{BC_2 \times \underline{\mathrm{Fin}}_{*}}\left(p^*(\mathrm{Assoc}_{\mathrm{naive}}^\otimes), p^*(\mathcal{C}^\otimes_{\mathrm{naive}} )\right)
    \end{equation*}
    where the left hand side denotes morphisms in generalized $ \infty $-operads over $ \mathrm{Free}_{C_2,*} $ and the right hand side denotes morphisms of $ BC_2 $-families of $ \infty $-operads in the sense of \cite[Definition 2.3.2.10]{LurHA}. 
    Now the latter is equivalent to $ \mathbb{E}_1 $-algebra objects in $ \left(\mathcal{C}^e\right)^{hC_2} $. 
\end{remark}
\begin{remark}
    Note that in the definition of $ \mathrm{Assoc}_{\mathrm{naive}} $ (\ref{diagram:C2_operad_generic_span}), $ f $ is equivalent to a fold map because $ X $ is acted upon freely by $ C_2 $. 
\end{remark}
\begin{definition}\label{defn:hermitian_mod_param_operad}
    Define a $ C_2 $-$ \infty $-operad $ \mathcal{HM}^\otimes \to \underline{\mathrm{Fin}}_{C_2, *} $ as having 
    \begin{itemize}
        \item objects consist of pairs $ (h : U \to V, \ell \colon U \to \{a, m\}) $ where $ h: U \to V $ is an object of $ \underline{\mathrm{Fin}}_{C_2, *} $ and $ \ell $ is a function which is constant on orbits (think `label') satisfying the condition that $ C_2 $ acts freely on $ \ell^{-1}(\{a\}) $. 
        \item morphisms from $ (h : U \to V, \ell_U \colon U \to \{a, m\}) $ to $ (h : X \to Y, \ell_X \colon X \to \{a, m\}) $ consists of 
        \begin{itemize}
            \item a span (\ref{diagram:C2_operad_generic_span}) in $ \underline{\mathrm{Fin}}_{C_2, *} $
            \item $ Z $ is equipped with a labeling $ \ell_Z $ satisfying $ \ell_U = g \circ \ell_Z $ 
            \item a t-ordering on $ f $ in the sense of Notation \ref{ntn:t_order_equivariant_map}  
            \item for all orbits $ T \subset X $, an ordering $ \leq $ on the orbits in the preimage $ f^{-1}(T) \subseteq Z $. 
        \end{itemize} 
        These are required to satisfy the conditions 
        \begin{itemize}
            \item if $ x \in X $ is so that $ \ell_X(x) = m $, there exists at exactly one orbit, call it $ Z_x $ in $ f^{-1}(x) $ which $ \ell_Z $ sends to $ m $, which is maximal with respect to the ordering.  
            \item let $ x \in X $ be so that $ \ell_X(x) = m $, and write $ Z_x $ for the orbit in $ f^{-1}(x) $ which $ \ell_Z $ sends to $ m $ of the previous bullet point. Then $ f $ restricts to an isomorphism $ f|_{Z_x} $. % if $ z \in Z $ is so that $ \ell_Z(z) = m $ and $ C_2 $ acts freely on the orbit containing $ z $, then $ C_2 $ acts freely on the orbit containing $ f(z) $
        \end{itemize}
    \end{itemize}
    there is a canonical map to $ \underline{\mathrm{Fin}}_{C_2, *} $ which is an inclusion on objects and on morphisms forgets the ordering. 
\end{definition}
\begin{remark}\label{rmk:param_operad_maps}
    There is a map of $ C_2 $-$ \infty $-operads $ s \colon \mathcal{HM}^\otimes \to \mathbb{E}_p^\otimes $ which forgets the data of the labelings and the orderings on the preimages of the orbits (but remembers the t-orderings!).  
    Furthermore, there is a map of $ C_2 $-$ \infty $-operads $ \iota_a \colon \mathrm{Assoc}_{\mathrm{naive}}^\otimes \to \mathcal{HM}^\otimes $ which sends $ U \to V $ to the pair $ (U \to V, U \to\{a\} \hookrightarrow \{a,m\}) $. 
\end{remark}
\begin{definition}\label{defn:genuine_hermitian_modules}
    Let $ \mathcal{C}^\otimes $ be a $ C_2 $-symmetric monoidal $ C_2 $-$ \infty $-category. 
    Write $ \Mod^h(\mathcal{C}) $ denote the $ C_2 $-$ \infty $-category $ \underline{\Alg}_{\mathcal{HM}}(\mathcal{C}) $. 
    Note that precomposition with the map $ \iota_a $ of Remark \ref{rmk:param_operad_maps} defines a map of $ C_2 $-$ \infty $-categories $ \theta \colon \Mod^h(\mathcal{C}) \to \underline{\mathrm{Assoc}_{\mathrm{naive}}}(\mathcal{C}) $. 
\end{definition}
\begin{remark}
    Let $ \mathcal{C}^\otimes $ be the $ C_2 $-symmetric monoidal $ C_2 $-$ \infty $-category of $ C_2 $-spectra equipped with the Hill--Hopkins--Ravenel norm.  
    In view of Remark \ref{rmk:naive_assoc_alg_description}, we may regard an $ \mathcal{O}^\op_{C_2} $-object of $ \Mod^h\left(\underline{\Spectra}^{C_2}\right) $ as consisting of a pair $ (A,M) $ where $ A $ is an $ \EE_1 $-algebra in $ \Spectra^{BC_2} $ and $ M $ is an $ N^{C_2}A $-module in $ \Spectra^{BC_2} $. 
    Precomposition with the maps of Remark \ref{rmk:param_operad_maps} induces a commutative diagram
    \begin{equation*}
    \begin{tikzcd}
        & \Mod^h \ar[d] \\
        \underline{\Alg}_{\EE_p}\left(\underline{\Spectra}^{C_2}\right) \ar[r] \ar[ru] & \underline{\Alg}_{\underline{\mathrm{Assoc}}_{\mathrm{naive}}}\left(\mathcal{C}\right)          
    \end{tikzcd}     
    \end{equation*} 
\end{remark}
\begin{construction}\label{cons:norm_in_families}
    Recall that by \cite[Proposition 2.8.7(1)]{NS22}, there is an equivalence $ \underline{\Alg}_{\underline{\mathrm{Assoc}}_{\mathrm{naive}}}\left(\underline{\Spectra}^{C_2}\right) \simeq \Fun^\otimes_{C_2}\left(\mathrm{Env}_{\underline{\Fin}_{C_2,*}}(\underline{\mathrm{Assoc}}_{\mathrm{naive}}), \left(\underline{\Spectra}^{C_2}\right)^\otimes\right) $. 
    Now recall that $ \mathrm{Env}_{\underline{\Fin}_{C_2,*}}(\underline{\mathrm{Assoc}}_{\mathrm{naive}}) \simeq \underline{\mathrm{Assoc}}_{\mathrm{naive}} \times_{\underline{\Fin}_{C_2,*}} Ar^{\mathrm{act}}(\underline{\Fin}_{C_2,*}) $ \cite[Definition 2.8.4]{NS22}. 
    In the following, we will be interested in objects of $ Ar^{\mathrm{act}}(\underline{\Fin}_{C_2,*}) $ in the fiber over $ C_2/C_2 $, so we omit it from notation. 
    Define objects
    \begin{equation*}
    \begin{split}
        X = \left(C_2 \to C_2/C_2, C_2 \to C_2/C_2, C_2 = C_2\right) \\
        Y = \left(C_2 \to C_2/C_2, C_2 \to C_2/C_2, C_2 \to C_2/C_2 \right)    
    \end{split}
    \end{equation*}
    in $ \mathrm{Env}_{\underline{\Fin}_{C_2,*}}(\underline{\mathrm{Assoc}}_{\mathrm{naive}}) $. 
    The collapse map gives a canonical morphism $ c \colon X \to Y $; observe that $ c $ is cocartesian over $ \underline{\Fin}_{C_2,*} $. 
    Using semiaddivity/universal properties\Lucy{todo; see e.g. \cite[Notation 4.1]{LYang_normedrings}}, define a functor which is uniquely characterized by 
    \begin{equation*}
    \begin{split}
        \Delta^1 \times \Fin_{*} &\to \mathrm{Env}_{\underline{\Fin}_{C_2,*}}(\underline{\mathrm{Assoc}}_{\mathrm{naive}}) \\
        (0,\langle 1 \rangle) &\mapsto X \\
        (0, \langle 1 \rangle) &\mapsto Y \\
        (0 <1, \mathrm{id}_{\langle 1 \rangle}) &\mapsto (c \colon X \to Y)\,.
    \end{split}
    \end{equation*}
    Precomposition with $ \Assoc \to \Fin_* $ defines a functor\footnote{with values in the fiber over $ C_2/C_2 $ of the target}
    \begin{equation}\label{eq:assoc_opd_to_envelope_param_assoc}
        q\colon \Delta^1 \times \Assoc \to \mathrm{Env}_{\underline{\Fin}_{C_2,*}}(\underline{\mathrm{Assoc}}_{\mathrm{naive}}) \,.
    \end{equation}
    Write $ p $ for the composite $ \Delta^1 \times \Assoc \to \mathrm{Env}_{\underline{\Fin}_{C_2,*}}(\underline{\mathrm{Assoc}}_{\mathrm{naive}}) \to \underline{\Fin}_{C_2,*} $. 
    Restriction along the functor of \ref{eq:assoc_opd_to_envelope_param_assoc} defines 
    \begin{equation}
        \Fun^\otimes_{C_2}\left(\mathrm{Env}_{\underline{\Fin}_{C_2,*}}\left(\underline{\mathrm{Assoc}}_{\mathrm{naive}}\right), \left(\underline{\Spectra}^{C_2}\right)^\otimes\right) \to \Fun_{\Delta^1 \times \Assoc}\left(\Delta^1 \times \Assoc, p^*\left(\underline{\Spectra}^{C_2}\right)\right) \,.
    \end{equation}
    Observe that $ p^*\left(\underline{\Spectra}^{C_2}\right) \to \Delta^1 \times \Assoc $ is a $ \Delta^1 $-cocartesian family of monoidal $ \infty $-categories in the sense of \cite[Definition 4.8.3.1]{LurHA}, and the image of the aforementioned restriction functor consists of $ \Delta^1 $-cocartesian families of associative algebra objects in $ p^*\left(\underline{\Spectra}^{C_2}\right) $. 
    In particular, by \cite[Remark 4.8.3.7]{LurHA}, the aforementioned restriction functor refines to 
    \begin{equation*}
         \Fun^\otimes_{C_2}\left(\mathrm{Env}_{\underline{\Fin}_{C_2,*}}\left(\underline{\mathrm{Assoc}}_{\mathrm{naive}}\right), \left(\underline{\Spectra}^{C_2}\right)^\otimes\right) \to \Fun\left(\Delta^1, \Cat^{\Alg}_\infty \right) \,. 
     \end{equation*} 
     Its image in $ \Fun\left(\Delta^1,\Alg\left(\Cat_\infty\right) \right) $ consists of the single arrow $ N^{C_2} \colon \Spectra \to \Spectra^{C_2} $. 
\end{construction}
\begin{definition}
    \Lucy{compare Higher Algebra Definition 4.8.3.7} 
    Let $ \Cat^{\Alg\Mod}_\infty $ denote the full subcategory of the fiber product 
    \begin{equation*}
        \Mon_{\Assoc}\left(\Cat_\infty\right) \times_{\Fun_{\Assoc}\left(\Assoc, \Assoc \times \Mon_{\Assoc}\left(\Cat_\infty\right) \right)} \Fun_{\Assoc}\left(\mathcal{LM}^\otimes, \widetilde{\Mon}_{\Assoc}\left(\Cat_\infty\right) \right)
    \end{equation*}
    on those triples $ (\mathcal{C}^\otimes, A, M) $ where $ \mathcal{C}^\otimes $ is a monoidal $ \infty $-category, $ A $ is an algebra object of $ \{\mathcal{C}^\otimes\} \times_{\Mon_{\Assoc}\left(\Cat_\infty\right) } \widetilde{\Mon}_{\Assoc}\left(\Cat_\infty\right) \simeq \mathcal{C}^\otimes $, and $ M $ is a left $ A $-module in $ \mathcal{C} $. 
\end{definition}
\begin{remark}
    \Lucyil{The \textbf{KEY} is getting an analogue of Remark 4.8.3.8 of Higher Algebra for $ \Cat^{\Alg\Mod}_\infty $ instead of $ \Cat^{\Alg} $. Here is an attempt.}
    For any simplicial set $ S $, there is a bijection between equivalence classes of maps $ S \to \Cat^{\Alg\Mod}_\infty $ and equivalence classes of diagrams
    \begin{equation*}
    \begin{tikzcd}
        \mathcal{LM}^\otimes \times S  \ar[r,"{M_A}"]& \mathcal{C}^\otimes \ar[d,"q"] \\
        \Assoc^\otimes \times S \ar[u] \ar[r,"{\mathrm{id}}"] \ar[r] & \Assoc^\otimes \times S
        % && \mathcal{C}^\otimes \ar[d,"q"] \\
        % \Assoc^\otimes \times S \ar[r] \ar[rr, bend right=30,"{\mathrm{id}}"] &\mathcal{LM} \times S \ar[r] \ar[ru,"{M_A}"] & \Assoc^\otimes \times S
    \end{tikzcd}
    \end{equation*}
    where $ q $ exhibits $ \mathcal{C}^\otimes $ as an $ S $-cocartesian family of monoidal $ \infty $-categories and $ M_A $ is an $ S $-family of $ \mathcal{LM} $-algebra objects (i.e. associative algebra objects and modules over them).\Lucy{to-do: cardinalities}
\end{remark}
\textbf{Next}: Define a forgetful functor $ \Cat^{\Alg\Mod}_\infty \to \Cat^{\Mod}_\infty $ which sends $ (\mathcal{C}^\otimes,A, M) $ to $ (\mathcal{C}^\otimes, \Mod_A) $. \Lucy{do I want a pointed version of $ \Cat^{\Mod}_\infty $? could probably even work with that directly?}

Define a functor $ \Modh \to \Fun\left(\Delta^1, \Cat^{\Mod}_\infty \right) \times_{\Cat^{\Mod}_\infty} \Cat^{\Alg\Mod}_\infty $ living in a commutative square
\begin{equation*}
\begin{tikzcd}
    \Modh \ar[rr,dashed,"{\exists ?}"] \ar[d] & & \Fun\left(\Delta^1, \Cat^{\Mod}_\infty \right) \times_{\Cat^{\Mod}_\infty} \Cat^{\Alg\Mod}_\infty \ar[d] \\
    \underline{\Alg}_{\underline{\mathrm{Assoc}}_{\mathrm{naive}}}\left(\underline{\Spectra}^{C_2}\right) \ar[r] & \Fun\left(\Delta^1, \Cat^{\Alg}_\infty \right) \ar[r,"{\Theta}"] & \Fun\left(\Delta^1, \Cat^{\Mod}_\infty \right)
\end{tikzcd}
\end{equation*}
where the lower arrow is Construction \ref{cons:norm_in_families} and $ \Theta $ is from \cite[Construction 4.8.3.24]{LurHA}. 
\begin{construction}    
    Recall that by \cite[Proposition 2.8.7(1)]{NS22}, there is an equivalence $ \Modh\left(\underline{\Spectra}^{C_2}\right) \simeq \Fun^\otimes_{C_2}\left(\mathrm{Env}_{\underline{\Fin}_{C_2,*}}(\mathcal{HM}^\otimes), \left(\underline{\Spectra}^{C_2}\right)^\otimes\right) $. 

    Choose a $ C_2 $-set $ U $ on which $ C_2 $ acts freely and transitively, and fix an ordering $ \leq $ on $ U $. 
    Define a functor from $ \mathcal{LM}^\otimes $ into the fiber of $ \mathcal{HM}^\otimes $ over $ C_2/C_2 $
    \begin{equation*}
         \left(\langle n \rangle, S \right) \mapsto \left((\langle n \rangle^\circ \setminus S) \times U \sqcup S_+ \to C_2/C_2, \ell(i) = \begin{cases}
             a & i \in (\langle n \rangle^\circ \setminus S) \times U \\
             m & i \in S 
         \end{cases} \right)
     \end{equation*} 
    \Lucy{explain how the ordering on $ U $ is used to define the map!}
    Now recall that $ \mathrm{Env}_{\underline{\Fin}_{C_2,*}}(\mathcal{HM}^\otimes) \simeq \mathcal{HM}^\otimes \times_{\underline{\Fin}_{C_2,*}} Ar^{\mathrm{act}}(\underline{\Fin}_{C_2,*}) $ \cite[Definition 2.8.4]{NS22}. 
    Notice that the fiber of $ \underline{\Fin}_{C_2,*} $ over $ C_2/C_2 $ has a final object given by the identity arrow at $ [C_2/C_2 = C_2/C_2] $ and for every object $ [U \to C_2/C_2] $ in $ \underline{\Fin}_{C_2,*} $, the unique arrow from $ [U \to C_2/C_2] $ to $ [C_2/C_2 \to C_2/C_2] $, hence the composite $ \mathcal{LM}^\otimes \to \mathcal{HM}^\otimes_{C_2/C_2} \to \left(\underline{\Fin}_{C_2,*}\right)_{C_2/C_2} $ lifts to $ \mathcal{LM}^\otimes \to \mathcal{HM}^\otimes_{C_2/C_2} \to \left(\underline{\Fin}_{C_2,*}\right)_{C_2/C_2/[\mathrm{id}_{C_2/C_2}]} \to Ar^{\mathrm{act}}(\underline{\Fin}_{C_2,*}) $. 
    This defines a functor \Lucy{observe that $ r $ sends inert maps to inert maps? What does it mean for a morphism to be inert in $ \mathrm{Env}_{\underline{\Fin}_{C_2,*}}(-) $?}
    \begin{equation}\label{lm_operad_to_genuine_fiber_hm_param_operad}
        r \colon \mathcal{LM}^\otimes \to \mathrm{Env}_{\underline{\Fin}_{C_2,*}}(\mathcal{HM}^\otimes) \simeq \mathcal{HM}^\otimes \times_{\underline{\Fin}_{C_2,*}} Ar^{\mathrm{act}}(\underline{\Fin}_{C_2,*}) \,.
    \end{equation}
    Writing $ \iota_a $ for the canonical inclusion $ \mathrm{Assoc}^\otimes \to \mathcal{LM}^\otimes $, there is an equivalence\footnote{Actually, need to postcompose right hand side with the canonical map $ \mathrm{Env}_{\underline{\Fin}_{C_2,*}}(\underline{\mathrm{Assoc}}_{\mathrm{naive}}) \to \mathrm{Env}_{\underline{\Fin}_{C_2,*}}\left(\mathcal{HM}^\otimes\right) $ induced by Remark \ref{rmk:param_operad_maps}.} 
    \begin{equation}
        r \circ \iota_a = p \circ \left(\iota_{1} \times \mathrm{id}_{\mathrm{Assoc}^\otimes}\right) =:p_1 \,.
    \end{equation}
    Restriction along the functor (\ref{lm_operad_to_genuine_fiber_hm_param_operad}) sends maps of $ C_2 $-$ \infty $-operads to maps of ordinary $ \infty $-operads, hence it defines 
    \begin{equation}
        \Fun^\otimes_{C_2}\left(\mathrm{Env}_{\underline{\Fin}_{C_2,*}}(\mathcal{HM}^\otimes), \left(\underline{\Spectra}^{C_2}\right)^\otimes\right) \to \Alg_{\mathcal{LM}/\mathrm{Assoc}}\left(p_1^*\left(\underline{\Spectra}^{C_2,\otimes}\right)\right) \to \Alg_{/\mathrm{Assoc}}\left(p_1^*\left(\underline{\Spectra}^{C_2,\otimes}\right)\right)
    \end{equation}
    in the notation of \cite[Definition 2.1.3.1]{LurHA}. 
    Now observe that $ p_1^*\left(\underline{\Spectra}^{C_2,\otimes}\right) $ is the ordinary $ \infty $-category of $ C_2 $-spectra, regarded as a monoidal $ \infty $-category via smash product, and there is a commutative diagram
    \begin{equation}
    \begin{tikzcd}
         \Fun^\otimes_{C_2}\left(\mathrm{Env}_{\underline{\Fin}_{C_2,*}}(\mathcal{HM}^\otimes), \left(\underline{\Spectra}^{C_2}\right)^\otimes\right) \ar[r] & \LMod\left(\Spectra^{C_2,\otimes}\right) \ar[r] \ar[d] & \widetilde{\Catex} \ar[d] \\
         &  \Alg_{\mathbb{E}_1}\left(\Spectra^{C_2}\right) \ar[r] & \Catex 
    \end{tikzcd}
    \end{equation}
    where $ \widetilde{\Catex} \to \Catex $ is the restriction of the universal cocartesian fibration to $ \Catex \subseteq \Cat_\infty $ and the right hand square is a pullback by \cite[Corollary 4.2.3.7(3)]{LurHA} and the straightening-unstraightening equivalence.  
\end{construction}
There is a canonical forgetful functor $ \Fun\left(\Delta^1, \Cat^{\Mod}_\infty \right) \times_{\Cat^{\Mod}_\infty} \Cat^{\Alg\Mod}_\infty \to \Fun\left(\Delta^1, \Cat_\infty \right) \times_{\Cat_\infty} \Cat_{\infty,*} $. 

Define a functor $ \Catex_{,*} \to \Catex_{,(-)^\op/\Spectra} $. 
\begin{lemma}
    Write $ \widetilde{\Catex} \to \Catex $ for the restriction of the universal cocartesian fibration to $ \Catex \subseteq \Cat_\infty $. 
    Then there is a functor $ \widetilde{\Catex} \to \Catex_{,(-)^\op/\Spectra} $. 
\end{lemma}
\begin{proof}
    \Lucyil{inspired by Yonatan's answer \href{https://mathoverflow.net/questions/385382/what-is-the-functoriality-of-the-infty-categorical-slice-construction}{here}}
    Write $ \mathrm{RFib} \subset \mathrm{Ar}\left(\Cat\right) \times_{\Cat} \Catex $ be the full subcategory of the arrow category spanned by the right fibrations. 
    Write $ \mathrm{RFib}_* \subseteq \mathrm{RFib} $ for the full subcategory on those arrows $ \mathcal{C} \to \mathcal{D} $ so that $ \mathcal{C} $ has a final object. 
    The maps $ \mathrm{RFib}_* \to \Catex $, $ \mathrm{RFib} \to \Catex $ which send an arrow to its target are cocartesian fibrations. \Lucy{Use Corollary A.32 \href{https://arxiv.org/pdf/1501.02161}{here}.}
    Given a right fibration $ c \colon \widetilde{\mathcal{C}} \to \mathcal{C} $, consider\Lucy{Technically I should consider the space of all diagrams and consider the projection which forgets $ t^*(c) $. This is a trivial Kan fibration, so we can choose an inverse.}
    \begin{equation*}
    \begin{tikzcd}
        &  t^*(c) \ar[d] \ar[r]\ar[rd,phantom,"\lrcorner", very near start] & \widetilde{\mathcal{C}} \ar[d,"c"] \\
        \mathcal{C}^\op & \mathrm{TwAr}(\mathcal{C}) \ar[r,"t"] \ar[l] & \mathcal{C} \,. 
    \end{tikzcd}
    \end{equation*}
    If $ \widetilde{\mathcal{C}} $ has a final object $ * $ and we write $ c = c(*) $, then the composite $ t^{-1}(c) \to \mathcal{C}^\op $ is a right fibration. 
    Furthermore, the right fibration $ t^*(c) $ straightens to the functor $ \hom_{\mathcal{C}}(-, c) \in \Fun'(\mathcal{C}^\op,\Spaces) \subseteq \Fun(\mathcal{C}^\op,\Spaces) $ where $ \Fun' $ denotes the full sub $ \infty $-groupoid on those functors which preserve finite limits. 
    Since $ \mathcal{C} $ is stable, $ \Omega^\infty $ induces an equivalence $ \Fun'(\mathcal{C}^\op,\Spectra) \simeq \Fun'(\mathcal{C}^\op,\Spaces) $ by \cite[Corollary 1.4.2.23]{LurHA}. 
    \Lucyil{In progress: Now use $ \Fun'(\mathcal{C}^\op,\Spectra) \to \Fun(\mathcal{C}^\op,\Spectra) $ to construct a map of cocartesian fibrations. } 
\end{proof}
% Then define a functor $ \Fun\left(\Delta^1, \Cat_\infty \right) \times_{\Cat_\infty} \Catex_{,*} \to \Fun\left(\Delta^1, \Cat_\infty \right) \times_{\Cat_\infty}\Catex_{,(-)^\op/\Spectra} \xrightarrow{\mathrm{compose}} \Cat_{(-)^\op/\Spectra} $. 
% \Lucyil{use a model for the former in terms of bifibrations and representable left fibrations! Finally use naturality of unstraightening in the base \href{https://arxiv.org/abs/1501.02161}{Proposition A.31 here}, esp. see how it is used in Paper I p.149}
\begin{construction}
    Suppose we have constructed a diagram\Lucy{replace $ \Catex $ by \emph{`large'} stable $ \infty $-categories}
    \begin{equation}\label{diagram:calgp_to_functor_cat_with_pointed_target}
    \begin{tikzcd}
        & & \int \Fun^{\mathrm{ex}}( (-)^\op, \Spectra) \ar[d] \\
        \CAlgp \ar[r] \ar[rru,dashed,bend left=15] & \left(\Cat^\epoly\right)^{\Delta^1} \times_{\Cat^\epoly} \Catex \ar[r] & \Catex
    \end{tikzcd}
    \end{equation}
    Define $ \Cat^\epoly $ to be a large $ \infty $-category of large stable $ \infty $-categories and reduced excisively polynomial functors between them. 
    There is a functor $ \Fun^\epoly((-)^\op, \Spectra ) \colon \Cat^\epoly \to \Cat $. 
    Under the non-full subcategory inclusion $ \Catex \to \Cat^\epoly $, for any $ \mathcal{C} \in \Catex $, there is an inclusion $ \Fun^{\mathrm{ex}}(\mathcal{C}^\op,\Spectra) \subseteq \Fun^\epoly(\mathcal{C}^\op, \Spectra) $, where $ \Fun^{\mathrm{ex}}(\mathcal{C}^\op,\Spectra) $ denotes the full subcategory on exact functors. % the full subcategory on reduced quadratic functors $ \mathcal{C}^\op \to \Spectra $ as in \cite[\S1]{CDHHLMNNSI}. 
    In particular, there is a map of cartesian fibrations 
    \begin{equation}\label{diagram:extended_fibration_Cat_epoly}
    \begin{tikzcd}
        \int \Fun^{\mathrm{ex}}( (-)^\op, \Spectra) \ar[r] \ar[d] & \int \Fun^{\epoly}( (-)^\op, \Spectra) \ar[d] \\
        \Catex \ar[r] & \Cat^\epoly\,.
    \end{tikzcd}
    \end{equation}
    Composing (\ref{diagram:calgp_to_functor_cat_with_pointed_target}) with (\ref{diagram:extended_fibration_Cat_epoly}) and noting that the map $ \left(\Cat^\epoly\right)^{\Delta^1} \to \Cat^\epoly $ defining the middle term in (\ref{diagram:calgp_to_functor_cat_with_pointed_target}) is given by evaluation at the target, we obtain a diagram 
    \begin{equation}\label{diagram:calgp_to_functor_cat_with_pointed_target_extended}
    \begin{tikzcd}[row sep=small]
        & & \int \Fun^{\epoly}( (-)^\op, \Spectra) \ar[dd] \\
        \CAlgp \times \{1\} \ar[r,"M"] \ar[rru,bend left=15,"{M_*}"] \ar[d] & \left(\Cat^\epoly\right)^{\Delta^1} \times \{1\} \ar[d] &  \\
        \CAlgp \times \Delta^1 \ar[r,"{M \times \mathrm{id}_{\Delta^1}}"] & \left(\Cat^\epoly\right)^{\Delta^1} \times \Delta^1 \ar[r,"{\mathrm{ev}}"] & \Cat^\epoly  \,.
    \end{tikzcd}
    \end{equation}
    Observe that in the bottom row of (\ref{diagram:calgp_to_functor_cat_with_pointed_target_extended}), given a Poincaré ring $ A $, the morphism $ (\mathrm{id}_A, 0\to 1) $ from $ (A,0) $ to $ (A, 1) $ on the lower left is sent to the morphism $ N^{C_2} \colon \Mod_{A^e}(\Spectra) \to \Mod_{N^{C_2}A}\left(\Spectra^{C_2} \right) $ in the lower right. 
    Now observe that $ \{1\} \to \Delta^1 $ and $ \CAlgp \times \{1\} \to \CAlgp \times \Delta^1 $ are right anodyne by \cite[Corollary 2.1.2.7]{HTT}. 
    It follows from \cites[Corollary 2.4.2.5]{HTT}[\href{https://kerodon.net/tag/01VF}{Tag 01VF} Theorem 5.2.1.1(1)]{kerodon} that there exists a functor $ \overline{M}_* \colon \CAlgp \times \Delta^1 \to \int \Fun^\epoly\left((-)^\op,\Spectra\right) $ in (\ref{diagram:calgp_to_functor_cat_with_pointed_target_extended}) extending $ M_* $. 
    Moreover, by \cite[\href{https://kerodon.net/tag/01VF}{Tag 01VF} Theorem 5.2.1.1(2)]{kerodon}, we see that $ \overline{M}(A,0) $ is the cartesian transport of $ M (A, 1) = \hom_{N^{C_2}A}(-,A) $ along the functor $$ \Fun^\epoly\left(\Mod_{N^{C_2} A}\left(\Spectra^{C_2}\right) ^\op, \Spectra\right) \to \Fun^\epoly\left(\Mod_{A^e}\left(\Spectra\right),\Spectra\right) $$ classified by $ \mathrm{ev} \circ \left(M \times \mathrm{id}_{\Delta^1}\right) $. 
    Consider the restriction of $ \overline{M}_* $ to $ \CAlgp \times \{0\} $. 
    Since the restriction of $ M \times \mathrm{id}_{\Delta^1} $ to $ \CAlgp \times \{0\} $ agrees with $ \Mod_{(-)^e} $ by Construction \ref{cons:norm_in_families}, we have that the image of $ \left.\left( M \times \mathrm{id}_{\Delta^1} \right)\right|_{\CAlgp \times \{0\}} $ factors through the inclusion $ \Catex \subseteq \Cat^\epoly $. \Lucy{todo: size issue} 
    It follows that we may regard $ \overline{M}_*|_{\CAlgp \times \{0\}} $ as having codomain the total space of the cartesian fibration $ \int \Fun^\epoly\left((-)^\op,\Spectra\right) \to \Catex $. 

    We claim that $ \overline{M}_*|_{\CAlgp \times \{0\}} $ admits an essentially unique factorization through the cartesian fibration $ \int \Fun^q(-) \to \Catex $. 
    Since the natural transformation $ \Fun^q(-) \subseteq \Fun^\epoly\left((-)^\op,\Spectra\right) $ of functors $ \Catex \to \Cat $ is given pointwise by full inclusions, the induced map $ \int \Fun^q(-) \to \int \Fun^\epoly\left((-)^\op,\Spectra\right) $ of cartesian fibrations over $ \Catex $ is a full inclusion by \cite[Proposition 2.4.4.2]{HTT}. 
    Therefore, to show that $ \overline{M}_*|_{\CAlgp \times \{0\}} $ factors through $ \int \Fun^q(-) $, it suffices to check that $ \overline{M}_*|_{\CAlgp \times \{0\}} $ sends objects of $ \CAlgp $ to objects of $ \int \Fun^q(-) $. 
    This is true because $ \overline{M}_*|_{\CAlgp \times \{0\}}(A,0) = \left(\Mod_{A^e}, \hom_{N^{C_2}A}\left(N^{C_2}(-), A\right)\right) $, $ N^{C_2} \colon \Mod-{A^e} \to \Mod_{N^{C_2}A} $ is quadratic, and a composite of a reduced quadratic functor and an exact functor is reduced and quadratic. \Lucy{reference for last fact?} 

    In sum, we have produced a commutative diagram like so: 
    \begin{equation}%\label{diagram:calgp_to_functor_cat_with_pointed_target_extended}
    \begin{tikzcd}[row sep=small]
        & & \int \Fun^{q}(-) \ar[d] \ar[rd] & \\
        \CAlgp \times \{0\} \ar[rr,"{\mathrm{ev}_0 \circ M \simeq \Mod_{(-)^e}}"] \ar[rru,bend left=15,"{\left.\overline{M}_*\right|_{\CAlgp \times \{0\}}}"] \ar[d] & & \Catex \ar[rd] & \int \Fun^\epoly\left((-)^\op,\Spectra\right) \ar[d] \\
        \CAlgp \times \Delta^1 \ar[r,"{M \times \mathrm{id}_{\Delta^1}}"] & \left(\Cat^\epoly\right)^{\Delta^1} \times \Delta^1 \ar[rr,"{\mathrm{ev}}"] & &\Cat^\epoly  \,.
    \end{tikzcd}
    \end{equation}
    \Lucyil{Todo: pass to compact/dualizable modules.}
\end{construction}

\subsection{An operadic description of Poincaré rings}
Finally, show that algebras over $ \mathbb{E}_p $ are equivalent to Poincaré rings. 

Let $ K $ denote the $ \infty $-categorical nerve of the $ 1 $-category
\begin{equation*}%\label{diagram:indexing_normedrings}
\begin{tikzcd}[cramped, row sep=tiny]
    & 3 \ar[dd] & \\
    & & 2 \ar[dd] \ar[lu] \\
    & 5 & \\
    1  \ar[rr] \ar[ru] \ar[ruuu, bend left=15] & & 4 \ar[lu]
\end{tikzcd}
\end{equation*}
in which all triangles and squares commute. 

\begin{definition}\label{defn:normedrings}
    Consider the diagram $ \mathcal{P} \colon K \to \Cat_\infty $ 
    \begin{equation}%\label{diagram:normedrings}
    \begin{tikzcd}[column sep=small]
        & \Fun\left(\Delta^1,\EE_\infty\Alg\left(\Spectra\right)^{BC_2}\right) \ar[dd,"{\mathrm{ev}_{0},\mathrm{ev}_{1}}"]  & \\
        & & \Fun\left(\Delta^1,\EE_\infty\Alg\left(\Spectra^{C_2}\right)\right) \ar[dd,"{\mathrm{ev}_{0}, \mathrm{ev}_{1}}"] \ar[lu,"{(-)^e}"'] \\
        &\EE_\infty\Alg(\Spectra)^{BC_2} \times \EE_\infty\Alg(\Spectra)^{BC_2} & \\
        \EE_\infty\Alg\left(\Spectra^{C_2}\right)  \ar[rr,"{N^{C_2}(-^e) \times \id}"'] \ar[ru] \ar[ruuu,"{m \circ (-^e)}", bend left=30] & &\EE_\infty\Alg\left(\Spectra^{C_2}\right) \times \EE_\infty\Alg\left(\Spectra^{C_2}\right) \ar[lu,"{(-)^e \times (-)^e}"]
    \end{tikzcd}
    \end{equation}
    where $ m $ is the functor of \cite[Construction 3.1]{LYang_normedrings} precomposed with the canonical map $ \Delta^1 \to \mathcal{O}_{C_2} $. 
    Observe that the right-hand trapezoid commutes essentially by definition, and the leftmost triangle commutes because $ \left(N^{C_2} A \right)^e \simeq (A^e)^{\otimes 2} $. 
    Consider the limit of the diagram 
    \begin{equation}
        \CAlgp'\left(\Spectra^{C_p}\right) := \lim_K \mathcal{P} .
    \end{equation} 

    There is a canonical forgetful functor $ G' \colon \CAlgp\left(\Spectra^{C_2}\right) \to \EE_\infty\Alg\left(\Spectra^{C_2}\right)  $ given by the canonical projection to the lower left corner of the diagram. 
\end{definition}
\begin{observation}\label{obs:limits_of_limits}
    Let $ F \colon K \to \mathcal{C} $ be any diagram and let $ \mathcal{C} $ be any $ \infty $-category with finite limits. 
    Then there is a canonical equivalence $ \displaystyle \lim_K F \simeq F(1) \times_{\left(F(3) \times_{F(5)} F(4)\right)} F(2) $. 
\end{observation}
\begin{observation}
    By the recollement decomposition of $ \Spectra^{C_2} $, $ \lim_K \mathcal{P} $ is equivalent to the limit of the diagram 
    \begin{equation}%\label{diagram:normedrings}
    \begin{tikzcd}[column sep=small]
        & \Fun\left(\Delta^1,\EE_\infty\Alg\left(\Spectra\right)\right) \ar[dd,"{\mathrm{ev}_{0},\mathrm{ev}_{1}}"]  & \\
        & & \Fun\left(\Delta^1 \times \Delta^1,\EE_\infty\Alg\left(\Spectra\right)\right) \ar[dd] \ar[lu,"{(-)^e}"'] \\
        &\EE_\infty\Alg(\Spectra) \times \EE_\infty\Alg(\Spectra) & \\
        \EE_\infty\Alg\left(\Spectra^{C_2}\right)  \ar[rr,"{N^{C_2}(-^e) \times \id}"'] \ar[ru] \ar[ruuu,"{m^{tC_2} \circ (-^e)}", bend left=30] & &\EE_\infty\Alg\left(\Spectra^{C_2}\right)^{\Delta^1} \times \EE_\infty\Alg\left(\Spectra^{C_2}\right)^{\Delta^1} \ar[lu,"{(-)^e \times (-)^e}"]
    \end{tikzcd}
    \end{equation} 
    Now $ \lim_K \mathcal{P} $ is equivalent to $ \CAlgp $ as defined in main.tex (see the limit of diagram \texttt{diagram:Poincare\_ring\_alternate\_diagram} in Remark 3.4) by Observation \ref{obs:limits_of_limits} and the fact that all commutative trapezoids below \emph{except} the upper left are pullbacks: 
    \begin{equation}
    \begin{tikzcd}[column sep=tiny]
        \EE\Alg(\Spectra)^{\Delta^1 \times \Delta^1} \ar[rr] \ar[d] & & \EE\Alg(\Spectra)^{\Delta^2} \ar[d] \ar[rr] & & \EE\Alg(\Spectra)^{\Delta^1} \ar[d] & \\
        \left(\EE\Alg(\Spectra)^{\Delta^1}\right)^{\times 2} \ar[rr] \ar[d,"{\pi_2}"] & &\left(\EE\Alg(\Spectra)^{\Delta^1}\right) \times \EE\Alg(\Spectra) \ar[rd,"{\pi_1}"] \ar[d,"{\pi_2}"] \ar[rr] & & \EE\Alg(\Spectra)^{\times 2} \ar[rd,"{\pi_1}"] & \\
        \EE\Alg(\Spectra)^{\Delta^1} \ar[rr] & & \EE\Alg(\Spectra) & \EE\Alg(\Spectra)^{\Delta^1} \ar[rr] & & \EE\Alg(\Spectra)\,.
    \end{tikzcd}
    \end{equation}
    Here $ \pi_1, \pi_2 $ denote projections onto the first and second factors, resp.
\end{observation}
\begin{remark}
    There is a map of $ C_2 $-$ \infty $-operads $ \mathrm{Com}_{\mathcal{O}_{C_2}^\simeq} \to \EE_p $. 
\end{remark}
\begin{construction}\label{cons:calgp_operadic_to_diagrammatic}
    Define a functor $ \gamma \colon \EE_p \Alg(\Spectra^{C_2}) \to \CAlgp $ (similar to \cite[\S4.1]{LYang_normedrings}). 
\end{construction}
\begin{theorem}
    The functor of Construction \ref{cons:calgp_operadic_to_diagrammatic} is an equivalence. 
\end{theorem}
\begin{proof}
    \Lucyil{Similar to proof of \cite[Theorem 4.21]{LYang_normedrings}, using Theorem \ref{thm:free_operadic_calgp_formula} and Proposition \ref{prop:unit_in_diagrammatic_calgp}. Include a remark/citation to Nardin--Shah that $ \EE_p\Alg $ is monadic over $ \EE_\infty\Alg $.}
\end{proof}
\begin{theorem}\label{thm:free_operadic_calgp_formula}
    Let $ A \in \EE_\infty\Alg\left(\Spectra^{C_2}\right) $ and consider the adjunction $ F \dashv G $ between $ \EE_p $ and $ \EE_\infty $-algebras in $ \Spectra^{C_2} $. % of Recollection \ref{recollection:freealg_colimit}. 
    \begin{enumerate}[label=(\arabic*)]
    \item The underlying $ C_2 $-spectrum of the free $ \EE_p $ algebra $ F(A) $ on $ A $ is given by 
    \begin{equation}\label{eq:free_operadic_calgp_formula}
    F(A) \simeq
    \begin{tikzcd}
        &  A^{\varphi C_2} \otimes A^e \ar[d,"{s_A \otimes \nu_A}"] \\
        A^e \ar[r,mapsto] & A^{tC_2}
    \end{tikzcd}
    \end{equation}
    where $ u $ is the unit, $ s_A : A^{\varphi C_2} \to A^{tC_2} $ is the structure map, and $ \nu_A $ is the twisted Tate-valued Frobenius. 
    \item There is a canonical $ \EE_\infty $ ring map $ \eta_A \colon A \to GF(A) $ given by $ \id_{A^{\varphi C_p}} \otimes (\eta_{A^e}:\mathbb{S}^0 \to A^e) $ on geometric fixed points and the identity on underlying.
    \end{enumerate}
\end{theorem}
\begin{proof}
    Similar to proof of \cite[Theorem 4.15]{LYang_normedrings}, substituting $ \mathrm{Env}_{\EE_p, \underline{\Fin}_{C_2,*}}\left(\mathrm{Com}_{\mathcal{O}_{C_2}^{\simeq,\op}}\right) $ for $ \mathrm{Com}_{\mathcal{T}^\simeq,act}^\otimes = \mathrm{Env}_{\underline{\Fin}_{C_2,*}, \underline{\Fin}_{C_2,*}}\left(\mathrm{Com}_{\mathcal{O}_{C_2}^{\simeq,\op}}\right) $. 
    Since the indexing diagram is different, one uses Lemma \ref{lemma:freeoperadic_calgp_colimdiagram} instead of Lemma 4.19 \emph{loc. cit}. 
\end{proof}

\begin{lemma}\label{lemma:freeoperadic_calgp_colimdiagram}
    Consider the (non-parametrized) fiber $ I $ of $ \mathrm{Env}_{\EE_p, \underline{\Fin}_{C_2,*}}\left(\mathrm{Com}_{\mathcal{O}_{C_2}^{\simeq,\op}}\right) $ over $ \mathrm{id}_{C_2/C_2} $. 
    An object of $ I $ can be thought of as a pair $ \left([U \to V], \alpha \colon i([U \to V]) \to [\mathrm{id}_{C_2/C_2}]\right)$ where $ [U \to V ] $ is an object in the fiber of $ \mathrm{Com}_{\mathcal{O}_{C_2}^{\simeq,\op}} $ over $ C_2/C_2 $, $ i $ denotes the inclusion $ \mathrm{Com}_{\mathcal{O}_{C_2}^{\simeq,\op}} \to \EE_p $, and $ \alpha $ is an active arrow in $ \EE_p $. 

    Choose a $ C_2 $-set $ U $ with free and transitive $ C_2 $-action, and fix an ordering $ \leq $ on $ U $. 
    This determines a t-ordering on the collapse map $ C_2/C_2 \sqcup U \to C_2/C_2 $, hence an active arrow $ \beta $ in $ \EE_p^\otimes $. 
    The inclusion
    \begin{equation*}
        \{*\} \xrightarrow{* \mapsto \beta} I
    \end{equation*}
    is cofinal. 
\end{lemma}
\begin{proof}
    
\end{proof}
\begin{proposition}\label{prop:unit_in_diagrammatic_calgp}
    Let $ A $ be an $ \EE_\infty $-ring in $ \Spectra^{C_p} $ and let $ (B, n_B \colon N^{C_p}B \to B) $ be a Poincaré ring in $ \Spectra^{C_p} $. 
    Then precomposition with the $ \EE_\infty $-map $ \eta_A : A \to GF(A) $ of Theorem \ref{thm:free_operadic_calgp_formula} induces an equivalence of morphism spaces
    \begin{equation}\label{eq:unit_in_normedalg}
    \begin{tikzcd}
        {\hom_{\CAlgp}\left(\left(\gamma F(A), n_{F(A)} \colon N^{C_p}\gamma F(A) \to \gamma F(A)\right),\left(B, n_B \colon N^{C_p} B \to B \right)\right)} \ar[d,"{G'}"] \ar[dd, bend left=75,looseness=2,"{\rotatebox{90}{$\sim$}}"] \\
        \hom_{\EE_\infty}(G'\gamma F(A), G'(B)) \ar[d,"{\eta^*}"] \\
         \hom_{\EE_\infty}(A, G'(B)).
    \end{tikzcd}
    \end{equation}
    where $ G' $ is the forgetful functor and $ \gamma $ is the functor of Construction \ref{cons:calgp_operadic_to_diagrammatic}. 
    That is, $ \eta_{(-)} $ is a unit for the functors $ (\gamma \circ F, G') $ in the sense of \cite[Definition 5.2.2.7]{HTT}. 
\end{proposition}
% \begin{cor}\label{cor:free_algebras_agree}
%     The natural transformation $ \eta_{(-)} $ exhibits $ \gamma \circ F $ as a left adjoint to $ G' $. 
% \end{cor}

\section{Comparing involutive classical Brauer and involutive higher Brauer}
\begin{question}
\begin{itemize}
    \item If, for a Poincaré $ \infty $-category $ (\mathcal{C},\Qoppa) $, there exists a Poincaré object $(E,q) $ so that $ E $ is a compact generator, can we rewrite both the category and its Poincaré structure in terms of $ \mathrm{End}_{\mathcal{C}}(E) $? 
    \item Can the \emph{property} of an existence of a Poincaré object $ (E,q) $ in $ \left(\mathrm{Perf}_X, \Qoppa_L \right) $ so that $ E $ is a compact generator be checked Zariski-locally? 
    See Toën's paper \S3.  
\end{itemize}
\end{question}

\section{Other} 

\begin{proposition}
    Assume that $ X $ has a \emph{good quotient} $ Y $ in the sense of \cite[Remark 4.20]{azumaya_involution}, and write $ p \colon X \to Y $ for the quotient map. 
    Let $ i \colon U \subseteq Y $ be the largest open subscheme on which $ \pi|_{X_U} $ is étale \cite[Proposition 4.45]{azumaya_involution}. 
    Write $ Z(\pi) $ for the closed complement of $ U $ regarded as a topological space, and let $ j \colon Z(\pi) \to Y $ denote the inclusion\footnote{$ Z(\pi) $ is referred to as the \emph{branch locus} in \cite{azumaya_involution}}. 
    Then $ \underline{\mathcal{O}}^{\varphi C_2} $ is in the essential image of $ j_* \colon \mathrm{Shv}_{\mathrm{Zar}}(Z(\pi)) \to \mathrm{Shv}_{\mathrm{Zar}}(Y) $. 
    In other words, there exists a sheaf $ \mathcal{Q} $ of $ \EE_\infty $-rings on $ Z(\pi) $ so that $ j_* \mathcal{Q} \simeq \underline{\mathcal{O}}^{\varphi C_2} $. \Lucy{hypothesis? move to main text?}
\end{proposition}
\Lucyil{How does $ \mathcal{Q} $ relate to the structure sheaf on the branch locus (as reduced subscheme of $ Y $) used in First--Williams?}
\begin{proof}
    Recall that the open-closed decomposition of $ Y $ induces a symmetric monoidal récollement
    \begin{equation*}
        \mathrm{Shv}_{\mathrm{Zar}}(U) \xleftarrow{i^*} \mathrm{Shv}_{\mathrm{Zar}}(Y) \xrightarrow{j^*} \mathrm{Shv}_{\mathrm{Zar}}(Z(\pi)) \,.
    \end{equation*}
    Therefore, to show that $ \underline{\mathcal{O}}^{\varphi C_2} $ is in the essential image of $ j_* $, it suffices to show that $ i^*\left( \underline{\mathcal{O}}^{\varphi C_2}\right) \simeq 0 $ as a sheaf on $ U $. 

    By \cite[Proposition 4.45]{azumaya_involution}, it suffices to show that if $ y $ is a point in $ U $, then $ \underline{\mathcal{O}}^{\varphi C_2}_{y} = 0 $. \Lucy{Need hypercompleteness to reduce to checking on points? Appeal to Clausen--Mathew and add finite Krull dim hypothesis?}  
    Since $ \underline{\mathcal{O}}^{\varphi C_2}_{y} = \tau_{\geq 0} \left(\Gamma\mathcal{O}_{X \times_Y \{y\}}^{tC_2}\right) $ where $ A = \Gamma\mathcal{O}_{Y,y} \to B = \Gamma\mathcal{O}_{X \times_Y \{y\}} $ is a quadratic étale map so that $ B $ has an involution $ \lambda $ and $ A = B^{\lambda} $ is a local ring with maximal ideal $ \mathfrak{m}_A $ (therefore $ B $ is semilocal by \cite[Proposition 3.15]{azumaya_involution}), it suffices to show that $ \pi_0 B^{tC_2} = 0 $. 
    By \cite[Lemma I.2.9]{NS}, we may without loss of generality replace $ A $ and $ B $ by their $ 2 $-completions. \Lucyil{this is unnecessary to the proof--but shows that the support of $ \mathcal{Q} $ intersects trivially with the open subscheme $ Y \left[\frac{1}{2}\right] $.}
    % Since $ B^{tC_2} $ is $ 2 $-complete, it suffices to show that $ B^{tC_2} \otimes \FF_2 $ vanishes. 
    By the recollement of $ A $-modules in terms of $ \mathfrak{m}_A $-complete and $ A[\mathfrak{m}_A^{-1}] $-modules, it suffices to show that $ \left(B_{\mathfrak{m}_A}^{\wedge}\right)^{tC_2} = 0 $ and $ \left(B[\mathfrak{m}_A^{-1}]\right)^{tC_2} = 0 $. 

    By \cite[Propositions 3.4 \& 3.15]{azumaya_involution}, $ B \mathfrak{m}_A = J \subseteq B $, where $ J $ denotes the Jacobson radical of $ B $. 
    We claim that $ B \simeq \lim_i B/J^i $ induces an equivalence $ B^{tC_2} \to \lim_i \left(B/J^i\right)^{tC_2} $. 
    Granting the claim, it suffices to show that $ \left(B[\mathfrak{m}_A^{-1}]\right)^{tC_2} = 0 $ and $ \left(B/J^i\right)^{tC_2} $ is zero for each $ i $. 
    Since $ (-)^{tC_2} $ is exact and lax symmetric monoidal and each $ B/J^i $ can be written as an extension of finitely many $ B/J $-modules, it suffices to show that $ \left(B[\mathfrak{m}_A^{-1}]\right)^{tC_2} $ and $ \left(B/J\right)^{tC_2} $ are zero. 
    % Now we have an exact sequence $ B/\mathfrak{m}_B \xrightarrow{\mathrm{Tr}} A/\mathfrak{m}_A = (B/\mathfrak{m}_B)^{C_2} \to \pi_0 (B/\mathfrak{m}_B)^{tC_2} \to 0 $. 

    Now observe that $ A/\mathfrak{m}_A $ (resp. $ A[\mathfrak{m}_A^{-1}] $-algebra) is a field and $ B/J $ (resp. $ B[\mathfrak{m}_A^{-1}] $) is a quadratic étale $ A/\mathfrak{m}_A $-algebra (resp. $ A[\mathfrak{m}_A^{-1}] $-algebra). 
    By \cite[Proposition 3.4(ii)]{azumaya_involution}, $ B/J $ (resp. $ B[\mathfrak{m}_A^{-1}] $) is either a separable quadratic field extension of $ A/\mathfrak{m}_A $-algebra (resp. $ A[\mathfrak{m}_A^{-1}] $-algebra), or it is isomorphic to $ \prod_{C_2} A/\mathfrak{m}_A $ (resp. $ \prod_{C_2} A[\mathfrak{m}_A^{-1}] $). 
    In the latter case, the action of $ C_2 $ on $ B/J $ (resp. $ B[\mathfrak{m}_A^{-1}] $) is manifestly free, hence $ (B/J)^{tC_2} = 0 $ (resp. $ B[\mathfrak{m}_A^{-1}]^{tC_2} = 0 $). 
    Suppose instead that $ B/J $ (resp. $ B[\mathfrak{m}_A^{-1}] $) is a separable quadratic field extension of $ A/\mathfrak{m}_A $-algebra (resp. $ A[\mathfrak{m}_A^{-1}] $-algebra). 
    By \cite[Proposition 3.4(ii)]{azumaya_involution}, $ \lambda \otimes_{B} B/J $ (resp. $ \lambda \otimes_B B[\mathfrak{m}_A^{-1}] $) is nontrivial, hence by \cite[Lemma 9.21.2, Tag 09DU]{stacks} the extension $ A/\mathfrak{m}_A \to B/J $ (resp. $ A[\mathfrak{m}_A^{-1}] \to B[\mathfrak{m}_A^{-1}] $) is Galois. 
    Since $ C_2 $ acts freely on $ B/J $ as an $ A/\mathfrak{m}_A $-module by the normal basis theorem, $ (B/J)^{tC_2} = 0 $ (resp. $ B[\mathfrak{m}_A^{-1}]^{tC_2} $).

    We conclude the proof by proving the claim. 
    Since homotopy fixed points commute with arbitrary limits, it suffices to show that $ B \simeq \lim_i B/\mathfrak{m}_B^i $ induces an equivalence $ B_{hC_2} \to \lim_i \left(B/\mathfrak{m}_B^i\right)_{hC_2} $. 
    This is true because the $ B/\mathfrak{m}_B^i $ are uniformly bounded below. \Lucy{compare \cite[Remark 2.8]{MR4280864}.}      
\end{proof}  
\begin{example}
    If $ \lambda = \mathrm{id}_X $ and $ Y = X $, then $ \pi_0 \underline{\mathcal{O}}^{\varphi C_2} = \mathcal{O}_Y /2 $. 
    On $ \pi_0 $, the norm map $ \underline{\mathcal{O}}^e \simeq \mathcal{O}_X \to \underline{\mathcal{O}}^{\varphi C_2} $ takes $ f \mapsto f^2 $. 
\end{example}

\paragraph{pushforwards along quotient maps} 
Two attempts to show the pushforward preserves filtered colimits. 
\Lucyil{ DEPRECATED JUNE 3RD: 
Here are some thoughts towards showing that the canonical map $ \colim_{[a,b]} \pi_* \mathbf{M}_{A^e}^{[a,b]} \to \pi_* \mathbf{M}_{A^e} $ is an equivalence. Later: see if \href{https://arxiv.org/pdf/2311.08051}{Example 3.1.2 here} could be useful?} 
\begin{enumerate}
    \item Each $ \mathbf{M}_{A^e}^{[a,b]} $ and $ \mathbf{M}_{A^e} $ is a hypersheaf. 
    For $ \mathbf{M}_{A^e} $ this follows from \cite[Lemma 5.4]{MR3190610}; need to prove for $ \mathbf{M}_{A^e}^{[a,b]} $. 
    \item $ \pi_* $ sends hypersheaves on the small \'etale site of $ X $ to hypersheaves on the small \'etale site of $ Y $; this follows from the proof of \cite[Proposition 6.5.2.13]{HTT}. \Lucy{which I learned from Recollection 1.14 of \href{https://math.berkeley.edu/~phaine/files/Homotopy_invariance_constructible_sheaves_published.pdf}{this paper}.} 
    \item Is $ \colim_{[a,b]} \pi_* \mathbf{M}_{A^e}^{[a,b]} $ still a hypersheaf? Hypersheaves are not in general closed under colimits, but maybe we can argue using an explicit model for this sheaf? 
    \item The hypercompletion of the \'etale $ \infty $-topos of $ Y $ has enough points; this follows from \cite[Theorem A.4.0.5]{Lurie_SAG} and Proposition 3.7.3 of Exodromy. 
    \item It suffices to show that the canonical map $ \colim_{[a,b]} \pi_* \mathbf{M}_{A^e}^{[a,b]} \to \pi_* \mathbf{M}_{A^e} $ is an equivalence on points; use the explicit model from \cite[Theorem 3.16]{azumaya_involution}?
\end{enumerate}

\Lucyil{As of June 3, I am suspicious of the next argument/think something has gone wrong--(\ref{eq:basechange}) is not supposed to hold in this level of generality. I'm not sure what the problem is yet--maybe that $ \pi_* $ does not in fact define a morphism of recollements? Will revisit later.}
\begin{lemma}
    Let $ (X,\sigma, Y,\pi) $ be a scheme with involution and good quotient. 
    Let $ \widetilde{U} \subseteq X $ be the largest open subscheme of $ X $ on which $ \pi $ is quadratic \'etale and let $ W \subseteq Y $ and $ Z = \pi^{-1}(W) \subseteq X $ be the branch and ramification loci of $ \pi $ in the sense of \cite[Proposition 4.45-4.47]{azumaya_involution} (in particular, $ W $ and $ Z $ are endowed with the reduced subscheme structure). 
    Assume that $ 2 \in \mathcal{O}_Y^\times $. 
    Then the pushforward $ \pi_* \colon \mathrm{Shv}_{\mathrm{\acute{e}t}}(X; \Spaces) \to \mathrm{Shv}_{\mathrm{\acute{e}t}}(Y; \Spaces) $ preserves filtered colimits. 
\end{lemma} 
\begin{proof}
    Since $ W \subseteq Y $, $ Z \subseteq X $ are closed immersions,\Lucy{this is automatic, but \cite[Lemma 5.36]{azumaya_involution} asserts that $ W \to Y $ is a closed embedding without proof} there exist r\'ecollements 
    \begin{equation*}
    \begin{tikzcd}[row sep=small,column sep=large]
        \mathrm{Shv}_{\mathrm{\acute{e}t}}(\widetilde{U}; \Spaces) \ar[r, bend right=10,"{j_{\widetilde{U}*}}"'] & \mathrm{Shv}_{\mathrm{\acute{e}t}}(X; \Spaces) \ar[l, bend right=10,"{j_{\widetilde{U}}^*}"'] \ar[r,bend right=10,"{i_{Z}^*}"']& \mathrm{Shv}_{\mathrm{\acute{e}t}}(Z; \Spaces)   \ar[l, bend right=10,"{i_{Z*}}"']  \\
        \mathrm{Shv}_{\mathrm{\acute{e}t}}(U; \Spaces) \ar[r, bend right=10,"{j_{U*}}"'] & \mathrm{Shv}_{\mathrm{\acute{e}t}}(Y; \Spaces) \ar[l, bend right=10,"{j_{U}^*}"'] \ar[r,bend right=10,"{i_{W}^*}"']& \mathrm{Shv}_{\mathrm{\acute{e}t}}(W; \Spaces)   \ar[l, bend right=10,"{i_{W*}}"']  
    \end{tikzcd}    \,.
    \end{equation*}
    Moreover, the pushforward functor $ \pi_* $ is a morphism of r\'ecollements in the sense of \cite[Definition 2.3]{ShahRS}. 
    In particular, the `components' of $ \pi_* $ (see Observation 2.4 of \emph{loc. cit.}) are 
    \begin{equation*}
    \begin{split}
        \left. \pi_* \right|_{\mathrm{Shv}_{\mathrm{\acute{e}t}}(\widetilde{U}; \Spaces)} = j_U^* \circ \pi_* \circ j_{\widetilde{U}*} = j_{U}^* \circ (\pi \circ j_{\widetilde{U}})_* = j_U^* \circ (j_U \circ \pi|_{\widetilde{U}})_* = j_U^* \circ j_{U*} \circ (\pi|_{\widetilde{U}})_* \\
        \left. \pi_* \right|_{\mathrm{Shv}_{\mathrm{\acute{e}t}}(Z; \Spaces)} = i_W^* \circ \pi_* \circ i_{Z*} = i_{W}^* \circ (\pi \circ i_{Z})_* = i_W^* \circ (i_W \circ \pi|_{Z})_* = i_W^* \circ i_{W*} \circ (\pi|_{Z})_* \,.
    \end{split}
    \end{equation*}
    Now $ j_U^* \circ j_{U*} \circ (\pi|_{\widetilde{U}})_* \simeq (\pi|_{\widetilde{U}})_* $ (resp. $ i_W^* \circ i_{W*} \circ (\pi|_{Z})_* \to (\pi|_{Z})_* $) induced by the counit of the adjunction $ (j_U^*, j_{U*}) $ (resp. $ (i_W^*, i_{W*}) $) is an equivalence because $ j_{U*} $ (resp. $ i_{W*}$) is fully faithful, thus
    \begin{equation}
         \left. \pi_* \right|_{\mathrm{Shv}_{\mathrm{\acute{e}t}}(\widetilde{U}; \Spaces)} \simeq (\pi|_{\widetilde{U}})_* \qquad \qquad  \left. \pi_* \right|_{\mathrm{Shv}_{\mathrm{\acute{e}t}}(Z; \Spaces)} \simeq (\pi|_{Z})_* \,. 
    \end{equation}
    We note for further reference the equivalences\Lucy{diagram instead?}
    \begin{equation}\label{eq:basechange}
         \left. \pi_* \right|_{\mathrm{Shv}_{\mathrm{\acute{e}t}}(\widetilde{U}; \Spaces)} \circ j_{\widetilde{U}}^* \simeq j_U^* \circ \pi_* \qquad \qquad \left. \pi_* \right|_{\mathrm{Shv}_{\mathrm{\acute{e}t}}(Z; \Spaces)}  \circ i_W^* \simeq i_Z^* \circ \pi_* \,.
    \end{equation}
    Suppose given a filtered diagram $ \mathcal{F}_\bullet $ in $ \mathrm{Shv}_{\mathrm{\acute{e}t}}(X; \Spaces) $ and write $ \mathcal{F} $ for its colimit. 
    We would like to show that the canonical map $ \colim_\bullet \pi_*(\mathcal{F}_\bullet) \to \pi_*(\mathcal{F}) $ is an equivalence.  
    Since $ j_U^*, i_W^* $ are jointly conservative (by definition of a r\'ecollement, see \cite[Definition A.8.1(e)]{LurHA}), it suffices to show that the canonical maps
    \begin{equation}\label{eq:colim_diagram_restricted}
    \begin{split}
        j^*_U \left(\colim_\bullet \pi_*(\mathcal{F}_\bullet) \right) & \to j^*_U \pi_*(\mathcal{F}) \\
        i^*_W \left(\colim_\bullet \pi_*(\mathcal{F}_\bullet) \right) & \to i_W^* \pi_*(\mathcal{F})      
    \end{split}
    \end{equation}
    are equivalences. 
    Since $ j^*_U $ and $ i^*_W $ preserve all colimits, the morphisms of (\ref{eq:colim_diagram_restricted}) can be identified with the canonical maps
    \begin{equation*}
    \begin{split}
        \colim_\bullet j^*_U \pi_*(\mathcal{F}_\bullet) \simeq \colim_\bullet (\pi|_{\widetilde{U}})_*(j^*_{\widetilde{U}}\mathcal{F}_\bullet) \to (\pi|_{\widetilde{U}})_*(j^*_{\widetilde{U}}\mathcal{F}_\bullet) \\
        \colim_\bullet i^*_W \pi_*(\mathcal{F}_\bullet) \simeq \colim_\bullet (\pi|_{W})_*(i^*_{W}\mathcal{F}_\bullet) \to (\pi|_{Z})_* i_Z^*(\mathcal{F})     \,,    
    \end{split}
    \end{equation*}
    respectively, where we have used (\ref{eq:basechange}). 
    Now $ \pi|_W $ is an equivalence and $ \pi|_{\widetilde{U}} $ is finite \'etale by \cite[Proposition 4.47]{azumaya_involution}, hence $ (\pi|_{\widetilde{U}})_* $ and $ (\pi|_{W})_* $ preserve filtered colimits. 
\end{proof}

\subsection{Algebras with genuine involution}
\begin{construction}\label{cons:pointed_left_module_cats}  
   Assume $ \mathcal{C} $ is a presentable monoidal $ \infty $-category such that the monoidal product $ - \otimes - \colon \mathcal{C} \times \mathcal{C} \to \mathcal{C} $ preserves small colimits separately in each variable. 
   Then there is an $ \infty $-category $ \LMod(\mathcal{C}) $ \cite[Example 4.2.1.18]{LurHA} whose objects are pairs $ (A, M) $ where $ A $ is an associative algebra object of $ \mathcal{C} $ and $ M $ is a left $ A $-module. 
   Write $ a, m $ respectively for the canonical forgetful functors $ \LMod(\mathcal{C})\to \mathrm{Alg}(\mathcal{C}) $, $ \LMod(\mathcal{C}) \to \mathcal{C} $ which send $ (A, M) $ to $ A $ and $ M $, resp. 
   Then $ a $ is a cocartesian fibration \cite[Corollary 4.2.3.7]{LurHA}, hence it is classified by a functor $ \mathrm{mod} \colon \mathrm{Alg}(\mathcal{C}) \to \Cat_\infty $. 

   The functor $ s $ of \cite[Example 4.2.1.17]{LurHA} determines a commutative diagram 
   %a natural transformation $ \eta \colon * \to \mathrm{mod} $, where $ * \colon \mathrm{Alg}(\mathcal{C}) \to \{*\} \hookrightarrow \Cat_\infty $ is the constant functor at the trivial category, or equivalently
   \begin{equation}
   \begin{tikzcd}
       & \mathcal{U} \ar[d] \\
        \mathrm{Alg}(\mathcal{C}) \ar[r,"\mathrm{mod}"] \ar[ru,"\eta"] & \Cat_\infty   
   \end{tikzcd}    
   \end{equation}
   where $ \mathcal{U} $ is the universal cocartesian fibration. 
   Now consider the functor $ o \colon \mathcal{U} \to \Cat_\infty $ which sends $ (\mathcal{D}, d \in \mathcal{D}) $ to the undercategory $ \mathcal{D}_{d/-} $. 
   Define $ \LMod(\mathcal{C})_{*/-} $ to be the cocartesian fibration over $ \mathrm{Alg}(\mathcal{C}) $ classified by $ o \circ \eta $. 
   Informally, an object of $ \LMod(\mathcal{C})_{*/-} $ lying over $ A \in \mathrm{Alg}(\mathcal{C}) $ is the data of a left $ A $-module $ M $ and a map of left $ A $-modules $ A \to M $. 
\end{construction}
\begin{variant}
    % Let $ \mathcal{C} $ be as in Recollection \ref{rec:left_module_cats}. 
    % There is a similar construction where left modules is replaced by \emph{bimodules} \cite[Definition 4.3.1.12]{LurHA}. 
    Let $ \mathcal{C} $ be an involutive monoidal $ \infty $-category in the sense of Definition \ref{defn:naive_involutive_algebras}. 
    There is a variation on Construction \ref{cons:pointed_left_module_cats} where left modules is replaced by \emph{bimodules} \cite[Definition 4.3.1.12]{LurHA}. \Lucy{involutive bimodules?}
\end{variant}
\begin{construction}\label{cons:alg_inv_is_bimod_canonically}
   Regard $ \mathbb{E}_1\mathrm{Alg}(\Spectra) $ as a category with $ C_2 $-action given by taking the opposite/reverse algebra.  
   There are functors $ b \colon \mathbb{E}_1\mathrm{Alg}\left(\Spectra\right)^{hC_2} \to \LMod(\Spectra) $ and $ b \colon \mathbb{E}_1\mathrm{Alg}\left(\Spectra\right)^{hC_2} \to \BiMod(\Spectra)_{*/-} $ so that $ (a, m) \circ b $ and $ (a, m)\circ b_* $ are (canonically) equivalent to $ (-^e) \otimes(-^ e)^\mathrm{op}, (-)^e $. 
   Informally, an $ \mathbb{E}_1 $-algebra with involution $ B $ can be regarded as a $ B \otimes B^\op $-module in a canonical way, and there is a canonical $ B \otimes B^\op $-module map $ B \otimes B^\op \to B $. 
\end{construction}
\begin{definition}
   The category of \emph{$\mathbb{E}_1$-algebras with genuine involution} is defined to be the limit of the $\Cat_\infty$-valued diagram 
   \begin{equation}\label{diagram:E1_alg_gen_involution}
   \begin{tikzcd}[column sep=small]
       & \LMod(\Spectra) \ar[dd,"{a,m}"]  & \\
       & & \LMod\left(\Spectra^{C_2}\right) \ar[dd,"{a, m}"] \ar[lu,"{(-)^e}"'] \\
       &\mathbb{E}_1\mathrm{Alg}(\Spectra) \times \Spectra & \\
       \mathbb{E}_1\mathrm{Alg}\left(\Spectra\right)^{hC_2} \times_{\Spectra} \Spectra^{\Delta^1}  \ar[rr,"{N^{C_2}(-^e) \times U}"'] \ar[ru,"{(-^e) \otimes(-^ e)^\mathrm{op}, (-)^e}"] \ar[ruuu,"{b \circ (-^e)}", bend left=30] & &\mathbb{E}_1\mathrm{Alg}\left(\Spectra^{C_2}\right) \times \Spectra^{C_2} \ar[lu,"{(-)^e \times (-)^e}"]
   \end{tikzcd}
   \end{equation}
   where
   \begin{itemize}
       \item $ b $ is the functor/section of Construction \ref{cons:alg_inv_is_bimod_canonically} 
       \item $ U $ is the `underlying' $C_2$-spectrum functor $ \mathbb{E}_1\mathrm{Alg}\left(\Spectra\right)^{BC_2} \times_{\Spectra} \Spectra^{\Delta^1} \to \Spectra^{BC_2} \times_{\Spectra} \Spectra^{\Delta^1} \simeq \Spectra^{C_2} $ 
       \item The upper right trapezoid commutes canonically by definition of $ \LMod $ (and the fact that the functors $ a, m $ are given by restriction to subcategories of $ LM^\otimes $). 
   \end{itemize}
   Write $ \mathbb{E}_1\mathrm{Alg}^{\mathrm{gi}}\left(\Spectra^{C_2}\right) $ for the $\infty $-category of $\mathbb{E}_1$-algebras with genuine involution. 
\end{definition}
\begin{definition}
   The category of \emph{$\mathbb{E}_\sigma$-algebras} is defined to be the limit of the $\Cat_\infty$-valued diagram 
   \begin{equation}\label{diagram:E_sigma_alg}
   \begin{tikzcd}[column sep=small]
       & \BiMod(\Spectra)_{*/-} \ar[dd,"{a,m}"]  & \\
       & & \BiMod\left(\Spectra^{C_2}\right)_{*/-} \ar[dd,"{a, m}"] \ar[lu,"{(-)^e}"'] \\
       &\mathbb{E}_1\mathrm{Alg}(\Spectra) \times \Spectra & \\
       \mathbb{E}_1\mathrm{Alg}\left(\Spectra\right)^{hC_2} \times_{\Spectra} \Spectra^{\Delta^1}  \ar[rr,"{N^{C_2}(-^e) \times U}"'] \ar[ru,"{(-^e) \otimes(-^ e)^\mathrm{op}, (-)^e}"] \ar[ruuu,"{b_* \circ (-^e)}", bend left=30] & &\mathbb{E}_1\mathrm{Alg}\left(\Spectra^{C_2}\right) \times \Spectra^{C_2} \ar[lu,"{(-)^e \times (-)^e}"]
   \end{tikzcd}\,.
   \end{equation}   
   Write $ \mathbb{E}_\sigma\mathrm{Alg}\left(\Spectra^{C_2}\right) $ for the $\infty $-category of $\mathbb{E}_\sigma$-algebras. 
\end{definition}
\begin{variant}
   Let the base be $ R $ an $ \mathbb{E}_\infty $-algebra or Poincaré ring instead of $ \mathbb{S}^0 $. 
\end{variant}
\begin{remarks}
\begin{enumerate}
   \item Compare \cite[Corollary 3.10]{AKGH_real_THH}. 
   \item There are canonical forgetful functors $  \mathbb{E}_\sigma\mathrm{Alg} \to \mathbb{E}_1\mathrm{Alg}^{\mathrm{gi}} \to \mathbb{E}_1\mathrm{Alg}^{hC_2} \to \mathbb{E}_1\mathrm{Alg}(\Spectra) $. 
\end{enumerate}
\end{remarks}
\begin{construction}\label{cons:Esigma_alg_to_R_lin_hermitian_cat}
   Let $ R, R^{\varphi C_2} \to R^{tC_2} $ be a Poincaré ring. 
   There is a functor $ \left(\Mod_{(-)}^\omega, \Qoppa_{(-)}\right) \colon \mathbb{E}_\sigma\mathrm{Alg}_R \to \left(\Cat^h_R\right)_{\left(\Mod_R^\omega,\Qoppa_R\right)/-} $. 
\end{construction}
\begin{lemma}\label{lemma:Esigma_alg_to_R_lin_hermitian_cat}
   Let $ R, R^{\varphi C_2} \to R^{tC_2} $ be a Poincaré ring. 
   \begin{enumerate}
       \item \label{lemma_item:Esigma_alg_to_R_lin_Poincare_cat} The functor of Construction \ref{cons:Esigma_alg_to_R_lin_hermitian_cat} factors through the subcategory $ \left(\Cat^p_R\right)_{\left(\Mod_R^\omega,\Qoppa_R\right)/-} $. 
       In other words, a map of $ \mathbb{E}_\sigma $-$ R $-algebras $ A \to B $ induces a duality-preserving map of $ R $-linear Poincaré $ \infty $-categories. 
       \item Write $ \Mod^p \colon \EE_1^{\mathrm{gi}} \Alg_R \to \left(\Cat^p_R\right)_{\left(\Mod_R^\omega,\Qoppa_R\right)/-} $ for the canonical factorization from part \ref{lemma_item:Esigma_alg_to_R_lin_Poincare_cat}. 
       Then $ \Mod^p $ is fully faithful. 
   \end{enumerate}
\end{lemma} 
\begin{proof}
    \Lucyil{Proof of the first point should be quite similar to/a relative variant on Corollary 3.4.2, Lemma 3.4.3 of \cite{CDHHLMNNSI}. }
    Let $ A, B $ be $ \EE_1 $-$ R $-algebras with genuine involution. 
    Then there is a fiber sequence
    \begin{equation*}
    \begin{tikzcd}[row sep=small]
        \hom_{\Mod^p_R /-}\left(\Mod^p_A, \Mod^p_B\right) \ar[r] & {\hom_{\Mod^p_R\text{-linear}}\left(\Mod^p_A, \Mod^p_B\right)} \ar[r] \ar[d,"\sim"] & {\hom_{\Mod^p_R\text{-linear}}\left(\Mod^p_R, \Mod^p_B\right)} \ar[d,"\sim"] \\
       &  \mathrm{Pn} \left(\Mod_{A^\op \otimes_R B}, \Qoppa_{A^\op \boxtimes B} \right) \ar[r] & \mathrm{Pn} \left(\Mod_{B}, \Qoppa_{B} \right) 
    \end{tikzcd}    
    \end{equation*}
    where we have used Corollary \ref{cor:poincare_fourier_mukai}. \Lucy{this refers to main text; will be fixed when we move it.} 
    The fiber of the horizontal map over the point $ (B, q_B) \in \mathrm{Pn} \left(\Mod_{B}, \Qoppa_{B} \right) $ ($q_B$ is the canonical nondegenerate form on $ B $) is %the space of $ A \otimes_R B $-module structures on $ B $ so that the induced $ N^{C_2}_R(A^\op) \otimes_{R^L}  N^{C_2}_R(B) $-module structures on $ N^{C_2}_R (B) $ is compatible with the canonical $ N^{C_2}_R(B) $-module structure on $ B $
    \Lucyil{IN PROGRESS: should be some sort of $ \EE_\sigma $ version of \cite[Proposition 3.1]{MR3190610}.}
\end{proof}

Now we observe that given a Poincaré object $ (x,q) $ of $ \left(\mathcal{C}, \Qoppa_\mathcal{C}\right) $, its endomorphism algebra admits a canonical lift to a $ \mathbb{E}_\sigma $-algebra. 
\begin{construction}\label{cons:endomorphism_of_hermitian_obj}
   There is a functor $ \mathrm{End}(-)\colon \left(\Cat^h_R\right)_{\left(\Mod_R^\omega,\Qoppa_R\right)} \to \mathbb{E}_\sigma\mathrm{Alg} $ lifting the functor $ \left(\Cat^h_R\right)_{\left(\Mod_R^\omega,\Qoppa_R\right)} \to \mathbb{E}_1\mathrm{Alg}^{hC_2} $ of \cite[Proposition 3.1.16]{CDHHLMNNSI}. 
\end{construction}
\begin{theorem}
   The functors of Construction \ref{cons:Esigma_alg_to_R_lin_hermitian_cat} and \ref{cons:endomorphism_of_hermitian_obj} form an adjoint pair. \Lucy{make everything Poincaré}
\end{theorem}
\begin{lemma}\label{lemma:poincare_endomorphisms_continuous}
    The right adjoint of Construction \ref{cons:endomorphism_of_hermitian_obj} preserves filtered colimits. 
    \Lucyil{similar to \cite[Lemma 3.4]{MR3190610}.}
\end{lemma}
\begin{proof}
    
\end{proof}
\begin{proposition}
    Let $ A $ be an $\mathbb{E}_1$-$ R $-algebra with genuine involution. 
    Then $ A $ is compact in $ \EE_1^{\mathrm{gi}} \Alg_R $ if and only if $ \Mod^p_A $ is compact in $ \Catp_R $. 
    \Lucyil{similar to \cite[Proposition 3.5]{MR3190610}.}
\end{proposition}
\begin{proof}
    The only if part of the statement follows from observing that $ \Mod^p $ admits a right adjoint which preserves filtered colimits (Lemma \ref{lemma:poincare_endomorphisms_continuous}) and \cite[Lemma 5.5.1.4]{HTT}. \Lucy{other half of the statement} 
\end{proof}
\begin{proposition}
    Let $ A $ be an $\mathbb{E}_1$-$ R $-algebra with genuine involution. 
    If $ \Mod^p_A $ is dualizable in $ \Catp_R $, then $ A $ is compact in $ \EE_1^{\mathrm{gi}} \Alg_R $.  
    \Lucyil{similar to \cite[Proposition 3.11]{MR3190610}.}
\end{proposition}
\begin{proof}
    
\end{proof}

\section{Speculative norm for Brauer group at the level of infinity categories}

	We would want a functor from $A^e$-linear stable idempotent complete infinity categories to $A^L$-linear stable idempotent complete infinity categories in order to get an extension of our exact sequence to the right. Here is what I think might do it:
	
	\begin{construction}
		Let $\lambda: A^e\to A^e$ denote the involution. Consider the functor \[\mathrm{Mod}_{\mathrm{Mod}_{A^{e}}^\omega}(\mathrm{Cat}_{\infty, \textrm{idem}}^{st})\xrightarrow{\left(-\otimes_{\mathrm{Mod}_{A^e}^\omega} \lambda^*-\right)^{hC_2}}\mathrm{Mod}_{(\mathrm{Mod}_{A^e}^\omega)^{hC_2}}(\mathrm{Cat}_{\infty, \textrm{idem}}^{st})\] which we will denote by $N_{A^{hC_2}/A}$. This functor is symmetric monoidal and we have that the composite with the base change functor gives \[\left(\mathcal{C}\otimes_{(\mathrm{Mod}_{A^e}^\omega)^{hC_2}}\mathrm{Mod}_{A^e}^\omega \otimes_{\mathrm{Mod}_{A^e}^\omega}\lambda^*(\mathcal{C}\otimes_{(\mathrm{Mod}_{A^e}^\omega)^{hC_2}}\mathrm{Mod}_{A^e}^\omega)\right)^{hC_2}\simeq \mathcal{C}^{\otimes_{(\mathrm{Mod}_{A^e}^\omega)^{hC_2}}2}\]\Noah{Maybe... We would want this to be true but I don't see a proof immediately... }
		For $\mathcal{C}\in \mathrm{Mod}_{A^e}$, we have that 
		\[(\mathcal{C}\otimes_{\mathrm{Mod}_{A^e}}\lambda^*\mathcal{C})^{hC_2}\otimes_{\mathrm{Mod}_{A^e}^{hC_2}}\mathrm{Mod}_{A^e}\simeq \mathcal{C}^{\otimes_{\mathrm{Mod}_{A^e}}2}\] via the functor which forgets the $C_2$-action. 
	\end{construction}
	
	\begin{lemma}
		The composite $\pnbr(A)\to \mathrm{br}(A^e)\to \mathrm{Pic}(\mathrm{Mod}_{\mathrm{Mod}_{A^e}^{hC_2}})$ is nullhomotopic. 
	\end{lemma}
	\begin{proof}
		The underlying category of a Poincar{\'e} invertible category is self-dual, and so its square will vanish. Since the functor is naturally nullhomotopic so too is the composite after applying the functor $\Pic(-)$.
	\end{proof}
	
	There is thus a map $\pnbr(-)\to \mathcal{F}(-)$, where $\mathcal{F}(-)$ is the fiber. Delooping both fiber sequences we see that we get a map of fiber sequences \[
	\begin{tikzcd}
		\Pic(\mathrm{Mod}_{A^L}(\mathrm{Sp}^{C_2})) \ar[r] \ar[d] & \pnbr(A) \ar[r] \ar[d] & \mathrm{br}(A^e)\ar[d,"="]\\
		\Pic(\mathrm{Mod}_{A^e}^{hC_2}) \ar[r] & \mathcal{F}(A) \ar[r] & \mathrm{br}(A^e)
	\end{tikzcd}
	\]
	from which we see that the middle horizontal map must be an equivalence whenever $\frac{1}{2}\in A^e$ and $A^{\varphi C_2}=0$.
\printbibliography

\end{document}
