\documentclass{article}
\usepackage[utf8]{inputenc}
%Packages Used------------------------------------------
% the following is to get qoppa and Qoppa
\DeclareFontFamily{T1}{cbgreek}{}
\DeclareFontShape{T1}{cbgreek}{m}{n}{<-6>  grmn0500 <6-7> grmn0600 <7-8> grmn0700 <8-9> grmn0800 <9-10> grmn0900 <10-12> grmn1000 <12-17> grmn1200 <17-> grmn1728}{}
\DeclareSymbolFont{quadratics}{T1}{cbgreek}{m}{n}
\DeclareMathSymbol{\qoppa}{\mathord}{quadratics}{19}
\DeclareMathSymbol{\Qoppa}{\mathord}{quadratics}{21}

\usepackage{amsmath, amssymb, amsthm}
\usepackage{longtable}
% \usepackage{amsfonts}
%\usepackage{mathtools}
%\usepackage{wasysym}
%\usepackage{MnSymbol}
%\usepackage{thmtools}
%\usepackage{stmaryrd}
\usepackage[letterpaper,margin=1in]{geometry}   
%\usepackage{slashed}
%\usepackage[english]{babel}				
%\usepackage[pdfencoding=auto, psdextra, draft=false]{hyperref}
\usepackage{bookmark}
\usepackage{url}		
\usepackage{lmodern}			
\usepackage[T1]{fontenc}
%\usepackage{xspace}		
%\usepackage{fancyhdr}
\usepackage{enumerate}
\usepackage{enumitem}
%\usepackage{mathrsfs}
\usepackage{graphicx}
\usepackage{soul,color}
\usepackage{tikz-cd}
\usepackage[maxbibnames=99,style=alphabetic]{biblatex}
\usepackage{csquotes}
\usepackage{chngcntr}
\usepackage[bbgreekl]{mathbbol}
\counterwithin{equation}{section}
\addbibresource{biblio.bib}
\usepackage{todonotes}

%hyperlink setup
\definecolor{darkred}{RGB}{128,0,0}
\definecolor{darkgreen}{RGB}{0,128,0}
\definecolor{darkblue}{RGB}{0,0,128}

\hypersetup{linktocpage,
	pdfborder = {0 0 0},
	colorlinks,
	citecolor=darkgreen,
	filecolor=darkred,
	linkcolor=darkblue,
	urlcolor=cyan!50!black!90}

%Greek and Latin black board bold-----------------------
\DeclareSymbolFontAlphabet{\mathbb}{AMSb}
\DeclareSymbolFontAlphabet{\mathbbl}{bbold}

%shortcut commands-------------------------------------------

\DeclareMathOperator{\Alg}{Alg} % Algebra objects
\DeclareMathOperator{\Br}{Br} % Brauer functor
\DeclareMathOperator{\Brp}{Br^p} % Poincare Brauer functor
\DeclareMathOperator{\Spec}{Spec}
\DeclareMathOperator{\CAlg}{CAlg} % Commutative Algebra objects
\DeclareMathOperator{\CAlgp}{CAlg^p} % Poincare ring spectra
\DeclareMathOperator{\Cat}{\mathcal{C}at} % Categories
\DeclareMathOperator{\Catex}{\Cat_\infty^{ex}} % stable categories with exact functors
\DeclareMathOperator{\Cath}{Cat^h_\infty} % Hermitian Categories
\DeclareMathOperator{\Cathidem}{Cat^h_{\infty,idem}} % Hermitian Categories
\DeclareMathOperator{\Catp}{Cat^p_\infty} % Poincare Categories
\DeclareMathOperator{\Catpidem}{Cat^p_{\infty, idem}} % idempotent complete Poincare Categories
\DeclareMathOperator{\Einfty}{\mathbf{E}_\infty} % E-infinity 
\DeclareMathOperator{\ex}{ex} % exact 
\DeclareMathOperator*{\fiberproduct}{\times}
\DeclareMathOperator{\Fun}{Fun} % Functors
\DeclareMathOperator{\gp}{gp} % grouplike
\DeclareMathOperator{\id}{id} % identity
\DeclareMathOperator{\idem}{idem} % idempotent
\DeclareMathOperator{\Mod}{Mod} % Modules
\DeclareMathOperator{\LMod}{LMod} % Left modules
\DeclareMathOperator{\BiMod}{BiMod} % Bimodules
\DeclareMathOperator{\Mon}{Mon} % monoids
\DeclareMathOperator{\Pic}{Pic} % Picard functor
\DeclareMathOperator{\Picp}{Pic^p} % Poincare Picard functor
\DeclareMathOperator{\Pn}{Pn} % Poincare space functor
\DeclareMathOperator{\Spectra}{Sp} % Spectra
\DeclareMathOperator{\Spaces}{\mathcal{S}} % Spaces
\DeclareMathOperator{\tee}{t} % tate/transpose
\DeclareMathOperator{\gmq}{\mathbb{G}_m^\Qoppa}


\newcommand{\pf}{{\bf Proof. \ }}
\renewcommand{\epsilon}{\varepsilon}
\renewcommand{\rho}{\varrho}
\renewcommand{\phi}{\varphi}
\newcommand{\NN}{\ensuremath{\mathbb{N}}\xspace}
\newcommand{\ZZ}{\mathbb{Z}}
\newcommand{\QQ}{\ensuremath{\mathbb{Q}}\xspace}
\newcommand{\RR}{\ensuremath{\mathbb{R}}\xspace}
\newcommand{\CC}{\mathbb{C}}
\newcommand{\FF}{\ensuremath{\mathbb{F}}\xspace}
\newcommand{\EE}{\mathbb{E}}
\newcommand{\TT}{\ensuremath{\mathbb{T}}\xspace}
\newcommand{\RP}{\ensuremath{\mathbb{RP}}\xspace}
\newcommand{\DD}{\ensuremath{\mathbbl{\Delta}}\xspace}
\newcommand{\op}{\mathrm{op}} % opposite functor
\newcommand{\sphere}{\mathbb{S}^0}
\newcommand{\tc}{\ensuremath{\mathrm{TC}}}
\newcommand{\thh}{\ensuremath{\mathrm{THH}}}
\newcommand{\tp}{\ensuremath{\mathrm{TP}}}
\newcommand{\tr}{\ensuremath{\mathrm{TR}}}
\newcommand{\pnpic}{\ensuremath{\mathrm{PnPic}}}
\newcommand{\pnbr}{\ensuremath{\mathrm{PnBr}}}
\newcommand{\pic}{\ensuremath{\mathrm{Pic}}}
\newcommand{\br}{\ensuremath{\mathrm{Br}}}
\DeclareMathOperator*{\colim}{\ensuremath{\operatorname{colim}}}
\newcommand{\aps}{\mathrm{APS}}
\newcommand{\psch}{\mathrm{PSch}}
\newcommand{\perf}{\mathrm{Perf}}
\newcommand{\perfpn}{\mathrm{Perf}^{\mathrm{Pn}}}

% Adjoint functors
\newcommand{\rlarrows}{\mathrel{\substack{\textstyle\longrightarrow\\[-0.6ex]
                                            \textstyle\longleftarrow}}}
\newcommand{\rrarrows}{\mathrel{\substack{\textstyle\longrightarrow\\[-0.6ex]
                                            \textstyle\longrightarrow}}}

%Theorem Environments ----------------------------------------------------------------
\newtheorem{theorem}[equation]{Theorem}
\newtheorem{proposition}[equation]{Proposition}
\newtheorem{lemma}[equation]{Lemma}
\newtheorem{corollary}[equation]{Corollary}

\theoremstyle{definition}
\newtheorem{definition}[equation]{Definition}
\newtheorem{construction}[equation]{Construction}
\newtheorem{remark}[equation]{Remark}
\newtheorem{remarks}[equation]{Remarks}
\newtheorem{observation}[equation]{Observation}
\newtheorem{notation}[equation]{Notation}
\newtheorem{example}[equation]{Example}
\newtheorem{recollection}[equation]{Recollection}
\newtheorem{variant}[equation]{Variant}

\newcommand{\Viktor}[1]{\todo{V: #1}}
\newcommand{\Noah}[1]{\todo[color=red]{N: #1}}
\newcommand{\Lucy}[1]{\todo[color=cyan!30]{\linespread{1}\footnotesize L: #1}}
\newcommand{\Lucyil}[1]{\todo[color=cyan!30,inline]{\linespread{1}\footnotesize L: #1}}

\title{Poincar{\'e} Schemes}
\author{Viktor Burghardt, Noah Riggenbach, Lucy Yang}
\date{}
\addbibresource{biblio.bib}

\begin{document}

\maketitle
\begin{abstract}
In this paper, we use the formalism of Poincar{\'e} $ \infty $-categories, as developed by \cite{CDHHLMNNSI}, to define and study derived moduli stacks of line bundles with hermitian metrics and of Azumaya algebras equipped with an involution. \Lucy{this is not strictly speaking true unless we add ``up to Morita equivalence''--revisit later? }
    Our moduli spaces give rise to cohomological invariants which we term the Poincar{\'e} Picard group and the Poincar{\'e} Brauer group; these are enhancements of the ordinary Picard and Brauer groups which incorporate the data of an involution on the base. 
    We show that in the \'etale case away from characteristic $2$, the Poincar{\'e} Brauer group recovers the involutive Brauer group of \cite{MR1162189}. 
    We also define the Poincar{\'e} Picard and Brauer groups for Poincar{\'e} rings in spectra, and compute these invariants for the sphere spectrum and other examples.
\end{abstract}
\tableofcontents

\section{Introduction} 
\subsection{Azumaya algebras and their involutions}
Let $ X $ be a scheme. 
Gabber showed that if $ X $ is quasi-compact and separated and admits an ample line bundle, the collection of sheaves of Azumaya $ \mathcal{O}_X $-algebras up to Morita equivalence is in bijection with the torsion subgroup of $ H^2_{\mathrm{\acute{e}t}}(X; \mathbb{G}_m) $ \cite{MR611868}. 
On the other hand, to $ \mathcal{A} $ we may associate the presentable $ \mathcal{O}_X $-linear $ \infty $-category $ \mathcal{D}(\mathcal{A}) $. \Lucy{AG notation or homotopy notation?} 
If $ \mathcal{A} $ and $ \mathcal{A}' $ are Morita equivalent, then $ \mathcal{D}(\mathcal{A}') $ and $ \mathcal{D}(\mathcal{A}) $ are equivalent in $ \mathrm{Pr}^L_{\mathcal{O}_X} $, and that $ \mathcal{A} $ is Azumaya implies that $ \mathcal{D}(\mathcal{A}) $ is invertible in $ \mathrm{Pr}^L_{\mathcal{O}_X} $ with respect to the $ \mathcal{O}_X $-linear tensor product. \Lucy{who is this observation due to?} \Lucy{cite Lieblich/de Jong? connect to twisted sheaves!} 

\begin{definition}
    The \emph{derived Brauer space} of $ X $ is $ \mathrm{dBr}(X) := \Pic\left(\mathrm{Pr}^L_{\mathcal{O}_X} \right) $. 
\end{definition}
Toën shows that the assignment $ X \mapsto \mathrm{dBr}(X) $ is an étale sheaf and there is an isomorphism $ \pi_0 \mathrm{dBr}(X) \simeq H^1_{\mathrm{\acute{e}t}}(X; \ZZ) \times H^2_{\mathrm{\acute{e}t}}(X; \mathbb{G}_m) $ \cite[Corollary 2.12]{MR2957304}. 
Furthermore, Toën shows that for any qcqs scheme $ X $, any invertible $ \mathcal{O}_X $-linear $ \infty $-category $ \mathcal{C} $ has a compact generator $ G $; thus we may write $ \mathcal{C} \simeq \Mod_{\mathrm{End}_{\mathcal{C}}(G)} \left(\Mod_{\mathcal{O}_X}\right) $ and $ \mathrm{End}_{\mathcal{C}}(G) $ is said to be a \emph{derived/generalized (sheaf of) Azumaya algebras} over $ \mathcal{O}_X $ (see Theorem 3.7 and Corollary 3.8 of \cite{MR2957304}). 
In particular, Toën's result gives concrete interpretations/realizations of all/not necessarily torsion classes in $ H^2_{\mathrm{\acute{e}t}}(X;\mathbb{G}_m) $, at the cost of using derived methods/considering derived objects. 

Antieau and Gepner extended Toën's work to connective ring spectra/affine spectral schemes with connective rings of functions in \cite{MR3190610}. \Lucy{cite \href{https://arxiv.org/abs/2210.15743}{this} too?} 

On the other hand, the theory of anti-involutions on Azumaya algebras is an essential tool for studying the behavior of Brauer classes under multiplication by 2 and/or norm maps/corestriction. 
An anti-involution on an Azumaya algebra over a ring $ R $ is an equivalence $ \sigma \colon A \xrightarrow{\sim} A^\op $ so that $ \sigma^\op \circ \sigma = \mathrm{id} $. 
The anti-involution is said to be of type $ 1 $ if it acts by the identity on $ R = $ the center of $ A $. 
If instead $ R $ is endowed with an involution $ \lambda $ so that the inclusion of the subring of fixed elements $ R^\lambda \to R $ exhibits $ R $ as a quadratic étale $ R^\lambda $-algebra and $ \sigma $ agrees with $ \lambda $ on the center of $ A $, then the involution $ \sigma $ is said to be of \emph{type 2}. 

If an Azumaya algebra $ A $ has an anti-involution, its Brauer class $ [A] \in H^2_{\acute{e}t}(-;\mathbb{G}_m) $ is manifestly 2-torsion. 
More surprisingly, if an Azumaya algebra $ A $ is such that $ [A] $ lies in the 2-torsion subgroup of $ H^2_{\acute{e}t}(-;\mathbb{G}_m) $, then there exists an Azumaya algebra $ A' $ in the Brauer class of $ A $ admitting an anti-involution; the result was proved for $ R $ a field by Albert (in fact Albert proves that one can take $ A' = A $), for an arbitrary ring by Saltman, and for schemes $ X $ with $ \frac{1}{2} \in \mathcal{O}_X $ by Parimala--Srinivas \cites[\S9 Theorem 19]{MR123587}[Theorem 3.1(a)]{MR495234}[Theorem 1]{MR1162189}. 

On the other hand, consider an étale cover $ X \to Y $ of degree 2 where $ \frac{1}{2} \in \mathcal{O}_Y $, and let $ \lambda $ denote the nontrivial $ C_2 $-Galois action on $ X $. 
There is an \emph{involutive Brauer group} $ \mathrm{Br}\left(X, \lambda \right) $ consisting of equivalence classes of sheaves of Azumaya $ \mathcal{O}_X $-algebras with involutions of the second kind \cite[p.216]{MR1162189}. 
Parimala--Srinivas showed that the involutive Brauer group sits in an exact sequence $ \Pic X \xrightarrow{N} \Pic Y \to \mathrm{Br}\left(X, \lambda \right) \to \mathrm{Br}\left(X\right) \xrightarrow{N} \mathrm{Br}\left(Y\right) $ \cite[Theorem 2]{MR1162189}.  

First and Williams observed that the aforementioned two cases comprise two extreme ends/special cases of a spectrum: multiplication by $ 2 $ on $ H^2_{\acute{e}t}(-;\mathbb{G}_m) $ may be regarded as a cohomological $ C_2 $-transfer map along the `quotient' map $ X = X $, where $ X $ is regarded as a scheme with trivial $ C_2 $-action \cite[\S1.2]{azumaya_involution}. 
In other words, the former comprises trivial $ C_2 $-actions with `everywhere ramified' quotient map, whereas the latter comprises free $ C_2 $-actions with nowhere ramified quotient maps. 
Moreover, First and William observe that a quotient involves a choice/is extra data; from the perspective of the stacky quotient $ X \to X//C_2 $, the $ C_2 $-action on $ X $ is free. \Lucy{connect to `many choices of different Poincaré structures' on a given $ \infty $-category eventually}  
Thus in order to study involutions systematically, it is necessary to specify a scheme with involution and the choice of a quotient. \Lucy{this is \emph{so close} to Spec of a Poincaré ring but also not quite.}

The study of involutions on Azumaya algebras goes hand-in-hand with/is inextricably linked to the study of symmetric bilinear/hermitian forms on vector bundles: étale-locally, (classical) Azumaya algebras can be described as endomorphism algebras of vector bundles $ \mathcal{A}\simeq \mathcal{E}\mathrm{nd}_X(V) $. 
Taking transposes and conjugating by an isomorphism $ q \colon V \to V^\vee $ adjoint to a nondegenerate \Lucy{insert adjectives} pairing $ q \colon V \otimes V \to \mathcal{O}_X $ comprise a `prototypical' example of an involution on $ \mathcal{E}\mathrm{nd}_X(V) $. 
In \cites{CDHHLMNNSI,CDHHLMNNSII,CDHHLMNNSIII}, Calmès--Dotto--Harpaz--Hebestreit--Land--Moi--Nardin--Nikolaus--Steimle introduce a framework for talking about objects of stable $ \infty $-categories equipped with the data of nondegenerate hermitian forms, expanding upon an idea introduced in \cite{Lurie_Ltheory_notes}. 
Motivated by this connection between involutions and duality (cf. \cite[\S3]{CDHHLMNNSI}), we use Poincaré $ \infty $-categories to define a derived enhancement of the involutive Brauer group.  
% In this work, we study a derived enhancement of the involutive Brauer group. \Lucy{later: make this more specific} 

\subsection{Main results}
\begin{theorem}
    Let $\mathcal{C}$ denote either the category of Poincar{\'e} rings (Definition~\ref{definition:poincare_ring_spectrum}) or the opposite category of schemes with involution and good quotients (Definition~\ref{defn:Category of good quotients}). Then there are functors \[\mathbb{G}_m^\Qoppa, \pnpic, \pnbr:\mathcal{C}\to \mathrm{Sp}_{\geq 0}\] such that 
    \begin{itemize}
        \item $\Omega \pnbr\simeq \pnpic$ and $\Omega \pnpic\simeq \mathbb{G}_m^\Qoppa$;
        \item $\mathbb{G}_m^\Qoppa$ is represented by the free $\mathbb{E}_\infty$ ring on one invertible generator $\mathbb{S}\{x^{\pm 1}\}$ with a certain genuine $C_2$-structure (see \ref{const: gmq});
        \item $\pnbr$ (and hence all the other functors) satisfy hyperdescent for an analogue of the {\'e}tale topology on $\mathcal{C}$ which incorporates the involution (see Notation~\ref{notation:poincare_ring_basechange} and Notation~\ref{notation:scheme_involution_basechange} for the exact constructions.) 
    \end{itemize}
\end{theorem}

These invariants naturally admit forgetful functors to the underlying Picard and Brauer spaces. The exact connection is the following:

\begin{theorem}
    Let $R$ be a Poincar{\'e} ring, and denote the underlying ring spectrum by $R^e$ and the underlying genuine $C_2$ ring spectrum by $R^L$. Then there is a fiber sequence \[\Pic(\mathrm{Mod}_{R^L}(\mathrm{Sp}^{C_2}))\to \pnbr(R)\to \mathrm{br}(R^e)\] and upon sheafification there is a similar fiber sequence for schemes with involutions and good quotients.
\end{theorem}

With this fiber sequence in hand we are able to make several computations of interest. As an example:

\begin{corollary}
    Let $\mathbb{S}^u$ denote the Poincar{\'e} ring spectrum associated to the initial Poincar{\'e} infinity category. Then \[\pnbr(\mathbb{S}^u)\simeq \ZZ\times B\mathrm{gl}_1(\mathrm{TR}^2(\mathbb{S};2))\]
\end{corollary}

In the case when $R$ is away from characteristic $2$ we are furthermore able to simplify the fiber of the comparison map $\pnbr(R)\to \mathrm{br}(R^e)$. As an application of this we obtain a comparison of the Poincar{\'e} Brauer group with the involuative Brauer group discussed earlier in the introduction:
\begin{theorem}[Theorem~\ref{thm: PS comparison in text}]~\label{thm: comparison to PS}
Let $(X,\lambda,Y,\pi)$ be a scheme with involution and good quotient such that $\pi:X\to Y$ is {\'e}tale and such that $\frac{1}{2}\in \Gamma(\mathcal{O}_Y)$. Then there is an equivalence \[\pnbr(X,\lambda,Y,\pi)\cong \mathrm{Br}(X,\lambda)\] where the second term is the involuative Brauer group of Parimala-Srinivas (\cite{MR1162189}).
\end{theorem}

\subsection{Outline}

\subsection{Acknowledgements} 
The authors wish to thank the Institute for Advanced Study and the organizers of the 2024 Park City Mathematics institute on motivic homotopy theory. 
Also thank: Columbia, Ben Antieau, James Hotchkiss.
\Lucyil{grants? people? travel funding?}

\subsection{Conventions}
\label{subsection:conventions} 
\Lucyil{Will `Azumaya algebra' refer to the classical/discrete ones or the derived/generalized ones? Note the distinction between higher (classical) stacks and spectral stacks!}
    \begin{longtable}{lll}
        $\Brp$ & Poincaré Brauer space\\
        $\CAlg$ & $\infty$-category of $\Einfty$-ring spectra\\
        $\CAlg(\Spaces)$ & $\infty$-category of $\Einfty$-spaces\\
        $\CAlg^{\gp}(\Spaces)$ & $\infty$-category of grouplike $\Einfty$-spaces\\
        $\CAlgp$ & $\infty$-category of Poincaré ring spectra\\
        $\Catex$ & $\infty$-category of small stable $\infty$-categories and exact functors\\
        $\Catp$ & $\infty$-category of Poincaré $\infty$-categories\\
        $\Catpidem$ & $\infty$-category of idempotent complete Poincaré $\infty$-categories\\
        $\Picp$ & Poincaré Picard space\\
        $\Spaces$ & $\infty$-category of spaces\\
        $\Spectra$ & $\infty$-category of spectra
    \end{longtable}

\section{Poincaré Structures on Compact Modules}
\label{section:poincare_structures_on_compact_modules}
We will use this section to recall notions and results about Poincaré $\infty$-categories which we require in the sections to follow. This section can safely be skipped by anyone with extensive knowledge of Poincaré $\infty$-categories, as found in \cite{CDHHLMNNSI}.

\begin{notation}
    \label{notation:omission_of_e_infty}
    Let $R$ be an $\mathbf{E}_\infty$-ring spectrum. We will drop $\mathbf{E}_\infty$ from our notation and simply call $R$ a \emph{ring spectrum}. Moreover, we will denote the $\infty$-category $\CAlg(\Spectra)$ of commutative algebra objects in the $\infty$-category of spectra $\Spectra$ by $\CAlg$. 
\end{notation}

Let $R$ be a ring spectrum and let $\Mod_R$ be the $\infty$-category of modules over $R$. We will study Poincaré structures on the $\infty$-category $\Mod_R^\omega$ of compact modules over $R$. 

\Viktor{-characterization in terms of modules with genuine involution, -characterization of symmetric monoidal structures, -Pn}

\section{Poincaré Schemes}
\label{section:poincare_ring_spectra}
We will now specify the objects which we are able to take the Poincar{\'e} Picard and Brauer spaces of. These will be schemes which are equipped with a Poincar{\'e} structure on their derived categories which is compatible with the symmetric monoidal structure. While this definition is simple and convenient, we find it both technically useful and psychologically comforting to have a  definition of such objects closer to a scheme with an involution. We will start by looking at affine objects.

\subsection{Poincar{\'e} rings}

\begin{definition}
    \label{definition:poincare_ring_spectrum}
    Let $R$ be a ring spectrum. 
    A \emph{Poincaré structure} on $R$ is the following data:
    \begin{itemize}
        \item A $C_2$-action on $R$ via maps of ring spectra, i.e. a functor $\lambda: BC_2\rightarrow \CAlg$.
        \item An $ \EE_\infty $-$R$-algebra $R\rightarrow C$.
        \item An $ \EE_\infty $-$R$-algebra map $C\rightarrow R^{tC_2}$. 
    \end{itemize}
    Here $R^{tC_2}$ is the Tate construction with respect to the given action. 
    Since the Tate construction is lax symmetric monoidal, $R^{tC_2}$ is naturally an $R$-algebra via the Tate-valued norm. A ring spectrum equipped with a Poincaré structure will be called a \emph{Poincaré ring spectrum}. 
\end{definition}

\begin{remark}
    \label{remark:notational_difference_to_nine-authored_papers}
    % Poincaré ring spectra, as defined in Definition \ref{definition:poincare_ring_spectrum}, were studied in \Viktor{cite 9 authored paper}. Note that we chose a different notation. 
    In \cite[discussion immediately preceding Examples 5.4.10]{CDHHLMNNSI}, Poincaré ring spectra are referred to as $\mathbf{E}_\infty$-\emph{ring spectra with genuine involution}.
\end{remark}

\begin{remark}
    \label{remark:poincare_ring_spectra_to_modules_with_Poincare_structure}
    Let $R$ be a ring spectrum. 
    % There is a natural equivalence between symmetric monoidal Poincaré structures on $\Mod_R^\omega$ and certain algebra objects over the genuine $C_2$-spectrum $NR$ . 
    By \cite[Corollary 5.4.8]{CDHHLMNNSI}, a Poincaré structure on $R$ gives rise to a symmetric monoidal lift of $ \Mod_R^\omega $ to the symmetric monoidal $ \infty $-category of Poincaré $\infty$-categories $ \Qoppa_R: (\Mod_R^\omega)^{\op}\rightarrow \Spectra $. 
    \Lucy{Better notation? Write $ R $ for the underlying $ \EE_\infty $-ring with $ C_2 $-action and $ q \colon C \to R^{tC_2} $ and $ \Qoppa_{(R,q)} $ or $ \Qoppa_q $ for the functor.}
    Furthermore, the structure map $ R \to C $ gives a canonical lift of $ R \in \Mod_R^\omega $ to a Poincaré object $ (R, q) \in \mathrm{Pn}\left(\Mod_R^\omega,\Qoppa_R \right) $. 
    % We call such a symmetric monoidal Poincaré $\infty$-category a \emph{Poincaré ring spectrum}. We will denote the full subcategory of $\CAlg(\Catp)$ spanned by Poincaré ring spectra by $\CAlgp$ and call it the \emph{$\infty$-category of Poincaré ring spectra}. 
\end{remark}

\begin{remark}
    \label{remark:poincare_structures_are_factorizations}
    A Poincaré structure on a ring spectrum $R$ with a $C_2$-action via maps of ring spectra is a factorization $R\rightarrow C \rightarrow R^{tC_2}$ in $\CAlg$ of the Tate Frobenius $R\rightarrow R^{tC_2}$.
\end{remark}
\begin{observation}\label{observation:normed_C2_ring_forget_to_Poincare_ring}
    There is a forgetful functor from $ C_2 $-$ \EE_\infty $-algebras to Poincaré rings which forgets the $ C_2 $-equivariance of the map $ R \to R^{\varphi C_2 } $. 
\end{observation}

\begin{remark}
    \label{remark:poincare_ring_spectra_as_algebra_objects}
    By \cite[\S5.1]{CDHHLMNNSI}, the assignment $ (R, R\to C \to R^{tC_2}) \mapsto \left(\Mod_R^\omega, \qoppa_R \right) $ promotes to a symmetric monoidal functor $ \CAlgp \to \CAlg(\Catp) $. 
    % Let $\mathcal{M}$ be the full subcategory of $\Catp$ spanned by Poincaré $\infty$-categories with underlying $\infty$-category $\Mod^\omega_R$ for some ring spectrum $R$. Then the symmetric monoidal structure of $\Catp $ restricts to a symmetric monoidal structure on $\mathcal{M}$ by Example \ref{example:universal_poincare_ring_spectrum} and \cite[\S5.1]{CDHHLMNNSI}. Then we have $\CAlgp\simeq \CAlg(\mathcal{M})$. In particular, the symmetric monoidal structure of $\CAlg(\Catp)$ restricts to a symmetric monoidal structure on $\CAlgp$.
\end{remark}

\begin{notation}
    \label{notation:spectrum_with_trivial_action}
    Let $R$ be an $ \mathbb{E}_\infty $-ring spectrum. We will denote by $\underline{R}$ the spectrum $R$ with trivial $ C_2 $-action. More precisely, $\underline{R}:BC_2\rightarrow \Spectra $ is the constant functor. \Lucy{This is commonly used for constant Mackey functors--could be ambiguous}
\end{notation}

\begin{example}
    \label{example:classification_of_poincare_structures_when_tate_vanishes}
    Let $R$ be a ring spectrum with a $ C_2 $-action. If $2\in \pi_0(R)$ is invertible, we have $\underline{R}^{tC_2}\simeq 0 $ by \cite[Lemma I.2.8]{NS}. 
    A Poincaré structure on $R$ is equivalent to the data of an $ \mathbb{E}_\infty $-$R$-algebra $R\rightarrow C$.
\end{example}

\begin{example}
    \label{example:tate_poincare_structure}
    Let $R$ be a ring spectrum equipped with a $C_2$-action via maps of ring spectra. The Tate-valued norm endows $ R^{tC_2} $ with a natural $R$-algebra structure, which induces a Poincaré structure on $R$ given by the factorization $R\xrightarrow{\id} R\rightarrow R^{tC_2}$. 
    We will call this Poincaré structure the \emph{Tate Poincaré structure on $R$} and will denote it by $(R,\Qoppa_R^{\tee})$.
\end{example}

\begin{example}
    \label{example:universal_poincare_ring_spectrum}
    The sphere spectrum $\mathbb{S}$ together with the Tate Poincaré structure will be called the \emph{universal Poincaré ring spectrum} (see \cite[\S4.1]{CDHHLMNNSI}). We will denote it by $(\mathbb{S},\Qoppa_u)$. 
\end{example}

\begin{remark}
    \label{remark:factorizations_of_tate_frobenius_through_invariants_induce_splittings_of_forms}
    Let $(R,\Qoppa)$ be a ring spectrum associated to a factorization $R\rightarrow C\rightarrow R^{tC_2}$. A factorization of the map $C\rightarrow R^{tC_2}$ through $R^{hC_2}$ induces a section of the canonical map $\Qoppa(R)\rightarrow \hom_R(R,C)\simeq C$. In that case, we have a splitting $\Qoppa(R)\simeq R_{hC_2}\oplus C$\Viktor{reference pullback that characterizes all quadratic functors}.
\end{remark}

\begin{example}
    \label{example:universal_tate_poincare_splits_at_unit}
    The Tate Frobenius for the sphere spectrum factors through $\mathbb{S}^{hC_2}$. Therefore, Remark \ref{remark:factorizations_of_tate_frobenius_through_invariants_induce_splittings_of_forms} implies $\Qoppa_u(\mathbb{S})\simeq \mathbb{S}_{hC_2}\oplus \mathbb{S}\simeq \Sigma^\infty(\mathbb{P}_\mathbb{R}^\infty) \oplus \mathbb{S}$.
\end{example}

\begin{example}
    \label{example:symmetric_poincare_structure}
    Let $R$ be a ring spectrum equipped with a $C_2$-action via maps of ring spectra. The identity map $\id: R^{tC_2}\rightarrow R^{tC_2}$ induces a Poincaré structure on $R$
    given by the factorization $R\rightarrow R^{tC_2}\xrightarrow{id} R^{tC_2}$. We will call this Poincaré structure the \emph{symmetric Poincaré structure on $R$}.
\end{example}

\begin{example}
    \label{example:genuine_symmetric_poincare_structure}
    Let $R$ be a connective ring spectrum equipped with a $C_2$-action via maps of ring spectra. The connective cover $\tau_{\geq 0}(R^{tC_2})\rightarrow R^{tC_2}$ of $R^{tC_2}$ induces a Poincaré structure on $R$ given by the factorization $R\rightarrow \tau_{\geq 0}(R^{tC_2})\rightarrow R^{tC_2}$. We will call this Poincaré structure the \emph{genuine symmetric Poincaré structure on $R$}.
\end{example}

\begin{example}
    \label{ex:fixpt_Mackey_functor}   
    Let $ R $ be a commutative ring endowed with an involution $ \sigma \colon R \xrightarrow{\sim} R $. 
    Write $ \underline{R}^\sigma $ for the $ C_2 $-Green functor with $ C_2 $-fixed points $ R^{C_2} $, where $ R^{C_2} $ denotes the strict fixed points of the $ C_2 $-action on $ R $, and underlying object $ R $. 
    The Mackey functor $ \underline{R}^\sigma $ is a $ C_2 $-$ \EE_\infty $ ring, therefore in particular we may regard it as a Poincaré ring by Observation \ref{observation:normed_C2_ring_forget_to_Poincare_ring}. 
    This is a special case of Example \ref{example:genuine_symmetric_poincare_structure}. 
\end{example}

\Viktor{copy more examples from notes}

%\subsection{Algebras with genuine involution}
%\begin{recollection}\label{rec:left_module_cats}  
%    Assume $ \mathcal{C} $ is a presentable monoidal $ \infty $-category such that the monoidal product $ - \otimes - \colon \mathcal{C} \times \mathcal{C} \to \mathcal{C} $ preserves small colimits separately in each variable. 
%    Then there is an $ \infty $-category $ \LMod(\mathcal{C}) $ \cite[Example 4.2.1.18]{LurHA} whose objects are pairs $ (A, M) $ where $ A $ is an associative algebra object of $ \mathcal{C} $ and $ M $ is a left $ A $-module. 
%    Write $ a, m $ respectively for the canonical forgetful functors $ \LMod(\mathcal{C})\to \mathrm{Alg}(\mathcal{C}) $, $ \LMod(\mathcal{C}) \to \mathcal{C} $ which send $ (A, M) $ to $ A $ and $ M $, resp. 
%    Then $ a $ is a cocartesian fibration \cite[Corollary 4.2.3.7]{LurHA}, hence it is classified by a functor $ \mathrm{mod} \colon \mathrm{Alg}(\mathcal{C}) \to \Cat_\infty $. 

%    The functor $ s $ of \cite[Example 4.2.1.17]{LurHA} determines a natural transformation $ \eta \colon * \to \mathrm{mod} $, where $ * \colon \mathrm{Alg}(\mathcal{C}) \to \{*\} \hookrightarrow \Cat_\infty $ is the constant functor at the trivial category, or equivalently
%    \begin{equation}
%    \begin{tikzcd}
%        & \mathcal{U} \ar[d] \\
%         \mathrm{Alg}(\mathcal{C}) \ar[r,"\mathrm{mod}"] \ar[ru,"\eta"] & \Cat_\infty   
%    \end{tikzcd}    
%    \end{equation}
%    where $ \mathcal{U} $ is the universal cocartesian fibration. 
%    Now consider the functor $ o \colon \mathcal{U} \to \Cat_\infty $ which sends $ (\mathcal{D}, d \in \mathcal{D}) $ to the undercategory $ \mathcal{D}_{d/-} $. 
%    Define $ \LMod(\mathcal{C})_{*/-} $ to be the cocartesian fibration over $ \mathrm{Alg}(\mathcal{C}) $ classified by $ o \circ \eta \circ \mathrm{mod} $. 
%\end{recollection}
%\begin{variant}
%    Let $ \mathrm{C} $ be as in Recollection \ref{rec:left_module_cats}. 
%    There is a similar construction where left modules is replaced by \emph{bimodules} \cite[Definition 4.3.1.12]{LurHA}. 
%\end{variant}
%\begin{construction}\label{cons:alg_inv_is_bimod_canonically}
%    Regard $ \mathbb{E}_1\mathrm{Alg}(\Spectra) $ as a category with $ C_2 $-action given by taking the opposite/reverse algebra.  
%    There are functors $ b \colon %\mathbb{E}_1\mathrm{Alg}\left(\Spectra\right)^{hC_2} \to \LMod(\Spectra) $ and $ b \colon \mathbb{E}_1\mathrm{Alg}\left(\Spectra\right)^{hC_2} \to \BiMod(\Spectra)_{*/-} $ so that $ (a, m) \circ b $ and $ (a, m)\circ b_* $ are (canonically) equivalent to $ (-^e) \otimes(-^ e)^\mathrm{op}, (-)^e $. 
%    Informally, an $ \mathbb{E}_1 $-algebra with involution $ B $ can be regarded as a $ B \otimes B^\op $-module in a canonical way, and there is a canonical $ B \otimes B^\op $-module map $ B \otimes B^\op \to B $. 
%\end{construction}
%\begin{definition}
%    The category of \emph{$\mathbb{E}_1$-algebras with genuine involution} is defined to be the limit of the $\Cat_\infty$-valued diagram 
%    \begin{equation}\label{diagram:E1_alg_gen_involution}
%    \begin{tikzcd}[column sep=small]
%        & \LMod(\Spectra) \ar[dd,"{a,m}"]  & \\
%        & & \LMod\left(\Spectra^{C_2}\right) \ar[dd,"{a, m}"] \ar[lu,"{(-)^e}"'] \\
%        &\mathbb{E}_1\mathrm{Alg}(\Spectra) \times \Spectra & \\
%        \mathbb{E}_1\mathrm{Alg}\left(\Spectra\right)^{hC_2} \times_{\Spectra} \Spectra^{\Delta^1}  \ar[rr,"{N^{C_2}(-^e) \times U}"'] \ar[ru,"{(-^e) \otimes(-^ e)^\mathrm{op}, (-)^e}"] \ar[ruuu,"{b \circ (-^e)}", bend left=30] & &\mathbb{E}_1\mathrm{Alg}\left(\Spectra^{C_2}\right) \times \Spectra^{C_2} \ar[lu,"{(-)^e \times (-)^e}"]
%    \end{tikzcd}
%    \end{equation}
%    where
%    \begin{itemize}
%        \item $ b $ is the functor/section of Construction \ref{cons:alg_inv_is_bimod_canonically} 
%        \item $ U $ is the `underlying' $C_2$-spectrum functor $ \mathbb{E}_1\mathrm{Alg}\left(\Spectra\right)^{BC_2} \times_{\Spectra} \Spectra^{\Delta^1} \to \Spectra^{BC_2} \times_{\Spectra} \Spectra^{\Delta^1} \simeq \Spectra^{C_2} $ 
%        \item The upper right trapezoid commutes canonically by definition of $ \LMod $ (and the fact that the functors $ a, m $ are given by restriction to subcategories of $ LM^\otimes $). 
%    \end{itemize}
%    Write $ \mathbb{E}_1\mathrm{Alg}^{\mathrm{gi}}\left(\Spectra^{C_2}\right) $ for the $\infty $-category of $\mathbb{E}_1$-algebras with genuine involution. 
%\end{definition}
%\begin{definition}
%    The category of \emph{$\mathbb{E}_\sigma$-algebras} is defined to be the limit of the $\Cat_\infty$-valued diagram 
%    \begin{equation}\label{diagram:E_sigma_alg}
%    \begin{tikzcd}[column sep=small]
%        & \BiMod(\Spectra)_{*/-} \ar[dd,"{a,m}"]  & \\
%        & & \BiMod\left(\Spectra^{C_2}\right)_{*/-} \ar[dd,"{a, m}"] \ar[lu,"{(-)^e}"'] \\
%        &\mathbb{E}_1\mathrm{Alg}(\Spectra) \times \Spectra & \\
%        \mathbb{E}_1\mathrm{Alg}\left(\Spectra\right)^{hC_2} \times_{\Spectra} \Spectra^{\Delta^1}  \ar[rr,"{N^{C_2}(-^e) \times U}"'] \ar[ru,"{(-^e) \otimes(-^ e)^\mathrm{op}, (-)^e}"] \ar[ruuu,"{b_* \circ (-^e)}", bend left=30] & &\mathbb{E}_1\mathrm{Alg}\left(\Spectra^{C_2}\right) \times \Spectra^{C_2} \ar[lu,"{(-)^e \times (-)^e}"]
%    \end{tikzcd}\,.
%    \end{equation}   
%    Write $ \mathbb{E}_\sigma\mathrm{Alg}\left(\Spectra^{C_2}\right) $ for the $\infty $-category of $\mathbb{E}_\sigma$-algebras. 
%\end{definition}
%\begin{variant}
%    Let the base be $ R $ an $ \mathbb{E}_\infty $-algebra or Poincaré ring instead of $ \mathbb{S}^0 $. 
%\end{variant}
%\begin{remarks}
%\begin{enumerate}
%    \item Compare \cite[Corollary 3.10]{AKGH_real_THH}. 
%    \item There are canonical forgetful functors $  \mathbb{E}_\sigma\mathrm{Alg} \to \mathbb{E}_1\mathrm{Alg}^{\mathrm{gi}} \to \mathbb{E}_1\mathrm{Alg}^{hC_2} \to \mathbb{E}_1\mathrm{Alg}(\Spectra) $. 
%\end{enumerate}
%\end{remarks}
%\begin{construction}\label{cons:Esigma_alg_to_R_lin_hermitian_cat}
%    Let $ R, R^{\varphi C_2} \to R^{tC_2} $ be a Poincaré ring. 
%    There is a functor $ \left(\Mod_{(-)}^\omega, \Qoppa_{(-)}\right) \colon \mathbb{E}_\sigma\mathrm{Alg}_R \to \left(\Cat^h_R\right)_{\left(\Mod_R^\omega,\Qoppa_R\right)/-} $. 
%\end{construction}
%\begin{lemma}
%    Let $ R, R^{\varphi C_2} \to R^{tC_2} $ be a Poincaré ring. 
%    The functor of Construction \ref{cons:Esigma_alg_to_R_lin_hermitian_cat} is fully faithful. 
%\end{lemma} 
%\begin{proof}
    
%\end{proof}

%Now we observe that given a hermitian object $ (x,q) $ of $ \left(\mathcal{C}, \Qoppa_\mathcal{C}\right) $, its endomorphism algebra admits a canonical lift to a $ \mathbb{E}_\sigma $-algebra. 
%\begin{construction}\label{cons:endomorphism_of_hermitian_obj}
%    There is a functor $ \mathrm{End}(-)\colon \left(\Cat^h_R\right)_{\left(\Mod_R^\omega,\Qoppa_R\right)} \to \mathbb{E}_\sigma\mathrm{Alg} $ lifting the functor $ \left(\Cat^h_R\right)_{\left(\Mod_R^\omega,\Qoppa_R\right)} \to \mathbb{E}_1\mathrm{Alg}^{hC_2} $ of \cite[Proposition 3.1.16]{CDHHLMNNSI}. 
%\end{construction}
%\begin{theorem}
%    The functors of Construction \ref{cons:Esigma_alg_to_R_lin_hermitian_cat} and \ref{cons:endomorphism_of_hermitian_obj} form an adjoint pair. 
%\end{theorem}

% \section{Modules over Poincaré Ring Spectra}
% \label{subsection:modules_over_poincare_ring_spectra}

\subsection{From schemes with involution to Poincaré structures on module categories}\label{subsection:Poincare_structures_module_cats} 

We will now turn our attention to the non-affine case. In this setting we will again want to work with schemes with some notion of a genuine $C_2$-structure as our model, and then show that this leads to the structure of a scheme together with a symmetric monoidal structure on its derived category. 

Philosophically, a scheme with genuine $C_2$-action should, via a recollement, be given by a scheme $X$ with an involution together with a choice of genuine $C_2$ quotient $X\to Y$ satisfying certain conditions. It turns out that such a notion has already appeared in the literature on Azumaya algebras with involution.

\begin{recollection} [{\cite[Remark 4.20]{azumaya_involution}}]
    Let $ X $ be a scheme with an involution $ \lambda \colon X \to X $. 
    A map $ \pi \colon X \to Y $ is called a good quotient of $ X $ relative to $ \lambda $ if $ \pi $ is $ C_2 $-invariant and affine and $ \pi_{\#} \colon \mathcal{O}_Y \to \pi_{*} \mathcal{O}_X $ induces an isomorphism $ \mathcal{O}_Y \simeq \left(\pi_{*} \mathcal{O}_X\right)^{C_2} $. 
    A good quotient of $ X $ exists if and only if every $ C_2 $-orbit is contained in an affine open subscheme, in which case it is unique up to isomorphism. 
\end{recollection}
\begin{definition}~\label{defn:Category of good quotients}
    Define a category $ \mathrm{qSch}^{C_2} $ so that
    \begin{itemize}
        \item an object of $ \mathrm{qSch}^{C_2} $ consists of the data of qcqs schemes $ X $ and $ Y $, an involution $ \lambda \colon X \to X $, and a morphism $ p \colon X \to Y $ which exhibits $ Y $ as a \emph{good quotient} of the involution on $ X $ in the sense of \cite[Remark 4.20]{azumaya_involution}. 
        \item a morphism from $ (X,\lambda, Y, p) $ to $ (Z,\nu, W, q) $ consists of a $ C_2 $-equivariant morphism $ X \to Z $ and a morphism $ Y \to W $ so that the diagram
        \begin{equation*}
        \begin{tikzcd}
            X \ar[d] \ar[r] & Z \ar[d] \\
            Y \ar[r] & W
        \end{tikzcd}
        \end{equation*}
        commutes. 
    \end{itemize}
\end{definition}
\begin{observation}\label{obs:fixpt_Mackey_functor_as_affine_C2_scheme}
    Suppose $ R $ is a discrete commutative ring with a $ C_2 $-action, and regard $ R $ as a $ C_2 $-Mackey functor via Example \ref{ex:fixpt_Mackey_functor}. 
    Then $ \Spec R \to \Spec (R^{C_2}) $ may be regarded as an object of $ \mathrm{qSch}^{C_2} $. 
\end{observation}
\begin{remark}\label{remark:restriction_of_schemes_with_involution}
    Suppose $ (X,\lambda, Y, p) $ is an object of $ \mathrm{qSch}^{C_2} $ and $ j \colon U \to Y $ is a flat map. %quasi-compact open subscheme %\Lucy{Do I need quasi-compact assumption?}
    Then $ (X_U, \lambda|_U, U, p|_U) $ is an object of $ \mathrm{qSch}^{C_2} $. 
    % In fact, if $ j $ is étale, $ (X_U, \lambda|_U, U, p|_U) $ is an object of $ \mathrm{qSch}^{C_2} $. 
    Affineness and invariance under the $ C_2 $-action are stable under pullback, so it suffices to show that $ j^*(\pi) $ satisfies $ \mathcal{O}_U \simeq \left(j^*(\pi)\right)_*(\mathcal{O}_{j^*U})^{C_2} $. 
    This follows from the proof of \cite[Theorem 4.35(i)]{azumaya_involution}.  
\end{remark}
\begin{proposition}
    Write $ U \colon \mathrm{qSch}^{C_2} \to \mathrm{qSch} $ for the functor so that $ U(X, \lambda, Y, p) = X $.  
    The category $ \mathrm{qSch}^{C_2} $ has a symmetric monoidal structure $ \boxtimes $ so that $ U $ is symmetric monoidal, where $ \mathrm{qSch} $ is endowed with the product symmetric monoidal structure. 
\end{proposition}
\begin{proof}
    If $ X $, $ Z $ are schemes with involutions $ \lambda_X $, $ \lambda_Z $, then $ \lambda_X \times \lambda_Z $ endows $ X \times Z $ with an involution. 
    It suffices to show that $ X \times Z $ admits a good quotient, as a good quotient is a categorical quotient and is therefore unique up to isomorphism. 
    By \cite[Remark 4.20]{azumaya_involution}, a good quotient exists if and only if every $ C_2 $-orbit is contained in an affine open subscheme. 
    Consider a $ C_2 $-orbit in $ X \times Z $. 
    Its image under the projection $ \pi_1 \colon X \times Z \to X $ (resp. $ \pi_2 \colon X \times Z \to Z $) is contained in an affine open subscheme $ U \subseteq X $ (resp. $ V \subseteq Z $). 
    Thus the orbit under consideration is contained in $ U \times Z $, which is affine. 
\end{proof}
\begin{construction}\label{cons:structure_sheaf_of_Green_functors}
    Assume that $ X $ has a \emph{good quotient} $ Y $ in the sense of \cite[Remark 4.20]{azumaya_involution}. 
    We write $ p \colon X \to Y $ for the quotient map.
    Let $ j \colon \Spec A  \simeq U \subseteq Y $ be an affine open subscheme of $ Y $. 
    Because $ p $ is an affine map, the fiber product $ \Spec B := \Spec A \times_{Y} X $ is an affine open of $ X $ which is invariant under the $ C_2 $-action. 
    In particular $ \Spec B $ inherits a $ C_2 $-action from $ X $ (hence so does its ring of functions $ B$).  
    Now $ A \to B $ acquires the structure of a $C_2$-Green functor $ \underline{\mathcal{O}}(U) $. 
    Regarding $ \underline{\mathcal{O}}(U) $ as a $ C_2 $-spectrum, by the isotropy separation sequence, we have an equivalence of $ A $-modules $ \underline{\mathcal{O}}(U)^{\varphi C_2} \simeq \mathrm{cofib} (\mathrm{tr} \colon B_{hC_2} \to A) $. 
\end{construction}
\begin{lemma} \label{lemma:identify_structure_sheaf_of_Green_func}
Let $ X $ be a scheme with an involution. 
Assume that $ X $ has a \emph{good quotient} $ Y $ in the sense of \cite[Remark 4.20]{azumaya_involution}, and write $ p \colon X \to Y $ for the quotient map. 
\begin{enumerate}[label=(\roman*)]
    \item \label{lem_item:structure_sheaf_of_Green_func_is_functor} The assignment of Construction \ref{cons:structure_sheaf_of_Green_functors} lifts to a contravariant functor from (the nerve of) the category of affine opens of $ Y $ to the $\infty $-category of Poincaré rings/$ C_2 $-$ \EE_\infty $-rings/Tambara functors. 
    \item \label{lem_item:structure_sheaf_of_Green_func_is_sheaf} The presheaf $ \underline{\mathcal{O}} $ of \ref{lem_item:structure_sheaf_of_Green_func_is_functor} defines a Zariski sheaf. 
    \item \label{lem_item:structured_pushforward_is_equiv} Write $ p_* \mathcal{O}_X $ for the sheaf of $ \EE_\infty $-$ \mathcal{O}_Y $-algebras (all functors are derived). 
    Then the pushforward $ p_* $ induces an equivalence $ \mathcal{D}(X) \xrightarrow{\sim} \Mod_{p_*\mathcal{O}_X} $. 
\end{enumerate}
\end{lemma}
\begin{proof}
    Part \ref{lem_item:structure_sheaf_of_Green_func_is_functor} follows from a similar argument to \cite[Theorem 5.1]{LYang_normedrings}; functoriality follows from noting that $ \tau_{\geq 0} $ is a functor.  
    Part \ref{lem_item:structure_sheaf_of_Green_func_is_sheaf} follows from Lemma \ref{lemma:limits_of_param_alg_detected_orbitwise}. 
    To prove part \ref{lem_item:structured_pushforward_is_equiv}, consider a Zariski cover $ \{j_i \colon U_i \to Y \} $ of $ Y $ by affine opens. 
    By Zariski descent, $ \displaystyle\mathcal{D}(X) \simeq \lim_{p^*(j_i) = p \times_Y j_i \colon U_i \times_Y X \to X} \Mod_{\mathcal{O}_X(U_i \times_Y X)} $ and $ \displaystyle\Mod_{p_*(\mathcal{O}_X)} \simeq \lim_{j_i \colon U_i \to Y} \Mod_{p_*\mathcal{O}_X(U_i)} $, hence the result follows. 
\end{proof}
\begin{lemma}\label{lemma:limits_of_param_alg_detected_orbitwise}
    Let $ K $ be a simplicial set, and let $ f \colon K^{\triangleleft} \to C_2 \EE_\infty\mathrm{Alg}(\Spectra^{C_2}) $ be a diagram. 
    Then $ f $ is a limit diagram if and only if $ f^e \colon K^{\triangleleft} \to \EE_\infty\mathrm{Alg}(\Spectra) $ and $ f^{C_2} \colon K^{\triangleleft} \to \EE_\infty\mathrm{Alg}(\Spectra) $ are both limit diagrams. 
\end{lemma}
\begin{proof}
    The result follows from the observation that limits in $ \EE_\infty \mathrm{Alg} \left(\Spectra^{C_2}\right) $ are computed in $ \Spectra^{C_2} $. 
\end{proof}
\begin{construction}\label{cons:C2_mod_over_sheaf_of_Green_func}
    Let $ p \colon X \to Y $ as before. 
    Consider the composites
    \begin{equation}\label{eq:functor_classifying_C2_mod_over_sheaf_of_Green_func}
    \begin{split}
        \Mod_{\underline{\mathcal{O}}} \colon & \mathrm{Op}(Y)^\op \xrightarrow{\underline{\mathcal{O}}} C_2\EE_\infty\mathrm{Alg}\left(\Spectra^{C_2}\right) \xrightarrow{\Mod_{(-)}} \Cat   \\
        \Mod_{\underline{\mathcal{O}}}^\otimes \colon & \mathrm{Op}(Y)^\op \xrightarrow{\underline{\mathcal{O}}} C_2\EE_\infty\mathrm{Alg}\left(\Spectra^{C_2}\right) \xrightarrow{\Mod_{(-)}^\otimes} C_2\otimes\Cat \,,  
    \end{split}    
    \end{equation}
    where $ C_2\otimes\Cat $ denotes the $ \infty $-category of (small)\Lucy{bleh...cardinals} $ C_2 $-symmetric monoidal $ C_2 $-$ \infty $-categories. 
    In the notation of Construction \ref{cons:structure_sheaf_of_Green_functors}, this functor sends the affine open $ \Spec A \subseteq Y $ to the category of modules in $ C_2 $-spectra over the $ C_2 $-$ \EE_\infty $-algebra which has underlying $ C_2 $-Mackey functor $ A \to B $. 
    Define\Lucy{invent better notation later} $ \Mod_{\underline{\mathcal{O}}} $, $ \Mod_{\underline{\mathcal{O}}}^\otimes $ to be the limits in $ \Cat $, $ C_2 \otimes \Cat $, resp. of the functors in (\ref{eq:functor_classifying_C2_mod_over_sheaf_of_Green_func}). 
    In particular, if we write $ s \colon \int \Mod_{\underline{\mathcal{O}}} \to \mathrm{Op}(Y)^\op $ for the cocartesian fibration obtained by taking the Grothendieck construction on (\ref{eq:functor_classifying_C2_mod_over_sheaf_of_Green_func}), an object of $ \Mod_{\underline{\mathcal{O}}} $ is a cocartesian section of $ s $. 
    In other words, it is a choice, for each affine open $ \Spec A $ of $ Y $ (same notation as before), of a module over the $ C_2 $-$ \EE_\infty $-algebra which has underlying $ C_2 $-Mackey functor $ A \to B $ which glue compatibly. 
\end{construction} 
Observe that for each $ A \to B $, there is a quadratic norm functor $ N^{C_2} \colon \Mod_{B}(\Spectra) \to \Mod_{N^{C_2}B}\left(\Spectra^{C_2}\right) $ and a quadratic relative norm functor $ N^{C_2} \colon \Mod_{B}(\Spectra) \to \Mod_{A\to B}\left(\Spectra^{C_2}\right) $. 
\begin{construction}\label{cons:globalize_relative_norm}
    Let $ X $ be a scheme with an involution, and let $ p \colon X \to Y $ exhibit $ Y $ as a good quotient of $ X $. 
    Assume that $ p $ is affine. 
    The norm functors (resp. relative norm functors) $ N^{C_2}_e $ assemble under Construction \ref{cons:structure_sheaf_of_Green_functors} to a `global' norm functor $ N^{C_2}_Y \colon \pi_{\#}\mathcal{O}_X \Mod \to N^{C_2} \pi_{\#} \mathcal{O}_X\Mod $ (resp. relative norm functor $ N^{C_2}_Y \colon \pi_{\#}\mathcal{O}_X \Mod \to \underline{\mathcal{O}}\Mod $). 
    Moreover, these functors are quadratic. 
\end{construction}
For each affine open $ j \colon \Spec A \subseteq Y $, write $ B = \Gamma (\mathcal{O}_{\Spec A \times_Y X}) $, consider the composite 
% Then the functor $ N^{C_2}_Y $ is the limit over all $ \Spec A \subseteq Y $ of the functors 
\begin{equation*}
     \pi_{\#}\mathcal{O}_X \Mod \xrightarrow{j^*} \Mod_B(\Spectra) \xrightarrow{N^{C_2}} \Mod_{N^{C_2}B}(\Spectra^{C_2}) \xrightarrow{ -\otimes_{N^{C_2}B} (A \to B)} \Mod_{A \to B}(\Spectra^{C_2})\,,
\end{equation*}  
where the last map is base change along the map $ N^{C_2}B \to (A\to B) $ which is a structure map for the $ C_2 $-$ \EE_\infty $-algebra structure on $ A \to B $. 
\Lucy{todo: use effective descent/limit definition for $ \underline{\mathcal{O}} $-modules.} 
Now since quadratic functors are closed under limits \cite[Theorem 6.1.1.10]{LurHA} and $ N^{C_2}_Y $ can be written as a limit of a diagram of quadratic functors, $ N^{C_2}_Y $ is also quadratic. 
\begin{definition}
    Varying $ X \to Y $, Constructions \ref{cons:C2_mod_over_sheaf_of_Green_func} and \ref{cons:globalize_relative_norm} define a functor \Lucy{want: codomain consists of $ C_2 $-stable $ C_2 $-presentable $ C_2 $-$ \infty $-categories}
    \begin{align*}
        \left(\mathrm{qSch}^{C_2}\right)^{\op} &\to C_2 \otimes \Cat  \\
        \left(X, \lambda, Y, p \right) &\mapsto \underline{\Mod}_{\underline{\mathcal{O}}}\left(\underline{\Spectra}^{C_2}\right)
    \end{align*}
\end{definition} 
\begin{definition}
    Suppose $ \mathcal{C} $ is a $ C_2 $-stable $ C_2 $-symmetric monoidal $ C_2 $-$ \infty $-category. 
    Define a functor
    \begin{align*}
       \mathrm{eInv} \colon  C_2 \otimes \Cat^{\mathrm{ex}} &\to \Spaces \\
        \left(\mathcal{C}, \otimes\right) & \mapsto \left(\mathrm{C}^{C_2}\right)^{\simeq} \times_{(\mathrm{C}^e)^{\simeq,hC_2}} \Pic (\mathrm{C}^e)^{\simeq,hC_2} \,.
    \end{align*}
    In other words, $ \mathrm{eInv} $ sends a $ C_2 $-symmetric monoidal $ C_2 $-$ \infty $-category to the full subgroupoid of $ \mathcal{C}^{C_2} $ on those objects $ L $ so that $ L^e $ is an invertible object in $ \mathcal{C}^e $. 

    Write $ \widetilde{\mathrm{eInv}} $ for the Grothendieck construction on $ \mathrm{eInv} $. 
\end{definition}
There is a functor
\begin{equation}\label{eq:C2_cat_to_Poincare_cat}
\begin{split}
    \widetilde{\mathrm{eInv}} &\to \EE_\infty \mathrm{Alg}\left(\mathrm{Cat}^h \right) \\
    (\mathcal{C}, L) &\mapsto (\mathcal{C}^e, \mathcal{C}^e \xrightarrow{N_{\mathcal{C}}} \mathcal{C}^{C_2} \xrightarrow{\hom_{\mathcal{C}^{C_2}}(-,L)} \Spectra)
\end{split}
\end{equation}
\Lucy{Pretty sure distributive norm functors are 2-excisive (results \href{https://arxiv.org/pdf/2010.09097}{here} should generalize readily)...if not, can define the problem away.}
\begin{lemma}
    The functor of (\ref{eq:C2_cat_to_Poincare_cat}) lifts to a functor $ \widetilde{\mathrm{eInv}}  \to \Catp $. 
\end{lemma}
\Lucy{Example: Special case where $ X $ has trivial $ C_2 $ action and $ X = Y $.}
\begin{lemma}\label{lemma:line_bundle_as_C2_sheaf}
    Let $ X $ be a scheme with involution $ \sigma \colon X \xrightarrow{\sim} X $ equipped with a good quotient $ \pi \colon X \to Y $. 
    Let $ L $ be a line bundle on $ Y $. 
    Then the canonical map 
    \begin{equation}\label{eq:line_bundle_as_C2_sheaf}
        L \to \pi_{\#} \pi^* L 
    \end{equation} 
    promotes (\ref{eq:line_bundle_as_C2_sheaf}) to a sheaf of $ \underline{\mathcal{O}} $-modules on $ Y $. 
    We will write $ \underline{L} $ for (\ref{eq:line_bundle_as_C2_sheaf}). 
\end{lemma}
\begin{proof}
    Follows from naturality of the unit and the canonical identification $ \pi^* \mathcal{O}_Y $ with $ \mathcal{O}_X $. 
\end{proof}
\begin{definition}\label{defn:quadratic_functor_from_good_quotient}
    Let $ X $ be a scheme with involution $ \sigma \colon X \xrightarrow{\sim} X $ equipped with a good quotient $ \pi \colon X \to Y $. 
    Let $ L $ be a line bundle on $ Y $. 
    Define $ \Qoppa_{\sigma, L} $ to be the functor
    \begin{equation*}
        \perf_X^\op \xrightarrow{\pi_{\#}} \pi_{\#}\mathcal{O}_X\Mod^{\omega,\op} \xrightarrow{N^{C_2}} N^{C_2}\pi_{\#}\mathcal{O}_X\Mod\left(\Spectra^{C_2}\right)^\op \xrightarrow{\hom_{N^{C_2}\pi_{\#}\mathcal{O}_X}(-,\underline{L})} \Spectra \,,
    \end{equation*}
    where $ \underline{L} $ is a $ \underline{\mathcal{O}} $-module by Lemma \ref{lemma:line_bundle_as_C2_sheaf} and $ \underline{\mathcal{O}} $ is a $ N^{C_2} \pi_{\#}\mathcal{O}_X $-algebra by Lemma \ref{lemma:identify_structure_sheaf_of_Green_func}. 
    By Construction \ref{cons:globalize_relative_norm} and the fact that the composite of an exact (1-excisive) functor and an $ m$-excisive functor is $m$-excisive (see \cite[\S2.2]{Kpoly}), $ \Qoppa_{\sigma,L} $ is quadratic. 
    % \Lucy{want to show that the assignment $ (\sigma:X \to X, \pi: X \to Y, L) \mapsto (\perf_X, \Qoppa_{\sigma,L}) $ defines a functor from some category of schemes with involution (+line bundle) to the $ \infty $-category of Poincaré $ \infty $-categories.}
\end{definition}
\begin{example} 
    Suppose $ L = \mathcal{O}_Y $. 
    Then we drop $ L $ from notation and the quadratic functor $ \Qoppa_{\sigma} $ of Definition \ref{defn:quadratic_functor_from_good_quotient} takes the form
    \begin{equation*}
        \perf_X^\op \xrightarrow{\pi_{\#}} \pi_{\#}\mathcal{O}_X\Mod^{\omega,\op} \xrightarrow{N^{C_2}} N^{C_2}\pi_{\#}\mathcal{O}_X\Mod\left(\Spectra^{C_2}\right)^\op \xrightarrow{\hom_{N^{C_2}\pi_{\#}\mathcal{O}_X}(-,\underline{\mathcal{O}})} \Spectra \,.
    \end{equation*}    
\end{example}
% \begin{lemma}\label{lemma:fully_faithful_pushforward}
%     Let $\pi \colon X \to Y $ be a finite étale map. 
%     Then the canonical functor $ \Mod_{\mathcal{O}_X} \to \Mod_{\pi_{\#}\mathcal{O}_X} $ is fully faithful. 
%     \Lucy{asked JH Nov 22nd: James says that this should definitely be true, and moreover in greater generality (for instance, if $ \pi$ is faithfully flat.); try, for instance, Barr--Beck.}
% \end{lemma}
% \begin{proof}
%     Affine-locally on $ Y $, I think this is an \emph{equivalence}. 
%     Conclude by descent. 
% \end{proof}
\begin{lemma}\label{lemma:bilinear_parts_agree}
    Let $ X $ be a scheme with involution $ \sigma \colon X \xrightarrow{\sim} X $, and let $ Y $ be a good quotient of $ X $. 
    % Assume that the quotient map $ q \colon X \to Y $ is étale. \Lucy{Weakest possible assumption? \href{https://mathoverflow.net/questions/169052/are-quotients-of-affine-schemes-by-finite-groups-faithfully-flat}{relevant MO post}.}
    Let $ L $ be a line bundle on $ Y $, and let $ \Qoppa_{\sigma,L} $ be the quadratic functor on $ \perf_X $ of Definition \ref{defn:quadratic_functor_from_good_quotient}. 
    Then the bilinear part of $ \Qoppa_{\sigma,L} $ agrees with that of Observation \ref{obs:cross_effect_twisted_poincare_structure}. 
    In particular, $ \left(\perf_X, \Qoppa_{\sigma,L} \right) $ is a Poincaré $ \infty $-category. 
\end{lemma}
\begin{proof}
    By definition of the bilinear part of a quadratic functor, it suffices to show that there is an equivalence $ \hom_{\pi_{\#}\mathcal{O}_X\Mod}\left(\pi_{\#}E \otimes_{\pi_{\#}\mathcal{O}_X} \pi_{\#}E, \pi_{\#}\mathcal{O}_X\right) \simeq \hom_{\mathcal{O}_X\Mod}\left(E \otimes_{\mathcal{O}_X} \sigma^*E, \mathcal{O}_X\right) $ for any perfect complex $ E $ on $ X $. 
    This follows from Lemma \ref{lemma:identify_structure_sheaf_of_Green_func}\ref{lem_item:structured_pushforward_is_equiv}. %\ref{lemma:fully_faithful_pushforward}.  
\end{proof}
\begin{remark}
    Compare the description of the space of bilinear forms in Lemma \ref{lemma:bilinear_parts_agree} with the description of a $ \delta $-hermitian form $ H $ in \cite[p. 216]{MR1162189}. 
\end{remark}

\section{The Poincaré Picard space}
\label{subsection:the_poincare_picard_group}

Recall that the Poincaré space functor $ \Pn \colon \Catp \to \CAlg(\Spaces) $ is lax symmetric monoidal with respect to tensor product of Poincaré $ \infty $-categories and smash product of $ \Einfty $-spaces \cite[Corollary 5.2.8]{CDHHLMNNSI}. In particular, we can consider invertible objects in $\Pn(A)$ for a Poincaré ring spectrum $A$.

\begin{definition}
    \label{definition:poincare_picard_space}
    Let $A$ be a Poincaré ring spectrum. We define the \emph{Picard space of $A$} to be $$\Picp(A):=\Pic(\Pn(A)).$$
    For $(X,\lambda, Y, p)\in \mathrm{qSch}^{C_2}$ we similarly define \[\Picp(X,\lambda,Y,p):= \Pic(\Pn(\perf_X,\Qoppa_{\sigma})).\]
\end{definition}

\begin{remark}
    \label{remark:poincare_picard_points_desc}
    Let $ \left(\Mod_R^\omega, \Qoppa_R \right)$ be a Poincaré ring spectrum, where $(M_R=R, N_R= R^{\varphi C_2}, R^{\varphi C_2}\to R^{tC_2})$ is the module with genuine involution associated to $ \Qoppa_R $. 
    Then a point in the Poincaré Picard space is the data of a pair $ (\mathcal{L}, q ) $, where $ \mathcal{L} $ is an invertible module in $ \Mod_R^\omega $ and $ q $ is a point in $ \Omega^\infty\Qoppa_R(\mathcal{L}) $. 
    By \cite[Proposition 1.3.11]{CDHHLMNNSI}, the data of $ q $ is equivalent to the data of points in the lower left and upper right corner of the square
    \begin{equation}
    \begin{tikzcd}
        \Qoppa(\mathcal{L}) \ar[r] \ar[d] & \hom_R(\mathcal{L}, R^{\varphi C_2}) \ni \ell(q) \ar[d] \\
        b(q) \in \hom_{R \otimes R}\left(\mathcal{L} \otimes \mathcal{L}, R\right)^{hC_2} \ar[r] & \hom_R(\mathcal{L}, R^{tC_2})
    \end{tikzcd}
    \end{equation} 
    and a path between their images in the lower right corner. 
    In particular, the adjoint of $ b(q) $ must define a nondegenerate hermitian form on $ \mathcal{L} $, that is, an equivalence $ \mathcal{L} \simeq \hom_{R}(\mathcal{L}, R^*) $ where $ R^* $ is considered as an $ R $-module via the action of the generator of $ C_2 $. \Lucy{add equivariance/symmetry data}

    Write $ (\mathcal{L}^\vee,q^\vee) $ is for the inverse of $ (\mathcal{L},q) $. 
    By definition of invertibility, there exists an $ R $-linear map $ \ell(q^\vee) \colon \mathcal{L}^\vee \to R^{\varphi C_2} $ so that the following diagram commutes
    \begin{equation}\label{diagram:pnpic_linear_part_condition}
    \begin{tikzcd}[column sep=huge]
        \mathcal{L} \otimes_R \mathcal{L}^\vee \ar[d,"\mathrm{ev}", "\sim"']  \ar[r,"{\ell(q) \otimes \ell(q^\vee)}"] & R^{\varphi C_2} \otimes_R R^{\varphi C_2} \ar[d,"\mathrm{multiplication}"] \\
        R \ar[r,"\mathrm{given}"] & N_R   
    \end{tikzcd}
    \end{equation} 
\end{remark}

\begin{lemma}
    \label{lemma:connectivity_of_qoppa_at_the_unit}
    Let $(R,\Qoppa)$ be a connective Poincaré ring spectrum\Viktor{define connectivity and make conditions here precise. As stated this works for R and C connective. More precisely, $conn(\Qoppa(\Sigma^n R))\leq \min(conn(\Sigma^{-2n}R),conn(\Sigma^{-n} C))$}. Then, for any integer $n$, the spectrum $\Qoppa(\Sigma^n R)$ is $(-2n)$-connective.
\end{lemma}

\begin{proof}
    This follows from the fiber sequence $$(\Sigma^{-2n} R)_{hC_2}\rightarrow \Qoppa(\Sigma^{n} R)\rightarrow \hom_R(\Sigma^n R,C)\simeq \Sigma^{-n} C.$$\Viktor{write out details}
\end{proof}

\begin{remark}
    \label{remark:poincare_picard_is_2-torsion}
    The functor $\Picp:\CAlgp\rightarrow \CAlg^{\gp} (\mathcal{S})$ preserves certain structures. Let $A$ be a Poincaré ring spectrum. Since $A$ is a module over $(\mathbb{S},\Qoppa_u)$, the space $\Picp(A)$ is something over $\Picp(\mathbb{S},\Qoppa_u)$.\Viktor{there is no truth in here yet. Work in progress. Noah had an example using Witt vectors which showed that $\pi_0$ does not need to be 2-torsion}
\end{remark}

Since the forgetful functor $\Pn(A)\to \mathrm{Mod}_A^\omega$ is symmetric monoidal we get an induced map \[ U:\Picp(A)\to \Pic(A)\] of spectra. For a point $(\mathcal{L},q)\in \pi_0(\Picp(A))$ we will refer to $\mathcal{L}:=U(\mathcal{L},q)$ as the \textit{underlying invertible module}.  Note that the $A$-module $A^*$ is (nonequivariantly) isomorphic to $A$ via the involution, and so the fact that $\mathcal{L}\simeq \mathrm{hom}_A(\mathcal{L},A^*)$ forces $\mathcal{L}$ to be $2$-torsion. In particular we get a refined map \[U:\Picp(A)\to \Pic(A)[2]\] which factors the underlying invertible module map. 

\begin{example}
    \label{example:picard_of_the_unit}
    Let $(\mathbb{S},\Qoppa_u)$ be the universal Poincaré ring spectrum from Example \ref{example:universal_poincare_ring_spectrum}. The only $2$-torsion element of $\Pic(\mathbb{S})\simeq \mathbf{Z}$ is $\mathbb{S}$. Therefore, any element in $\Picp(\mathbb{S},\Qoppa_u)$ lies above $\mathbb{S}$ under $U$. With Remark \ref{example:universal_tate_poincare_splits_at_unit}, we conclude $\pi_0(\Picp(\mathbb{S},\Qoppa_u))\simeq \pi_0(\mathbb{S}_{hC_2}\oplus \mathbb{S}^\times)^\times \simeq(\mathbf{Z}\times\mathbf{Z/2})^\times\simeq \mathbf{Z}/2\times \mathbf{Z}/2$. \todo{V:did we mod out by isomorphisms here?}
\end{example}

\begin{remark}
One might hope that the map $\Picp(A)\to \Pic(A)[2]$ is close to an equivalence. This however is quite far from being true. Let $k$ be a finite field of characteristic $2$, and let $\mathbb{S}_{W(k)}$ be the spherical Witt vectors on $k$ in the sense of \cite[Example 5.2.7]{lurie-elliptic-2}. Then by \cite[Example 3.4]{Nikolaus-Frob} we know that $\mathbb{S}_{W(k)}$ must satisfy that the map $\phi_2:\mathbb{S}_{W(k)}\to \mathbb{S}_{W(k)}^{tC_2}$ is an equivalence where the action is trivial.

Consider now the Poincar{\'e} ring $(\mathrm{Mod}_{\mathbb{S}_{W(k)}}^\omega, \Qoppa_{\mathbb{S}_{W(k)}}^u)$ where $\Qoppa^u_{\mathbb{S}_{W(k)}}$ is the Tate Poincar{\'e} structure. We have that $\pi_0(\Pic(\mathbb{S}_{W(k)}))\cong \mathbb{Z}$ and is generated by $\Sigma \mathbb{S}_{W(k)}$. To see this note that for $\mathcal{L}$ an invertible module over $\mathbb{S}_{W(k)}$, $\mathcal{L}$ must be bounded below since otherwise it would not be perfect. Then for $\pi_n(\mathcal{L})$ its bottom homotopy group, we have that $\pi_n(\mathcal{L}/2)\cong k$ since it must be an invertible $k$-module and $k$ is a field. Thus we get a map $\mathbb{S}^{n}\to \mathcal{L}$ lifting a generator of $k$, and by adjunction an $\mathbb{S}_{W(k)}$-module map $\Sigma^{n}\mathbb{S}_{W(k)}\to \mathcal{L}$ which on $\pi_n((-)/2)$ gives an isomorphism $k\cong k$. Therefore \[\mathbb{S}_{W(k)}[n]\otimes k \simeq k[n] \to k[n] \simeq \mathcal{L}\otimes k\] is an equivalence, where the equivalence $k[n]\simeq \mathcal{L}\otimes k$ follows from the fact that base change preserves invertible objects. The map $\mathbb{S}_{W(k)}[n]\to\mathcal{L}$ is then a $k$-local, and therefore an $\mathbb{F}_p$-local, equivalence. Both sides are connective and $p$-complete so it follows that the map $\mathbb{S}_{W(k)}[n]\to \mathcal{L}$ is an equivalence.\Noah{There is probably a reference for this fact, I'll look around for one.}

Thus $\pi_0(\Pic(\mathbb{S}_{W(k)}))=0$. On the other hand, we have that the unit map $\mathbb{S}_{W(k)}\to \Qoppa^u_{\mathbb{S}_{W(k)}}(\mathbb{S}_{W(k)})$ is split by the map $\Qoppa^u_{\mathbb{S}_{W(k)}}(\mathbb{S}_{W(k)})\to \mathbb{S}_{W(k)}^{\phi C_2}=\mathbb{S}_{W(k)}$. Consequently $\pi_0(\Qoppa^u_{\mathbb{S}_{W(k)}}(\mathbb{S}_{W(k)}))\cong \pi_0(\mathbb{S}_{W(k)}\oplus (\mathbb{S}_{W(k)})_{hC_2})\cong W(k)\times W(k)$. As a ring this is $W_2(W(k))$, and in order for $q\in W_2(W(k))$ to induce a Poincar{\'e} structure we must have that $q\in W_2(W(k))^\times \cong W(k)^\times \times W(k)^\times$. 

We then have that $\pi_0(\Picp(\mathbb{S}_{W(k)}))\cong W(k)^\times \times W(k)^\times/H$ where $H$ is the subgroup of Poincar{\'e} structures $q$ on $\mathbb{S}_{W(k)}$ which are identified by some automorphism $f:\mathbb{S}_{W(k)}\to \mathbb{S}_{W(k)}$. By the defining property of spherical Witt vectors there is an equivalence $\mathrm{Maps}_{\mathrm{CAlg}}(\mathbb{S}_{W(k)}, \mathbb{S}_{W(k)})\simeq \mathrm{Maps}_{Perf}(k,k)=\mathrm{Gal}(k/\mathbb{F}_2)$ and the action on $W(k)^\times \times W(k)^\times$ is given by $g\in \mathrm{Gal}(k/\mathbb{F}_2)$ acts via $W(g)\times W(g)$. Consequently \[\pi_0(\Picp(\mathbb{S}_{W(k)}))\cong (W(k)^\times \times W(k)^\times) /\mathrm{Gal}(k/\mathbb{F}_2)\] which even for $k=\mathbb{F}_2$ is not zero and in fact not even $2^\infty$-torsion. 
\end{remark}

In the usual Picard spectrum one has the relationship $\pic = B\mathbb{G}_m$, where $\mathbb{G}_m$ is the spectral algebraic group scheme sending a ring spectrum $E$ to the spectrum of $E$-linear equivalences of $E$ $\mathrm{gl}_1E:=\mathrm{Aut}_E(E)$.\footnote{Normally the automorphism space of an object is only $\mathbb{A}_\infty$, but as the unit in a symmetric monoidal category, the automorphisms of $E$inherit a canonical and in fact functorial $\mathbb{E}_\infty$ structure and this construction makes sense.} 
Equivalently, $\mathbb{G}_m$ is the affine group scheme given by $\mathbb{G}_m=\mathrm{Sp}\textrm{\'et}(\mathbb{S}\{x^{\pm 1}\})$, where $\mathbb{S}\{x^{\pm 1}\}$ is the free $\mathbb{E}_\infty$ ring on the $\mathbb{E}_\infty$ space $\mathbb{Z}$. This relationship between $\pic$ and $\mathbb{G}_m$ has many important applications, for example relating the higher homotopy groups of $\pic(A)$ with those of $A$. 
We will spend the rest of this section on establishing such an equivalence in the Poincar\'e setting.

\begin{construction}~\label{const: gmq}
	The underlying $\mathbb{E}_\infty$ ring of $\gmq$ will again be $\mathbb{S}\{x^{\pm 1}\}$, but in order to promote this ring to a Poincare ring it will be helpful to write it as \[\mathbb{S}\{x^{\pm 1}, y^{\pm 1}\}\otimes_{\mathbb{S}\{z^{\pm 1}\}}\mathbb{S}\] where the map $\mathbb{S}\{z^{\pm 1}\}\to \mathbb{S}\{x^{\pm 1}, y^{\pm 1}\}$ is induced by $z\mapsto xy$, This ring naturally lifts to a Borel $C_2$-ring given by $C_2$ swaps $x$ and $y$ and does nothing to $z$. Now take $\gmq$ to be the Poincar{\'e} ring with underlying Borel $C_2$ structure as described above and geometric fixed points $(\gmq)^{\phi C_2}=\mathbb{S}$ and the map $(\gmq)^{\phi C_2}\to (\gmq)^{tC_2}$ given by the unit map. Endowing $(\gmq)^{\phi C_2}$ with the $\gmq$-module structre given by $x,y\mapsto 1$, it remains to show that the unit map $(\gmq)^{\phi C_2}\to (\gmq)^{tC_2}$ factors the Tate valued Frobenius $\gmq\to (\gmq)^{tC_2}$ in order to promote $\gmq$ to a Poincar{\'e} ring.
	
	By construction of $\gmq$ this amounts to showing that on $\pi_0$ the Tate valued Frobenius sends $x,y\mapsto 1$ in $\pi_0((\gmq)^{tC_2})$. This map sends both $x$ and $y$ to $xy\in \pi_0((\gmq)^{tC_2})$. These are equal to $1$ in $\pi_0((\gmq)^{tC_2})$ since the functor $(-)^{tC_2}$ is lax-monoidal so $(\gmq)^{tC_2}$ is a modules over $\mathbb{S}\{x^{\pm 1}, y^{\pm 1}\}^{tC_2}\otimes_{\mathbb{S}\{z\}^{tC_2}}\mathbb{S}^{tC_2}$ which has the image of $xy$ equal to $1$.
\end{construction}

\begin{theorem}\label{theorem:loops_Poincare_pic_is_Gm_Qoppa}
	There is a natural equivalence of \[\Omega \Picp(-)\simeq \gmq\] of functors on Poincar{\'e} rings.
\end{theorem}
\begin{proof}
	This amounts to identifying the space $\mathrm{Aut}_{\mathrm{Pn(\mathrm{Mod}_A)}}(A,u)$ functorially, where $(A,u)$ is the Poincar{\'e} object $A$ with bilinear form given by the unit map $\mathbb{S}\to \Qoppa_A(A)$. Note that any automorphism of Hermetian objects will automatically be Poincar{\'e} and so we may instead describe the automorphisms as a Hermetian object. We then have that $\mathrm{He}(\mathrm{Mod}_A)\to \mathrm{Mod}_A$ is a cocartesian fibration by definition, and classified by the functor which takes a module $M$ to the groupoid $\Omega^\infty \Qoppa_A(M)$. Thus we get that $\mathrm{Aut}_{\mathrm{He}(\mathrm{Mod}_A)}((A,u))$ is exactly the fiber of the map \[\mathrm{Aut}_{\mathrm{Mod}_A}(A)\to \Qoppa_A(A)\] or in other words an automorphism $(A,u)\to (A,u)$ is the data of an automorphism $a\in \mathrm{Aut}(A)$ together with a path $q:u\mapsto a^*u$ in $\Omega^{\infty +1}\Qoppa_A(A)$.  
	
	There is a natural transformation $\gmq(-)\to \Omega\Picp(-)$ given as follows: we get a map $\gmq((\mathrm{Mod}_A, \Qoppa_A))\to \mathrm{Aut}_A(A)$ given by forgetting the Poincar{\'e} structure everywhere, and so it is enough to see that on $\pi_0$ the automorphisms of $A$ coming from $\gmq$ preserve $u$. By using the linear and quadratic decomposition of $\Qoppa_A$, for an element $a\in \pi_0(A)^\times$ send $u$ to $u$ is must be sent to $1\in \pi_0(A^{\phi C_2})$ and must act by $1$ on $A^{hC_2}$. By the following Lemma this second condition is equivalent to $a\sigma(a)\in \pi_0(A)^\times$ being equal to $1$, but then these two conditions are exactly describing a map out of $\gmq$ as desired. 
	
	Consequently we have a comparison map $\gmq(\mathrm{Mod}_A, \Qoppa_A)\to \Omega\Picp(\mathrm{Mod}_A,\Qoppa_A)$, and the above argument in fact shows that this map is an equivalence on $\pi_0$. To finish the argument, note that the pushout description of $\gmq$ induces a pullback of mapping spaces 
	\[
	\begin{tikzcd}
		\gmq(\mathrm{Mod}_A, \Qoppa_A) \ar[d] \ar[r] & \mathrm{Maps}_{\CAlg(\mathrm{Sp}^{C_2})}(\mathbb{S}\{x^{\pm 1}, y^{\pm 1}\}, A)\simeq \mathrm{gl}_1(A)\ar[d]\\
		* \ar[r] & \mathrm{Maps}_{\CAlg(\mathrm{Sp^{C_2}})}(\mathbb{S}\{z\},A)\simeq \Omega^\infty\Qoppa_A(A) 
	\end{tikzcd}
	\] which finishes the proof.
\end{proof}

\begin{lemma}
	Let $A\in \mathrm{CAlg}(\mathrm{Sp}^{BC_2})$ and $s\in \pi_0(A)^\times$. Then $a\sigma(a)=1$ in $\pi_0(A)$ if and only if $(a\otimes a)^*$ acts by $1$ on $\pi_0(A^{hC_2})=\pi_0(\mathrm{Hom}_{A\otimes A}(A\otimes A, A)^{hC_2})$.
\end{lemma}
\begin{proof}
	The only if direction follows from the fact that the evaluation map $\mathrm{Hom}_{A\otimes A}(A\otimes A, A)\to A$ is an $A\otimes A$-module map. Now suppose that $a\sigma(a)=1$ in  $A$. Then before taking homotopy fixed points the induced map $a^*=id$ because $A$ is $\mathbb{E}_\infty$.\footnote{Or just $\mathbb{E}_2$.} 
\end{proof}

\subsection{prime factorization and the picard group of hearts}\Viktor{still in development}
We establish a Poincaré analogue of Fausk's result which describes the picard group of the derived category of a scheme $X$ in terms of connected components of $X$ and the classical picard group of $X$.

Let $X$ be a simplicial set. The set $\pi_0(X)$ is the set of connected compoents of $X$, i.e. simplicial subsets which are connected and form $X$ via a coproduct. In other words, the functor $\pi_0$ records a unique and maximal decomposition of $X$ into coproducts. To establish the the result mentioned above, we study the dual analogue of connected components in the sense of a maximal decomposition of $X$ into products. %Later on, we will generalize this to arbitrary symmetric monoidal structures in a given $\infty$-category.

\begin{definition}\label{definition:factor}
    Let $X$ be a simplicial set. We call a map of simplicial sets $X\rightarrow Y$ a factor of $X$, if there is an isomorphism $X\simeq Y_\times Z$, for some simplicial set $Z$, such that $Y \times Z\simeq X\rightarrow Y$ is a structure map of the given product.
\end{definition}

\begin{definition}\label{definition:prime_indecomposable}
    Let $X$ be a simplicial set. We say that $X$ is prime, or indecomposable, if it is nonempty and every factor of $X$ is isomorphic to  either $\Delta^0$ or $X$. We let $\operatorname{prin}(X)$ denote the set of prime factors of $X$.
\end{definition}

\begin{proposition}\label{proposition:prime_factorization}
    Let $X_\bullet$ be a simplicial set, then $X_\bullet$ is the product of its prime factors.
\end{proposition}
\begin{proof}
    \Viktor{}
\end{proof}

\begin{proposition}
    Let $X$ be a simplicial set. Then $\operatorname{prim}(X)\simeq \operatorname{prim}(X^\simeq)$.
\end{proposition}
\begin{proof}
    Let $f:X\rightarrow Y$ be a weak equivalence of simplicial sets. Then $f^\simeq: X^\simeq\rightarrow Y^\simeq$ is a weak equivalence of spaces. \Viktor{}
\end{proof}

\begin{proposition}
    Let $X$ be a simplicial set. Then $\operatorname{prim}(X)\simeq \pi_0(\operatorname{Spec}(X)),$ where $\operatorname{Spec}(X^\simeq)$ is the Balmer spectrum of $X$ with respect to the symmetric monoidal structure given by cartesian product.
\end{proposition}
\begin{proof}
    \Viktor{}
\end{proof}

\begin{remark}
    Let $R$ be a commutative ring. Then the scheme $\operatorname{Spec}(R)$ is isomorphic to the Balmer spectrum of $\operatorname{Mod}^\heartsuit_R$. When we view $R$ as a discrete simplicial set, we thus have $$\operatorname{prim}(\operatorname{Mod}_R^\heartsuit)\simeq \pi_0(\operatorname{Spec}(R)).$$ \Viktor{does this need a proof?}
\end{remark}

\begin{definition}
    Let $X$ be a prime simplicial set. A $c$-structure on $X$ is a map of simplicial sets $X\rightarrow \mathbf{Z}$ satisfying (todo). Let $Y$ be a simplicial set, then a $c$-structure on $Y$ is a product of $c$-structures on each of its prime components. We write $X_{\geq n}$ for the homotopy pullback of $\mathbf{Z}_{\geq n}$ along $c$, $X_{\leq n}$ for the homotopy pullback of $\mathbf{Z}_{\leq n}$ along $c$, and  $X^\heartsuit$ for the pullback of $X_{\leq n}$ along $X_{\geq n}\rightarrow X$. \Viktor{when X is a stable infinity category and prime, then a c-structure should be a t-structure on it}
\end{definition}

\begin{theorem}
    Let $X$ be a prime simplicial set and $c: X\rightarrow \mathbf{Z}$ a c-structure.  Then we have a fiber sequence of monoids \Viktor{what kind exactly}
>$$X^\heartsuit\rightarrow X^\simeq \rightarrow \mathbf{Z}.$$
\end{theorem}
\begin{proof}
    \Viktor{}
\end{proof}

\begin{corollary}[Fausk]
    Let $R$ be a discrete ring. Then we have a short exact sequence:
>$$0\rightarrow \operatorname{Pic}(\operatorname{Mod}(R)^\heartsuit) \rightarrow\pi_0(\operatorname{Pic}(\operatorname{Mod}(R)))\rightarrow H^0(\operatorname{Spec}(R);\mathbf{Z})\rightarrow 0.$$
\end{corollary}
\begin{proof}
    \Viktor{apply pic to the previous sequence and take $\pi_0$}
\end{proof}

\subsection{Hermitian line bundles}
\begin{definition}
    Let $ R $ be a commutative discrete ring with a $ C_2 $-action $ \sigma \colon R \to R $. 
    Write $ \sigma_*R $ for the $ R $-module with underlying abelian group $ R $ and action $ r \cdot m = \sigma(r) \cdot m $. 
    Let $ M $ be an $ R $-module. 
    Define the \emph{adjoint} of $ M $ to be the $ R $-module $ M^\dag := \hom_R \left(M, \sigma_* R\right)$. \Lucy{see 3.8-3.11 in \href{https://arxiv.org/pdf/2009.09124}{this paper}}
    \Lucyil{From meeting May 29, 2025: Missing a $ \sigma_* $ before $ \hom $? clarify the $ R $-action!} 
    Also recall that there is a canonical $ R $-linear isomorphism $ \left(M^\dag\right)^\dag \simeq M $. 
    Note that given two $ R $-modules $ M, N $, the adjoint satisfies $ M^\dag \otimes N^\dag \simeq (M\otimes N)^\dag $. 
    Let $ I $ be a projective $ R $-module (in particular, there is a canonical identification $ (I^\dag)^\dag \simeq I $). 
    A \emph{$ \sigma $-hermitian form on $ I $} is an $ R $-linear isomorphism $ \varphi \colon I \xrightarrow{\sim} I^\dag $ so that $ \varphi^\dag = \varphi $. 
\end{definition}
\begin{observation}\label{obs:tensoring_hermitian_forms}    
    Let $ R $ be a commutative discrete ring with a $ C_2 $-action $ \sigma \colon R \to R $. 
    Given two discrete $ R $-modules $ M, N $ equipped with $ \sigma $-hermitian forms $ \varphi, \psi $, respectively, $ \varphi \otimes \psi $ defines a $ \sigma $-hermitian form on $ M \otimes_R N $. 
    Using the canonical isomorphism mentioned above, if $ \varphi $ is a $ \sigma $-hermitian form on $ M $, then $ \varphi^\dag $ induces a $ \sigma $-hermitian form on $ M^\dag $.  
    Finally, observe that $ R $ has a canonical $ \sigma $-hermitian form which is the adjoint of the map $ R \otimes R \to R $, $ r \otimes s \mapsto r \sigma(s) $. 
\end{observation}  
\begin{definition}
    Let $ R $ be a commutative discrete ring with a $ C_2 $-action $ \sigma \colon R \to R $. 
    Define the \emph{hermitian Picard group of $ R $}\Lucy{workshop the name later} to have underlying set consisting of pairs $ (I, \varphi) $ where $ I $ is an invertible $ R $-module and $ \varphi $ is a $ \sigma $-hermitian form on $ I $. 
    
    By Observation \ref{obs:tensoring_hermitian_forms}, this set inherits a group structure. 
    We write $ \mathrm{hPic}(R) $ for the group of $\sigma$-hermitian line bundles on $ \Spec R $.  
\end{definition}     
\begin{theorem}\label{theorem:Poincare_Pic_of_fixpt_Mackey_functor}
    Let $ R $ be a discrete commutative ring with a $ C_2 $-action $ \sigma \colon R \xrightarrow{\sim} R $ via ring maps. 
    Regard $ R $ as a Poincaré ring via Example \ref{ex:fixpt_Mackey_functor}. 
    Then there is a split short exact sequence of abelian groups
    \begin{equation*}
        0 \to \mathrm{hPic}(R) \to \pi_0\pnpic(\underline{R}^\sigma) \to C_{C_2}(\Spec R, \ZZ^{-}) \to 0
    \end{equation*}
    where $ R $ is endowed with the genuine symmetric Poincaré structure and $ \ZZ^{-} $ is endowed with the $ C_2 $-action given by multiplication by $ -1 $ and $ C_{C_2} $ denotes continuous functions which are moreover $ C_2 $-equivariant. 
    Moreover, forgetting the hermitian form (resp. forgetting the $ C_2 $-action) induces a commutative diagram
    \begin{equation*}
    \begin{tikzcd}
        0 \ar[r] & \mathrm{hPic}(R) \ar[r]\ar[d] & \pi_0\pnpic(R) \ar[r] \ar[d] & C_{C_2}(\Spec R, \ZZ^{-}) \ar[r] \ar[d] & 0 \\
        0 \ar[r] & \mathrm{Pic}^{\mathrm{cl}}(R) \ar[r] & \pi_0 \mathrm{Pic}\left(\perf_R\right) \ar[r] & C(\Spec R, \ZZ) \ar[r] & 0 
    \end{tikzcd}    
    \end{equation*}
    where the bottom row is that of \cite[Theorem 3.5]{MR1966659}. 
\end{theorem}
\begin{proof} 
    An object of $ \pi_0 \pnpic(R) $ is a pair $ (I, q) $ where $ I $ is an invertible $ R$-module and $ q $ is a point in $ \pi_0 \Omega^\infty \Qoppa_{R^{gs}}(I) $. 
    By the proof of \cite[Theorem 3.5]{MR1966659}, $ I $ induces a continuous map $ \Psi(I) \colon \Spec R \to \ZZ $. 
    Write $ \sigma $ for the involution on $ R $. 
    Now $ q $ in particular induces an equivalence $ q \colon I \xrightarrow{\sim} I^\dag \simeq (\sigma_*I)^\vee $. 
    For each point $ \mathfrak{p} \in \Spec R $, localizing $ q $ gives an equivalence
    \begin{equation*}
        q_{\mathfrak{p}} \colon I_{\mathfrak{p}} \xrightarrow{\sim} (\sigma_*I)^\vee_{\mathfrak{p}} \simeq \left(\sigma_*(I_{\sigma(\mathfrak{p})})\right)^\vee \,.
    \end{equation*}
    Since $ I_{\mathfrak{p}} $ is an invertible module over a local ring, \cite[Proposition 3.2]{MR1966659} implies that $ q_{\mathfrak{p}} $ induces an equivalence
    \begin{equation*}
        I_{\mathfrak{p}} \simeq R_{\mathfrak{p}}[\phi(\mathfrak{p})] \xrightarrow{\sim} \left(\sigma_*(R_{\sigma(\mathfrak{p})}[\phi(\sigma(\mathfrak{p}))])\right)^\vee \simeq (\sigma_*(R_{\sigma(\mathfrak{p})}))^\vee [-\phi(\sigma(\mathfrak{p}))] \,.
    \end{equation*}
    Since $ R $ is discrete, this implies in particular that $ \Psi(I)(\sigma(\mathfrak{p})) = -\Psi(I)(\mathfrak{p}) $, i.e. that $ \Psi(I) $ is $ C_2 $-equivariant. 
    It follows immediately from \cite[Theorem 3.5]{MR1966659} that $ \Psi $ is a homomorphism and that an element of the kernel of $ \Psi $ lifts to $ \mathrm{hPic}(R) $. 

    Now consider a $ C_2 $-equivariant map $ g \colon \Spec R \to \ZZ $. 
    As in \emph{loc. cit.}, the image of $ g $ is finite and $ C_2 $-invariant, say $ \{n_1, -n_1, \ldots, n_m, -n_m\} $ or $ \{0, n_1, -n_1, \ldots, n_m, -n_m\} $ for some $ n_i \neq 0 $. 
    As in \emph{loc. cit.}, the disjoint subsets $ U_{\pm n_i} := g^{-1}(\pm n_i) $ correspond to an orthogonal basis of idempotents $ e_{U_{\pm n_i}} $ in $ R $. 
    Since $ g $ is $ C_2 $-equivariant with respect to the sign action on $ \ZZ $, we have $ \sigma(U_{n_i}) = U_{-n_i} $. 
    Moreover, it follows from Lemma 3.4 \emph{ibid.} that $ \sigma(e_{U_{n_i}}) = e_{U_{-n_i}} $. 
    Consider the $ R $-module $ \Phi(g):= \bigoplus_{n \in \mathrm{Im}(g)} e_{g^{-1}(\{n\})} R[n] $. 
    In other words, $ \Phi(g) := \bigoplus_{i=1}^m \left(e_{U_{n_i}}R[n_i] \oplus e_{U_{-n_i}}R[-n_i] \right) $ if $ 0 $ is not in the image of $ g $ and $ \Phi(g) := e_{U_0} \oplus \bigoplus_{i=1}^m \left(e_{U_{n_i}}R[n_i] \oplus e_{U_{-n_i}}R[-n_i] \right) $ otherwise. 
    Observe that $ \left(e_{U_{-n_i}}R[-n_i]\right)^\dag = \hom_R(e_{U_{-n_i}}R[-n_i], \sigma_* R)= \hom_R\left(\sigma_*(e_{U_{-n_i}}R), R\right)[n_i] = \hom_R\left(e_{U_{n_i}}R, R\right)[n_i] $. 
    Finally, we claim that there is a canonical $ \sigma $-hermitian form $ q_g \in \Omega^\infty \Qoppa_{R^{\mathrm{gs}}}(\Phi(g)) $ whose adjoint $ q_g^\dag \colon \Phi(g) \xrightarrow{\sim} \Phi(g)^\dag $ corresponds to the identity. 
    That $ q_g $ defines a point of $ \hom_{R^{\otimes 2}}(\Phi(g)^{\otimes 2}, R)^{hC_2} $ is evident. 
    Observe that to give a lift of $ q_g $ to $ \Qoppa_{R^{\mathrm{gs}}}(\Phi(g)) = \hom_{N^{C_2}R}\left(N^{C_2}\Phi(g), R\right) $ is equivalent to giving a commutative diagram
    \begin{equation}\label{diagram:lifting_sym_form_to_gen_sym_form}
    \begin{tikzcd}
        \Phi(g) \otimes_R R^{\varphi C_2} \ar[d] \ar[r,dashed,"{\exists ?}"] & R^{\varphi C_2} \ar[d] \\
        \left(\Phi(g)^{\otimes 2}\right)^{tC_2} \ar[r,"{q_g^{tC_2}}"] & R^{tC_2}
    \end{tikzcd} 
    \end{equation} 
    of $ R^{\varphi C_2} $-modules. 
    Let us write $ \eta \colon R \to \pi_0 R^{\varphi C_2} $ for the ring map induced by the structure map. 
    Since $ R $ is a $ C_2 $-$ \EE_\infty $-ring, $ \eta $ is invariant with respect to the given action on $ R $ and the trivial action on $ \pi_0 R^{\varphi C_2} $. 
    Consider $ e_{U_{n_i}} $ an idempotent corresponding to an element of the image of $ g $ so that $ n_i \neq 0 $. Then
    \begin{equation*}
    \begin{split}
        \eta(e_{U_{n_i}}) &= \eta(e_{U_{n_i}})^2 \qquad \text{ ring maps preserve idempotents } \\
        &= \eta(e_{U_{n_i}}) \cdot \eta(e_{U_{-n_i}}) \qquad \text{ $C_2$-invariance of }\eta \\
        &= \eta(e_{U_{n_i}}e_{U_{-n_i}}) \qquad \text{ $ \eta $ is a ring map } \\
        &= 0 \qquad \text{ orthogonality and }n_i \neq 0 \,.  
    \end{split}
    \end{equation*}
    In particular, if $ 0 $ is not in the image of $ g $, $ \Phi(g) \otimes_R R^{\varphi C_2} \simeq 0 $ and (\ref{diagram:lifting_sym_form_to_gen_sym_form}) commutes vacuously. 
    If $ 0 $ is in the image of $ g $, then $ e_{U_0}R $ is a discrete/projective $ e_{U_0}R $-module and $ q_g $ evidently defines a genuine hermitian form on $ e_{U_0}R $ (compare \cite[Remark 4.2.21]{CDHHLMNNSI}). 

    Thus, $ g \mapsto (\Phi(g), q_g) $ defines a splitting of $ \Psi $ which agrees with the splitting constructed in \cite[Theorem 3.5]{MR1966659} on underlying objects. 
\end{proof}

%\section{Poincaré structures on schemes with involution}
Let $ X $ be a scheme with an involution $ \sigma \colon X \xrightarrow{\sim} X $. 
We want to introduce a Poincaré structure $ \Qoppa $ on $ \perf(X) $ so that the duality is given by $ E \mapsto E^\vee \otimes \sigma_*(\mathcal{O}_X) $ (contrast with \S3 of \href{https://arxiv.org/abs/2402.15136}{this paper}). 
\Noah{I will comment out the Poincar{\'e} structures from rigid symmetric monoidal categories for the moment since we do not yet have an application for these ideas.}
\subsection{Poincaré structures associated to rigid symmetric monoidal \texorpdfstring{$ \infty $}{∞}-categories with involution}
Let $ \mathcal{C} $ be a stably symmetric monoidal $\infty $-category with an involution, i.e. an exact autoequivalence $ \sigma \colon \mathcal{C} \xrightarrow{\sim} \mathcal{C} $ and a functor $ BC_2 \to \EE_\infty \mathrm{Alg} \Catex $ sending $ * \mapsto \mathcal{C} $ and a generator of  \mathrm{End}_{BC_2}(*) \simeq C_2 $ to $ \sigma $. 
Then $ \sigma $ induces a $ C_2 $-action on the $ \infty $-groupoid of $ \otimes $-invertible objects $ \Pic(\mathcal{C}) $. 
Let $ L \in \Pic(\mathcal{C})^{hC_2} $ be a homotopy fixed point of this action. 
In other words, $ L $ is endowed with the choice of an equivalence $ \phi \colon L \simeq \sigma(L) $, a homotopy from $ \sigma(\phi) \circ \phi $ to the identity on $ L $, and higher coherences. 

Consider a functor $ f \colon \mathcal{C}^\op \to \Spectra $ which is $ C_2 $-equivariant with respect to the $ \sigma $-action on $ \mathcal{C} $ and the trivial action on $ \Spectra $. 
In particular, the ``$ C_2 $-equivariance'' of $ f $ is additional data: for each $ x \in \mathcal{C} $, an equivalence of spectra $ c_x \colon f(x) \simeq f(\sigma(x)) $ which is natural in $ x $, a homotopy from $ c_{\sigma(x)} \circ c_x $ to $ \mathrm{id}_x $, and higher coherences. 
\begin{lemma}\label{lemma:maps_into_fixed_object_are_equivariant}
    Let $ \mathcal{C} $ be a stably symmetric monoidal $\infty $-category with an involution, and suppose given $ L \in \Pic(\mathcal{C})^{hC_2} $ a homotopy fixed point of this action. 
    Then the functor $ \hom_{\mathcal{C}}(-, L) $ promotes canonically to a $ C_2 $-equivariant functor $ \mathcal{C}^\op \to \Spectra $ in the sense of the previous paragraph. 
\end{lemma}
\begin{proof}
    Note that the Yoneda embedding $ y \colon \mathcal{C} \to \mathrm{Fun}\left(\mathcal{C}^\op, \Spectra\right) $ is equivariant with respect to the given action on $ \mathcal{C} $ and the action of $ C_2 $ on the functor category via $ F \mapsto \sigma^*F = F \circ \sigma $. \Lucy{This may be `overkill,' but this is true because $ \mathcal{C} $ can be regarded as a $ \mathcal{O}^\op_{C_2} $-parametrized $\infty$-category (with empty fiber over $ C_2/C_2 $). Then there is a parametrized Yoneda embedding. }
    Now since $ \Pic(\mathcal{C}) \subseteq \mathcal{C} $ induces $ \Pic(\mathcal{C})^{hC_2} \to \mathcal{C}^{hC_2} $, we may take the image of $ L $ under the Yoneda embedding: $ y(L) \in \mathrm{Fun}\left(\mathcal{C}^\op, \Spectra\right)^{hC_2} \simeq \mathrm{Fun}_{C_2}\left(\mathcal{C}^\op, \Spectra\right) $. 
\end{proof}
Now fix a presentably symmetric monoidal $ \infty $-category $ \mathcal{D} $, and regard it as having the trivial $ C_2 $-action. 
Recall the $ \mathrm{Fin}_* $-cartesian fibration $ \left(\Cat^{BC_2}_{\op//q^*\mathcal{D}} \right)^{\otimes} \simeq \left(\left(\Cat_{\op//\mathcal{D}}\right)^{BC_2} \right)^{\otimes}\to (\Cat^{BC_2})^\times $ from \cite[p. 13]{CHN2024}. 
Set 
\begin{equation*}
    \mathcal{W}_{\mathcal{D}}^\otimes := \EE_\infty \mathrm{Alg} \left(\Cat^{BC_2}\right)^\times \times_{(\Cat^{BC_2})^\times} \left(\Cat^{BC_2}_{\op//q^*\mathcal{D}} \right)^{\otimes}
\end{equation*}
This is a $ \mathrm{Fin}_* $-cartesian fibration classified by the lax symmetric monoidal functor
\begin{equation}\label{eq:functor_classifying_eqvt_functors}
\begin{split}
     \EE_\infty \mathrm{Alg}(\Cat)^{BC_2} \to \Cat \\
     \mathcal{C}^\otimes \mapsto \Fun_{C_2}(\mathcal{C}^\op, q^* \mathcal{D}) \,.
\end{split}
\end{equation} 
In particular, an object of the underlying $ \infty $-category of $ \mathcal{W}_{\mathcal{D}}^\otimes $ is a pair $ (\mathcal{C}, f) $ where $ \mathcal{C} $ is a symmetric monoidal $ \infty $-category with a $ C_2 $-action (via a symmetric monoidal functor) and $ f \colon \mathcal{C}^\op \to q^*\mathcal{D} $ is a $ C_2 $-equivariant functor. 
\begin{construction}\label{cons:Poincare_structure_from_equivariant_functor}
    Given a $ C_2 $-equivariant functor $ f \colon \mathcal{C}^\op \to \mathcal{D} $, we may regard the data of the $ C_2 $-equivariance of $ f $ as a commutative diagram
    \begin{equation*}
    \begin{tikzcd}
        \widetilde{\mathcal{C}^\op} \ar[r, "{\widetilde{f}}"] \ar[d] & \mathcal{D} \times BC_2 \ar[d] \\
        BC_2\ar[r,equals] & BC_2
    \end{tikzcd}
    \end{equation*}
    where the vertical maps are cocartesian fibrations and the restriction of $ \widetilde{f} $ to the fiber over the point $ * \in BC_2 $ recovers $ f $. 
    The diagram induces a map on cocartesian sections
    \begin{equation*}
        \overline{f} \colon \Fun_{BC_2}^{\mathrm{cocart}}(BC_2, \widetilde{\mathcal{C}^\op}) \to \mathcal{D}^{BC_2} \,.
    \end{equation*}
    Now if $ \mathrm{C} $ is a symmetric monoidal $ \infty $-category, we can associate to $ f $ the composite
    \begin{equation*}
        T_f \colon \mathcal{C}^\op \xrightarrow{x \mapsto x \otimes \sigma(x)} \Fun_{BC_2}^{\mathrm{cocart}}(BC_2, \widetilde{\mathcal{C}^\op}) \xrightarrow{\overline{f}} \mathcal{D}^{BC_2} \,. 
    \end{equation*}
    Finally, if $ \mathcal{D} $ admits $ BC_2 $-indexed limits, we can take homotopy fixed points of $ C_2 $-objects in which case we define the functor $ \Qoppa_f^s \colon \mathcal{C}^\op \to \mathcal{D} $ as\Lucy{I'm just using the same notation as Harpaz-Nardin-Shah here, but $(-)^s$ is maybe a little weird because it should be `hermitian,' not `symmetric.'} the composite
    \begin{equation*}
         \Qoppa_f^s \colon \mathcal{C}^\op \xrightarrow{T_f} \mathcal{D}^{BC_2} \xrightarrow{(-)^{hC_2}} \mathcal{D} \,. 
    \end{equation*}
\end{construction}
\begin{observation}\label{obs:cross_effect_twisted_poincare_structure}
    Let $ \mathcal{C} $, $ L $, be as before. 
    Then the cross effect $ B_{L} $ of $ \Qoppa^s_L$ is given by $ B_L(x,y) = \mathrm{hom}_{\mathcal{C}}(x \otimes \sigma (y), L) $. 
\end{observation}
\begin{proposition}\label{prop:multiplicativity_of_Poincare_struct_from_eqvt_functor}
    Let $ \mathcal{D} $ be a symmetric monoidal $ \infty $-category which admits $ BC_2 $-indexed limits; endow $ \mathcal{D} $ with the trivial $ C_2 $-action.   
    Then the assignment $ (\mathcal{C}, f) \mapsto ( \mathcal{C}, \Qoppa^s_{f}) $ of Construction \ref{cons:Poincare_structure_from_equivariant_functor} assembles to form a lax symmetric monoidal functor 
    \begin{equation*}
        \mathcal{W}_{\mathcal{D}}^\otimes \to \left(\Cat^{BC_2}\right)_{\op//\mathcal{D}}^\otimes 
    \end{equation*}
    sitting in a commutative diagram
    \begin{equation}\label{diagram:mult_of_Poincare_struct_from_eqvt_functor}
    \begin{tikzcd}
         \mathcal{W}_{\mathcal{D}}^\otimes \ar[r] \ar[d] & \left(\Cat^{BC_2}\right)_{\op//\mathcal{D}}^\otimes \ar[d] \\
         \EE_\infty \mathrm{Alg}(\Cat^{BC_2})^\times \ar[r,"{\mathcal{C}^\otimes \mapsto \mathcal{C}}"] & \left(\Cat^{BC_2}\right)^\times 
    \end{tikzcd}
    \end{equation}
    in which both vertical arrows are both $ \mathrm{Fin}_* $-cartesian fibrations and cocartesian fibrations of $ \infty $-operads. 
\end{proposition}

\begin{construction}\label{cons:functor_classifying_twisted_diagonal} \Lucy{This is basically Construction 3.1.4 of \cite{CHN2024} with minor edits.}
    Consider the functors $ r \colon \EE_\infty\mathrm{Alg}(\Cat^{BC_2}) \xrightarrow{\mathcal{C}^\otimes \mapsto \mathcal{C}} \Cat^{BC_2} $ and $ p \colon \EE_\infty\mathrm{Alg}(\Cat^{BC_2}) \xrightarrow{\mathrm{forget}} \Cat \xrightarrow{q^*} \Cat^{BC_2} $ where $ q \colon BC_2 \to * $. 
    We construct a symmetric monoidal natural transformation $ \tau \colon r^\times \implies p^\times $ whose component at a given category $ \mathcal{C} $ with $ C_2 $-action $ \sigma \colon \mathcal{C} \simeq \mathcal{C}$ is the $ C_2 $-equivariant functor $ \mathcal{C} \xrightarrow{x \mapsto x \otimes \sigma(x)} p(\mathcal{C}) $. 

    Let $ \mathrm{Span}\left(\mathrm{Fin}_{C_2}^{\mathrm{free}}\right) $ be the span $ \infty $-category of finite sets with free $ C_2 $-action. 
    For an $ \infty $-category with finite products $ \mathcal{E} $, there is a natural equivalence $ \Fun^\times\left(\mathrm{Span}\left(\mathrm{Fin}_{C_2}^{\mathrm{free}}\right), \mathcal{E}\right) \simeq \mathrm{CMon}(\mathcal{E})^{BC_2} $ between product-preserving functors $ \mathrm{Span}\left(\mathrm{Fin}_{C_2}^{\mathrm{free}}\right) \to \mathcal{E} $ and commutative monoids in $ \mathcal{E} $ with $ C_2 $-action. 
    Taking $ \mathcal{E} \simeq \Cat $, we may identify the functor $ r $ as restriction along the inclusion $ i \colon BC_2 \to \mathrm{Span}\left(\mathrm{Fin}_{C_2}^{\mathrm{free}}\right) $ of the maximal subgroupoid in the full subcategory on a finite $ C_2 $-set with a single orbit. 
    On the other hand, we can identify the functor $ p $ as restriction along the map $ j \colon BC_2 \to \{*\} \xrightarrow{* \mapsto C_2} \mathrm{Span}\left(\mathrm{Fin}_{C_2}^{\mathrm{free}}\right) $. 
    Now the span $ C_2 \xleftarrow{\pi_1} C_2 \times C_2 \xrightarrow{\pi_2} C_2 $ determines a morphism in $ \mathrm{Span}\left(\mathrm{Fin}_{C_2}^{\mathrm{free}}\right) $ which is equivariant with respect to the given action on the source $ C_2 $ and the \emph{trivial action} on the target $ C_2 $. 
    This morphism determines a functor $ \Delta^{1} \times BC_2 \to \mathrm{Span}\left(\mathrm{Fin}_{C_2}^{\mathrm{free}}\right) $ whose restriction to $ \{0\} \times BC_2 $ agrees with $ i $ and whose restriction to $ \{1\} \times BC_2 $ agrees with $ j $. 
    This determines a natural transformation $ i^* \implies j^* $ of functors $ \Fun \left(\mathrm{Span}\left(\mathrm{Fin}_{C_2}^{\mathrm{free}}\right), \Cat\right) \to \Fun(BC_2, \Cat) $, and precomposing with the inclusion $ \Fun^\times \left(\mathrm{Span}\left(\mathrm{Fin}_{C_2}^{\mathrm{free}}\right), \Cat\right) \subset \Fun \left(\mathrm{Span}\left(\mathrm{Fin}_{C_2}^{\mathrm{free}}\right), \Cat\right)$ gives the desired natural transformation $ \tau \colon r \implies p $. 
    Since $ r $ and $ p $ preserve products, we may lift them to symmetric monoidal functors $ r^\times, p^\times \colon \EE_\infty\mathrm{Alg}(\Cat^{BC_2})^\times \to (\Cat^{BC_2})^\times $, and $ \tau $ refines to a symmetric monoidal natural transformation $ \tau^\times \colon r^\times \implies p^\times $.
\end{construction}
\begin{proof}
    [Proof of Proposition \ref{prop:multiplicativity_of_Poincare_struct_from_eqvt_functor}] 
    \Lucy{Pretty similar to proof of \cite[Proposition 3.1.3]{CHN2024}, the main thing is the fact (stated after proof of Lemma 3.1.1 of \emph{op. cit.}) that $ \mathrm{Fin}_* $-cartesian fibrations are classified by lax symmetric monoidal functors.}  
    Horizontally composing the natural transformation of Construction \ref{cons:functor_classifying_twisted_diagonal} with the functor \ref{cons:Poincare_structure_from_equivariant_functor} and unstraightening induces a commutative diagram
    \begin{equation}
    \begin{tikzcd}
         \mathcal{W}_{\mathcal{D}}^\otimes \ar[d] \ar[r] & \left(\Cat^{BC_2}_{\op//\mathcal{D}}\right)^\otimes \ar[d] \\ 
         \left(\Cat^{BC_2}\right)_{\op//\mathcal{D}}^\otimes \ar[r,"{\mathcal{C}^\otimes \mapsto \mathcal{C}}"] & \left(\Cat^{BC_2}\right)^\times      
    \end{tikzcd}    \,,
    \end{equation}
    where we have used that $ (q^* \mathcal{D})^{hC_2} \simeq \mathcal{D}^{BC_2} $. 
\end{proof}
\begin{definition}
    Define 
    \begin{equation*}
        \mathcal{W}_{\mathrm{ex}}^\otimes \subseteq \mathcal{W}_{\Spectra}^\otimes \times_{\EE_\infty\mathrm{Alg}(\Cat)} \EE_\infty\mathrm{Alg}(\Catex)
    \end{equation*}
    to be the full sub-operad on those colors $ (\mathcal{C}, f) $ so that $ f $ is exact. 
\end{definition}
\begin{observation}
    The commutative square (\ref{diagram:mult_of_Poincare_struct_from_eqvt_functor}) restricts to a commutative square of $ \infty $-operads 
    \begin{equation}\label{diagram:from_eqvt_functor_to_Cath}
    \begin{tikzcd}
        \mathcal{W}_{\mathrm{ex}}^\otimes \ar[r,"{(\mathcal{C},f) \mapsto (\mathcal{C}, \Qoppa_f^s)}"] \ar[d] & \Cath^\otimes \ar[d] \\
        \left(\EE_\infty\mathrm{Alg}(\Catex)^{BC_2}\right)^\otimes \ar[r] & \Catex^\otimes  
    \end{tikzcd}
    \end{equation}
    where $ \Qoppa^s_f $ was defined in Construction \ref{cons:Poincare_structure_from_equivariant_functor}. 
\end{observation}
\begin{lemma}
    Both vertical maps in \ref{diagram:from_eqvt_functor_to_Cath} are cocartesian fibrations of $ \infty $-operads. 
    In particular, $ \mathcal{W}_{\mathrm{ex}}^\otimes $ is a symmetric monoidal $ \infty $-category. 
\end{lemma}
\begin{proof}
    That $ \Cath^\otimes $ is symmetric monoidal and the right-hand projection is a symmetric monoidal functor is \cite[Theorem 5.2.7]{CDHHLMNNSI}. 
    Furthermore, by Proposition \ref{prop:multiplicativity_of_Poincare_struct_from_eqvt_functor} and base change, we have a cocartesian fibration of $ \infty $-operads:
    \begin{equation*}
       \pi \colon \left(\EE_\infty\mathrm{Alg}(\Catex)^{BC_2}\right)^\otimes \times_{\left(\Cat^{BC_2}\right)^\times} \left(\Cat^{BC_2}_{\op//\mathcal{D}}\right)^\otimes \to \left(\EE_\infty\mathrm{Alg}(\Catex)^{BC_2}\right)^\otimes 
    \end{equation*}
    so that $ \mathcal{W}_{\mathrm{ex}}^\otimes $ includes as a full sub-operad on the fiber product on the left on those colors $ (\mathcal{C}, f \colon \mathcal{C}^\op \to \Spectra) $ so that $ f $ is exact. 
    It suffices to show that if $ \alpha \colon (\mathcal{C}_i, f_i \colon \mathcal{C}_i^\op \to \Spectra)_{i \in I} \to (\mathcal{D}, f \colon \mathcal{D}^\op \to \Spectra) $ is a $ \pi $-cocartesian arrow so that $(\mathcal{C}_i, f_i \colon \mathcal{C}_i^\op \to \Spectra)_{i \in I} $ is in $ \mathcal{W}_{\mathrm{ex}}^\otimes $, then $ (\mathcal{D}, f \colon \mathcal{D}^\op \to \Spectra) $ is also in $ \mathcal{W}_{\mathrm{ex}}^\otimes $. 
    This holds because the left Kan extension of the multi-exact functor $ \Pi_i  f_i \colon \Pi_i \mathcal{C}_i^\op \to \Spectra $ along $ \Pi_i \mathcal{C}_i^\op \to \bigotimes_i \mathcal{C}_i^\op \xrightarrow{\alpha} \mathcal{D}^\op $ is exact. 
\end{proof}
As in \cite[p. 15]{CHN2024}, we can identify objects of the underlying $ \infty $-category of $ \mathcal{W}_{\mathrm{ex}}^\otimes $ with pairs $ \left(\mathcal{C}^\otimes, \sigma_\mathcal{C}, L, \lambda\right) $ where $ \mathcal{C} $ is a symmetric monoidal stable $ \infty $-category with involution $ \sigma_{\mathcal{C}} $, and $ (L, \lambda) \in \mathrm{Ind}(\mathcal{C})^{hC_2} $ is a fixed point with respect to the induced action on $ \mathrm{Ind}(\mathcal{C})$. 
\begin{recollection}
    A symmetric monoidal $ \infty $-category $ \mathcal{C} $ is said to be \emph{rigid} if every object in $ \mathcal{C} $ is dualizable. 
    \Lucy{\href{https://mathoverflow.net/a/337430}{reference} for when derived categories of connective objects over more general objects are compactly generated}
\end{recollection}
\begin{proposition}
    % Let $  \left(\mathcal{C}^\otimes, \sigma_\mathcal{C}, L, \lambda\right) \in \mathcal{W}_{\mathrm{ex}} $ be as above. 
    Suppose $ \mathcal{C} $ is a rigid stably symmetric monoidal $ \infty $-category with a $ C_2 $-action $ \sigma_{\mathcal{C}} $ via symmetric monoidal functors, and let $ (L,\lambda) \in \mathrm{Ind}(\mathcal{C})^{BC_2} $. 
    Then the hermitian structure $ \Qoppa^s_{L} $ is non-degenerate if and only if $ L $ belongs to $ \mathcal{C} $, and it is furthermore Poincaré if and only if the underlying object $ L $ is tensor-invertible in $ \mathcal{C} $. 
    In addition, if $ g \colon \mathcal{C} \to \mathcal{C}' $ is a symmetric monoidal $ C_2 $-equivariant exact functor, $ (L,\lambda) \in \mathcal{C}^{BC_2} $ and $ (L',\lambda') \in (\mathcal{C}')^{BC_2} $ are tensor-invertible, and $ g(L,\lambda) \simeq (L',\lambda') $ an equivalence in $ (\mathcal{C}')^{BC_2} $, then the induced hermitian functor $ \left(\mathcal{C}, \Qoppa^s_{L}\right) \to \left(\mathcal{C}', \Qoppa^s_{L'}\right) $ is Poincaré. 
\end{proposition}
\begin{proof}
    If the bilinear part $ B_L $ is represented by $ D \colon \mathcal{C}^\op \to \mathcal{C} $, then $ D(1_{\mathcal{C}}) = \sigma_{\mathcal{C}}(L) $; in particular, $ \sigma_{\mathcal{C}}(L) \simeq L $ is an object of $ \mathcal{C} $.  
    On the other hand, if $ L $ belongs to $ \mathcal{C} $, then for $ x , y \in \mathcal{C} $, the bilinear part $ B_L $ is $ B_L(x,y) = \hom_{\mathrm{Ind}(\mathcal{C})}\left(x \otimes \sigma_{\mathcal{C}}(y), L \right) \simeq \hom_{\mathcal{C}}\left(x \otimes \sigma_{\mathcal{C}}(y), L \right) \simeq \hom_{\mathcal{C}}\left(x , L \otimes \sigma_{\mathcal{C}}(y)^\vee \right) $. 
    The natural transformation $ \mathrm{id}_{\mathcal{C}} \to D^{\op} \circ D $ is given by $ y \to L \otimes \sigma_{\mathcal{C}} \left(L \otimes \sigma_{\mathcal{C}}(y)^\vee \right)^\vee $ induced by the adjoint of the equivalence $ L \simeq \sigma_{\mathcal{C}} (L) $, which is an equivalence if and only if $ L $ is invertible. 

    Finally, the natural transformation $ g \circ D_L \Rightarrow D_{L'} \circ g $ is $ g\left(L \otimes \sigma_{\mathcal{C}}(y)^\vee\right) \simeq g(L) \otimes g\left(\sigma_{\mathcal{C}}(y)^\vee\right) \to L' \otimes \sigma_{\mathcal{C}'}(g(y))^\vee $ which is an equivalence if and only if the map $ g(L) \to L' $ is an equivalence\Lucy{how much equivariance on $ g(L) \to L' $ is necessary?}. 
\end{proof}

\Noah{I'm commenting out section 5 since we do not yet have an application for it. Feel free to put it back in if you want.}
\section{The Poincaré Brauer Group}
\label{subsection:the_poincare_brauer_group}
Let $A$ be a Poincaré ring spectrum. 
By Remark \ref{remark:poincare_ring_spectra_to_modules_with_Poincare_structure}, $ \left(\Mod_A^\omega,\Qoppa_A\right) $ promotes to a commutative algebra object in the $\infty$-category of Poincaré $\infty$-categories $\Catp$ , and we may thus consider modules over it. 
In this section, we will use modules over Poincaré ring spectra to define derived analogues of the involutive Brauer group for Poincaré ring spectra.

Recall that a Poincaré $\infty$-category is called idempotent complete if the underlying stable $\infty$-category is idempotent complete. The full subcategory of $\Catp$ spanned by idempotent complete Poincaré $\infty$-categories is denoted by $\Catpidem$ \cite[Definition 1.3.2]{CDHHLMNNSII}.

\begin{definition}
    \label{definition:poincare_brauer_space}
    Let $A$ be a Poincaré ring spectrum. We define the \emph{Poincaré Brauer space of $A$} as $$\Brp(A):=\Pic\left(\Mod_{\left(\Mod_A^\omega,\Qoppa_A\right)}(\Catpidem)\right).$$
    The assignment $ A \mapsto \Brp(A) $ defines a functor
    \begin{equation*}
        \Brp \colon \CAlgp \to \CAlg^{\gp}(\Spaces)
    \end{equation*}
    valued in grouplike $ \Einfty $-spaces. 

    Let $ (X, \sigma, Y, \pi) $ be a scheme with involution and a good quotient. 
    We define the \emph{Poincaré Brauer space of $ X $ with respect to $ \pi $} as  
    \begin{equation*}
        \Brp(X) := \lim_{ j \colon \Spec R \to Y } \Brp(\underline{\Gamma(j^*(X))}^{j^*\sigma} )
    \end{equation*}
    where the limit is over all étale $ j $ and $ \Gamma(j^*(X)) $ is a ring with involution by affineness of $ \pi $, hence $ R \to \Gamma(j^*(X)) $ is a Poincaré ring via Example \ref{ex:fixpt_Mackey_functor}. 
\end{definition}
\Lucyil{Can replace good quotient $ Y $ by the Deligne--Mumford stack $ X//C_2 $ \cite[Construction 4.41]{azumaya_involution}. This should correspond to the symmetric Poincaré structure of Example \ref{example:symmetric_poincare_structure}. }
\begin{remark}
    The symmetric monoidal forgetful functor $ \Mod_A(\Catpidem) \to \Mod_A(\Cat^{\mathrm{ex}}_\infty) $ induces a map $ \Brp(A) \to \Br(A) $ of grouplike $ \Einfty $-spaces, where $ \Br(A) $ is the Brauer space $ \mathrm{br}_{\mathrm{alg}}(A) $ of \cite[1154-1155]{MR3190610}. 
\end{remark}
\begin{proposition}
    Let $A$ be a Poincaré ring spectrum. Then we have a canonical equivalence $$\Omega \Brp(A) \simeq \Picp(A).$$
\end{proposition}
\begin{proof} 
    Since $ \Omega\Brp(R) $ is given by the space of automorphisms of any object in $ \Brp(R) $, it suffices to determine the space of autoequivalences of $ \left(\Mod_R^\omega, \Qoppa_R \right) $. 
    By Proposition \ref{prop:relative_poincare_cats_basic_properties}, an autoequivalence is the data of a pair $ (f, \eta) $ where $ f \colon \Mod_R^\omega \xrightarrow{\simeq} \Mod_R^\omega $ is an exact $ R $-linear autoequivalence and $ \eta \colon \Qoppa_R \xrightarrow{\sim} \Qoppa_R \circ f^{\mathrm{op}} $ is a $ \Qoppa_R $-linear equivalence. 
    Since $ \Catp_R\to \Catex_R $ is symmetric monoidal (and hence $f$ will be $\Mod_R^\omega$-linear), $ f $ is of the form $ - \otimes_R \mathcal{L} $ where $ \mathcal{L} $ is an invertible $ R $-module. 
    Since taking bilinear and linear parts is functorial by \cite[Proposition 1.3.11]{CDHHLMNNSI}, $ \eta $ is equivalently the data of a pair of equivalences
    \begin{align*}
        b(\eta) &\colon \hom_{R \otimes R}((-\otimes \mathcal{L}) \otimes (-\otimes \mathcal{L}), R)^{hC_2} \simeq \hom_{R \otimes R}(-\otimes -, R)^{hC_2} \\
        \ell(\eta) &\colon \hom_R( -\otimes \mathcal{L}, R^{\varphi C_2}) \simeq \hom_R( -, R^{\varphi C_2})
    \end{align*} 
    plus a path between their images in $ \hom_R(\mathcal{L}, R^{tC_2}) $. 
    The transformation $ b(\eta) $ is equivalent to the data of an $ R $-bilinear equivalence $ R \simeq \mathcal{L}^\vee \otimes \sigma^* \mathcal{L}^\vee $, and the transformation $ \ell(\eta) $ is equivalent to the data of an $ R^{\varphi C_2} $-linear\Lucy{is the $R^{\varphi C_2}$-linearity of this $\simeq$ correct?} equivalence $ \ell (\eta)\colon R^{\varphi C_2}\otimes_R \mathcal{L}^\vee \xrightarrow{\sim} R^{\varphi C_2} $. 

    Now consider the composites 
    \begin{align*}
        R \otimes_R \mathcal{L}^\vee \xrightarrow{\mathrm{unit}\,\otimes \mathrm{id}} R^{\varphi C_2} \otimes \mathcal{L}^\vee \xrightarrow{ \ell(\eta)} R^{\varphi C_2} \\
        R \otimes_R \mathcal{L} \xrightarrow{\mathrm{unit}\,\otimes \mathrm{id}} R^{\varphi C_2} \otimes \mathcal{L} \xrightarrow{ \ell(\eta)^{-1} \otimes \mathrm{id}_{\mathcal{L}}} R^{\varphi C_2} \,.
    \end{align*}
    These correspond to the $ \ell(q^\vee), \ell(q) $ of Remark \ref{remark:poincare_picard_points_desc}, respectively. 
    In particular, the condition that $ \ell(q^\vee), \ell(q) $ make the diagram (\ref{diagram:pnpic_linear_part_condition}) commute is equivalent to the condition that $ \ell(\eta) $ is an equivalence by an adjunction argument. The data of the path in $\hom_R(\mathcal{L},R^{tC_2})$ is exactly the data needed to show that the maps $\ell(q)$ and $b(q)$ glue together to give a form on $\mathcal{L}$.

    The above thus produces a natural transformation $\Omega \Brp(-)\to \Picp(-)$. In the other direction, to any $(\mathcal{L},q)\in \Pn(\Mod_{A}^\omega)$ invertible, we may define an autoequivalence $(\Mod_A^\omega,\Qoppa)\to (\Mod_A^\omega,\Qoppa)$ via tensoring with $(\mathcal{L},q)$, which will be an autoequivalence by the assumption that $(\mathcal{L},q)$ is invertible. We have that these two natural transformations are inverse to each other, hence the result.
\end{proof}

\subsection{Generalities on $R$-linear Poincar\'e \texorpdfstring{$\infty$}{∞}-categories}
\begin{proposition}\label{prop:relative_poincare_cats_basic_properties}
    Let $ (R, C \to R^{tC_2})$ be a Poincaré ring spectrum and write $ (\Mod_R^\omega, \Qoppa_R) $ for the Poincaré $ \infty $-category of Remark \ref{remark:poincare_ring_spectra_to_modules_with_Poincare_structure}. 
    \begin{enumerate}[label=(\arabic*)]
        \item \label{propitem:Rlin_Poincare_cats_is_symm_mon} The $ \infty $-category $ \Mod_{\left(\Mod_R^\omega, \Qoppa_R \right)}(\Catpidem) $ admits all small limits and colimits, and it inherits a canonical symmetric monoidal structure, and for every morphism $ \left(R, R^{\varphi C_2} \to R^{tC_2}\right) \to (S, S^{\varphi C_2} \to S^{tC_2}) $, the functor $ \Mod_{\left(\Mod_R^\omega, \Qoppa_R \right)}(\Catpidem) \to \Mod_{\left(\Mod_S^\omega, \Qoppa_S \right)}(\Catpidem) $ is a symmetric monoidal left adjoint. 
        \item \label{propitem:classify_R_lin_hermitian_struct} Let $ A $ be an $ \EE_1 $-$ R $-algebra in spectra, and regard the category of compact right $ A $-modules $ \Mod_A^\omega $ as left-tensored over $ \Mod_R^\omega $ in the canonical way. 
        Then the pullback
        \begin{equation}
        \begin{tikzcd}
            & \Mod_{(\Mod_R^\omega, \Qoppa_R)}\left(\Cath\right) \ar[d] \\
            \{\Mod_A^\omega\} \ar[r] & \Catex_R
        \end{tikzcd}
        \end{equation}
        is canonically equivalent to $ \Mod_{N_R A \otimes_{N_R R} R^L }\left(\Spectra^{C_2}\right) $ where $ R^L $ is the $ \EE_\infty $-$ N_R R $-algebra with $ (R^L)^e \simeq R $ and $ (R^L)^{\varphi C_2}  \simeq C $. 

        A $ N_R A \otimes_{N_R R} R^L $-module classifies a $ (\Mod_R^\omega, \Qoppa_R) $-module in Poincaré $ \infty $-categories if its underlying $ A $-module is invertible in the sense of \cite[Definition 3.1.4]{CDHHLMNNSI}. 
        \item \label{propitem:Rlin_Poincare_cats_tensor_mod_gen_inv} Let $ A,B $ be $ R $-algebras with associated ($R$-linear) modules with genuine involution $ (M_A, N_A, N_A \to M_A^{tC_2}) $ and $ (M_B, N_B, N_B \to M_B^{tC_2}) $, respectively so that (under item \ref{propitem:classify_R_lin_hermitian_struct}) $ \left(\Mod_A^\omega, \Qoppa_A \right) $ and $ \left(\Mod_B^\omega, \Qoppa_B \right) $ are objects of $ \Mod_{\left(\Mod_R^\omega, \Qoppa_R \right)}(\Catpidem) $. 
        Then the symmetric monoidal structure of item \ref{propitem:Rlin_Poincare_cats_is_symm_mon} is so that the underlying $ R$-linear $ \infty $-category with perfect duality $ \left(\Mod_A^\omega, \Qoppa_A \right) \otimes_{\left(\Mod_R^\omega, \Qoppa_R \right)} \left(\Mod_B^\omega, \Qoppa_B \right) $ is $ \Mod_A^\omega \otimes_{\Mod_R^\omega} \Mod_B^\omega \simeq \Mod_{A \otimes_R B}^\omega $, and the associated module with genuine involution is given by $ M_A \otimes_R M_B $, $ N_A \otimes_{R^{\varphi C_2}} N_B $, and the structure map is $ N_A \otimes_{R^{\varphi C_2}} N_B \to M_A^{tC_2} \otimes_{R^{tC_2}} M_B^{tC_2} \to (M_A \otimes_R M_B)^{tC_2} $ where the latter map arises canonically from lax monoidality of the Tate construction. 
        % Commented out Aug 14th--just leaving this here just in case: \Lucy{What if it should be $ N_A \otimes_{R^{\varphi C_2}} N_B \to M_A^{tC_2} \otimes_{R^{tC_2}} M_B^{tC_2} \to (M_A \otimes_R M_B)^{tC_2} $ instead?} 

        \item \label{prop_item:R_linear_poincare_cats_maps} Let $ \left(\mathcal{C}, \Qoppa_{\mathcal{C}}\right),  \left(\mathcal{D}, \Qoppa_{\mathcal{D}}\right) $ be objects of $ \Mod_{\left(\Mod_R^\omega, \Qoppa_R \right)}(\Cath) $. 
        Then the forgetful functor induces $ \hom_{\Cath_R} \left(\left(\mathcal{C}, \Qoppa_{\mathcal{C}}\right),  \left(\mathcal{D}, \Qoppa_{\mathcal{D}}\right)\right) \to \hom_{\Catex_R}\left(\mathcal{C}, \mathcal{D}\right) $ on mapping spaces so that the fiber over an $ R $-linear functor $ F \colon \mathcal{C} \to \mathcal{D} $ is the mapping space $ \mathrm{map}_{\Qoppa_R}\left(F_! \Qoppa_\mathcal{C}, \Qoppa_{\mathcal{D}}\right) \simeq \mathrm{map}_{\Qoppa_R}(\Qoppa_{\mathcal{C}}, \Qoppa_{\mathcal{D}} \circ F^\op) $, where the mapping space is taken in $ \Fun_{\Qoppa_R}^q(\mathcal{D}^{\op},\Spectra) $ and $ \Fun_{\Qoppa_R}^q(\mathcal{C}^{\op},\Spectra) $, respectively.\footnote{The proof of \ref{propitem:classify_R_lin_hermitian_struct} in particular shows that $ \Fun^q(\mathcal{C}^{\op},\Spectra) $ is left-tensored over $ \Fun^q(\Mod_R^{\omega,\op},\Spectra) $ in the sense of \cite[Definition 4.2.1.19]{LurHA}, so this makese sense.} 

        \item The symmetric monoidal forgetful functor $ \theta \colon \Mod_{\left(\Mod_R^\omega, \Qoppa_R \right)}(\Cath) \to \Mod_{\Mod_R^\omega}(\Catex) $ is a (co)cartesian fibration. \Lucy{what is it classified by?}
    \end{enumerate}
\end{proposition}
\begin{remark}
    A special case of part \ref{propitem:classify_R_lin_hermitian_struct} is \cite[Example 5.4.13]{CDHHLMNNSI}.
\end{remark}
\begin{proof}
    \begin{enumerate}[label=(\arabic*)]
        \item The first part of the statement follows from \cite[\S6.1]{CDHHLMNNSI} and \cite[\S4.2.3]{LurHA}. 
        \item Let $ \mathcal{LM}^\otimes $ denote the $ \infty $-operad of \cite[Definition 4.2.1.7]{LurHA}. 
        Our strategy of proof will be similar to that of \cite[\S5.3]{CDHHLMNNSI}: First, we show that an $ \mathcal{LM}^\otimes $-algebra object in $ \Cath $ is equivalent to an $ \mathcal{LM}^\otimes $-algebra object in an operad of functor categories. 
        Then, we use a (suitably coherent version of) the classification of hermitian structures on module categories as categories of modules over the Hill--Hopkins--Ravenel norm \cite[Theorem 3.3.1]{CDHHLMNNSI} to conclude. 
        Recall that the action of $ \Mod_R^{\omega} $ on $ \Mod_A^\omega $ is given by a functor $ \mathcal{LM}^\otimes \to \Cat_\infty^\times $, and define $ \Fun_{\Mod_R^{\omega,\op}}(\Mod_A^{\omega,\op},\Spectra)^\otimes $ via the following pullback square of $ \infty $-operads: 
        \begin{equation}
        \begin{tikzcd}
            \Fun_{\Mod_R^{\omega,\op}}(\Mod_A^{\omega,\op},\Spectra)^\otimes \ar[r,"{p}"]\ar[d] & \mathcal{LM}^\otimes \ar[d,"{\Mod_R^\omega,\Mod_A^\omega}"] \ar[d] \\
            \left(\Cat_\infty\right)_{\op/-/\Spectra}^\otimes \ar[r] & \Cat_\infty^\times 
        \end{tikzcd}\,.
        \end{equation}
        Informally, an object $ F \in \Fun_{\Mod_R^{\omega,\op}}(\Mod_A^{\omega,\op},\Spectra)^\otimes_{\mathfrak{a}} $ is a functor $ F \colon \Mod_R^{\omega,\op} \to \Spectra $ and an object $ G $ over the fiber of $ \mathfrak{m} $ is a functor $ G \colon \Mod_A^{\omega,\op} \to \Spectra $. 
        The $p $-cocartesian edge over the canonical map $ (\mathfrak{a},\mathfrak{m}) \to \mathfrak{m} $ in $ \mathcal{LM}^\otimes $ sends $ (F,G) $ to the lower arrow in the diagram
        \begin{equation*}
        \begin{tikzcd}[column sep=4.5em]
            \Mod_R^{\omega,\op}\times \Mod_A^{\omega,\op} \ar[r,"{F\times G}"] \ar[d,"{-\otimes_R -}"'] & \Spectra \times \Spectra \ar[d,"{\otimes_{\Spectra}}"] \\
            \Mod_A^{\omega,\op} \ar[r,"{F \otimes G := \mathrm{LKE}_{\otimes_R}( \otimes_{\Spectra} \circ (F \times G))}"] & \Spectra
        \end{tikzcd}\,.
        \end{equation*}
        Now define $ \Fun_{\Mod_R^{\omega,\op}}^q(\Mod_A^{\omega,\op},\Spectra)^\otimes $ to consist of the full subcategory of $ \Fun_{\Mod_R^{\omega,\op}}(\Mod_A^{\omega,\op},\Spectra)^\otimes $ consisting of those tuples of functors which are all quadratic. 
        The inclusion $ \Fun_{\Mod_R^{\omega,\op}}^q(\Mod_A^{\omega,\op},\Spectra)^\otimes \to \Fun_{\Mod_R^{\omega,\op}}(\Mod_A^{\omega,\op},\Spectra)^\otimes $ exhibits the former as an $ \infty $-operad, and moreover the localization is compatible with the $ \mathcal{LM}^\otimes $-monoidal structure in the sense of \cite[Definition 2.2.1.6]{LurHA}. 
        We can extend the previous diagram to 
        \begin{equation}\label{diagram:module_structure_on_quadratic_functors}
        \begin{tikzcd}
            \Fun_{\Mod_R^{\omega,\op}}^q(\Mod_A^{\omega,\op},\Spectra)^\otimes \ar[r]\ar[d] & \Fun_{\Mod_R^{\omega,\op}}(\Mod_A^{\omega,\op},\Spectra)^\otimes \ar[r,"{p}"]\ar[d] & \mathcal{LM}^\otimes \ar[d,"{\Mod_R^\omega,\Mod_A^\omega}"] \ar[d] \\
           \Cath^\otimes \ar[r]& \left(\Cat_\infty\right)_{\op/-/\Spectra}^\otimes \ar[r] & \Cat_\infty^\otimes 
        \end{tikzcd}\,.
        \end{equation}
        Modifying \cite[Construction 5.3.15 \& Lemma 5.3.15]{CDHHLMNNSI} slightly (note that Corollary 5.1.4 did not assume the tensor factors to be equivalent), we obtain an analogous commutative diagram of $ \infty $-operads
        \begin{equation}\label{diagram:module_structure_on_pb_functors}
        \begin{tikzcd}
            \Fun_{\Mod_R^{\omega,\op}}^p(\Mod_A^{\omega,\op},\Spectra)^\otimes \ar[d] \ar[r] & \Fun_{\Mod_R^{\omega,\op}}^q(\Mod_A^{\omega,\op},\Spectra)^\otimes \ar[r]\ar[d] & \Fun_{\Mod_R^{\omega,\op}}(\Mod_A^{\omega,\op},\Spectra)^\otimes \ar[r,"{p}"]\ar[d] & \mathcal{LM}^\otimes \ar[d,"{\Mod_R^\omega,\Mod_A^\omega}"] \ar[d] \\
           \Catp^\otimes \ar[r] &\Cath^\otimes \ar[r]& \left(\Cat_\infty\right)_{\op/-/\Spectra}^\otimes \ar[r] & \Cat_\infty^\otimes 
        \end{tikzcd}
        \end{equation}
        in which all squares are pullbacks. 
        Now suppose $ A $ is given a module with genuine involution $ (M_A, N_A,N_A \to M_A^{tC_2}) $ and call the associated Poincaré $\infty$-category $ \overline{\Mod}_A $.  
        Then to lift $ \overline{\Mod}_A $ to a module over $ \left(\Mod_R^\omega,\Qoppa_R\right) $ compatibly with the $ \Mod_R^\omega $-module structure on $ \Mod_A^\omega $ is to give a map of $ \infty $-operads $ \mathcal{LM}^\otimes \to \Cath^\otimes $ so that the restriction along the canonical inclusion $ \mathrm{Assoc}^\otimes \to \mathcal{LM}^\otimes $ gives the algebra object $ \left(\Mod_R^\omega,\Qoppa_R\right) $ and postcomposing with the canonical projection to $ \Catex^\times $ recovers the given $ \Mod_R^\omega $-module structure on $ \Mod_A^\omega $. 
        By the pullback square (\ref{diagram:module_structure_on_quadratic_functors}), this is equivalent to giving an object of $ \mathrm{Alg}_{\mathcal{LM}/\mathcal{LM}}\left(\Fun_{\Mod_R^{\omega,\op}}^q(\Mod_A^{\omega,\op},\Spectra)^\otimes\right) $. 
        Now let us identify the bilinear functor $ \Mod_R^\omega \times \Mod_A^\omega \xrightarrow{ - \otimes_R -} \Mod_A^\omega $ with the exact functor $ \Mod_R^\omega \otimes \Mod_A^\omega \simeq \Mod_{R \otimes A}^\omega \to \Mod_A^\omega $ which is induction along the action map $ R \otimes A \to A $. 
        Using \cite[Corollary 3.4.1]{CDHHLMNNSI} and unravelling definitions gives the claim for $ R$-linear hermitian structures. 
        The proof for $ R $-linear Poincaré structures considers (\ref{diagram:module_structure_on_pb_functors}) instead but otherwise proceeds in an identical fashion.  
        \item By \cite[Theorem 4.4.2.8]{LurHA}, the relative tensor product $ \left(\Mod_A^\omega, \Qoppa_A \right) \otimes_{\left(\Mod_R^\omega, \Qoppa_R \right)} \left(\Mod_B^\omega, \Qoppa_B \right) $ is computed as the geometric realization of the bar construction 
        \begin{align*}
            p \colon \Delta^\op & \to \Cath \\
            [n] &\mapsto \left(\Mod_A^\omega, \Qoppa_A \right) \otimes \left(\Mod_R^\omega, \Qoppa_R \right)^{\otimes n} \otimes \left(\Mod_B^\omega, \Qoppa_B \right)
        \end{align*} 
        Write $ f \colon \Cath \to \Catex $ for the forgetful functor. 
        Then $ f \circ p $ has a colimit with value $ \Mod_A^\omega \otimes_{\Mod_R^\omega} \Mod_B^\omega \simeq \Mod_{A \otimes_R B}^\omega $. 
        Writing $ g \colon \Catex \to \{*\} $, by Example 4.3.1.3 of \cite{HTT} we see that $ f \circ p $ is a $ g $-colimit. 
        By Proposition 4.3.1.5(2) and Example 4.3.1.3 of \cite{HTT}, $ p $ admits a colimit in $ \Cath $ if and only if it admits an $ f $-colimit. 
        Now recall that $ f $ is a cocartesian fibration with pushforward given by left Kan extension \cite[Corollary 1.4.2]{CDHHLMNNSI}. 
        We show that $ f $ satisfies the conditions of \cite[Corollary 4.3.1.11]{HTT}. 
        \begin{itemize}
            \item Condition (1) follows from Theorem 6.1.1.10 of \cite{LurHA} applied to $ \Spectra^\op $ (see the end of \cite[Construction 1.1.26]{CDHHLMNNSI}). 
            \item Condition (2) follows from \cite[Corollary 1.4.2]{CDHHLMNNSI}, the adjoint functor theorem, and presentability of $ \Fun^q(\mathcal{C}) $, which is discussed in the proof of \cite[Lemma 5.3.3]{CDHHLMNNSI} (also see \cite[Remark 6.1.1.11]{LurHA}). 
        \end{itemize}
        Thus the preceding discussion shows that there exists a map of simplicial sets $ p' $ making the diagram commute
        \begin{equation*}
        \begin{tikzcd}
            \Delta^\op \ar[d] \ar[r,"p"] & \Cath \ar[d,"f"] \\
            \left(\Delta^\op\right)^{\triangleright} \ar[r]\ar[ru,"{p'}"] & \Catex 
        \end{tikzcd}\,.
        \end{equation*}
        Since $ \{0\} \to \Delta^1 $ is left anodyne, by \cite[Corollary 2.1.2.7]{HTT} the inclusions
        \begin{align*}
            \{0\} \times \Delta^\op & \to \Delta^1 \times \Delta^\op \\
            \iota \colon \left(\{0\} \times (\Delta^\op)^\triangleright\right) \sqcup_{\{0\} \times \Delta^\op}\left(\Delta^1 \times \Delta^\op \right) & \to \Delta^1 \times \left(\Delta^\op\right)^\triangleright 
        \end{align*}
        are left anodyne. 
        The former implies that there exists a map $ p'' $ making the diagram 
        \begin{equation*}
        \begin{tikzcd}
            \{0\} \times \Delta^\op \ar[r,"p"]\ar[d] &\Cath \ar[d,"f"] \\
            \Delta^1 \times \Delta^\op \ar[r] \ar[ru,"{p''}"] & \Catex
        \end{tikzcd}
        \end{equation*}
        commute. 
        The maps $ p' $ and $ p'' $ assemble to give a map $ p''' := p' \sqcup_p p'' $ making the diagram 
        \begin{equation*}
        \begin{tikzcd}[arrows={crossing over},row sep=large]
             \{0\} \times \Delta^\op \ar[rr,"p"]\ar[d] &  &\Cath \ar[d,"f"] \\
            \left(\{0\} \times (\Delta^\op)^\triangleright\right) \sqcup_{\{0\} \times \Delta^\op}\left(\Delta^1 \times \Delta^\op \right) \ar[r,"\iota"] \ar[rru,"{p'''}", near start] & \Delta^1 \times \left(\Delta^\op\right)^\triangleright \ar[r] \ar[ru,"{\overline{p}}"', bend right=10] & \Catex
        \end{tikzcd}
        \end{equation*}
        commute, and likewise $ \overline{p} $ exists making the diagram commute since $ \iota $ is left anodyne. 
        Now we show that $ \overline{p} $ satisfies the conditions of \cite[Proposition 4.3.1.9]{HTT}. 
        By (the opposite/dual/cocartesian version of) \cite[Remark 3.1.1.10]{HTT} and Proposition 3.1.1.5(2'') \emph{ibid.} and the fact that $ f $ is a cocartesian fibration, we can choose $ \overline{p} $ so that for all $ k \in (\Delta^\op)^\triangleright $, $ \overline{p}|_{\Delta^1 \times \{k\}} $ is $f$-cocartesian. 
        Furthermore, since we can choose $ \Delta^\op, \,\left(\Delta^\op\right)^\triangleright $ to have the markings $ (-)^\flat $ in \cite[Remark 3.1.1.10]{HTT}, $ f \circ \overline{p}|_{\Delta^1 \times \{\infty\}} $ is a degenerate edge in $ \Catex $. 

        Now \cite[Proposition 4.3.1.9]{HTT} implies that $ \overline{p}_0 $ is an $ f $-colimit diagram if and only if $ \overline{p}_1 $ is an $ f $-colimit diagram. 
        Now notice that $ \overline{p}|_{\{1\} \times \left(\Delta^\op\right)^{\triangleright}} $ has image contained in the fiber of $ f $ over $ \Mod_{A \otimes_R B}^\omega $. 
        By \cite[Proposition 4.3.1.10]{HTT}, it suffices to show that $ \overline{p}_1 $ is a colimit diagram in $ \Fun^q\left(\Mod_{A \otimes_R B}^\omega\right) $. 
        Write $ \overline{M}_A \in \Mod_{N^{C_2}A} $ and $ \overline{M}_B \in \Mod_{N^{C_2}B} $ for the corresponding modules (see introduction to \S3.3 of \cite{CDHHLMNNSI}). 
        Unraveling definitions and using \cite[Theorem 3.3.1 \& Corollary 3.4.1 \& Lemma 5.4.6]{CDHHLMNNSI}, it follows that the diagram $ \overline{p}_1|_{\{1\} \times \Delta^\op} $ is the bar construction 
        \begin{align*}
            [n] & \mapsto \overline{M}_A \otimes_{N^{C_2}R} R^{\otimes_{N^{C_2}R} n} \otimes_{N^{C_2} R} \overline{M}_B \,.
        \end{align*}
        This proves the result. 
        %\Lucy{Point is: $ \iota $ is \emph{marked anodyne} and . Then compute the colimit in $ \Fun^q \left(\Mod_{A\otimes_R B}^\omega \right) $.}

        \item  Let $ \left(\mathcal{C}, \Qoppa_{\mathcal{C}}\right) $ be an object of $ \Mod_{\left(\Mod_R^\omega, \Qoppa_R \right)}(\Cath) $ and let $ F \colon \mathcal{C} = \theta \left(\mathcal{C}, \Qoppa_{\mathcal{C}}\right)\to \mathcal{D} $ be an $ R $-linear functor. 
        Now define $ \Qoppa_{\mathcal{D}} \colon \mathcal{D}^\op \to \Spectra $ to be the left Kan extension of $ \Qoppa_{\mathcal{C}} $ along $ F^\op $. 
        Now $ \left(\mathcal{D}, \Qoppa_{\mathcal{D}}\right) \in \Cath $ and there is a canonical map $ (f, \eta )\colon \left(\mathcal{C}, \Qoppa_{\mathcal{C}}\right) \to \left(\mathcal{D}, \Qoppa_{\mathcal{D}}\right)$. 
        Now $ F $ is classified by a functor $ \Delta^1 \times \mathcal{LM}^\otimes \to \Catex^\otimes $, and we may form the pullback
        \begin{equation}
        \begin{tikzcd}
            \mathcal{N} \ar[r] \ar[d] & \Delta^1 \times \mathcal{LM}^\otimes \ar[d]  \\
            \Cath^\otimes \ar[r,"{p}"] & \Catex^\otimes
        \end{tikzcd}\,.
        \end{equation}
        Since $ p $ is a cocartesian fibration \cite[Theorem 5.2.7]{CDHHLMNNSI}, $ \mathcal{N} \to \Delta^1 \times \mathcal{LM}^\otimes $ is a cocartesian fibration, and the nontrivial morphism in $ \Delta^1 $ classifies a map $ F_! \colon \Fun_{\Mod_R^{\omega,\op}}^q(\mathcal{C}^{\op},\Spectra)^\otimes \to \Fun_{\Mod_R^{\omega,\op}}^q(\mathcal{D}^{\op},\Spectra)^\otimes $ of $ \infty $-operads over $ \mathcal{LM}^\otimes $. 
        Passing to algebra objects, we obtain the desired result on mapping spaces. 

        \item By \cite[Proposition 2.4.2.8]{HTT}, it suffices to show that $ \theta $ is a locally (co)cartesian fibration, and that locally (co)cartesian edges are closed under composition. 
        We give the proof that $ \theta $ is a cocartesian fibration; the proof that $ \theta $ is a cartesian fibration is formally dual and will be left to the reader. 

        Let $ \left(\mathcal{C}, \Qoppa_{\mathcal{C}}\right) $ be an object of $ \Mod_{\left(\Mod_R^\omega, \Qoppa_R \right)}(\Cath) $ and let $ F \colon \mathcal{C} = \theta \left(\mathcal{C}, \Qoppa_{\mathcal{C}}\right)\to \mathcal{D} $ be an $ R $-linear functor. 
        Now define $ \Qoppa_{\mathcal{D}} \colon \mathcal{D}^\op \to \Spectra $ to be the left Kan extension of $ \Qoppa_{\mathcal{C}} $ along $ F^\op $. 
        By the proof of \ref{prop_item:R_linear_poincare_cats_maps}, we see that the image of $ \Qoppa_{\mathcal{C}} $ under $ F_! $ is a lift of $ \left(\mathcal{D}, \Qoppa_{\mathcal{D}}\right) $ to an object of $ \Mod_{\left(\Mod_R^\omega, \Qoppa_R \right)}(\Cath) $ and $ (f, \eta) $ to a morphism in $ \Mod_{\left(\Mod_R^\omega, \Qoppa_R \right)}(\Cath) $. 

        Now by Lemma 2.4.4.1 and the locally cocartesian version of Proposition 2.4.1.10 of \cite{HTT}, we must show that for all choices $ \Qoppa_{\mathcal{D}}' $ of an $ R $-linear Hermitian structure on $ \mathcal{D} $, precomposition with $ F_! $ induces a pullback square
        \begin{equation}
        \begin{tikzcd}
            \hom_{\Cath_R}\left(\left(\mathcal{D}, \Qoppa_{\mathcal{D}}\right), \left(\mathcal{D}, \Qoppa_{\mathcal{D}}'\right)\right) \ar[d] \ar[r] & \hom_{\Cath_R}\left(\left(\mathcal{C}, \Qoppa_{\mathcal{C}}\right), \left(\mathcal{D}, \Qoppa_{\mathcal{D}}'\right)\right) \ar[d] \\
            \hom_{\Catex_R}\left(\mathcal{D}, \mathcal{D}\right) \ar[r] & \hom_{\Catex_R}\left(\mathcal{C}, \mathcal{D} \right)
        \end{tikzcd}\,.
        \end{equation}
        By \ref{prop_item:R_linear_poincare_cats_maps}, $ F_! $ induces equivalences on the fibers of the vertical maps, hence $ (f, \eta) $ is locally $ \theta $-cocartesian. 
        The locally $ \theta $-cocartesian maps are manifestly closed under composition, hence we are done.        \qedhere
    \end{enumerate}
\end{proof}
\begin{corollary}\label{cor:poincare_fourier_mukai}
    % Will want to use this to show that some functor: ``algebras'' to Poincaré infinity categories is fully faithful; see proof of Proposition 3.1 in Antieau--Gepner. 
    Let $ R $ be a Poincaré ring, and let $ A, B $ be $ \mathbb{E}_1 $-$ R $-algebras with genuine involution. 
    Then there is an equivalence $ \hom_{\Catpidem_R}\left(\left(\Mod_A^\omega, \Qoppa_A\right), \left(\Mod_B^\omega, \Qoppa_B\right)\right) \simeq \left(\BiMod_{A \otimes_R B^\op}\right)_{A^{\varphi C_2} \otimes_R B^{\varphi C_2} /-} $.   
\end{corollary}
\begin{proof}
    \Lucy{todo--probably need to fix the statement with \emph{duals} when the proof is written}
\end{proof}

\begin{definition}
    Let $ R $ be a Poincar\'e ring, and let $ A $ be an $ \EE_1 $-$ R $-algebra with an anti-involution. 
    We will refer to the data of $ (M_A, N_A, N_A \to M_A^{tC_2}) $ of Proposition \ref{prop:relative_poincare_cats_basic_properties}\ref{propitem:classify_R_lin_hermitian_struct} as an \emph{$ R $-linear $ A $-module with genuine involution}. 
\end{definition}
\begin{remark}
    When $ R = \sphere $ is the initial Poincar\'e ring, then a $ \sphere $-linear $ A $-module with genuine involution is simply an $ A $-mmodule with genuine involution in the sense of \cite[Defintiion 3.2.3]{CDHHLMNNSI}. 
\end{remark}
\begin{proposition}\label{prop:classify_R_linear_Poincare_functors}
    Let $ R $ be a Poincar\'e ring, and let $ A, B $ be $ \EE_1 $-$ R $-algebras with anti-involutions and let $ (M_A, N_A, \alpha \colon N_A \to M_A^{tC_2} ) $, $ (M_B, N_B, \beta \colon N_B \to M_B^{tC_2}) $ be $ R $-linear modules with genuine involution over $ A $ and $ B $, respectively. 
    Suppose given a map $ f \colon A \to B $ of $ \EE_1 $-$ R $-algebras with anti-involution, and write $ f_* $ for the functor $ B \otimes_A - \colon \LMod_A^\omega \to \LMod_B^\omega $. \Lucy{`geometrically' this should be $ f^* $}
    Then 
    \begin{enumerate}
        \item the data of a $ R $-linear hermitian functor $ \left(\LMod_A^\omega, \Qoppa_{M_A}^\alpha \right) \to \left(\LMod_B^\omega, \Qoppa_{M_B}^\beta \right) $ covering the base change functor $ f_* $ can be encoded by a triple $ (\delta, \gamma, \sigma ) $ where $ \delta \colon M_A \to M_B $ is a morphism in $ \LMod_{A \otimes_R A}^{hC_2} $, $ \gamma \colon N_A \to N_B $ is a morphism in $ \LMod_{A \otimes_R R^{\varphi C_2}} $, and $ \sigma $ is a homotopy making the square
        \begin{equation*}
        \begin{tikzcd}
            N_A \ar[d,"\alpha"] \ar[r,"\gamma"] & N_B \ar[d,"\beta"] \\
            M_A^{tC_2} \ar[r,"{\delta^{tC_2}}"] & M_B^{tC_2}
        \end{tikzcd}
        \end{equation*}
        commute. \Lucy{This is an $ R $-linear version of \cite[Corollary 3.4.2]{CDHHLMNNSI}}

        \item $ (\delta, \gamma, \sigma) $ defines an $ R $-linear Poincar\'e functor if the maps
        \begin{equation*}
        \begin{split}
            B \otimes_A M_A \to (B \otimes_R B) \otimes_{A \otimes_R A} M_A \to M_B \\
            B \otimes_A N_A \to N_B    
        \end{split}    
        \end{equation*}
        are equivalences. 
    \end{enumerate}
\end{proposition}

As in the Picard group case, the symmetric monoidal forgetful functor $\theta \colon \Catp_{R} \to \Catex_R $ induces a map of spectra $ \theta \colon \Brp(A)\to \Br(A^e)$. 
When $ A^e $ is endowed with the trivial action, $ \theta $ will factor through the $2$-torsion on $\pi_0$. 
As a consequence of Proposition~\ref{prop:relative_poincare_cats_basic_properties}\ref{propitem:classify_R_lin_hermitian_struct} we can identify the fiber of this map.
\begin{corollary}\label{cor:Poincare_Brauer_to_Brauer_fiber}
Let $(\mathrm{Mod}_A^\omega, \Qoppa_A)$ be a Poincar{\'e} ring with underlying genuine $C_2$ spectrum $A^L$ as in Proposition~\ref{prop:relative_poincare_cats_basic_properties}\ref{propitem:classify_R_lin_hermitian_struct}. 
Write $ \sigma \colon A^e \simeq A^e $ for the $ C_2 $-action on the underlying $ \EE_\infty $-ring associated to $ A $. 
Then the fiber of the map \[\theta \colon \Brp(A)\to \Br(A^e)\] can be naturally identified with $ \Pic \left(\mathrm{Mod}_{A^L}\left(\mathrm{Sp}^{C_2}\right) \right)  $. 
Moreover, the connecting map $ \Omega \Br(A^e) \simeq \Pic(A^e) \to \mathrm{fib}(\theta) $ is induced by the norm $ \Mod_{A^e}^\op \to \Mod_{A^L}\left(\Spectra^{C_2}\right) $, $ X \mapsto N^{C_2} (X^\vee) \otimes_{N^{C_2}A^e} A^L $, which on underlying spectra is given by $ X \mapsto X^\vee \otimes_A \sigma^* X^\vee $. 
% Then the fiber of the map \[\Brp(A)\to \Br(A^e)\] can be naturally identified with the cofiber $ \mathrm{cofib}\left(\Pic\left(\Mod_{A^e}(\Spectra)^{hC_2}\right) \to \Pic \left(\mathrm{Mod}_{A^L}\left(\mathrm{Sp}^{C_2}\right) \right) \right) $, where the former denotes  the latter cofiber is taken in connective spectra.
\end{corollary}
\begin{proof}[Proof of Corollary \ref{cor:Poincare_Brauer_to_Brauer_fiber}]
Since $\theta:\mathrm{Mod}_{(\Mod_A^\omega, \Qoppa_A)}(\Catpidem)\to \mathrm{Mod}_{\mathrm{Mod}_{A^e}^\omega}(\Catex)$ is symmetric monoidal and conservative, it induces a map $ \theta^\simeq \colon \pnbr(A) \to \mathrm{br}(A^e) $ on the groupoid core of invertible objects. 
Now observe that $ \theta $ is an isofibration; it follows that $ \theta^\simeq $ is a Kan fibration by \cite[\href{https://kerodon.net/tag/01EZ}{Proposition 01EZ}]{kerodon}. 
Consequently, to identify the homotopy fiber of $ \theta $, it suffices to identify the fiber of $ \theta $ over a single point. 
Consider $ \left(\Mod_{A^e}^\omega, \Qoppa\right) $ a point in the fiber of $ \theta $ over $ \Mod_{A^e}^\omega $. 
By Proposition~\ref{prop:relative_poincare_cats_basic_properties}\ref{propitem:classify_R_lin_hermitian_struct}, $ \Qoppa $ is associated to an $ A $-linear invertible module with involution $ \left(M, N, N \to M^{tC_2} \right) $. 
By Proposition~\ref{prop:relative_poincare_cats_basic_properties}\ref{propitem:Rlin_Poincare_cats_tensor_mod_gen_inv}, invertibility of $ \left(\Mod_{A^e}^\omega, \Qoppa\right) $ implies that $ \left(M, N, N \to M^{tC_2}\right) $ is invertible as a module over $ A^L $. 

Now we give the description of the connecting map. 
Write $ \left(\Mod_{A^e}^\omega, \Qoppa_A\right) $ for the identity element in the fiber of $ \theta $ over $ \Mod_{A^e}^\omega $, and let $ \gamma \colon S^1 \to \Br(A^e) $. 
Write $ \mathcal{L}_\gamma $ for $ \gamma $ regarded as a point in $ \Pic(A^e) \simeq \Omega \Br(A^e) $. 
Lift $ \gamma $ to a path $ \widetilde{\gamma} $ in $ \pnbr(A) $ starting at $ \left(\Mod_{A^e}^\omega, \Qoppa_A\right) $, and write $ \left(\Mod_{A^e}^\omega, \Phi\right) $ for the other endpoint of $ \widetilde{\gamma} $. 
By Proposition~\ref{prop:relative_poincare_cats_basic_properties}\ref{propitem:classify_R_lin_hermitian_struct}, $ \Qoppa_A $ is associated to the invertible $ A^L $-module with involution $ A^L $ and $ \Phi $ is associated to some invertible $ A^L $-module with involution $ (M, N, N \to M^{tC_2}) $. 
We may regard $ \widetilde{\gamma} $ as an $ A $-linear hermitian equivalence from $ \left(\Mod_{A^e}^\omega, \Qoppa_A\right) $ to $ \left(\Mod_{A^e}^\omega, \Phi\right) $, which by Proposition \ref{prop:relative_poincare_cats_basic_properties}\ref{prop_item:R_linear_poincare_cats_maps} consists of an $ A $-linear hermitian functor $ (F,\eta \colon \Qoppa_A \to \Phi \circ F) $ so that $ F, \eta $ are both equivalences. 
Since $ \widetilde{\gamma} $ projects to $ \gamma $, we must have that $ F = - \otimes \mathcal{L}_\gamma $. 
Now the natural equivalence $ \eta $ classifies an equivalence $ A^L \simeq \hom_{A^L}(N^{C_2}_A(L), (M,N, N \to M^{tC_2})) $ of $ A^L $-modules (Proposition \ref{prop:classify_R_linear_Poincare_functors}), hence the result. 
\end{proof}

\begin{example}\label{ex: pnbr of sphere}
    Let $\mathbb{S}^u$ denote the universal Poincar{\'e} structure on the sphere spectrum, or equivalently $\mathbb{S}^u$ is the Poincar{\'e} ring associated to the genuine equivariant sphere spectrum. By Corollary~\ref{cor:Poincare_Brauer_to_Brauer_fiber} we have a fiber sequence \[\Pic(\mathrm{Sp}^{C_2})\to \pnbr(\mathbb{S}^u)\to \mathrm{br}(\mathbb{S})\] and by \cite[Corollary 7.17]{Antieau_Gepner_Bruaer} we have that $\pi_0(\mathrm{br}(\mathbb{S}))=0$. Therefore we get a long exact sequence \[
\begin{tikzcd}
\ldots \arrow[r] & \pi_1(\pnbr(\mathbb{S}^u)) \arrow[r] \arrow[d, "\cong"] & \pi_1(\mathrm{br}(\mathbb{S})) \arrow[r] \arrow[d, "\cong"] & \pi_0(\pic(\mathrm{Sp}^{C_2})) \arrow[r] \arrow[d, "\cong"] & \pi_0(\pnbr(\mathbb{S}^u)) \ar[r] & 0 \\
                 & \mathbb{Z}/2\mathbb{Z}\times\mathbb{Z}/2\mathbb{Z}      & \mathbb{Z}                                                  & \mathbb{Z}\times\mathbb{Z}                                  &                            &  
\end{tikzcd}
    \] where the third term is identified via \cite[Section 8.1]{Krause_picard}. From this and the fact that $\mathbb{Z}\to \mathbb{Z}\times \mathbb{Z}$ is the identity on the first component since they are both suspension of the underlying spectrum we see that $\pi_0(\pnbr(\mathbb{S}^u))\simeq \mathbb{Z}$. In fact this allows us to identify the space $\pnbr(\mathbb{S}^u)\simeq \mathbb{Z}\times B\mathrm{TR}^2(\mathbb{S};2)^\times$.
\end{example}
\begin{example}\label{ex:pnbr_closed_point_ramified}
    Let $ k $ be an algebraically closed field, and regard $ k $ as a Poincar\'e ring $ \underline{k} $ via Example \ref{ex:fixpt_Mackey_functor} with the trivial involution.  
    Then 
    \begin{equation*}
        \pi_0\pnbr(\underline{k}) \simeq \begin{cases}
            \ZZ/2\times \mu_2(k) & \text{ if }\mathrm{char}\, k \neq 2 \\
            \ZZ & \text{ if }\mathrm{char}\, k = 2 \,.
        \end{cases} 
    \end{equation*}
    To see this, note that by \cite[Proposition 1.9]{MR2957304}, $ \pi_0 \mathrm{br}(k^e) \simeq 0 $. 
    Thus by Corollary \ref{cor:Poincare_Brauer_to_Brauer_fiber}, it suffices to understand the fiber sequence \[\pic(k)\to \pic(\Mod_{\underline{k}}(\mathrm{Sp}^{C_2})) \to \pnbr(\underline{k})\]. 

    Now if char $ k \neq 2 $, then $ \pic(\underline{k}) \simeq \pic(k)^{hC_2} \simeq (\ZZ \times B k^\times)^{hC_2} $ where the generator of $ C_2 $ acts on $ \pi_1 B k^\times $ trivially. We then have that $H^1(C_2,k^\times)=\mu_2(k)\cong \mathbb{Z}/2\mathbb{Z}$ since $k$ is algebraically closed. 
    %\Lucyil{what is the $ C_2 $-action on $ k^\times $? Come to think of it it might just be trivial.}
    We can then deduce that
    \begin{equation*}
        \pi_0 \pic(\Mod_{\underline{k}}(\mathrm{Sp}^{C_2})) \simeq \begin{cases}
            \ZZ \times \ZZ & \text{ if }\mathrm{char}\, k = 2 \\
            \ZZ\times \mu_2(k) & \text{ otherwise }
        \end{cases}
    \end{equation*}
    where the $\mathrm{char}(k)=2$ case is handled by an argument similar to \cite[Section 8.1]{Krause_picard}. 
    If char $ k \neq 2 $ then the map $ \pi_0 \pic(k) \to \pi_0\pic(\underline{k}) $ is $ \ZZ \xrightarrow{(-\cdot 2,0)} \ZZ\times \mu_2(k) $. 
    On the other hand, if char $ k = 2 $, then the map $ \pi_0 \pic(k) \to \pi_0\pic(\underline{k}) $ is $ \ZZ \xrightarrow{n \mapsto (2n, n)} \ZZ \times \ZZ $. 
    % -----------------------------------------------------------------------------------
    % Leaving this here for ease of comparison/just in case I'm wrong. (--Lucy, May 23rd) 
    % -----------------------------------------------------------------------------------
    % Thus by Corollary \ref{cor:Poincare_Brauer_to_Brauer_fiber}, it suffices to understand the fiber sequence \[\pic(k)\to \mathrm{cofib}\left(\pic(\Mod_{k}(\mathrm{Sp})^{hC_2})\to \pic(\Mod_{\underline{k}}(\mathrm{Sp}^{C_2}))\right)\to \pnbr(\underline{k})\]. 
    % On $\pi_\star$, we obtain the exact sequence \[\ldots\to \mathbb{Z}\to \mathrm{coker}(\mathbb{Z}\to \mathbb{Z}\times \mathbb{Z})\to \pi_0(\pnbr(\underline{k}))\to 0\] where $\pic(\Mod_{\underline{k}}(\mathrm{Sp}^{C_2}))\cong \mathbb{Z}\times\mathbb{Z}$ via the map sending an invertible $\underline{k}$-module $\mathcal{L}$ to the pair $(n_1,n_2)$ where $\mathcal{L}^e\simeq k[n_1]$ and $\mathcal{L}^{\phi C_2}=k[n_2]$, see \cite[Section 8.1]{Krause_picard} for an argument. 
    % The $\mathbb{Z}$-term coming from $\pi_0(\Pic(\Mod_k^{hC_2}))$ maps to the diagonal, and therefore we get a short exact sequence \[0\to \mathbb{Z}\to \mathbb{Z}\to \pi_0(\pnbr(\underline{k}))\to 0\] where the left hand map is injective since $\pnbr(\underline{k})=0$.
    \Lucyil{This and Example \ref{ex:pnbr_closed_pt_unramified} are a pair; if we move one, also move the other.}
\end{example}
\begin{example}\label{ex:pnbr_closed_pt_unramified}
    Let $ k $ be an algebraically closed field, and consider the Poincar\'e ring associated to $ \prod_{C_2} k $ (i.e. $ k \times k $ with the swap action) via Example \ref{ex:fixpt_Mackey_functor}. 
    Similarly to Example \ref{ex:pnbr_closed_point_ramified}, by \cite[Proposition 1.9]{MR2957304} we have $ \pi_0 \mathrm{br}(\Spec k \sqcup \Spec k) \simeq \pi_0 \mathrm{br}(\Spec k)^{\times 2} = 0 $. 
    Thus by Corollary \ref{cor:Poincare_Brauer_to_Brauer_fiber}, it suffices to understand the cokernel of the connecting homomorphism $ \pi_1 \mathrm{br}(k \times k) \to \pi_0 \Pic(\prod_{C_2} k) $. 
    Now since $ \prod_{C_2} k $ is Borel and $(\prod_{C_2}k)^{tC_2}=0$, 
    \begin{equation*}
        \pic\left(\Mod_{\prod_{C_2} k}\left(\Spectra^{C_2}\right) \right) \simeq \pic \Mod_{k \times k}(\Spectra)^{hC_2} \simeq \left(\prod_{C_2} (\ZZ \oplus  k^\times[1])\right)^{hC_2} \simeq \ZZ \oplus k^\times [1] 
    \end{equation*}
    thus $ \pi_0 \pic\left(\Mod_{\prod_{C_2} k}\left(\Spectra^{C_2}\right) \right) \simeq \ZZ $. 
    On the other hand, $ \pi_1 \mathrm{br}(k \times k) \simeq \pi_0 \Pic(k \times k) \simeq \ZZ^{\times 2} $ and the connecting homomorphism is $ (n, m) \mapsto n+m $, whence $ \pi_0\pnbr\left(\prod_{C_2} k\right) = 0 $. 
\end{example}
Write $ \mathrm{Fm} $, resp. $ \mathrm{Pn} $ for the composite $ \Cath_R \xrightarrow{U} \Catp \xrightarrow{\mathrm{Fm}} \Spaces $, resp. $ \Catp_R \xrightarrow{U} \Catp \xrightarrow{\mathrm{Pn}} \Spaces $. 
\begin{proposition}
    Let $ (R, R^{\varphi C_2} \to R^{tC_2}) $ be a Poincaré ring. 
    Then $ \left(\Mod_R^\omega, \Qoppa_R \right) $ corepresents the functors $ \mathrm{Fm} \colon \Cath_R \to \Spaces $ and $ \mathrm{Pn} \colon \Catp_R \to \Spaces $.
\end{proposition}
\begin{proof}
    We prove the statement for $ \mathrm{Pn} $; the proof for $ \mathrm{Fm} $ is similar and is left to the reader. 
    Recall that Proposition \ref{prop:relative_poincare_cats_basic_properties}.\ref{propitem:Rlin_Poincare_cats_is_symm_mon} furnishes an adjoint pair $ \Catp_R \rlarrows \Catp $ of functors. 
    Write $ \overline{\mathcal{C}} = (\mathcal{C},\Qoppa_{\mathcal{C}}) \in \Catpidem_R $. 
    Then
    \begin{equation*}
        \mathrm{Pn}(\mathcal{C}) = \hom_{\Catp}\left((\Spectra^f,\Qoppa^u), U(\overline{\mathcal{C}})\right) \simeq \hom_{\Catp_R}\left(\left(\Mod_R^\omega,\Qoppa_R\right)\otimes(\Spectra^f,\Qoppa^u), \overline{\mathcal{C}}\right) \,,
    \end{equation*}
    where the first equivalence is \cite[Proposition 4.1.3]{CDHHLMNNSI}. 
\end{proof}

\subsection{The Parimala-Srinivas fiber sequence}

It would be helpful to extend the fiber sequence of Corollary~\ref{cor:Poincare_Brauer_to_Brauer_fiber} to the right, both in order to use Poincar{\'e} Brauer groups to compute ordinary Brauer groups and to get a comparison to the involutive Brauer group of \cite{MR1162189}. We will first show that the fiber sequence we have already constructed in Corollary~\ref{cor:Poincare_Brauer_to_Brauer_fiber} does in fact recover the first half of the long exact sequence constructed in \cite[Theorem 2]{MR1162189}.

\begin{lemma}
    Let $A$ be a Poincar{\'e} ring which is has underlying genuine $C_2$-spectrum Borel and such that $A^e$ is connective and $\frac{1}{2}\in \pi_0(A^e)$. Then the zigzag of maps \[\pic(A^{hC_2})\leftarrow \Pic(\mathrm{Mod}_{A}(\mathrm{Sp}^{BC_2}))\rightarrow \pic(\mathrm{Mod}_{A^L}(\mathrm{Sp}^{C_2}))\] are equivalences. 
    If $A^e$ is discrete, this equivalence is given by sending an equivariant discrete module $M$ to $M^{C_2}$ (since $ 2 $ is invertible, strict and homotopy fixed points agree).
\end{lemma}
\begin{proof}
    By the assumption that $\frac{1}{2}\in \pi_0(A^e)$ we have that $A^{tC_2}=0$. Furthermore since $A$ is Borel it also follows that $A^{\phi C_2}=0$, and consequently any module $M$ over $A^L$ has $M^{\phi C_2}=M^{tC_2}=0$. Thus $\mathrm{Mod}_A(\mathrm{Sp}^{BC_2})\to \mathrm{Mod}_{A^L}(\mathrm{Sp}^{C_2})$ is an equivalence even before taking the Picard space.

    It remains to show that the functor $\mathrm{Mod}_A(\mathrm{Sp}^{BC_2})\xrightarrow{(-)^{hC_2}}\mathrm{Mod}_{A^{hC_2}}$ induces an equivalence on the picard space. Note that $(-)^{hC_2}$ commutes with finite colimits and therefore on the thick subcategory generated by $A$ the functor $(-)^{hC_2}$ will be symmetric monoidal, which for $A^e$ connective will include the invertible modules. In this case we also have that $(-)^{hC_2}$ admits an inverse on the thick subcategory generated by $A^{hC_2}$, given by $-\otimes_{A^{hC_2}}A$. Note that since $\frac{1}{2}\in \pi_0(A^e)$ we have that $A^{hC_2}$ will be connective and so the thick subcategory generated by $A^{hC_2}$ will also contain all invertible modules and the result follows. 
\end{proof}

Writing this out for $A$ a discrete ring with $\frac{1}{2}\in A$, we have a fiber sequence \[\pic(A)\xrightarrow{N_{A^{C_2}/A}}\pic(A^{C_2})\to \pnbr(A)\] where we have that $N_{A^{C_2}/A}$ is given by the map $\mathcal{L}\mapsto (\mathcal{L}^{\vee}\otimes \sigma^*\mathcal{L}^\vee)^{C_2}$ which is (the negative of) the usual norm map. In order to extend this fiber sequence to the right we will need to categorify this map.
	
	\begin{construction}
		Let $A$ be a Poincar{\'e} ring and let $\sigma: A^e\to A^e$ denote the involution. Consider the functor \[\mathrm{Mod}_{\mathrm{Mod}_{A^{e}}^\omega}(\mathrm{Cat}_{\infty, \textrm{idem}}^{st})\xrightarrow{\left(-^{op}\otimes_{\mathrm{Mod}_{A^e}^\omega} \sigma^*-^{op}\right)^{hC_2}}\mathrm{Mod}_{(\mathrm{Mod}_{A^e}^\omega)^{hC_2}}(\mathrm{Cat}_{\infty, \textrm{idem}}^{st})\] which we will denote by $N_{A^{hC_2}/A}$. Note that this functor is symmetric monoidal, which can be checked on underlying infinity categories. Here if $\mathcal{C}$ is a module over $\mathrm{Mod}_{A^e}^{\omega}$, then $\mathcal{C}^{op}$ is a module via the equivalence $D_\Qoppa:\mathrm{Mod}_{A^e}^\omega\xrightarrow{\simeq}(\mathrm{Mod}_{A^e}^\omega)^{op}$ induced by the duality from the Poincar{\'e} structure. 
	\end{construction}
	
	\begin{lemma}
		The composite $\pnbr(A)\to \mathrm{br}(A^e)\to \mathrm{Pic}(\mathrm{Mod}_{\mathrm{Mod}_{A^e}^{hC_2}})$ is nullhomotopic. 
	\end{lemma}
	\begin{proof}
		The underlying category of a Poincar{\'e} invertible category is self-dual, and so its square will vanish. Since the functor is naturally nullhomotopic so too is the composite after applying the functor $\Pic(-)$.
	\end{proof}

    \begin{lemma}~\label{lem: surjection onto kernel PS sequence}
        Let $\mathcal{C}\in \mathrm{Mod}_{\mathrm{Mod}_{A^e}^\omega}(\mathrm{Cat}^{st}_{\infty,\textrm{idem}})$ be an invertible category such that $N_{A^{hC_2}/A}(\mathcal{C})=0$. Suppose that $A$ is Borel and that $\frac{1}{2}\in \pi_0(A^e)$. Then $\mathcal{C}$ admits an invertible $A$-linear Poincar{\'e} infinity category structure.
    \end{lemma}
    \begin{proof}
        By assumption there is an equivalence $(\mathcal{C}^{op}\otimes_{\mathrm{Mod}_{A^e}^\omega}\sigma^*\mathcal{C}^{op})^{hC_2}\simeq \mathrm{Mod}_{A}(\mathrm{Sp}^{BC_2})$, fix such an equivalence. We then may define a Poincar{\'e} structure via \[\mathcal{C}^{op}\xrightarrow{x\mapsto x\otimes \sigma^*x}\left(\mathcal{C}^{op}\otimes_{\mathrm{Mod}_{A^e}^\omega}\sigma^*\mathcal{C}^{op}\right)^{hC_2}\simeq \mathrm{Mod}_A(\mathrm{Sp}^{BC_2})\xrightarrow{(-)^{hC_2}}\mathrm{Sp}\] and under the assumptions on $A$ this defines an invertible Poincar{\'e} structure on $\mathcal{C}$ as desired.
    \end{proof}
	
	There is thus a map $\pnbr(-)\to \mathcal{F}(-)$, where $\mathcal{F}(-)$ is the fiber. Delooping both fiber sequences twice we see that we get a map of fiber sequences \[
	\begin{tikzcd}
		\Pic(A^e)\ar[r] \ar[d] & \Pic(\mathrm{Mod}_{A^L}(\mathrm{Sp}^{C_2})) \ar[r] \ar[d] & \pnbr(A) \ar[d]\\
		\Pic(A^e)\ar[r,"N_{A^{hC_2}/A}"] &\Pic(\mathrm{Mod}_{A^e}^{hC_2}) \ar[r] & \mathcal{F}(A) 
	\end{tikzcd}
	\]
	and when $A$ is Borel, connective, and $\frac{1}{2}\in \pi_0(A^e)$ we see that the left and middle vertical maps must be equivalences. This together with Lemma~\ref{lem: surjection onto kernel PS sequence} gives that the right hand vertical map is in fact an equivalence of $\mathbb{E}_\infty$ spaces.

    Summarizing this situation, we get the following:
    \begin{theorem}
        Let $A$ be a Poincar{\'e} ring which is Borel, connective, and $\frac{1}{2}\in \pi_0(A^e)$. 
        Then there is a fiber sequence \[\pnbr(A)\to \mathrm{br}(A^e)\to \mathrm{br}(A^{hC_2}) \] of connective spectra. 

        If $A$ is furthermore discrete, then the map $\mathrm{Br}(A)\to \mathrm{Br}(A^{hC_2})$ is the {\'e}tale cohomological transfer map. 
        \Lucyil{If $ \Spec A $, $ \Spec A^{C_2} $ are not assumed to be normal, we have $ \mathrm{Br}(-) \simeq H^1_{\acute{e}t}(-;\mathbb{Z}) \times H^2_{\acute{e}t}(-;\mathbb{G}_m) $ \cite[Corollary 7.14]{MR3190610}. Is it obvious what the map does on the $ H^1 $ factor?}
    \end{theorem}
    \begin{proof}
        All the remains to show is the identification of $\mathrm{Mod}_{(\mathrm{Mod}_{A^e}^\omega)^{hC_2}}(\mathrm{Cat}^{st}_{\infty,\textrm{idem}})\simeq \mathrm{Mod}_{\mathrm{Mod}_{A^{hC_2}}^\omega}(\mathrm{Cat}^{st}_{\infty,\textrm{idem}})$, which follows from the monoidal equivalence $(\mathrm{Mod}_{A^e}^{\omega})^{hC_2}\simeq \mathrm{Mod}_{A^{hC_2}}^\omega$, and the fact that under this equivalence an Azumaya algebra over $B$ is sent to $(B\otimes_A\sigma^*B)^{C_2}$ which is exactly the cohomological transfer map (see \cite[Example 6.2.3]{azumaya_involution}). \Lucy{Definition 6.2?}
    \end{proof}

\subsection{Azumaya algebras with genuine involution} 
In this section, we introduce the notion of a derived/generalized Azumaya algebra with involution. 
Recall that a classical/discrete Azumaya algebra $ \mathcal{A} $ is étale-locally given by the endomorphism algebra of a vector bundle $ V $, and a derived/generalized Azumaya algebra over $ X $ is étale-locally given by the endomorphism algebra of a perfect complex. 
Observe that if $ \mathcal{A} \simeq \mathrm{End}_X(V) $, then $ \sigma^* \mathcal{A}^\op \simeq \mathrm{End}_X(\sigma^*(V^\vee)) $. 
The prototypical anti-involution on a classical/discrete Azumaya algebra $ \mathcal{A} $ is given by taking transposes and conjugating by an isomorphism $ V \simeq \sigma^*(V^\vee) \otimes \mathcal{L} $, where $ \mathcal{L} $ is a line bundle on $ X $ \cite[\S1.1(ii), p.209; \S1.2, p.216]{MR1162189}. 
If we now consider a derived/generalized Azumaya algebra $ \mathcal{A} \simeq \mathrm{End}_X(P) $ where $ P $ is a perfect complex on $ X $, then the prototypical anti-involution on $ \mathcal{A} $ arises from transposition and conjugation by an equivalence $ P \simeq \sigma^*(P^\vee) \otimes \mathcal{L} [n] $. \Lucy{Example? $ R\Gamma $ of an oriented family of smooth proper schemes over $ X $? relate to Serre duality!} 

Let $ R $ be an $ \EE_\infty $-ring spectrum. 
\begin{recollection}\label{rec:Azumaya_alg_wo_involution} 
    Recall \cites{MR2927172,MR3190610} that an $ \EE_1 $-$ R $-algebra $ A $ is said to be \emph{Azumaya} if it is a compact generator of $ \Mod_R $ and if the natural $ R $-algebra map giving the bimodule structure on $ A $
    \begin{equation*}
        A \otimes_R A^{\mathrm{op}} \to \mathrm{End}_R(A)
    \end{equation*}
    is an equivalence of $ R $-algebras. 
\end{recollection}

\begin{definition}\label{defn:Azumaya_genuine_anti-inv}
    Let $ (R, R \to R^{\varphi C_2} \to R^{tC_2}) $ be a Poincaré ring spectrum, and write $ \sigma \colon R \xrightarrow{\sim} R $ for the involution on $ R $. 
    An \emph{Azumaya algebra with genuine (anti-)involution} over $ R $ is the data of 
    \begin{enumerate}[label=(\alph*)]
        \item \label{defnitem:Azumaya_alg_gi_the_involution} An $ \EE_1 $-$ R $-algebra $ A $ equipped with an anti-involution $ \tau \colon A \to \sigma^* A^\op $ so that the underlying $ \EE_1 $-$ R $-algebra $ A $ is an Azumaya $ R $-algebra in the sense of Recollection \ref{rec:Azumaya_alg_wo_involution} 
        \item \label{defnitem:Azumaya_alg_gi_underlying} an $ (A \otimes_R \sigma^* A)^{\otimes_R 2} $-linear equivalence $ \hom_{R \otimes R}(A \otimes_R A, R) \simeq A \otimes_R \sigma^* A^\op $
        \item A left $ A \otimes_{R} R^{\varphi C_2} $-module $ P $ and an $ A^\op \otimes_R R^{\varphi C_2} $-module $ \overline{P} $
        \item \label{defn_item:Azumaya_gen_inv_conn_map} An $ A \otimes_R R^{\varphi C_2} $-linear map $ P \to A^{tC_2} $ and an $ A^\op \otimes_R R^{\varphi C_2} $-linear map $ \overline{P} \to A^{tC_2} $. 
        Here we regard $ A^{tC_2} $, which is canonically a $ (A \otimes_R \sigma^* A)^{tC_2} $-module, as an $ A \otimes_R R^{\varphi C_2} $-module (resp. $ A^\op \otimes_R R^{\varphi C_2} $-module) via the twisted Tate-valued diagonal $ A \to (A \otimes \sigma^* A)^{tC_2} $ (resp. $ A^\op \to (\sigma^* A^\op \otimes A^\op)^{tC_2} $). 
        \item \label{defn_item:Azumaya_gen_inv_geom_fixpt} An equivalence of $ \left(A \otimes_R \sigma^* A^\op\right) \otimes_R R^{\varphi C_2} $-modules 
        \begin{equation*}
            \hom_R(A, R^{\varphi C_2}) \simeq P \otimes_{R^{\varphi C_2}} \overline{P}
        \end{equation*}
        and a homotopy making the diagram 
        \begin{equation*}
        \begin{tikzcd}
            \hom_R(A, R^{\varphi C_2}) \ar[r] \ar[d] & P \otimes_{R^{\varphi C_2}} \overline{P} \ar[d] \\
            \hom_R(A, R^{tC_2}) \simeq \hom_R(A \otimes A^\op, R)^{tC_2} \ar[r] & \hom_{(A \otimes A^\op)^{\otimes 2}}\left((A \otimes A^\op)^{\otimes 2}, A \otimes A^\op\right)^{tC_2} \simeq (A \otimes A^\op)^{tC_2}
        \end{tikzcd}    
        \end{equation*}
        commute, where the lower horizontal arrow is induced by \ref{defnitem:Azumaya_alg_gi_underlying} and the right vertical arrow is induced by \ref{defn_item:Azumaya_gen_inv_conn_map}. 
    \end{enumerate}
    % Similarly, an \emph{Azumaya algebra with genuine type 2 (anti-)involution} over $ R $ is obtained by replacing $ \tau $ in item \ref{defnitem:Azumaya_alg_gi_the_involution} with $ \tau \colon A \to \sigma_R^* A^\op $, where $ \sigma_R \colon R \xrightarrow{\sim} R $ is the given involution on $ R $. 
\end{definition}
\begin{observation}\label{obs:azumaya_geninv_base_change}
    Let $ A $ be an Azumaya algebra with genuine involution over $ R $, and suppose given a map $ R \to S $ of Poincaré rings. 
    Then $ (A \otimes_R S, P \otimes_{R^{\varphi C_2}} S^{\varphi C_2}) $ is an Azumaya algebra with genuine involution over $ S $. 
\end{observation}
\begin{observation}
    Let $ R $ be a discrete commutative $ \mathbb{F}_2 $-algebra and assume that $ R $ is perfect (i.e. the Frobenius $ r \mapsto r^2 $ is an isomorphism), and endow $ R $ with the trivial (= identity) involution. 
    Regard $ \underline{R} $ as a Poincaré ring spectrum via \ref{ex:fixpt_Mackey_functor}, and let $ A $ be an $ \EE_\sigma $-algebra over $ R $ whose underlying $ \EE_1 $-$ R $-algebra is (generalized) Azumaya. 
    By the isotropy separation sequence $ R_{hC_2} \to R \to R^{\varphi C_2} $ and the Tate orbit lemma \cite[Lemma I.2.1]{NS}, the canonical map $ R^{C_2} \to R^{\varphi C_2} $ induces an equivalence $ R^{tC_2} \to (R^{\varphi C_2})^{tC_2} $.  
    \Lucyil{this map $ R^{C_2} \to R^{\varphi C_2} $ (`mod 2' on $ \pi_0 $) is not the same map $ R^e \to R^{\varphi C_2} $ as the lift along the Tate valued Frobenius! The following commutative triangle commutes on $ \pi_0 $ but I am not sure about higher homotopy yet. Use \cite[\S IV.1.15-16]{NS} or naturality of the Tate-valued Frobenius?}
    \begin{equation*}
    \begin{tikzcd}[row sep=tiny]
        R^{C_2} \ar[dd,"{\sqrt{-}}","\sim"'] \ar[rd] & \\
        & R^{\varphi C_2} \\
        R^e \ar[ru] &    
    \end{tikzcd}    
    \end{equation*} 
    Assume for the moment that the lift $ R \to R^{\varphi C_2} $ of the Tate-valued Frobenius induced an equivalence $ R^{tC_2} \xrightarrow{\sim} (R^{\varphi C_2})^{tC_2} $. 
    Granting this, we obtain a map of Poincaré rings $ \underline{R} \to (R^{\varphi C_2})^t $, where the latter is endowed with the Tate Poincaré structure of Example \ref{example:tate_poincare_structure}. 

    Note that $ A \otimes_{\underline{R}} (R^{\varphi C_2})^t $ is an Azumaya algebra with genuine involution over $ (R^{\varphi C_2})^t $. 
    Suppose given a trivialization of $ A \otimes_{\underline{R}} (R^{\varphi C_2})^t $ over $ (R^{\varphi C_2})^t $, i.e. an equivalence of Poincaré categories $ \left(\Mod_{(R^{\varphi C_2})^t}, \Qoppa\right) \simeq \left(\Mod^\omega_{A\otimes_{\underline{R}} (R^{\varphi C_2})^t},\Qoppa_A \right) $. 
    By the argument of Example \ref{ex:endomorphisms_of_poincare_object}\footnote{By exactness of the Tate construction and the fact that $ A $ is a compact $ R $-module, $ \left(A \otimes_{R} R^{\varphi C_2}\right)^{tC_2} \simeq A^{tC_2} $.}, the data of this trivialization gives a lift of $ A $ to an Azumaya algebra with genuine involution over $ \underline{R} $, i.e. parts (d) and (e) of Definition \ref{defn:Azumaya_genuine_anti-inv}. 
    \Lucyil{to-do: use deformation theory/nilcompleteness of the stack $ \mathbf{M}^p_A $ to produce a trivialization.}
    % $ A \otimes_R R^{\varphi C_2} $-module $ P $ which is compact $ R^{\varphi C_2} $-module $ P $, and an $ (A \otimes_R \sigma^* A^\op) \otimes_R R^{\varphi C_2} $-linear equivalence $ t_A \colon A \otimes_R R^{\varphi C_2} \simeq P \otimes_{R^{\varphi C_2}} P^\vee $, where $ (-)^\vee $ refers to $ R^{\varphi C_2} $-linear dual. 
\end{observation}
\begin{remark}\label{rmk:azumaya_geninv_gives_module_geninv}
    If $ A $ is an Azumaya algebra with genuine involution over $ R $, then in particular $ M_A = A $, $ N_A = P $ is a module with genuine involution over $ A $ in the sense of \cite[Definition 3.2.3]{CDHHLMNNSI}. 
\end{remark}
\begin{definition}
    Let $ (X, \sigma) $ be a scheme with an involution and let $ \pi \colon X \to Y $ exhibit $ Y $ as a good quotient of $ X $. 
    Recall that there is a sheaf of $ C_2 $-$ \EE_\infty $-algebras $ \underline{\mathcal{O}} $ (Construction \ref{cons:structure_sheaf_of_Green_functors}), and write $ \sigma $ for the involution $ \pi_* \mathcal{O}_X \xrightarrow{\sim} \pi_* \mathcal{O}_X $. 
    An \emph{Azumaya algebra with genuine (anti-)involution} over $ X $ is the data of 
    \begin{enumerate}[label=(\alph*)]
        \item An $ \EE_1 $-$ \underline{\mathcal{O}}^e = \pi_* \mathcal{O}_X $-algebra $ A $ equipped with an anti-involution $ \tau \colon A \to \sigma^* A^\op $ (i.e. $ \sigma^*(\tau^\op) \circ \tau \simeq \id_A $, and higher coherences) so that the underlying $ \EE_1 $-$ \underline{\mathcal{O}}^e $-algebra $ A $ is a [generalized] Azumaya $ \underline{\mathcal{O}}^e $-algebra in the sense of \cite[Definition 2.11]{MR2957304}. 
        \item \label{defnitem:global_Azumaya_alg_gi_underlying} an $ (A \otimes_{\underline{\mathcal{O}}^e} \sigma^* A)^{\otimes_{\underline{\mathcal{O}}^e} 2} $-linear equivalence $ \hom_{\underline{\mathcal{O}}^e}(A \otimes_{\underline{\mathcal{O}}^e} A, R) \simeq A \otimes_{\underline{\mathcal{O}}^e} \sigma^* A^\op $
        \item A left $ A \otimes_{\underline{\mathcal{O}}^e} \underline{\mathcal{O}}^{\varphi C_2} $-module $ P $ and an $ A^\op \otimes_{\underline{\mathcal{O}}^e} \underline{\mathcal{O}}^{\varphi C_2} $-module $ \overline{P} $
        \item \label{defn_item:global_Azumaya_gen_inv_conn_map} An $ A \otimes_{\underline{\mathcal{O}}^e} {\underline{\mathcal{O}}}^{\varphi C_2} $-linear map $ P \to A^{tC_2} $ and an $ A^\op \otimes_{\underline{\mathcal{O}}^e} {\underline{\mathcal{O}}}^{\varphi C_2} $-linear map $ \overline{P} \to A^{tC_2} $. 
        Here we regard $ A^{tC_2} $, which is canonically a $ (A \otimes_{\underline{\mathcal{O}}^e} \sigma^* A)^{tC_2} $-module, as an $ A \otimes_{\underline{\mathcal{O}}^e} \underline{\mathcal{O}}^{\varphi C_2} $-module (resp. $ A^\op \otimes_{\underline{\mathcal{O}}^e} \underline{\mathcal{O}}^{\varphi C_2} $-module) via the twisted Tate-valued diagonal $ A \to (A \otimes \sigma^* A)^{tC_2} $ (resp. $ A^\op \to (\sigma^* A^\op \otimes A^\op)^{tC_2} $). 
        \item \label{defn_item:global_Azumaya_gen_inv_geom_fixpt} An equivalence of $ \left(A \otimes \sigma^* A^\op\right) \otimes_{\underline{\mathcal{O}}^e} \underline{\mathcal{O}}^{\varphi C_2} $-modules 
        \begin{equation*}
            \hom_{\underline{\mathcal{O}}^e}(A, \underline{\mathcal{O}}^{\varphi C_2}) \simeq P \otimes_{\underline{\mathcal{O}}^{\varphi C_2}} \overline{P}
        \end{equation*}
        and a homotopy making the diagram 
        \begin{equation*}
        \begin{tikzcd}
            \hom_{\underline{\mathcal{O}}^e}(A, \underline{\mathcal{O}}^{\varphi C_2}) \simeq P \otimes_{\underline{\mathcal{O}}^{\varphi C_2}} \overline{P} \ar[r] \ar[d] & P \otimes_{\underline{\mathcal{O}}^{\varphi C_2}} \overline{P} \ar[d] \\
            \hom_{\underline{\mathcal{O}}^e}(A, {\underline{\mathcal{O}}^e}^{tC_2}) \simeq \hom_{\underline{\mathcal{O}}^e}(A \otimes A^\op, \underline{\mathcal{O}}^e)^{tC_2} \ar[r] & \hom_{(A \otimes A^\op)^{\otimes 2}}\left((A \otimes A^\op)^{\otimes 2}, A \otimes A^\op\right)^{tC_2} \simeq (A \otimes A^\op)^{tC_2}
        \end{tikzcd}    
        \end{equation*}
        commute, where the lower horizontal arrow is induced by \ref{defnitem:global_Azumaya_alg_gi_underlying} and the right vertical arrow is induced by \ref{defn_item:global_Azumaya_gen_inv_conn_map}. 
    \end{enumerate}
\end{definition}
% Commented out May 28th, 2025: Is the following redundant?
% With ordinary Azumaya algebras, the prototypical Azumaya algebra with anti-involution arises from endomorphism rings of perfect modules. 
% Choosing a (nondegenerate symmetric bilinear) form on a perfect module $ P $ endows its endomorphism algebra with additional structure. 
\begin{example}\label{ex:endomorphisms_of_poincare_object}
    Let $ (R, R \to R^{\varphi C_2} \to R^{tC_2} ) $ be a Poincaré ring, and let $ (P,q ) \in \mathrm{Pn}\left(\Mod_R^\omega, \Qoppa_R \right) $. \Lucyil{What if I replaced $ \Qoppa_R $ by a shift $ \Qoppa_R^{[n]} $? Or any $ R $-linear Poincaré structure $ \Qoppa' $ so that $(\Mod_R^\omega, \Qoppa') $ is in $ \pnbr(R) $? see \cite[\S3.5]{CDHHLMNNSI}, and furthermore see \cite[p.216]{MR1162189}.}

    Then $ A:= \mathrm{End}_R(P) $ admits a canonical lift to an $ \EE_1 $ algebra with genuine involution over $ R $ with $ A^{\varphi C_2}:= \hom_R(P, R^{\varphi C_2}) $. 
    If $ P $ is a generator of $ \Mod_R^\omega $, then $ A $ is furthermore Azumaya. 

    By \cite[Proposition 3.1.16]{CDHHLMNNSI}, $ A $ inherits a canonical anti-involution. 
    %\Lucy{todo: an $ R $-linear enhancement of Proposition 3.1.16?} 
    To exhibit \ref{defnitem:Azumaya_alg_gi_underlying}, observe that $ q^\dag $ induces a canonical $ A \otimes A^\op $-linear equivalence $ A = \mathrm{End}_R(P) \xrightarrow{D} \mathrm{End}_{\Mod_R^\mathrm{op}}\left(P^\vee\right) \simeq \mathrm{End}(P^\vee)^\op \xrightarrow{f \mapsto q^{-1} \circ f \circ q} \mathrm{End}_{R}\left(P\right)^\op A^\vee $.  
    If $ P $ is a generator, $ \hom_{R}(P,-) $ induces an equivalence $ \Mod_R^\omega \simeq \Mod_{A}^\omega $, thus we can regard $ \Mod_A^\omega $ as equipped with a Poincaré structure. 
    By the classification of $ R $-linear Poincaré structures of Proposition \ref{prop:relative_poincare_cats_basic_properties}\ref{propitem:classify_R_lin_hermitian_struct}, the Poincaré structure on $ \Mod_A^\omega $ is associated to an $ A $-module with genuine involution $ (M_A, N_A, N_A \to M_A^{tC_2}) $. 
    We claim that $ M_A \simeq A $ with the canonical $ A $-$ A $-bimodule structure: By \cite[Proposition 3.1.6]{CDHHLMNNSI}, as an $ A^\op $-module $ M_A $ is the image of $ A $ under the composite
    \begin{equation*}
        \Mod_A^\omega \xrightarrow{\hom_R(P,-)^{-1}} \Mod_R^\omega \xrightarrow{D_R = \hom_R(-,R)} \Mod_R^{\omega, \op} \xrightarrow{\hom_R(P,-)} \Mod_A^{\omega,\op} \,.
    \end{equation*}
    Observe that the image of $ A $ in $ \Mod_R^{\omega, \op} $ is $ D_R(P) $ and $ q^\dag $ induces an equivalence $ D_R(P) \simeq P $, hence $ M_A \simeq A $ as $ A^\op $-modules. 

    A similar argument with the linear part of $ \Qoppa $ shows that we have an equivalence $ N_A \simeq \hom_{R}(P, R^{\varphi C_2}) $ of $ A $-modules and a commutative square 
    \begin{equation*}
    \begin{tikzcd}
        N_A \ar[r,"\sim"] \ar[d] &  \hom_{R}(P, R^{\varphi C_2}) \ar[d] \\
        A^{tC_2} \ar[r,"\sim"] & \hom_R(P, R^{tC_2}) \simeq \hom(P \otimes_R P, R)^{tC_2}      
    \end{tikzcd}     
    \end{equation*} 
    of $ A $-modules, where $ A $ acts on $ A^{tC_2} $ via the Tate-valued norm. 
    \Lucyil{Reference for Tate-valued norm for $ \mathbb{E}_\sigma $-algebras? continue. discuss consequences of being Morita trivial (distinguished `point' in $ N_A $)? can we identify $ \overline{P} $?}
    % DEPRECATED MAY 8, 2025; PLEASE DO NOT DELETE FOR NOW -- Lucy
    % Since $ P $ is compact, it is dualizable with respect to the symmetric monoidal structure on $ \Mod_R^\omega $ \cite[Theorem III.7.9]{elmendorf2007rings}. 
    % Since $ \otimes_R R^{\varphi C_2} $ is symmetric monoidal, in particular it takes $ P $ to a dualizable object--call it $ \overline{P} $. 
    % Now there is a canonical choice of equivalence \ref{defn_item:Azumaya_gen_inv_geom_fixpt} since both sides are canonically equivalent to $ \overline{P} \otimes_{R^{\varphi C_2}} \overline{P}^\vee $. 
\end{example}

\begin{proposition}\label{prop:classical_inv_to_genuine_inv}
    Let $ R $ be a discrete ring with a given $ C_2 $-action $ \sigma $, regard $ R $ as a Poincaré ring spectrum $ \underline{R}^\lambda $ via Example \ref{ex:fixpt_Mackey_functor}. 
    Let $ A $ be a classical Azumaya algebra over $ R $ with an involution of type 2, i.e. an equivalence of associative $ R $-algebras $ \lambda \colon A \to \sigma^* A^{\op} $. 
    Suppose that either:
    \begin{itemize}
        \item the branch locus in $ \Spec (R)/C_2 = \Spec(R^{C_2}) $ is empty, or 
        \item $ 2 $ is invertible in $ R $.
    \end{itemize} 
    Then there is a canonical Azumaya algebra with genuine involution over $ \underline{R}^{\lambda} $ whose underlying Azumaya algebra is $ A $. 
    \Lucyil{What data is needed to give a lift, when $ R^{\varphi C_2} $ is not necessarily zero (as is have assumed here)? 
    In what other cases does a lift exist trivially? wrt Tate Poincaré structure, suffices to take a trivialization of $ A $?}
\end{proposition} 
\begin{remark}
    If $ \frac{1}{2} \in R $, then $ \mathrm{Br}(\Spec R, \lambda) $ is defined \cite[p. 216]{MR1162189}. 
    In view of Propositions \ref{prop:classical_inv_to_genuine_inv} and \ref{prop:mod_over_azumaya_geninv_is_invertible}, there is a homomorphism $ \mathrm{Br}(\Spec R, \lambda) \to \pi_0 \pnbr(\underline{R}^\lambda) $. \Lucy{rewrite for scheme with involution}
\end{remark}
\begin{remark}
    Equip $ R $ with the trivial involution, and consider the composite $ R \to R^{\varphi C_2 } \to \pi_0 R^{\varphi C_2} \simeq R/2 $. 
    This is equivalent to the composite $ R \to R/2 \xrightarrow{\mathrm{Frobenius}} R/2 $, and since the Frobenius is multiplication by 2 in $ \mathbb{G}_m $ and $ A $ is 2-torsion (because of the involution, there must be a trivialization of $ A \otimes_R \pi_0 R^{\varphi C_2} $, and the set of trivializations is described by some coset space in the exact sequence $ \cdots H^1 (-;\mathbb{G}_m) \to H^1(-;\mathrm{GL}_n) \to H^1(-; \mathrm{PGL}_n) \to \cdots $). 
    Now $ R^{\varphi C_2 } \to \pi_0 R^{\varphi C_2} $ is a limit of nilpotent extensions; use deformation theory to try to lift any given trivialization of $  A \otimes_R \pi_0 R^{\varphi C_2} $ to a trivialization of $ A \otimes_{R} R^{\varphi C_2} $. 
    Finally, use that the space of lifts of $ A $ is controlled by Corollary \ref{cor:Poincare_Brauer_to_Brauer_fiber}? 
    \Lucy{tried to put this in a comment but TeX would not compile :/}
\end{remark}
\begin{proof} [Proof of Proposition \ref{prop:classical_inv_to_genuine_inv}]
    % By \cite[Examples 3.1.9 \& 3.2.9]{CDHHLMNNSI}, it suffices to exhibit $ N $ and an $ A $-linear map $ N \to A^{tC_2} $. 
    Since $ R^{\varphi C_2} = 0 $, conditions (c)-(e) of Definition \ref{defn:Azumaya_genuine_anti-inv} are vacuous. 
\end{proof}

\begin{proposition}\label{prop:mod_over_azumaya_geninv_is_invertible}
    Let $ (R, R \to R^{\varphi C_2} \to R^{tC_2} ) $ be a Poincaré ring, and let $ (A, A^{\varphi C_2} \to A^{tC_2}) $ be an Azumaya algebra with genuine involution over $ R $. 
    Then 
    \begin{enumerate}
        \item $ \left(\Mod_A^\omega, \Qoppa_A \right) $ defines an $ R $-linear Poincaré $ \infty $-category. 
        \item $ \left(\Mod_A^\omega, \Qoppa_A \right) $ is an invertible object in $ \Mod_{\left(\Mod_R^\omega, \Qoppa_R\right)}\left(\Catpidem\right) $. 
    \end{enumerate}
\end{proposition}
\begin{proof}[Proof of Proposition \ref{prop:mod_over_azumaya_geninv_is_invertible}]
    The first statement follows from Proposition \ref{prop:relative_poincare_cats_basic_properties}\ref{propitem:classify_R_lin_hermitian_struct}; we prove the second statement. 
    First, by \cite[Example 3.2.9]{CDHHLMNNSI}, we see that $ \left(\Mod_A^\omega, \Qoppa_A \right) $ is indeed an $ R$-linear Poincaré $ \infty $-category (and not merely hermitian). 
    To show that the associated Poincaré $ \infty $-category is invertible, we must identify a dual $ \left(\Mod_A^\omega, \Qoppa_A \right)^\vee $ and exhibit an equivalence $ \left(\Mod_A^\omega, \Qoppa_A \right) \otimes \left(\Mod_A^\omega, \Qoppa_A \right)^\vee \simeq \left(\Mod_R^\omega, \Qoppa_R \right) $. 
    Since $ \Catpidem_R \to \Catex_R $ is symmetric monoidal, we see that the underlying $ R $-linear $ \infty $-category associated to the dual must be $ \Mod_{A^\op}^\omega $. 
    Moreover, the canonical evaluation map $ \mathrm{ev} \colon \Mod_A^\omega \otimes \Mod_{A^\op}^\omega \xrightarrow{\simeq} \Mod_R^\omega $ sends $ A \otimes A^\op $ to $ A $. 
    Endow $ \Mod_{A^\op}^\omega $ with a Poincaré structure corresponding to the module with genuine involution $ M_{A^\op}:= A^\op $, $ N_{A^\op} := \overline{P} $. 
    It remains to exhibit a natural equivalence 
    \begin{equation}\label{eq:invertible_quadratic_compatibility}
        \eta \colon \left(\Qoppa_A \otimes \Qoppa_{A^\op}\right) \xrightarrow{\simeq} \mathrm{ev}^* \Qoppa_R 
    \end{equation} 
    of [quadratic] functors $ \Mod_A^\omega \otimes \Mod_{A^\op}^\omega \to \Spectra $. 
    By \cite[Theorem 3.3.1]{CDHHLMNNSI}, it suffices to exhibit equivalences on the bilinear and linear parts of (\ref{eq:invertible_quadratic_compatibility}) which glue compatibly. 
    By Proposition \ref{prop:relative_poincare_cats_basic_properties}\ref{propitem:Rlin_Poincare_cats_tensor_mod_gen_inv}, on linear parts, it suffices to exhibit an $ A \otimes_R A^\op $-linear equivalence 
    \begin{equation*}
        \hom_R(A, R^{\varphi C_2}) \simeq N_A \otimes_{R^{\varphi C_2}} N_{A^\op}
    \end{equation*}
    and on bilinear parts, it suffices to exhibit an $ (A \otimes_R A^\op)^{\otimes_R 2} $-linear equivalence 
    \begin{equation*}
        \hom_{R \otimes R}(A \otimes_R A, R) \simeq M_A \otimes_R M_{A^\op} 
    \end{equation*}
    which glue compatibly. 
    This follows from the definitions, concluding the proof. 
\end{proof}
\begin{observation}
    Let $ k $ be an algebraically closed field, and regard $ \underline{k} $ as a Poincar\'e ring spectrum with the trivial involution via Example \ref{ex:fixpt_Mackey_functor}. 
    Let $ (A, A^{\varphi C_2}, A^{\varphi C_2} \to A^{tC_2}) $ be an Azumaya algebra with genuine involution over $ \underline{k} $. 
    By Proposition \ref{prop:mod_over_azumaya_geninv_is_invertible}, $ \left(\Mod_A^\omega, \Qoppa_A\right) \in \pnbr(\underline{k}) $. 
    By \cite[Corollary 1.15]{MR2957304}, $ A $ is equivalent to $ \mathrm{End}_k(P) $ for some compact $ k $-module $ P $. 
    Observe that there is a canonical identification $ A^\op \simeq \mathrm{End}_k(P^\vee) $. 
    By the derived Skolem--Noether theorem \cite[Theorem 5.1.5]{MR2579390}, there exists a unique $ n \in \ZZ $ so that the involution $ A \simeq A^\op $ is induced by an equivalence $ P \simeq P^\vee [n] $ (which is unique up to multiplication by a unit in $ k $). 

    \Lucyil{the next part is contingent on the computation in Example \ref{ex:pnbr_closed_point_ramified}; omit if the computation in the example works out to be zero. If the computation is nonzero, consider this a numerical invariant of derived Azumaya algebras w involution and put it in a definition?} 
    By Example \ref{ex:pnbr_closed_point_ramified}, we may associate to $ \left(\Mod_A^\omega, \Qoppa_A\right) \in \pi_0 \pnbr(\underline{k}) $ an integer $ n $. 
\end{observation}
\begin{proposition}\label{prop:Poincare_cat_as_module_cat} \Lucy{later: add case of schemes with involution}
    Let $ (R, R \to R^{\varphi C_2} \to R^{tC_2}) $ be a Poincaré ring. 
    Let $ \left(\mathcal{C}, \Qoppa\right) $ be an invertible idempotent-complete $ R $-linear Poincaré $ \infty $-category. 
    Suppose given a Poincaré object $ (P, q) $ of $ \left(\mathcal{C}, \Qoppa\right) $ so that $ P $ is a generator for $ \mathcal{C} $.  
    Then $ \left(\mathcal{C}, \Qoppa\right) $ is of the form $ \left(\Mod^\omega_A, \Qoppa_A \right) $ for some Azumaya algebra over $ R $ with genuine involution.  
\end{proposition} 
\begin{corollary}
    Let $ (R, R \to R^{\varphi C_2} \to R^{tC_2}) $ be a Poincaré ring.  
    Let $ \left(\mathcal{C}, \Qoppa\right) $ be an invertible idempotent-complete $ R $-linear Poincaré $ \infty $-category. 
    Then $ \left(\mathcal{C}, \Qoppa\right) $ is of the form $ \left(\Mod^\omega_A, \Qoppa_A \right) $ for some Azumaya algebra over $ R $ with genuine involution.      
\end{corollary}
\begin{proof}
    By Proposition \ref{prop:Poincare_cat_as_module_cat}, it suffices to exhibit a Poincaré object $ (P, q) $ whose underlying object is a generator of $ \mathcal{C} $. Since the underlying $R$-linear stable infinity category $\mathcal{C}$ is dualizable it follows that it is an $R$-linear category which staisfies {\'e}tale hyperdescent by \cite[Example 4.4]{Antieau_Gepner_Bruaer}.
    %\Lucy{Need to show $ \mathcal{C} $ (or $ \mathrm{Ind}(\mathcal{C})$) defines a $ R $-linear category with descent?}
    By \cite[Theorem 6.1]{MR3190610}, $ \mathcal{C} $ has a generator $ G $. 
    Now $ G \oplus D_{\Qoppa}G $ promotes canonically to a Poincaré object of $ \left(\mathcal{C}, \Qoppa\right) $ by \cite[Proposition 2.2.5]{CDHHLMNNSI}. 
\end{proof}
\begin{proof}[Proof of Proposition \ref{prop:Poincare_cat_as_module_cat}]
    Example~\ref{ex:endomorphisms_of_poincare_object} produces from this data an Azumaya algebra with genuine involution $A$. Since $P$ generates $\mathcal{C}$ we then get an equivalence $\mathcal{C}\simeq \mathrm{Mod}_A^\omega$ which by the construction of the genuine involution on $A$ will in fact be an equivalence of $R$-linear Poincar{\'e} infinity categories.
\end{proof}

\section{Stacks associated to Poincaré-\texorpdfstring{$ \infty $}{∞}-categories}

\subsection{Moduli of hermitian objects}
\begin{notation}\label{notation:poincare_ring_basechange}
    Let $ (R, R\to R^{\varphi C_2} \to R^{tC_2}) $ be a Poincaré ring. 
    There is a functor 
    \begin{equation*}
    \begin{split}
        \EE_\infty\Alg^{BC_2}_{R/} &\to \CAlgp_{R/-} \\
        S & \mapsto (S, S \to R^{\varphi C_2} \otimes_{R} S \to S^{t C_2}) =: (R, R \to R^{\varphi C_2} \to R^{tC_2}) \otimes S \,,
    \end{split}    
    \end{equation*}
    where the map $ R^{\varphi C_2} \otimes_R S \to R^{tC_2} \otimes_{R^{tC_2}} S^{tC_2} \simeq S^{tC_2} $ is given by base change along the Tate-valued norm composed with the structure map $ R^{\varphi C_2} \to R^{tC_2} $. 
    Composing the aforementioned functor with the functor that sends a Poincaré ring to its category of compact modules equipped with the canonical Poincaré structure defines a functor
    \begin{equation*}
        \Mod^p \colon \EE_\infty\Alg^{BC_2}_{R/-} \to \EE_\infty\Alg\left(\Catpidem_{R}\right) \,.
    \end{equation*}
\end{notation}
\begin{notation}\label{notation:scheme_involution_basechange}
    Let $ X $ be a scheme with an involution $ \sigma $ and let $ \pi \colon X \to Y $ exhibit $ Y $ as a good quotient of $ X $. 
    If $ j \colon U \to Y $ is flat, let us write $ \pi^*U $ for the tuple $ (X \times_Y U, j^*(\sigma), U, j^*(\pi)) $ of Remark \ref{remark:restriction_of_schemes_with_involution}. 
    Then the assignment $ (j \colon U \to Y ) \mapsto \left(\Mod^\omega_{j^*X}, \Qoppa_{j^*\underline{\mathcal{O}}}\right) $ defines a functor %defined on the category of schemes equipped with an étale map to $ Y $ valued in symmetric monoidal Poincaré $ \infty $-categories over $ \left(\Mod^\omega_X, \Qoppa_{\underline{\mathcal{O}}}\right) $. 
    \begin{equation*}
        \Mod^p \colon \mathrm{\acute{E}t}_Y^\op \to \EE_\infty \Alg\left(\Mod_{\left(\Mod^\omega_X, \Qoppa_{\underline{\mathcal{O}}}\right)}(\Catpidem)\right) \,.
    \end{equation*}
\end{notation}
\begin{observation}
    Let $ R $ be a discrete commutative ring with a $ C_2 $-action, and recall that $ X:= \Spec R \to \Spec(R^{C_2}) = Y $ may be regarded as a $ C_2 $-scheme (Observation \ref{obs:fixpt_Mackey_functor_as_affine_C2_scheme}). 
    Then base change along $ R^{C_2} \to R $ defines a map from étale covers of $ Y $ of Notation \ref{notation:scheme_involution_basechange} to $ C_2 $-equivariant étale covers of $ X $ of Notation \ref{notation:poincare_ring_basechange}. 
    However, not all $ C_2 $-equivariant étale covers of $ X $ arise from this construction: consider the étale cover\Lucy{verify later} associated to the map of rings $ R \xrightarrow{r \mapsto (r, \sigma(r))} R \times R $, where $ R \times R $ is endowed with the flip action $ (r, s) \mapsto (s,r) $. 
\end{observation}
\Lucy{I think Propositions \ref{prop:Poincare_modules_as_etale_sheaf_affine_spectral} and \ref{prop:Poincare_modules_as_etale_sheaf_inv_scheme} have essentially the same proof--present them together?}
%\Lucy{recall étale hypersheaf/hypercovering}
\begin{proposition}
    $ (R, R\to R^{\varphi C_2} \to R^{tC_2}) $ be a Poincaré ring and assume that $ R^{\varphi C_2} $ and $ R $ are connective. Let $ (X, \lambda, Y, \pi) $ be a scheme with involution $ X $ and a good quotient $ Y $. Then:
    \begin{enumerate}
        \item~\label{prop:Poincare_modules_as_etale_sheaf_affine_spectral} The assignment of Notation~\ref{notation:poincare_ring_basechange} is a hypersheaf on the small $C_2$-equivariant étale site of $ R $. \Lucy{Antieau--Gepner write \emph{stack} for `sheaf of categories' and \emph{sheaf} for `sheaf of spaces.' Do we want to use this convention?}
        \item~\label{prop:Poincare_modules_as_etale_sheaf_inv_scheme} The assignment of Notation~\ref{notation:scheme_involution_basechange} is a hypersheaf on the small étale site of $ Y $. 
    \end{enumerate}
\end{proposition}
\begin{proof}
    For now, use the notational shorthand $ R^p = (R, R\to R^{\varphi C_2} \to R^{tC_2}) $. 
    Since limits in categories of algebras and modules are computed at the level of underlying objects, it suffices to show that the functor sends an étale hypercovering $ j_\bullet \colon S \to T^\bullet $ to a limit diagram in $ \Catpidem $.  
    By Proposition 6.1.4 of \cite{CDHHLMNNSI}, it suffices to show that the relevant diagram is a limit diagram in $ \Cathidem $.  
    The proof of Lemma 5.4 in \cite{MR3190610} implies that the diagram defines a limit diagram on underlying $ \infty $-categories. 
    Thus by Remark 6.1.3 of \cite{CDHHLMNNSI}, it suffices to show that $ j_\bullet^* \colon \Mod^\omega_{R^e \otimes_{R^{C_2}} S} \to \Mod^\omega_{R^e \otimes_{R^{C_2}} T^\bullet} $ induces an equivalence $ \Qoppa_{R^p \otimes S} \xrightarrow{\sim} \lim_{\Delta} \Qoppa_{R^p \otimes T^\bullet} \circ \left(j_\bullet^*\right)^{\mathrm{op}} $ of quadratic functors $ \Mod^{\omega,\op}_{R^e \otimes_{R^{C_2}} S} \to \Spectra $. 
    This follows from our assumption on $ S \to T^\bullet $ and \cite[Theorem 3.3.1]{CDHHLMNNSI}. 

    The proof of the second point is similar.
\end{proof}

\begin{corollary}\label{cor:pnbr_as_etale_sheaf_affine_spectral}
    Let $\mathcal{C}$ denote either the category of schemes with good quotients or the category of Poincar{\'e} rings with connective underlying and geometric fixed-points ring. Write $ \pnbr $ for the composite functor $ \mathrm{\acute{E}t}_R \xrightarrow{\Mod^p} \EE_\infty\Alg\left(\Catpidem_{R}\right) \xrightarrow{\pnbr} \Spectra_{\geq 0} $, where $ \Mod^p $ is from Notation \ref{notation:poincare_ring_basechange} or Notation~\ref{notation:scheme_involution_basechange} and $ \pnbr $ is Definition \ref{definition:poincare_brauer_space}. 
    Then $ \pnbr $ is a $C_2$-\'etale hypersheaf. 
\end{corollary} 
\begin{proof}
    The case of Poincar{\'e} rings is handled by the fiber sequence of Corollary~\ref{cor:Poincare_Brauer_to_Brauer_fiber}. Note further that if we know that $\pnbr(-)$ is a $C_2$-\'etale sheaf in the case of schemes with good quotients, then we may reduce the hypersheaf question to Poincar{\'e} rings which we already know. Thus we need only check that the Poincar{\'e} Brauer space is an \'etale sheaf. 

    Since the Picard space commutes with limits, it is enough to know that the functor \[\mathrm{Mod}_{\mathrm{Mod}^p}:\mathrm{\acute{E}t}_Y^\op \to \mathrm{LinCat}^p \] is a $C_2$-\'etale sheaf, where $\mathrm{LinCat}^p$ is the Grothendieck construction on the functor sending a $C_2$-{\'e}tale extension $X\to X'$ to the category $\mathrm{Mod}_{\mathrm{Mod}_{(X',\Qoppa_{\underline{\mathcal{O}}_{X'}}})}(\Catpidem)$. This follows by the same argument as in \cite[Theorem 5.13]{Lurie2011Descent} (note that the argument is formal once one has that the functors of Notation~\ref{notation:poincare_ring_basechange} and Notation~\ref{notation:scheme_involution_basechange} are \'etale sheaves, which we have already handled.)
\end{proof} 

Now that we have \'etale descent we are able to prove Theorem~\ref{thm: comparison to PS}.

\begin{theorem}~\label{thm: PS comparison in text}
    Let $(X,\lambda,Y,\pi)$ be a scheme with good quotient. Suppose further that $\pi:X\to Y$ is \'etale and that $\frac{1}{2}\in \Gamma(\mathcal{O}_Y)$. Then the natural map \[\mathrm{Br}(X,\lambda)\to \pnbr(X,\lambda,Y,\pi)\] is an isomorphism.
\end{theorem}
\begin{proof}
    By \'etale descent we have that $\pnbr(X,\lambda,Y,\pi)$ fits into an exact sequence \[\pic(X)\to \pic(Y)\to \pnbr(X,\lambda,Y,\pi)\to \mathrm{Br}(X)\to \mathrm{Br}(Y)\] compatible with the Parimala-Srinivas exact sequence of \cite[Theorem 1.2]{MR1162189}. 
    \Lucyil{Minor modification needed? If $ \mathrm{Br}(-) $ refers to derived Brauer (`prime'), there is an additional factor of $ H^1_{\acute{e}t}(-;\mathbb{G}_m) $ (see \cite[Corollary 7.14]{MR3190610}). 
    Now, this term goes away if we apply the hypothesis that $ X ,Y $ are normal \cite[Lemma 7.15]{MR3190610}, so we could also add that assumption and call it a day. }
    Hence by the 5-lemma we have an equivalence as desired.
\end{proof}
\begin{notation} [Moduli (pre)sheaf of hermitian objects] \label{ntn:moduli_of_hermitian_objects_affine}
    Let $ (R, R^{\varphi C_2} \to R^{tC_2} ) $ be a Poincar\'e ring and let $ (\mathcal{C}, \Qoppa_{\mathcal{C}}) $ be an $ R $-linear Poincar\'e $ \infty $-category. \Lucy{Do we need Poincar\'e or does it suffice to take $ (\mathcal{C}, \Qoppa_{\mathcal{C}}) $ to be hermitian here?} 
    Define a presheaf $ \mathbf{M}_{\left(\mathcal{C},\Qoppa_{\mathcal{C}}\right)}^h \colon \left(\EE_\infty \Alg_R^{BC_2}\right) \to \Spaces $ whose value at $ R \to S $ is the $ \infty $-groupoid of $ R $-linear hermitian functors from $ (\mathcal{C}, \Qoppa_{\mathcal{C}}) $ to $ \left(\Mod_R^\omega, \Qoppa_R \right) \otimes_R S $. 
    \Lucyil{Antieau--Gepner consider the \emph{$ \infty $-category} of functors, then pass to the maximal sub-groupoid \cite[\S5.2]{MR3190610}. 
    Do we need to consider an enhancement of hermitian functors to $ \infty $-categories?}
    Now suppose $ A $ is an $ \EE_1 $-$ R $-algebra with genuine involution. 
    Then we will write $ \mathbf{M}^h_A $ for $ \mathbf{M}^h_{\left(\Mod_A^\omega, \Qoppa_A\right)} $. 
\end{notation}
\begin{observation}
    Let $ (R, R^{\varphi C_2} \to R^{tC_2} ) $ be a Poincar\'e ring and suppose $ A $ is an $ \EE_1 $-$ R $-algebra with genuine involution. 
    A point of $ \mathbf{M}^h_A(S) $ is classified by pair $ (F,\eta ) $ where $ F $ is an $ R $-linear functor $ F \colon \Mod_A^\omega \to S $ and $ \eta $ is a natural transformation $ \eta \colon \Qoppa_A \to \Qoppa_S \circ F^\op $ of $ R $-linear hermitian structures. 
    Now $ F $ is classified by an $ A \otimes_R S $-module $ P $ which is compact as an $ S $-module (cf. \cite[Proposition 5.10]{MR3190610}), and $ \eta $ is classified by a map of $ N^{C_2}_S (A \otimes_R S) $-modules $ A \otimes_R S \to N^{C_2}_S(P^\vee) $ (Proposition \ref{prop:relative_poincare_cats_basic_properties}\ref{propitem:classify_R_lin_hermitian_struct}). 
    Thus $ \mathbf{M}^h_A(S) $ may be identified with the full sub-$ \infty $-groupoid of the pullback of the diagram
    \begin{equation*}
    \begin{tikzcd}
        & \Mod_{A \otimes_R S}\left(\Spectra\right) \ar[d,"{N_S^{C_2}}"] \\
        \Mod_{N^{C_2}_S(A \otimes_R S)}\left(\Spectra^{C_2}\right)_{A \otimes_R S/} \ar[r] &  \Mod_{N^{C_2}_S(A \otimes_R S)}\left(\Spectra^{C_2}\right)   \,.
    \end{tikzcd}    
    \end{equation*}
\end{observation}
\begin{lemma}\label{lemma:moduli_of_hermitian_objects_is_etale_sheaf}
    Let $ (R, R^{\varphi C_2} \to R^{tC_2} ) $ be a Poincar\'e ring and let $ (\mathcal{C}, \Qoppa_{\mathcal{C}}) $ be an $ R $-linear Poincar\'e $ \infty $-category. 
    Suppose that $ R $ is connective. \Lucy{maybe want $ R^{\varphi C_2} $ to be connective too?}
    Then the assignment $ \mathbf{M}^h_{(\mathcal{C}, \Qoppa_{\mathcal{C}})} $ of Notation \ref{ntn:moduli_of_hermitian_objects_affine} satisfies \'etale hyperdescent.      
\end{lemma}
\begin{notation} [Moduli (pre)sheaf of hermitian objects] \label{ntn:moduli_of_hermitian_objects_global}
    Let $ (X,\sigma, Y,\pi) $ be a scheme with involution and good quotient and let $ (\mathcal{C}, \Qoppa_{\mathcal{C}}) $ be an $ \left(\Mod_X^\omega, \Qoppa_{\underline{O}}\right) $-linear Poincar\'e $ \infty $-category.  
    Define a presheaf $ \mathbf{M}_{\left(\mathcal{C},\Qoppa_{\mathcal{C}}\right)}^h \colon \mathrm{\acute{E}t}_Y^\op \to \Spaces $ whose value at $ \Spec S \to Y $ is the $ \infty $-groupoid of $ \left(\Mod_X^\omega, \Qoppa_{\underline{O}}\right) $-linear hermitian functors from $ (\mathcal{C}, \Qoppa_{\mathcal{C}}) $ to $ \left(\Mod_{X_S}^\omega, \Qoppa_{\underline{O}|_S}\right) $. 
    \Lucyil{same comment about \emph{$ \infty $-category} of functors vs $ \infty $-groupoid.}
    Now suppose $ A $ is an Azumaya algebra over $ X $ with genuine $ \sigma $-linear (anti-)involution. 
    Then we will write $ \mathbf{M}^h_A $ for $ \mathbf{M}^h_{\left(\Mod_A^\omega, \Qoppa_A\right)} $. 
\end{notation}
\begin{lemma}
    Let $ (X,\sigma, Y,\pi) $ be a scheme with involution and good quotient and let $ (\mathcal{C}, \Qoppa_{\mathcal{C}}) $ be an $ \left(\Mod_X^\omega, \Qoppa_{\underline{O}}\right) $-linear Poincar\'e $ \infty $-category. 
    Then the functor $ \mathbf{M}_{\left(\mathcal{C},\Qoppa_{\mathcal{C}}\right)}^h $ defines a hypersheaf on the small \'etale site of $ Y $. 
\end{lemma}
\begin{observation}\label{obs:from_moduli_of_hermitian_objects_to_moduli_of_objects}
    Let $ (X,\sigma, Y,\pi) $ be a scheme with involution and good quotient and let $ A $ be an Azumaya algebra over $ X $ with genuine $ \sigma $-linear (anti-)involution. 
    Recall that there is a sheaf $ \mathbf{M}_{A^e} $ on the small \'etale site of $ X $ \cites[\S5.2]{MR3190610}. \Lucy{cite To\"en? \S2.2?} 
    The forgetful functor $ \Cath_R \to \Catex_R $ induces a morphism of (pre)stacks $ \mathbf{M}^h_A \to \pi_* \mathbf{M}_{A^e} $ on the small \'etale site of $ Y $, where $ \mathbf{M}^h_A $ is from Notation \ref{ntn:moduli_of_hermitian_objects_global}. 
    Given a point $ \Spec S \to Y $, the map  
    \begin{equation*}
    \begin{tikzcd}[row sep=small]
        \mathbf{M}^h_A(\Spec S) \ar[d,phantom,"{\rotatebox[origin=c]{270}{$\simeq$}}"] \ar[r] & \pi_*\mathbf{M}^h_{A^e}(\Spec S) \ar[d,phantom,"{\rotatebox[origin=c]{270}{$\simeq$}}"] \\
        \hom_{\left(\Mod_X^\omega, \Qoppa_{\underline{O}}\right)-\Mod\left( \Cath\right)}\left(\left(\Mod_A^\omega, \Qoppa_A \right), \left(\Mod_{X_S}^\omega, \Qoppa_{\underline{O}|_S}\right)\right) & \hom_{\left(\Mod_X^\omega\right)-\Mod\left( \Catex\right)}\left(\Mod_A^\omega, \Mod_{X_S}^\omega\right) 
    \end{tikzcd}
    \end{equation*}
    sends a hermitian functor $ (F, \eta) $ to its underlying $ \mathcal{O}_X $-linear functor $ F $. 
\end{observation}
\Lucyil{I think we can combine Lemma \ref{lemma:pushforward_moduli_of_objects_remains_geometric}, Lemma \ref{lemma:hermitian_moduli_to_ordinary_moduli_is_geometric}, and \cite[Lemma 4.25]{MR3190610} to show that $ \mathbf{M}^h_A $ is locally geometric.} 
\begin{recollection}\label{recollection:quotient_maps_are_quasicompact}
    An affine morphism of schemes is always quasi-compact \cite[\href{https://stacks.math.columbia.edu/tag/01S5}{Tag 01S5}]{stacks}. 
    In particular, if $ (X, \sigma, Y, \pi) $ is a scheme with involution and good quotient, then $ \pi $ is quasi-compact.    
\end{recollection}
\begin{lemma}\label{lemma:pushforward_moduli_of_objects_remains_geometric}
    Let $ (X,\sigma, Y,\pi) $ be a scheme with involution and good quotient and let $ A $ be an Azumaya algebra over $ X $ with genuine $ \sigma $-linear (anti-)involution. 
    Assume that $ Y $ is quasi-compact. 
    Then the sheaf $ \pi_* \mathbf{M}_{A^e} $ is locally geometric. 
    \Lucyil{Other desired properties: locally of finite presentation, smoothness. Possibly need to impose different assumptions on $ \pi $ for different properties to be preserved.}
\end{lemma}
\begin{proof}
    Recall \cite[Theorem 5.8]{MR3190610} that the sheaf $ \mathbf{M}_{A^e} $ is locally geometric, i.e. there exists a filtered system $ \mathbf{M}_{A^e} = \colim_{[a,b]} \mathbf{M}_{A^e}^{[a,b]} $ where each $ \mathbf{M}_{A^e}^{[a,b]} \to \mathbf{M}_{A^e} $ is a monomorphism and each $ \mathbf{M}_{A^e}^{[a,b]} $ is $ n_i $-geometric for some $ n_i $. 

    While $ \pi_* $ does not preserve filtered colimits in general, quasi-compactness of $ \pi $ (Recollection \ref{recollection:quotient_maps_are_quasicompact}) and of $ Y $ implies that $ \colim_{[a,b]} \pi_* \mathbf{M}_{A^e}^{[a,b]} \to \pi_*\mathbf{M}_{A^e} $ is an equivalence \cite[\S3.1.2]{Thomason_Trobaugh}. 

    Since pushforward is left exact (and assuming that $ \pi_* $ preserves this particular filtered colimit), it suffices to show that the pushforward of an $ n $-geometric sheaf on $ X $ is $ m $-geometric for some $ m $. 
    \Lucyil{Incomplete! Bootstrap Lemma \ref{lemma:pushforward_rep'able_n_geometric_sheaves} and Lemma \ref{lemma:pushforward_of_n_geometric_morphisms} to show this? 

    % NOTE: prove as little as necessary! Sufficient to show that at every step in the proof of local geometricity of $ \mathbf{M}_{A^e} $, $ \pi_* $ preserves the relevant properties. 
    Note: By definition of exact quotient and \cite[Theorem 4.35(i)]{azumaya_involution} and \cite[Proposition 7.2.1.14]{LurHA}, $ \pi_* $ preserves epimorphisms. Therefore, to show that $ \pi_* $ sends submersions (=smooth and surjective) to submersions, it suffices to show that $ \pi_* $ sends smooth maps to smooth maps.}
\end{proof}
\begin{lemma}\label{lemma:pushforward_rep'able_n_geometric_sheaves}
    Let $ R \to S $ be a map of connective $ \EE_\infty $-rings and write $ \pi \colon \Spec S \to \Spec R $. 
    Assume that 
    \begin{itemize}
        \item The map $ R \to S $ is an effective descent morphism, and 
        \item $ S $ is dualizable as an $ R $-module; write $ S^\vee $ for the $ R $-linear dual of $ S $, and suppose it has Tor-amplitude contained in $ [d,c] $ for some $ d \leq 0 $. 
    \end{itemize}  
    Then: 
    \begin{enumerate}[label=(\arabic*)]
        \item \label{lemma_item:pushforward_free_alg} Let $ M $ be a perfect $ S $-module and consider the sheaf $ \Spec \mathrm{Sym}_S(M) $ on the \'etale site of $ S $. 
        Assume that $ M $ has Tor-amplitude in $ [a,b] $ (so the stack $ \Spec \mathrm{Sym}_S(M) $ is $ \mathrm{max}(-a,0) $-geometric by \cite[Theorem 5.2]{MR3190610}). 
        Then $ \pi_* \Spec \mathrm{Sym}_S(M) $ is a quasiseparated $ (\mathrm{max}(-a,0) - d) $-geometric sheaf on the \'etale site of $ R $. 
        \Lucyil{pretty sure that if $ S $ is flat over $ R $ and $ b \leq 0 $, then $ \pi_* \Spec \mathrm{Sym}_S(M) $ is furthermore smooth. Check later!}
        
        \item \label{lemma_item:pushforward_pres_loc_fin_pres} Let $ T $ be a connective $ S $-algebra which is locally of finite presentation. 
        Then $ \pi_* \Spec T $ is a $ (-d) $-geometric sheaf on the \'etale site of $ R $ which is locally of finite presentation. 
    \end{enumerate}
    % Commented this out because it is trivial (but still true!): $ \pi_* $ is a right adjoint, hence preserves all limits and the final object is the limit of the empty diagram. 
    % ------------------------------------------------------------------------
    % In particular, $ \pi_* 1_{\mathrm{Shv}_{\acute{e}t}(S;\Spaces)} $ is $ (-d) $-geometric. 
\end{lemma}
\begin{proof}
    We begin with the proof of part \ref{lemma_item:pushforward_free_alg}. 
    Using the morphisms $ M \to M \otimes_R S \to M $ \Lucy{that these are morphisms of $ S $-modules requires that $ S $ is $ \EE_\infty $.}, we know that $ \Spec \mathrm{Sym}_S (M) $ is a retract of $ \Spec \mathrm{Sym}_S(M \otimes_R S) $, hence $ \pi_* \Spec \mathrm{Sym}_S (M) $ is a retract of $ \pi_* \Spec \mathrm{Sym}_S(M \otimes_R S) $. 
    Since $ S $ is a perfect $ R $-module and $ M $ is a perfect $ S $-module, $ M $ is also perfect when regarded as an $ R $-module. 
    % --------------------------------------------------
    % Here is the previous argument (which I replaced June 8th with the preceding 2 sentences): 
    % Since $ R \to S $ is an effective descent morphism, we have an equivalence of $ S $-modules $ M \simeq \colim_{\Delta^\op} M \otimes_R S^{\otimes_R \bullet + 1} $. 
    % Since $ \mathrm{Sym}_S $ preserves sifted colimits and $ \Spec $ is contravariant, it follows that $ \Spec \mathrm{Sym}_S (M) \simeq \lim_{\Delta} \Spec \mathrm{Sym}_S \left(M \otimes_R S^{\otimes_R \bullet + 1} \right) $ as \'etale sheaves over $ \Spec S $. 
    % Since $ \pi_* $ preserves all small limits, it follows that $ \pi_* \Spec \mathrm{Sym}_S (M) \simeq \lim_{\Delta} \pi_* \Spec \mathrm{Sym}_S \left(M \otimes_R S^{\otimes_R \bullet + 1} \right) $ as \'etale sheaves over $ \Spec R $. 
    % Since $ S $ is a perfect $ R $-module and $ M $ is a perfect $ S $-module, each $ M \otimes_R S^{\otimes_R \bullet} $ is a perfect $ R $-module whose tor-amplitude (over $ R $) is bounded below \emph{independently of $ \bullet$}. 
    % The earlier argument used \cite[Lemma 4.36]{MR3190610}.
    % -------------------------------------------------------
    By \cite[Lemma 4.38]{MR3190610}, it suffices to show that if $ N $ is a perfect $ R $-module, then
    \begin{enumerate}[label=(\arabic*)]
        \item the sheaf $ \pi_* \Spec \mathrm{Sym}_S \left(N \otimes_R S \right) $ is quasiseparated 
        \item If $ N $ has tor-amplitude contained in $ [e, f] $ with $ e \leq 0 $, then $ \pi_* \Spec \mathrm{Sym}_S \left(N \otimes_R S \right) $ is $ (e - d) $-geometric. 
    \end{enumerate}
    Now if $ T $ is an $ R $-algebra, 
    \begin{equation}\label{eq:pushforward_free_alg_formula}
    \begin{split}       
         \pi_* \Spec \mathrm{Sym}_S \left(N \otimes_R S \right)(\Spec T) & \simeq \hom_{S\mathrm{Alg}}\left(\mathrm{Sym}_S \left(N \otimes_R S \right), T \otimes_R S \right) \\
         &\simeq \hom_{S\Mod}(N \otimes_R S, T \otimes_R S) \\
         &\simeq \hom_{R\Mod}(N, T \otimes_R S) \\
         &\simeq \hom_{R \Mod }(N \otimes_R S^\vee, T)\qquad \text{ since $ S $ is dualizable over $ R $} \\
         &\simeq \hom_{R\mathrm{Alg}}\left(\mathrm{Sym}_R (N \otimes_R S^\vee), T\right)\,.
    \end{split}
    \end{equation}
    Since $ N \otimes_R S^\vee $ has Tor-amplitude\Lucy{\cite[Proposition 2.13(4)]{MR3190610} or find ref in [TT90] later} in $ [e+d, f+c] $, it follows that $ \pi_* \Spec \mathrm{Sym}_S \left(N \otimes_R S \right) $ is quasiseparated and $ (e-d) $-geometric by \cite[Theorem 5.2]{MR3190610}. 

    % The final statement follows from taking $ M = 0 $. 
    We now prove part \ref{lemma_item:pushforward_pres_loc_fin_pres}. 
    By \cites[Proposition 7.2.4.27(b) \& (c)]{LurHA}[Lemma 5.4.2.4]{HTT}, $ \Spec T $ is a retract of a finite limit of sheaves of the form $ \Spec \mathrm{Sym}_S (M) $ where $ M $ is a compact connective $ S $-module. 
    By \ref{lemma_item:pushforward_free_alg}, each $ \pi_* \Spec \mathrm{Sym}_S(M) $ is quasi-separated and $ n $-geometric for some $ n $. 
    Since the collection of quasi-separated $ n $-geometric sheaves which are locally of finite presentation is closed under finite limits and retracts by Lemmas 4.35 and 4.38, resp. of \cite{MR3190610} and $ \pi_* $ preserves arbitrary limits (in particular finite limits and retracts), it suffices to show that $ \pi_* \Spec \mathrm{Sym}_S(M) $ where $ M $ is a compact connective $ S $-module is locally of finite presentation as an $ n $-geometric sheaf on the \'etale site of $ R $. 
    Since $ \pi_* \Spec \mathrm{Sym}_S(M) $ is a retract of $ \pi_* \Spec \mathrm{Sym}_S(M \otimes_R S) $ by the proof of part \ref{lemma_item:pushforward_free_alg}, by \cite[Lemma 4.38]{MR3190610}, it suffices to show that the latter is locally of finite presentation. 
    The result now follows from \cite[Theorem 5.2]{MR3190610} and (\ref{eq:pushforward_free_alg_formula}). 
\end{proof}
\begin{lemma}\label{lemma:pushforward_of_n_geometric_morphisms}
    Let $ R $ be a discrete ring with involution and consider $ \pi \colon \Spec R \to \Spec (R^{C_2}) $ as a scheme with involution and good quotient. 
    Suppose that $ \pi_* \colon \mathrm{Shv}_{\acute{e}t}(\Spec R; \Spaces) \to \mathrm{Shv}_{\acute{e}t}(\Spec (R^{C_2}); \Spaces) $ sends disjoint unions of affines to disjoint unions of affines and sends smooth maps to smooth maps. 
    \Lucyil{Can we relax/get rid of the assumption that $ \pi_* $ preserves disjoint unions of affines by imposing that $ f $ quasicompact and other finiteness conditions?}
    Then if $ f \colon X \to Y $ is an $ n $-geometric morphism in $ \mathrm{Shv}_{\acute{e}t}(\Spec R; \Spaces) $, then $ \pi_*(f) $ is $ m $-geometric for some $ m $.  
\end{lemma}
\begin{proof}
    We will induct on $ n $. 
    Let $ S $ be a connective $ R^{C_2} $-algebra and fix a map $ p \colon \Spec S \to \pi_* Y $ of \'etale sheaves over $ R^{C_2} $. 
    The map $ p $ admits a canonical factorization $$ \Spec S \to \pi_* \pi^* \Spec S \xrightarrow{\pi_*(p^\dagger)} \pi_*Y \,, $$ where $ p^\dagger$ corresponds to $ p $ under the $ (\pi^*, \pi_*) $-adjunction. 
    Furthermore note that $ \pi^* \Spec S \simeq \Spec (S \otimes_{R^{C_2}} R ) $. 

    Base case: If $ f $ is $ 0 $-geometric, $ (\pi^* \Spec S) \fiberproduct_{Y} X \simeq \bigsqcup_{i \in I} \Spec T_i $ for some connective commutative $ R $-algebras $ T_i $. \Lucy{if $ f $ is quasi-compact, can take $ I $ to be \emph{finite}.} 
    \Lucyil{very much a work in progress/incomplete!!! To show that $ \pi_* f$ is $ d $-geometric, it suffices to show that $ \Spec S \fiberproduct_{\pi_* \pi^* \Spec S}\left(\bigsqcup_i \Spec T_i \right) $ admits a smooth surjective $ (d-1) $-geometric morphism from a disjoint union of affines. Impose more assumptions and try to reduce to Lemma \ref{lemma:pushforward_rep'able_n_geometric_sheaves}?} 

    Inductive step: Since $ f $ is $ n $-geometric, there exists a smooth $ (n-1) $-geometric surjection $ \varphi \colon U \simeq \bigsqcup_{i \in I} \Spec T_i \to (\pi^* \Spec S) \fiberproduct_{Y} X $. 
    Now form the following diagram in which all squares are pullbacks
    \begin{equation*}
    \begin{tikzcd}
        V\ar[d,"\psi"] \ar[r] & \pi_* U \ar[d,"{\pi_*(\varphi)}"] & \\
        {\Spec S \fiberproduct_{\pi_* Y} \pi_* X} \ar[r] \ar[d] & {\pi_*\left( (\pi^* \Spec S) \fiberproduct_Y X\right)} \ar[d] \ar[r] & \pi_* X \ar[d,"{\pi_*(f)}"] \\
        \Spec S \ar[r] & \pi_*\pi^*(\Spec S) \ar[r] & \pi_* Y    \,.
    \end{tikzcd}    
    \end{equation*}
    By inductive hypothesis, $ \pi_*(\varphi) $ is a smooth $ m $-geometric surjection for some $ m = m(n-1) $ depending only on $ (n-1) $. \Lucy{tried to use function notation but it looks like multiplication.}
    It follows from \cite[Lemma 4.23]{MR3190610} that $ \psi $ is a smooth $ m $-geometric surjection. 
    By assumption, $ \pi_* U $ is a disjoint union of affines. 
    Recall that $ \mathrm{Shv}_{\acute{e}t}(\Spec (R^{C_2}); \Spaces) $ is an $ \infty $-topos, hence colimits are universal. 
    \Lucyil{Need to show that $ V $ is a disjoint union of affines or admits a smooth $ m $-geometric surjection from disjoint union of affines (then apply \cite[Lemma 4.25(1)]{MR3190610})! 
    
    By assumption that $ \pi_* $ preserves arbitrary disjoint union, suffices to prove the result for $ U = \Spec T $ (so $V = \Spec S \fiberproduct_{\pi_*\pi^* \Spec S} \pi_* \Spec T $). Impose more assumptions and try to reduce to Lemma \ref{lemma:pushforward_rep'able_n_geometric_sheaves}?}

    We have shown that there exists a smooth surjective $ m $-geometric morphism to $ \Spec S \fiberproduct_{\pi_* Y} \pi_* X $ from a disjoint union of affines, hence $ \pi_*(f) $ is $ (m+1) $-geometric. 
\end{proof}
\begin{lemma}\label{lemma:hermitian_moduli_to_ordinary_moduli_is_geometric}
    Let $ (X,\sigma, Y,\pi) $ be a scheme with involution and good quotient and let $ A $ be an Azumaya algebra over $ X $ with genuine $ \sigma $-linear (anti-)involution. 
    Then the morphism $ \mathbf{M}^h_A \to \pi_* \mathbf{M}_{A^e} $ of Observation \ref{obs:from_moduli_of_hermitian_objects_to_moduli_of_objects} is locally geometric in the sense of \cite[\S4.3]{MR3190610}. 
\end{lemma}
\begin{proof}
    It suffices to show that for all $ p \colon \Spec S \to \pi_* \mathbf{M}_{A^e} $, the fiber product $ \Spec S \fiberproduct_{\pi_* \mathbf{M}_{A^e}} \mathbf{M}^h_{A} $ is locally geometric over $ \Spec S $. 
    Write $ \overline{p} $ for the composite $ \Spec S \to \pi_* \mathbf{M}_{A^e} \to Y $; there is a pullback diagram 
    \begin{equation*}
    \begin{tikzcd}
        \Spec R \ar[d] \ar[r,"{\widetilde{p}}"] \ar[rd,phantom,"\lrcorner", very near start] & X \ar[d,"\pi"] \\
        \Spec S \ar[r,"{\overline{p}}"] & Y 
    \end{tikzcd}
    \end{equation*}
    where the pullback $ \Spec R $ is affine by assumption. 
    Recall that the map $ p $ classifies an $ \widetilde{p}^*A $-module $ M $ which is compact as an $ R $-module. 
    Note that the fiber product $ \Spec S \fiberproduct_{\pi_* \mathbf{M}_{A^e}} \mathbf{M}^h_{A} $classifies lifts of the $ R $-linear functor $ \otimes M \colon \Mod_{\widetilde{p}^*A}^\omega \to \Mod^\omega_R $ to an $ R $-linear hermitian functor. 
    A lift of $ \otimes M $ to an $ R $-linear hermitian functor is equivalent to the data of a map $ \widetilde{p}^*A \to N^{C_2}_R(M^\vee) $ of $ N_R^{C_2}(\widetilde{p}^*A) $-modules in $ \Spectra^{C_2} $, where we write $ M^\vee $ for the $ R $-linear dual of $ M $. 
    Thus the fiber product $ \Spec S \fiberproduct_{\pi_* \mathbf{M}_{A^e}} \mathbf{M}^h_{A} $ is given by the moduli sheaf $ \Spec \mathrm{Sym}_S \tau_{\geq 0} \hom_{N_R^{C_2}(\widetilde{p}^*A)}\left(\widetilde{p}^*A, N^{C_2}_R(M^\vee)\right) $. 
    \Lucy{Similar to \cite[Theorem 5.6, Proposition 5.7]{MR3190610}.}
\end{proof}
\begin{notation}\label{ntn:subsheaf_Poincare_Morita_equiv}
    Let $ (X,\sigma, Y,\pi) $ be a scheme with involution and good quotient and let $ A $ be an Azumaya algebra over $ X $ with genuine $ \sigma $-linear (anti-)involution; write $ \lambda \colon A \to \sigma^* A^\op $ for the involution on $ A $. 
    Write $ \mathbf{Mor}^p_{A} \to \mathbf{M}^h_{A} $ for the subsheaf of \emph{Poincaré Morita equivalences}, i.e. $ \mathbf{Mor}^h_{A}(\Spec S \to Y) $ is the full sub-groupoid of $ \mathbf{M}^h_{A}(\Spec S \to Y) $ on those $ \underline{\mathcal{O}} $-linear hermitian functors $ (F,\eta) \colon \left(\Mod^\omega_{A}, \Qoppa_A\right) \to \left(\Mod_{\underline{\mathcal{O}}|_{\Spec S}}^\omega, \Qoppa_{\underline{\mathcal{O}}}|_{\Spec S}\right) $ so that $ F $ is an equivalence and $ (F,\eta) $ is duality-preserving.    
\end{notation}
\begin{lemma}\label{lemma:poincare_morita_eq_is_open_subsheaf}
    Suppose given the setup of Notation \ref{ntn:subsheaf_Poincare_Morita_equiv}. 
    Supppose that the quotient map $ \pi \colon X \to Y $ is \emph{quasi-perfect} in the sense of \cite[Definition A.6.3]{CHN2024}, i.e. the pushforward $ \pi_* $ of quasicoherent sheaves preserves perfect complexes. 
    Suppose further that $ \pi $ is flat. 
    Then the inclusion $ \mathbf{Mor}^p_{A} \to \mathbf{M}^h_{A} $ exhibits $ \mathbf{Mor}^h_{A} $ as a quasicompact open subsheaf of $ \mathbf{M}^h_{A} $. 
    % \Lucyil{To loosen the assumption, need to know that the image of a [$C_2$-invariant?] quasicompact Zariski open $ U $ of $ \Spec R $ under the quotient map $ \pi \colon \Spec R \to \Spec S $ is a quasicompact Zariski open of $ \Spec S $. 
    % Revisit later and peruse \href{https://math.stackexchange.com/questions/83631/quotient-of-an-affine-variety-by-a-finite-group}{this}. 
    % Better yet: If $ \sigma \colon R \to R $ is the action and $ U = \cup_{i=1}^n D(f_i) $ for $ f_i \in R$, then $ \pi(U) = \cup_{i=1}^n (f_i \sigma(f_i)) $?}         
\end{lemma}
\begin{remark}
    If either $ \pi \colon X \to Y $ is an isomorphism or quadratic étale on each connected component, then $ \pi_* $ satisfies the assumptions of Lemma \ref{lemma:poincare_morita_eq_is_open_subsheaf}. 

    Here is an example which doesn't fall under either of the preceding cases: If $ X = \Spec \CC [x] $ with the action $ x \mapsto -x $, then $ Y = \Spec \CC[x^2] $ and the quotient map $ X \to Y $ is quasi-perfect because $ \CC[x] $ is a free module of finite rank over $ \CC[x^2] $. 
    In general, one may check quasi-perfection Zariski-locally on the quotient \cite[Corollary A.6.5]{CHN2024}. 
    However, not all schemes with involution and good quotient are quasi-perfect; for a counterexample, see \cite[Example 3.9]{azumaya_involution}. 
    \Lucy{general conditions for this to hold? References: \href{https://arxiv.org/abs/1207.3648}{IV.Proposition 2.2.3} and \href{https://arxiv.org/abs/1505.00754}{\S4.3 here}.}    
\end{remark}
\begin{proof} [Proof of Lemma \ref{lemma:poincare_morita_eq_is_open_subsheaf}]
    Use the notation of Lemma \ref{lemma:hermitian_moduli_to_ordinary_moduli_is_geometric}. 
    By the proof of Lemma \ref{lemma:hermitian_moduli_to_ordinary_moduli_is_geometric}, we know that $ (F,\eta) $ is given by $ (-\otimes_R M, \widetilde{p}^*A \to N^{C_2}_R(M^\vee)) $ where $ M $ is a compact $ R $-module. 
    By quasiperfection of $ \pi_* $ and flatness of $ \pi $, $ \Spec R \to \Spec S $ is quasiperfect, i.e. $ R $ is a compact $ S $-module. 
    It follows that $ M $ is a compact $ S $-module. 
    \Lucyil{quasiperfection is not preserved under arbitrary base change, only tor-independent base change (which is guaranteed under flatness assumption, see \href{https://arxiv.org/pdf/math/0611760}{p.5 here} or \cite[Lemma A.6.4]{CHN2024} which is a special case of the former). OTOH, if we only work with the small \'etale site of $ Y $, then all $ \Spec S \to Y $ would be flat and we can remove flatness assumption on $ \pi $ (but still need quasi-perfection).} 
    By the proof of \cite[Proposition 5.10]{MR3190610}, the subsheaf of points of $ \Spec S $ on which the functor $ F $ is an equivalence is a quasicompact Zariski open $ U \subseteq \Spec S $. 
    % \Lucy{Actually, I think I need to know that $ (\widetilde{p}^*A)^{C_2} $ and $ (N^{C_2}_R(M^\vee))^{C_2} $ are compact $ S $-modules.}
    The hermitian functor $ (F,\eta) $ is duality-preserving if and only if the canonical map $ \tau_{\eta} \colon M \otimes_{\widetilde{p}^*A} \lambda^* \widetilde{p}^*A \to \hom_{R}(M, \sigma^*R) $ is an equivalence (cf. \cite[Lemma 3.4.3]{CDHHLMNNSI}). 
    By \cite[Proposition 2.14]{MR3190610} (compare the proof of Proposition 5.10 of \emph{loc.cit.}), the subsheaf of points of $ \Spec S $ on which the map $ \tau_\eta $ is an equivalence is a quasicompact Zariski open $ V \subseteq \Spec S $. 

    Finally $ (F,\eta) $ defines a Poincar\'e Morita equivalence if, in addition to $ F $ being an equivalence and $ \eta $ being duality-preserving, $ \eta $ is an equivalence of $ N^{C_2}_R(\widetilde{p}^*A) $-modules. 
    Since categorical fixed points are jointly conservative, $ \eta $ is an equivalence if and only if $ \eta^{C_2} $ and $ \eta^e $ are equivalences. 
    By Lemma \ref{lemma:norm_preserves_compactness}, $ (\widetilde{p}^*A)^{C_2} $ and $ (N^{C_2}_R(M^\vee))^{C_2} $ are compact $ S $-modules, hence by a similar argument to the above, there exists a quasicompact Zariski open $ W \subseteq \Spec S $ on whicn $ \eta $ is an equivalence. 
    Taking the intersection $ U \cap V \cap W $ gives the quasicompact Zariski open subsheaf of $ \Spec S $ on which $ (F,\eta ) $ defines a Poincaré Morita equivalence. 
\end{proof}
\begin{lemma}\label{lemma:norm_preserves_compactness}
    Let $ R = (R^{C_2} \to R^e) $ be a $ C_2 $-$ \EE_\infty $-ring in $ C_2 $-spectra and suppose that $ R $ is connective (i.e. $ R^e $ and $ R^{C_2} $ are both connective). 
    Then the composite $ \Mod_{R^e}\left(\Spectra\right) \xrightarrow{N^{C_2}_R} \Mod_R\left(\Spectra^{C_2}\right) \xrightarrow{(-)^{C_2}} \Mod_{R^{C_2}}\left(\Spectra\right) $ sends compact $ R^e $-modules to compact $ R^{C_2} $-modules if and only if the restriction map $ R^{C_2} \to R^e $ exhibits $ R^e $ as a perfect $ R^{C_2} $-module. 
    Here $ N^{C_2}_R $ denotes the relative norm. 
\end{lemma}
\begin{proof}
    Observe that the relative norm satisfies $ N^{C_2}_R(P \oplus Q) \simeq N^{C_2}_R(P) \oplus N^{C_2}_R(Q) \oplus C_2 \otimes (P \oplus Q) $ for all $ P, Q \in \Mod_{R^e} $, where $ C_2 \otimes - \colon \Mod_{R^e} \to \Mod_{R} $ is the left adjoint to the restriction functor. \Lucyil{This is essentially implied by \cites[Example 3.17 \& Corollary 3.28]{Nardinthesis} but we should find an earlier reference if possible (Maybe \S~A.3.3 of Hill--Hopkins--Ravenel?). Same for the other references in this proof. }

    To prove the `only if' direction, observe that
    \begin{equation*}
        \left(N^{C_2}_R(R^e \oplus R^e)\right)^{C_2} = \left(N^{C_2}_R(R^e) \oplus N^{C_2}_R(R^e) \oplus C_2 \otimes (R^e \oplus R^e)\right)^{C_2} \simeq R^{C_2} \oplus R^{C_2} \oplus R^e \oplus R^e \,.
    \end{equation*}
    Since perfect $ R^{C_2} $-modules contain $ R^{C_2} $ and are closed under cofibers and taking summands, this implies that $ R^{e} $ must be a perfect $ R^{C_2} $-module. 

    Note that the above argument also implies that if $ R^e $ is a perfect $ R^{C_2} $-module, then the exact functor $ (-)^{C_2} \colon \Mod_R\left(\Spectra^{C_2}\right) \to \Mod_{R^{C_2}}\left(\Spectra\right) $ sends perfect $ R $-modules (which are generated as a thick subcategory by $ R $ and $ C_2 \otimes R^e $) to perfect $ R^{C_2} $-modules. 
    To prove the `only if' direction, it suffices to show that if the restriction map $ R^{C_2} \to R^e $ exhibits $ R^e $ as a perfect $ R^{C_2} $-module, then $ N^{C_2}_R $ sends perfect $ R^e $-modules to perfect $ R $-modules in $ \Spectra^{C_2} $. 
    Recall that any perfect $ R^e $-module $ P $ can be written as a finite extension of $ \left\{\Sigma^n R^e \right\}_{n \in \ZZ} $; write $ \ell(P) $ for the minimum $ \ell $ so that $ P $ can be written as an extension of $ \Sigma^{n_i}R^e $ for some $ n_1, \ldots, n_\ell \in \ZZ $. 
    The result follows from induction on $ \ell(P) $ and the following observations:  
    \begin{itemize} 
        \item $ N^{C_2}_R(\Sigma^a R^{e}) = \Sigma^{\rho a} N^{C_2}_R(R^e) = \Sigma^{\rho a} R $ is a perfect $ R $-module  
        \item If $ Q, S $ are perfect $ R^e $-modules so that $ N^{C_2}_R(Q) $, $ N^{C_2}_R(S) $ are perfect $ R $-modules, then $ N^{C_2}_R(Q \oplus S) $ is a perfect $ R $-module. 
        \item If $ \ell(P) > 1 $, then there exists a perfect $ R^e $-module $ Q $ with $ \ell(Q) < \ell(P) $ and an exact sequence of $ R^e $-modules $ P \to Q \to \Sigma^a R^e $ for some $ a \in \ZZ $. 
        \item $ N^{C_2}_R(-) $ is a quadratic functor and the following cube
        \begin{equation*}
        \begin{tikzcd}[row sep=tiny, column sep=tiny]
            & Q \ar[rr] \ar[dd] & & Q \oplus \Sigma^a R^e \ar[dd] \\
            P \ar[rr,crossing over] \ar[dd] \ar[ru] & & Q \ar[ru] & \\
            & \Sigma^a R^e \ar[rr] & & Q \oplus \Sigma^a R^e \oplus \Sigma^a R^e \\
            0 \ar[rr] \ar[ru] & & \Sigma^a R^e \ar[from=2-3,crossing over] \ar[ru] & 
        \end{tikzcd}     
        \end{equation*} 
        is strongly cartesian \cite[Corollary 6.1.1.16]{LurHA}. 
        Therefore, $ N^{C_2}_R $ sends the cube to a cartesian cube in $ \Mod_R\left(\Spectra^{C_2}\right) $. 
        By the inductive hypothesis and the second point, $ N^{C_2} $ sends the vertices of the cube excluding $ P $ to perfect $ R $-modules. 
        Thus we have shown that $ N^{C_2}_R(P) $ can be written as a finite limit of perfect $ R $-modules. \qedhere
    \end{itemize}  
\end{proof}

\begin{lemma}
    Let $ (X,\sigma, Y,\pi) $ be a scheme with involution and good quotient and let $ A $ be an Azumaya algebra over $ X $ with genuine $ \sigma $-linear (anti-)involution. 
    Assume further that $ \pi $ is quasi-perfect and flat. \Lucy{in progress June 18th! might need more assumptions later}
    Then 
    \begin{itemize}
        \item the composite $ \mathbf{M}^p_A \to \mathbf{M}^h_A \to \pi_* \mathbf{M}_{A^e} $ of Observation \ref{obs:from_moduli_of_hermitian_objects_to_moduli_of_objects} factors through the inclusion $ \pi_*\mathbf{Mor}_{A^e} \to \pi_* \mathbf{M}_{A^e} $ of Notation \ref{ntn:subsheaf_Poincare_Morita_equiv}, i.e. there is a commutative square 
        \begin{equation*}
        \begin{tikzcd}
            \mathbf{M}^p_A \ar[r] \ar[d,dashed,"\exists"'] & \mathbf{M}^h_A \ar[d] \\
            \pi_* \mathbf{Mor}_{A^e} \ar[r] & \pi_* \mathbf{M}_{A^e}
        \end{tikzcd}
        \end{equation*}
        \item if $ P $ is compact as an $ A \otimes_R R^{\varphi C_2} $-module, then the morphism $ \mathbf{M}^p_A \to \pi_* \mathbf{Mor}_{A^e}$ is locally of finite presentation in the sense of \cite[\S4.3]{MR3190610}. 
        \Lucyil{modify definition of Azumaya w genuine involution to make $ P $, $ \overline{P} $ compact? }
    \end{itemize}
\end{lemma}
\begin{proof}
    % We now turn to the proof of the latter half of the statement. 
    Keep the notation of the proof of Lemma \ref{lemma:hermitian_moduli_to_ordinary_moduli_is_geometric}. 
    By \cite[Theorem 5.2]{MR3190610}, it suffices to show that $ \hom_{N_R^{C_2}(\widetilde{p}^*A)}\left(\widetilde{p}^*A, N^{C_2}_R(M^\vee)\right) $ is a perfect $ S $-module. 
    By Lemma \ref{lemma:norm_preserves_compactness}, it suffices to show that the $ C_2 $-mapping spectrum $ \hom_{N_R^{C_2}(\widetilde{p}^*A)}\left(\widetilde{p}^*A, N^{C_2}_R(M^\vee)\right) $ is a compact $ R^L $-module in $ C_2 $-spectra. 
    Now recall that the mapping $ C_2 $-spectrum arises as
    \begin{equation*}
        \mathrm{ev} \colon \Mod_{N_R^{C_2}(\widetilde{p}^*A)}\left(\Spectra^{C_2}\right) \otimes_{\Mod_{R^L}(\Spectra^{C_2})} \Mod_{N_R^{C_2}(\widetilde{p}^*A)^\op}\left(\Spectra^{C_2}\right) \to \Mod_{R^L}(\Spectra^{C_2}) \,;
    \end{equation*}
    in particular $ \mathrm{ev}\left(N_R^{C_2}(\widetilde{p}^*A) \otimes_{R^L} N_R^{C_2}(\widetilde{p}^*A)^\op \right) \simeq N_R^{C_2}(\widetilde{p}^*A) $.  
    To show that $ \hom_{N_R^{C_2}(\widetilde{p}^*A)}\left(\widetilde{p}^*A, N^{C_2}_R(M^\vee)\right) $ is a compact $ R^L $-module, it suffices to prove that
    \begin{itemize}
        \item $ N_R^{C_2}(\widetilde{p}^*A) $ and $ N_R^{C_2}(\widetilde{p}^*A)[C_2] $ are compact $ R^L $-modules (compare the proof of \cite[Theorem 3.15]{MR3190610}), 
        \item $ \widetilde{p}^*A $ is a compact $ N_R^{C_2}(\widetilde{p}^*A) $-module, and 
        \item $ N_R^{C_2}(M^\vee) $ is a compact $ N_R^{C_2}(\widetilde{p}^*A) $-module. 
    \end{itemize}
    The first point follows from Lemma \ref{lemma:norm_preserves_compactness}. 
    By the claim at the end of the proof, it suffices to check compactness/the latter two points separately on underlying and geometric fixed points. 
    Since $ p $ classifies a point in $ \pi_* \mathbf{Mor}_{A^e} $, $ M^\vee $ is compact as an $ A \otimes_{R} R^{\varphi C_2} $-module. 
    By assumption on $ A $, $ (\widetilde{p}^*A)^{\varphi C_2} $ is a compact $ (N_R^{C_2}(\widetilde{p}^*A))^{\varphi C_2} $-module. 
    That $ (\widetilde{p}^*A)^e $ is a compact $ (N_R^{C_2}(\widetilde{p}^*A))^e $-module follows from our assumption that $ A $ is Azumaya. 
    \Lucyil{WORK IN PROGRESS: map is locally of finite presentation. REMAINING: $ (M^\vee)^{\otimes 2} $ is a compact $ (\widetilde{p}^*A)^{\otimes 2} $-module} 

    \paragraph{Claim} Let $ B $ be an $ \mathbb{E}_1 $-algebra in $ \Spectra^{C_2} $, and assume that $ B^e $, $ B^{\varphi C_2} $ are connective. 
    Then a $ B $-module $ M $ is compact if and only if $ M^e $ and $ M^{\varphi C_2} $ are compact as $ B^e $ and $ B^{\varphi C_2} $-modules, respectively. 
    \Lucyil{Forgot that a generalized Azumaya algebra cannot be assumed to be connective. Fix later. Use that $ A $ is bounded below? or other `niceness' properties.}

    \emph{Proof.} By the connectivity hypotheses on $ B $, a $ B $-module in $ \Spectra^{C_2} $ is compact if and only if it is dualizable. 
    The result follows from noting that $ (-)^e $, $ (-)^{\varphi C_2} $ detect dualizability and for $ B^e $-modules and $ B^{\varphi C_2} $-modules, dualizability is equivalent to compactness. 
    % To prove the `only if' part of the statement, it suffices to show that $ (-)^e $ and $ (-)^{\varphi C_2} $ preserve compact objects. 
    % This follows from observing that they both admit right adjoints which commute with filtered colimits. \Lucy{here it is important that $ (-)^e $ takes values in $ \Mod_{B^e}(\Spectra) $ and not $ \Mod_{B^e}(\Spectra^{BC_2}).} 
\end{proof}

\subsection{The Poincaré Brauer space spectral sequence} % (fold)
\label{sub:the_poincaré_brauer_space_spectral_sequence}

\Lucy{next bit is sketchy; working towards a spectral sequence like the one in \cite[\S7]{MR3190610}}
\begin{notation}
    Let $ (X, \lambda, Y, \pi) $ be a scheme with involution $ X $ and a good quotient $ Y $. 
    Let $ \ZZ_X $ be the \'etale sheafification of the constant presheaf. 
    Write $ \ZZ_X^\sigma $ for the equalizer
    \begin{equation*}
        \ZZ_X^\sigma := \mathrm{Eq}\left(\ZZ_X \overset{\lambda, \cdot (-1)}{\rrarrows} \ZZ_X\right) \,.
    \end{equation*}\Noah{How is $\lambda$ inducing a map on $\mathbb{Z}_X$?}
    Write $ \mathrm{disc} $ for the cokernel
    \begin{equation}
        \mathrm{disc} := \mathrm{coKer} \left(\mathcal{O}_X^\times \xrightarrow{f \mapsto f \cdot \lambda(f)} \mathcal{O}_X^\times \right) \,;
    \end{equation}
    here the cokernel is taken in the category of sheaves of abelian groups on the small \'etale site of $ X $. \Noah{Why the name disc? In any event I think as a sheaf this this vanishes because for strict henselian rings there is no contribution here.} 
    \Lucy{inspired by `discriminant.' To the second point: I think if $ X = \Spec k \times k $ with the flip action, then the cokernel of $ (u,v) \mapsto (u,v) \cdot (v,u) = (uv, uv) $ is nontrivial.}
    Consider the assignment
    \begin{equation*}
    \begin{split}
        U_1 \colon \mathrm{\acute{E}t}_Y &\to \EE_\infty \mathrm{Mon}(\Spaces) \\
       \left( V =\Spec A \to Y \right) & \mapsto \mathrm{Ker} \, \left(R\Gamma(\mathcal{O}_{X_V}(X_V))^\times \xrightarrow{f \mapsto f \cdot \lambda(f)} R\Gamma(\mathcal{O}_{X_V}(X_V))^\times \right) \,.
    \end{split}
    \end{equation*}
    Then $ U_1 $ is a sheaf of groups on the small \'etale site of $ Y $. \Lucy{check later} 
    By Theorem \ref{theorem:loops_Poincare_pic_is_Gm_Qoppa}, there is a natural map of sheaves $ BU_1 \to \pnpic $ given by inclusion of the identity component. 
\end{notation}
\begin{example}
    If $ X $ has the trivial involution $ \lambda = \id $, then $ \ZZ_X^\sigma $ is the trivial sheaf and $ \mathrm{disc}_X $ is the units in $ \mathcal{O}_X $ mod 2. 

    If $ \pi $ is quadratic \'etale, then $ \pi_* \ZZ_X^\sigma $ is a $ \ZZ $-torsor.  
\end{example}
\begin{corollary}
    Let $ (X, \lambda, Y, \pi) $ be a scheme with involution $ X $ and a good quotient $ Y $. 
    The homotopy sheaves of $ \pnbr $ are
    \begin{equation*}
        \pi_* \pnbr = \begin{cases}
            ? & * = 0 \\
            \pi_* \ZZ^\sigma_X \times \pi_* \mathrm{disc} & * = 1 \\
            U_1 & * = 2 \\
            0 & \text{else. }
        \end{cases}
    \end{equation*}
    \Lucyil{If we believe what's in Example \ref{ex:pnbr_closed_point_ramified}, then $ \pi_0 \pnbr $ should be supported on the branch locus (terminology from \cite{azumaya_involution}); maybe to simplify presentation we could state it as: Assume $ X $ is over a field $ k $. Then the $ \pi_0 \pnbr $ is the pushforward of either (sheafification of constant sheaves) $ \ZZ $ or $ \ZZ/2 $ from the branch locus (depending on char $k$).}
\end{corollary}

%\section{Poincar{\'e} schemes}
%\begin{defn}
Let $\aps$ be the $(\infty,1)$-category defined by the pullback \[
\begin{tikzcd}
\aps \arrow[rr]\arrow[d] & & \operatorname{Fun}(\Delta^2, \calg(\Sp))\arrow[d,"d_1^*"]\\
\calg(\Sp^{BC_2})\arrow[rr,"U(-)\to (-)^{tC_2}"] & & \operatorname{Fun}(\Delta^1, \calg(\Sp))
\end{tikzcd}
\] where $U:\Sp^{BC_2}\to \Sp$ is the functor which forgets the $C_2$-action.
\end{defn}

We record here a few structural results about this category.

\begin{thm}
The following statements about $\aps$ hold:
\begin{enumerate}
\item The category $\aps$ is a cocomplete and symmetric monoidal infinite category;
\item the pullback diagram above is homotopy Cartesian;
\item the functor $\aps\to \calg(\Sp^{BC_2})$ is symmetric monoidal and (co)continuous;
\item the functor $\aps\to \calg(\Sp)^{\Delta^2}$ is lax symmetric monoidal;
\item and the functor $\aps\to \calg(\Sp)^{\Delta^2}\xrightarrow{ev_{[1]}} \calg(\Sp)$ is symmetric monoidal.
\end{enumerate}
\end{thm}
\begin{proof}
    For (2) it is enough to show that $d_1^*$ is a cartesian fibration which follows from \cite[Corollary 2.4.6.5]{HTT}. 

    For (3), let $p:K\to \aps$ be a map of simplicial sets, $K$ a small simplicial set. Suppose the $K^\vartriangleright\to \aps$ be an extension such that $K^\vartriangleright\to \aps\to \calg(\Sp^{BC_2})$ is a colimit diagram. By \cite[Proposition 2.4.3.2]{HTT} the diagram \[\begin{tikzcd}
        \aps_{p/}\arrow[r]\arrow[d] &\calg(\Sp)^{\Delta^2}_{p/-}\arrow[d]\\
        \calg(\Sp^{BC_2})_{p/}\arrow[r] & \calg(\Sp)^{\Delta^1}_{p/-}
    \end{tikzcd}\] is again homotopy cartesian. Then 
    \begin{align*}
        \hom_{\aps}(p(\infty), -)&\simeq \hom_{\calg(\Sp^{BC_2})}(p(\infty), -)\times_{\hom_{\calg(\Sp)^{\Delta^1}}(p(\infty),-)}\hom_{\calg(\Sp)^{\Delta^2}}(p(\infty))\\
        &\simeq 
    \end{align*}
\end{proof}

We will denote elements of $\aps$ by $\underline{A}=(A,s:A^{\Phi C_2}\to A^{tC_2})$. Here $s:A^{\Phi C_2}\to A^{tC_2}$ is the image of $\underline{A}$ under the top horizontal map above. The use of the notation $A^{\Phi C_2}$ is justified by the following.

\begin{lem}
Let $\aps\to \calg(\Sp)$ be the composition of the functors \[\aps\to \operatorname{Fun}(\Delta^2,\calg(\Sp))\xrightarrow{ev_{[1]}}\calg(\Sp).\] Then this functor factors as a composition $\aps\to \calg(\Sp^{C_2})\xrightarrow{(-)^{\Phi C_2}}\calg(\Sp)$. 
\end{lem}
\begin{proof}
The commutativity of the diagram
\[
\begin{tikzcd}
 & & \operatorname{Fun}(\Delta^2, \calg(\Sp)) \arrow[rd, "d_0^*"] \arrow[dd, "d_1^*"] & \\
 & & & \operatorname{Fun}(\Delta^1,\calg(\Sp)) \arrow[dd,"ev_{[1]}"]\\
\calg(\Sp^{BC_2}) \arrow[rr, "U(-)\to (-)^{tC_2}"] \arrow[rrd, "id"] & & \operatorname{Fun}(\Delta^1, \calg(\Sp)) \arrow[rd, "ev_{[1]}"] & \\
  & & \calg(\Sp^{BC_2}) \arrow[r, "(-)^{tc_2}"] & \calg(\Sp)
\end{tikzcd}
\] induces a functor on the pullback infinity categories $\aps\to \calg(\Sp^{C_2})$ which makes the corresponding cube commute. The functor $ev_{[1]}:\operatorname{Fun}(\Delta^2, \calg(\Sp))\to \calg(\Sp)$ factors through $d_0^*$ and so $\aps\to \operatorname{Fun}(\Delta^2, \calg(\Sp))\to \calg(\Sp)$ is equivalent to the composition \[\aps\to \calg(\Sp^{C_2})\to \operatorname{Fun}(\Delta^1,\calg(\Sp))\to \calg(\Sp)\] and the composition of the last two maps is the geometric fixed point functor as desired.
\end{proof}

The following Lemma gives the justification of the name Poincar{\'e} scheme.

\begin{lem}
    There is a symmetric monoidal functor \[\perfpn:\aps\to \mathrm{Cat}_{\infty}^{\mathrm{Pn}}\] to the category of Poincar{\'e} infinity categories which has essential image the subcategory spanned by objects $(\perf(R),\Qoppa)$ which are $\mathbb{E}_\infty$-algebras.
\end{lem}

\begin{defn}
     A map $f:\underline{A}\to \underline{B}\in \aps$ is faithfully flat if the underlying map $f:A\to B$ is faithfully flat and the map $f^{\Phi C_2}:A^{\Phi C_2}\to B^{\Phi C_2}$ is also faithfully flat.
\end{defn}

\begin{lem}
    The fpqc covers on $\aps$ form a Grothendieck site. 
\end{lem}

%\begin{defn}
%    The infinity category of Poincar{\'e} schemes, denoted $\psch$, is the infinity category \[\psch:= \operatorname{Ind}(\aps^{op})\]
%\end{defn}

\addcontentsline{toc}{section}{\protect\numberline{\thesection}References}

\printbibliography
\end{document}
