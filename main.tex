\documentclass{article}
\usepackage[utf8]{inputenc}
%Packages Used------------------------------------------
% the following is to get qoppa and Qoppa
\DeclareFontFamily{T1}{cbgreek}{}
\DeclareFontShape{T1}{cbgreek}{m}{n}{<-6>  grmn0500 <6-7> grmn0600 <7-8> grmn0700 <8-9> grmn0800 <9-10> grmn0900 <10-12> grmn1000 <12-17> grmn1200 <17-> grmn1728}{}
\DeclareSymbolFont{quadratics}{T1}{cbgreek}{m}{n}
\DeclareMathSymbol{\qoppa}{\mathord}{quadratics}{19}
\DeclareMathSymbol{\Qoppa}{\mathord}{quadratics}{21}

\usepackage{amsmath, amssymb, amsthm}
\usepackage{longtable}
%\usepackage{amsfonts}
%\usepackage{mathtools}
%\usepackage{wasysym}
%\usepackage{MnSymbol}
%\usepackage{thmtools}
%\usepackage{stmaryrd}
\usepackage[letterpaper,margin=1in]{geometry}   
%\usepackage{slashed}
%\usepackage[english]{babel}				
%\usepackage[pdfencoding=auto, psdextra, draft=false]{hyperref}
\usepackage{bookmark}
\usepackage{url}		
\usepackage{lmodern}			
\usepackage[T1]{fontenc}
%\usepackage{xspace}		
%\usepackage{fancyhdr}
\usepackage{enumerate}
\usepackage{enumitem}
%\usepackage{mathrsfs}
\usepackage{graphicx}
\usepackage{soul,color}
\usepackage{tikz-cd}
\usepackage[maxbibnames=99,style=alphabetic]{biblatex}
\usepackage{csquotes}
\usepackage{chngcntr}
\usepackage[bbgreekl]{mathbbol}
\counterwithin{equation}{section}
\addbibresource{biblio.bib}
\usepackage{todonotes}

%hyperlink setup
\definecolor{darkred}{RGB}{128,0,0}
\definecolor{darkgreen}{RGB}{0,128,0}
\definecolor{darkblue}{RGB}{0,0,128}

\hypersetup{linktocpage,
	pdfborder = {0 0 0},
	colorlinks,
	citecolor=darkgreen,
	filecolor=darkred,
	linkcolor=darkblue,
	urlcolor=cyan!50!black!90}

%Greek and Latin black board bold-----------------------
\DeclareSymbolFontAlphabet{\mathbb}{AMSb}
\DeclareSymbolFontAlphabet{\mathbbl}{bbold}

%shortcut commands-------------------------------------------

\DeclareMathOperator{\Alg}{Alg} % Algebra objects
\DeclareMathOperator{\Br}{Br} % Brauer functor
\DeclareMathOperator{\Brp}{Br^p} % Poincare Brauer functor
\DeclareMathOperator{\Spec}{Spec}
\DeclareMathOperator{\CAlg}{CAlg} % Commutative Algebra objects
\DeclareMathOperator{\CAlgp}{CAlg^p} % Poincare ring spectra
\DeclareMathOperator{\Cat}{\mathcal{C}at} % Categories
\DeclareMathOperator{\Catex}{\Cat_\infty^{ex}} % stable categories with exact functors
\DeclareMathOperator{\Cath}{Cat^h_\infty} % Hermitian Categories
\DeclareMathOperator{\Catp}{Cat^p_\infty} % Poincare Categories
\DeclareMathOperator{\Catpidem}{Cat^p_{\infty, idem}} % idempotent complete Poincare Categories
\DeclareMathOperator{\Einfty}{\mathbf{E}_\infty} % E-infinity 
\DeclareMathOperator{\ex}{ex} % exact 
\DeclareMathOperator{\Fun}{Fun} % Functors
\DeclareMathOperator{\gp}{gp} % grouplike
\DeclareMathOperator{\id}{id} % identity
\DeclareMathOperator{\idem}{idem} % idempotent
\DeclareMathOperator{\Mod}{Mod} % Modules
\DeclareMathOperator{\LMod}{LMod} % Left modules
\DeclareMathOperator{\BiMod}{BiMod} % Bimodules
\DeclareMathOperator{\Mon}{Mon} % monoids
\DeclareMathOperator{\Pic}{Pic} % Picard functor
\DeclareMathOperator{\Picp}{Pic^p} % Poincare Picard functor
\DeclareMathOperator{\Pn}{Pn} % Poincare space functor
\DeclareMathOperator{\Spectra}{Sp} % Spectra
\DeclareMathOperator{\Spaces}{\mathcal{S}} % Spaces
\DeclareMathOperator{\tee}{t} % tate/transpose
\DeclareMathOperator{\gmq}{\mathbb{G}_m^\Qoppa}


\newcommand{\pf}{{\bf Proof. \ }}
\renewcommand{\epsilon}{\varepsilon}
\renewcommand{\rho}{\varrho}
\renewcommand{\phi}{\varphi}
\newcommand{\NN}{\ensuremath{\mathbb{N}}\xspace}
\newcommand{\ZZ}{\mathbb{Z}}
\newcommand{\QQ}{\ensuremath{\mathbb{Q}}\xspace}
\newcommand{\RR}{\ensuremath{\mathbb{R}}\xspace}
\newcommand{\CC}{\ensuremath{\mathbb{C}}\xspace}
\newcommand{\FF}{\ensuremath{\mathbb{F}}\xspace}
\newcommand{\EE}{\mathbb{E}}
\newcommand{\TT}{\ensuremath{\mathbb{T}}\xspace}
\newcommand{\RP}{\ensuremath{\mathbb{RP}}\xspace}
\newcommand{\DD}{\ensuremath{\mathbbl{\Delta}}\xspace}
\newcommand{\op}{\mathrm{op}} % opposite functor
\newcommand{\tc}{\ensuremath{\mathrm{TC}}}
\newcommand{\thh}{\ensuremath{\mathrm{THH}}}
\newcommand{\tp}{\ensuremath{\mathrm{TP}}}
\newcommand{\tr}{\ensuremath{\mathrm{TR}}}
\newcommand{\pnpic}{\ensuremath{\mathrm{PnPic}}}
\newcommand{\pnbr}{\ensuremath{\mathrm{PnBr}}}
\newcommand{\pic}{\ensuremath{\mathrm{Pic}}}
\newcommand{\br}{\ensuremath{\mathrm{Br}}}
\DeclareMathOperator*{\colim}{\ensuremath{\operatorname{colim}}}
\newcommand{\aps}{\mathrm{APS}}
\newcommand{\psch}{\mathrm{PSch}}
\newcommand{\perf}{\mathrm{Perf}}
\newcommand{\perfpn}{\mathrm{Perf}^{\mathrm{Pn}}}

% Adjoint functors
\newcommand{\rlarrows}{\mathrel{\substack{\textstyle\longrightarrow\\[-0.6ex]
                                            \textstyle\longleftarrow}}}

%Theorem Environments ----------------------------------------------------------------
\newtheorem{theorem}[equation]{Theorem}
\newtheorem{proposition}[equation]{Proposition}
\newtheorem{lemma}[equation]{Lemma}
\newtheorem{corollary}[equation]{Corollary}

\theoremstyle{definition}
\newtheorem{definition}[equation]{Definition}
\newtheorem{construction}[equation]{Construction}
\newtheorem{remark}[equation]{Remark}
\newtheorem{remarks}[equation]{Remarks}
\newtheorem{observation}[equation]{Observation}
\newtheorem{notation}[equation]{Notation}
\newtheorem{example}[equation]{Example}
\newtheorem{recollection}[equation]{Recollection}
\newtheorem{variant}[equation]{Variant}

\newcommand{\Viktor}[1]{\todo{V: #1}}
\newcommand{\Noah}[1]{\todo[color=red]{N: #1}}
\newcommand{\Lucy}[1]{\todo[color=cyan]{\linespread{1}\footnotesize L: #1}}

\title{Poincar{\'e} Schemes}
\author{Ben Antieau, Viktor Burghardt, Noah Riggenbach, Lucy Yang}
\date{}
\addbibresource{biblio.bib}

\begin{document}

\maketitle
\begin{abstract}
    We do stuff \Noah{Change this}
\end{abstract}
\tableofcontents

\section{Introduction}

\begin{theorem}
Let $\underline{A}$ be an affine Poincar{\'e} scheme with underlying $\mathbb{E}_\infty$-ring spectrum with involution $A$. Then the natural maps \[\pi_i(\pnpic(\underline{A}))\to \pi_i(\pic(A))\] \Noah{I think there is some interaction with the homotopy fixed points, or maybe even the genuine fixed points}are surjective on $2$-torsion.
\end{theorem}

\begin{theorem}
    Let $A$ be an $\mathbb{E}_\infty$ ring with involution, and let $\underline{NA}$ be the associated Tate affine Poincar{\'e} scheme. Let $\br_\nu(A)$ be the Brauer group of Azumaya algebras over $A$ with involution. \Noah{I think we need to define this for ring spectra. For $A$ discrete this is done in \cite{azumaya_involution}.} Then the natural map \[\pnbr(\underline{NA})\to \br_\nu(A)\] is an equivalence\Noah{probably of $\mathbb{E}_\infty$ dodads}
\end{theorem}

\begin{theorem}
    The functors $\pnpic,\pnbr:\mathrm{APS}\to \Spectra$ are fppf sheaves.
\end{theorem}

\begin{theorem}
    There is a Poincar{\'e} group scheme $\mathbb{G}_m^\Qoppa$ such that \[B\mathbb{G}_m^\Qoppa\simeq \pnpic\] as fppf stacks.
\end{theorem}

\subsection{Acknowledgements} 
The authors wish to thank the Institute for Advanced Study and the organizers of the 2024 Park City Mathematics institute on motivic homotopy theory. 

\subsection{Conventions}
\label{subsection:conventions}
    \begin{longtable}{lll}
        $\Brp$ & Poincaré Brauer space\\
        $\CAlg$ & $\infty$-categoriy of $\Einfty$-ring spectra\\
        $\CAlg(\Spaces)$ & $\infty$-categoriy of $\Einfty$-spaces\\
        $\CAlg^{\gp}(\Spaces)$ & $\infty$-categoriy of grouplike $\Einfty$-spaces\\
        $\CAlgp$ & $\infty$-categoriy of Poincaré ring spectra\\
        $\Catex$ & $\infty$-category of small stable $\infty$-categories and exact functors\\
        $\Catp$ & $\infty$-category of Poincaré $\infty$-categories\\
        $\Catpidem$ & $\infty$-category of idempotent complete Poincaré $\infty$-categories\\
        $\Picp$ & Poincaré Picard space\\
        $\Spaces$ & $\infty$-category of spaces\\
        $\Spectra$ & $\infty$-category of spectra
    \end{longtable}

\section{Poincaré Structures on Compact Modules}
\label{section:poincare_structures_on_compact_modules}
We will use this section to recall notions and results about Poincaré $\infty$-categories which we require in the sections to follow. This section can safely be skipped by anyone with extensive knowledge of Poincaré $\infty$-categories, as found in \cite{CDHHLMNNSI}.

\begin{notation}
    \label{notation:omission_of_e_infty}
    Let $R$ be an $\mathbf{E}_\infty$-ring spectrum. We will drop $\mathbf{E}_\infty$ from our notation and simply call $R$ a \emph{ring spectrum}. Moreover, we will denote the $\infty$-category $\CAlg(\Spectra)$ of commutative algebra objects in the $\infty$-category of spectra $\Spectra$ by $\CAlg$. 
\end{notation}

Let $R$ be a ring spectrum and let $\Mod_R$ be the $\infty$-category of modules over $R$. We will study Poincaré structures on the $\infty$-category $\Mod_R^\omega$ of compact modules over $R$. 

\Viktor{-characterization in terms of modules with genuine involution, -characterization of symmetric monoidal structures, -Pn}

\section{Poincaré Ring Spectra}
\label{section:poincare_ring_spectra}
In this section we will define the ring theoretic building blocks of Poincaré schemes and the corresponding category they live in. Affine Poincaré Schemes will then be the dual objects, similar to how affine schemes are dual to commutative rings.

\begin{definition}
    \label{definition:poincare_ring_spectrum}
    Let $R$ be a ring spectrum. 
    A \emph{Poincaré structure} on $R$ is the following data:
    \begin{itemize}
        \item A $C_2$-action on $R$ via maps of ring spectra, i.e. a functor $\lambda: BC_2\rightarrow \CAlg$.
        \item An $R$-algebra $R\rightarrow C$.
        \item An $R$-algebra map $C\rightarrow R^{tC_2}$. 
    \end{itemize}
    Here $R^{tC_2}$ is the Tate construction with respect to the above action. Since the Tate construction is lax symmetric monoidal, $R^{tC_2}$ is naturally an $R$-algebra via the Tate-valued norm. A ring spectrum equipped with a Poincaré structure will be called a \emph{Poincaré ring spectrum}. 
\end{definition}

\begin{remark}
    \label{remark:notational_difference_to_nine-authored_papers}
    % Poincaré ring spectra, as defined in Definition \ref{definition:poincare_ring_spectrum}, were studied in \Viktor{cite 9 authored paper}. Note that we chose a different notation. 
    In \cite[discussion immediately preceding Examples 5.4.10]{CDHHLMNNSI}, Poincaré ring spectra are referred to as $\mathbf{E}_\infty$-\emph{ring spectra with genuine involution}.
\end{remark}

\begin{remark}
    \label{remark:poincare_ring_spectra_to_modules_with_Poincare_structure}
    Let $R$ be a ring spectrum. 
    % There is a natural equivalence between symmetric monoidal Poincaré structures on $\Mod_R^\omega$ and certain algebra objects over the genuine $C_2$-spectrum $NR$ . 
    By \cite[Corollary 5.4.8]{CDHHLMNNSI}, a Poincaré structure on $R$ gives rise to a symmetric monoidal lift of $ \Mod_R^\omega $ to the symmetric monoidal $ \infty $-category of Poincaré $\infty$-categories $ \qoppa_R: (\Mod_R^\omega)^{\op}\rightarrow \Spectra $. 
    Furthermore, the structure map $ R \to C $ gives a canonical lift of $ R \in \Mod_R^\omega $ to a Poincaré object $ (R, q) \in \mathrm{Pn}\left(\Mod_R^\omega,\qoppa_R \right) $. 
    % We call such a symmetric monoidal Poincaré $\infty$-category a \emph{Poincaré ring spectrum}. We will denote the full subcategory of $\CAlg(\Catp)$ spanned by Poincaré ring spectra by $\CAlgp$ and call it the \emph{$\infty$-category of Poincaré ring spectra}. 
\end{remark}

\begin{remark}
    \label{remark:poincare_structures_are_factorizations}
    A Poincaré structure on a ring spectrum $R$ with a $C_2$-action via maps of ring spectra is a factorization $R\rightarrow C \rightarrow R^{tC_2}$ in $\CAlg$ of the Tate Frobenius $R\rightarrow R^{tC_2}$.
\end{remark}

\begin{remark}
    \label{remark:poincare_ring_spectra_as_algebra_objects}
    By \cite[\S5.1]{CDHHLMNNSI}, the assignment $ (R, R\to C \to R^{tC_2}) \mapsto \left(\Mod_R^\omega, \qoppa_R \right) $ promotes to a symmetric monoidal functor $ \CAlgp \to \CAlg(\Catp) $. 
    % Let $\mathcal{M}$ be the full subcategory of $\Catp$ spanned by Poincaré $\infty$-categories with underlying $\infty$-category $\Mod^\omega_R$ for some ring spectrum $R$. Then the symmetric monoidal structure of $\Catp $ restricts to a symmetric monoidal structure on $\mathcal{M}$ by Example \ref{example:universal_poincare_ring_spectrum} and \cite[\S5.1]{CDHHLMNNSI}. Then we have $\CAlgp\simeq \CAlg(\mathcal{M})$. In particular, the symmetric monoidal structure of $\CAlg(\Catp)$ restricts to a symmetric monoidal structure on $\CAlgp$.
\end{remark}

\begin{notation}
    \label{notation:spectrum_with_trivial_action}
    Let $R$ be an $ \mathbb{E}_\infty $-ring spectrum. We will denote by $\underline{R}$ the spectrum $R$ with trivial $ C_2 $-action. More precisely, $\underline{R}:BC_2\rightarrow \Spectra $ is the constant functor. \Lucy{This is commonly used for constant Mackey functors--could be ambiguous}
\end{notation}

\begin{example}
    \label{example:classification_of_poincare_structures_when_tate_vanishes}
    Let $R$ be a ring spectrum with a $ C_2 $-action. If $2\in \pi_0(R)$ is invertible, we have $\underline{R}^{tC_2}\simeq 0 $. 
    A Poincaré structure on $R$ is equivalent to the data of an $ \mathbb{E}_\infty $-$R$-algebra $R\rightarrow C$.
\end{example}

\begin{example}
    \label{example:tate_poincare_structure}
    Let $R$ be a ring spectrum equipped with a $C_2$-action via maps of ring spectra. The Tate-valued norm endows $ R^{tC_2} $ with a natural $R$-algebra structure, which induces a Poincaré structure on $R$ given by the factorization $R\xrightarrow{\id} R\rightarrow R^{tC_2}$. 
    We will call this Poincaré structure the \emph{Tate Poincaré structure on $R$} and will denote it by $(R,\Qoppa_R^{\tee})$.
\end{example}

\begin{example}
    \label{example:universal_poincare_ring_spectrum}
    The sphere spectrum $\mathbb{S}$ together with the Tate Poincaré structure will be called the \emph{universal Poincaré ring spectrum} (see \cite[\S4.1]{CDHHLMNNSI}). We will denote it by $(\mathbb{S},\Qoppa_u)$. 
\end{example}

\begin{remark}
    \label{remark:factorizations_of_tate_frobenius_through_invariants_induce_splittings_of_forms}
    Let $(R,\Qoppa)$ be a ring spectrum associated to a factorization $R\rightarrow C\rightarrow R^{tC_2}$. A factorization of the map $C\rightarrow R^{tC_2}$ through $R^{hC_2}$ induces a section of the canonical map $\Qoppa(R)\rightarrow \hom_R(R,C)\simeq C$. In that case, we have a splitting $\Qoppa(R)\simeq R_{hC_2}\oplus C$\Viktor{reference pullback that characterizes all quadratic functors}.
\end{remark}

\begin{example}
    \label{example:universal_tate_poincare_splits_at_unit}
    The Tate Frobenius for the sphere spectrum factors through $\mathbb{S}^{hC_2}$. Therefore, Remark \ref{remark:factorizations_of_tate_frobenius_through_invariants_induce_splittings_of_forms} implies $\Qoppa_u(\mathbb{S})\simeq \mathbb{S}_{hC_2}\oplus \mathbb{S}\simeq \Sigma^\infty(\mathbb{P}_\mathbb{R}^\infty) \oplus \mathbb{S}$.
\end{example}

\begin{example}
    \label{example:symmetric_poincare_structure}
    Let $R$ be a ring spectrum equipped with a $C_2$-action via maps of ring spectra. The identity map $\id: R^{tC_2}\rightarrow R^{tC_2}$ induces a Poincaré structure on $R$
    given by the factorization $R\rightarrow R^{tC_2}\xrightarrow{id} R^{tC_2}$. We will call this Poincaré structure the \emph{symmetric Poincaré structure on $R$}.
\end{example}

\begin{example}
    \label{example:genuine_symmetric_poincare_structure}
    Let $R$ be a connective ring spectrum equipped with a $C_2$-action via maps of ring spectra. The connective cover $\tau_{\geq 0}(R^{tC_2})\rightarrow R^{tC_2}$ of $R^{tC_2}$ induces a Poincaré structure on $R$ given by the factorization $R\rightarrow \tau_{\geq 0}(R^{tC_2})\rightarrow R^{tC_2}$. We will call this Poincaré structure the \emph{genuine symmetric Poincaré structure on $R$}.
\end{example}

\begin{example}
    \label{ex:fixpt_Mackey_functor}   
    Let $ R $ be a commutative ring endowed with an involution $ \sigma \colon R \xrightarrow{\sim} R $. 
    Write $ \underline{R}^\sigma $ for the $ C_2 $-Green functor with $ C_2 $-fixed points $ R^{C_2} $, where $ R^{C_2} $ denotes the strict fixed points of the $ C_2 $-action on $ R $, and underlying object $ R $. 
    The Mackey functor $ \underline{R}^\sigma $ is a $ C_2 $-$ \EE_\infty $ ring, therefore in particular we may regard it as a Poincaré ring. 
    This is a special case of Example \ref{example:genuine_symmetric_poincare_structure}. 
\end{example}

\Viktor{copy more examples from notes}

\subsection{Algebras with genuine involution}
\begin{recollection}\label{rec:left_module_cats}  
    Assume $ \mathcal{C} $ is a presentable monoidal $ \infty $-category such that the monoidal product $ - \otimes - \colon \mathcal{C} \times \mathcal{C} \to \mathcal{C} $ preserves small colimits separately in each variable. 
    Then there is an $ \infty $-category $ \LMod(\mathcal{C}) $ \cite[Example 4.2.1.18]{LurHA} whose objects are pairs $ (A, M) $ where $ A $ is an associative algebra object of $ \mathcal{C} $ and $ M $ is a left $ A $-module. 
    Write $ a, m $ respectively for the canonical forgetful functors $ \LMod(\mathcal{C})\to \mathrm{Alg}(\mathcal{C}) $, $ \LMod(\mathcal{C}) \to \mathcal{C} $ which send $ (A, M) $ to $ A $ and $ M $, resp. 
    Then $ a $ is a cocartesian fibration \cite[Corollary 4.2.3.7]{LurHA}, hence it is classified by a functor $ \mathrm{mod} \colon \mathrm{Alg}(\mathcal{C}) \to \Cat_\infty $. 

    The functor $ s $ of \cite[Example 4.2.1.17]{LurHA} determines a natural transformation $ \eta \colon * \to \mathrm{mod} $, where $ * \colon \mathrm{Alg}(\mathcal{C}) \to \{*\} \hookrightarrow \Cat_\infty $ is the constant functor at the trivial category, or equivalently
    \begin{equation}
    \begin{tikzcd}
        & \mathcal{U} \ar[d] \\
         \mathrm{Alg}(\mathcal{C}) \ar[r,"\mathrm{mod}"] \ar[ru,"\eta"] & \Cat_\infty   
    \end{tikzcd}    
    \end{equation}
    where $ \mathcal{U} $ is the universal cocartesian fibration. 
    Now consider the functor $ o \colon \mathcal{U} \to \Cat_\infty $ which sends $ (\mathcal{D}, d \in \mathcal{D}) $ to the undercategory $ \mathcal{D}_{d/-} $. 
    Define $ \LMod(\mathcal{C})_{*/-} $ to be the cocartesian fibration over $ \mathrm{Alg}(\mathcal{C}) $ classified by $ o \circ \eta \circ \mathrm{mod} $. 
\end{recollection}
\begin{variant}
    Let $ \mathrm{C} $ be as in Recollection \ref{rec:left_module_cats}. 
    There is a similar construction where left modules is replaced by \emph{bimodules} \cite[Definition 4.3.1.12]{LurHA}. 
\end{variant}
\begin{construction}\label{cons:alg_inv_is_bimod_canonically}
    Regard $ \mathbb{E}_1\mathrm{Alg}(\Spectra) $ as a category with $ C_2 $-action given by taking the opposite/reverse algebra.  
    There are functors $ b \colon \mathbb{E}_1\mathrm{Alg}\left(\Spectra\right)^{hC_2} \to \LMod(\Spectra) $ and $ b \colon \mathbb{E}_1\mathrm{Alg}\left(\Spectra\right)^{hC_2} \to \BiMod(\Spectra)_{*/-} $ so that $ (a, m) \circ b $ and $ (a, m)\circ b_* $ are (canonically) equivalent to $ (-^e) \otimes(-^ e)^\mathrm{op}, (-)^e $. 
    Informally, an $ \mathbb{E}_1 $-algebra with involution $ B $ can be regarded as a $ B \otimes B^\op $-module in a canonical way, and there is a canonical $ B \otimes B^\op $-module map $ B \otimes B^\op \to B $. 
\end{construction}
\begin{definition}
    The category of \emph{$\mathbb{E}_1$-algebras with genuine involution} is defined to be the limit of the $\Cat_\infty$-valued diagram 
    \begin{equation}\label{diagram:E1_alg_gen_involution}
    \begin{tikzcd}[column sep=small]
        & \LMod(\Spectra) \ar[dd,"{a,m}"]  & \\
        & & \LMod\left(\Spectra^{C_2}\right) \ar[dd,"{a, m}"] \ar[lu,"{(-)^e}"'] \\
        &\mathbb{E}_1\mathrm{Alg}(\Spectra) \times \Spectra & \\
        \mathbb{E}_1\mathrm{Alg}\left(\Spectra\right)^{hC_2} \times_{\Spectra} \Spectra^{\Delta^1}  \ar[rr,"{N^{C_2}(-^e) \times U}"'] \ar[ru,"{(-^e) \otimes(-^ e)^\mathrm{op}, (-)^e}"] \ar[ruuu,"{b \circ (-^e)}", bend left=30] & &\mathbb{E}_1\mathrm{Alg}\left(\Spectra^{C_2}\right) \times \Spectra^{C_2} \ar[lu,"{(-)^e \times (-)^e}"]
    \end{tikzcd}
    \end{equation}
    where
    \begin{itemize}
        \item $ b $ is the functor/section of Construction \ref{cons:alg_inv_is_bimod_canonically} 
        \item $ U $ is the `underlying' $C_2$-spectrum functor $ \mathbb{E}_1\mathrm{Alg}\left(\Spectra\right)^{BC_2} \times_{\Spectra} \Spectra^{\Delta^1} \to \Spectra^{BC_2} \times_{\Spectra} \Spectra^{\Delta^1} \simeq \Spectra^{C_2} $ 
        \item The upper right trapezoid commutes canonically by definition of $ \LMod $ (and the fact that the functors $ a, m $ are given by restriction to subcategories of $ LM^\otimes $). 
    \end{itemize}
    Write $ \mathbb{E}_1\mathrm{Alg}^{\mathrm{gi}}\left(\Spectra^{C_2}\right) $ for the $\infty $-category of $\mathbb{E}_1$-algebras with genuine involution. 
\end{definition}
\begin{definition}
    The category of \emph{$\mathbb{E}_\sigma$-algebras} is defined to be the limit of the $\Cat_\infty$-valued diagram 
    \begin{equation}\label{diagram:E_sigma_alg}
    \begin{tikzcd}[column sep=small]
        & \BiMod(\Spectra)_{*/-} \ar[dd,"{a,m}"]  & \\
        & & \BiMod\left(\Spectra^{C_2}\right)_{*/-} \ar[dd,"{a, m}"] \ar[lu,"{(-)^e}"'] \\
        &\mathbb{E}_1\mathrm{Alg}(\Spectra) \times \Spectra & \\
        \mathbb{E}_1\mathrm{Alg}\left(\Spectra\right)^{hC_2} \times_{\Spectra} \Spectra^{\Delta^1}  \ar[rr,"{N^{C_2}(-^e) \times U}"'] \ar[ru,"{(-^e) \otimes(-^ e)^\mathrm{op}, (-)^e}"] \ar[ruuu,"{b_* \circ (-^e)}", bend left=30] & &\mathbb{E}_1\mathrm{Alg}\left(\Spectra^{C_2}\right) \times \Spectra^{C_2} \ar[lu,"{(-)^e \times (-)^e}"]
    \end{tikzcd}\,.
    \end{equation}   
    Write $ \mathbb{E}_\sigma\mathrm{Alg}\left(\Spectra^{C_2}\right) $ for the $\infty $-category of $\mathbb{E}_\sigma$-algebras. 
\end{definition}
\begin{variant}
    Let the base be $ R $ an $ \mathbb{E}_\infty $-algebra or Poincaré ring instead of $ \mathbb{S}^0 $. 
\end{variant}
\begin{remarks}
\begin{enumerate}
    \item Compare \cite[Corollary 3.10]{AKGH_real_THH}. 
    \item There are canonical forgetful functors $  \mathbb{E}_\sigma\mathrm{Alg} \to \mathbb{E}_1\mathrm{Alg}^{\mathrm{gi}} \to \mathbb{E}_1\mathrm{Alg}^{hC_2} \to \mathbb{E}_1\mathrm{Alg}(\Spectra) $. 
\end{enumerate}
\end{remarks}
\begin{construction}\label{cons:Esigma_alg_to_R_lin_hermitian_cat}
    Let $ R, R^{\varphi C_2} \to R^{tC_2} $ be a Poincaré ring. 
    There is a functor $ \left(\Mod_{(-)}^\omega, \Qoppa_{(-)}\right) \colon \mathbb{E}_\sigma\mathrm{Alg}_R \to \left(\Cat^h_R\right)_{\left(\Mod_R^\omega,\Qoppa_R\right)/-} $. 
\end{construction}
\begin{lemma}
    Let $ R, R^{\varphi C_2} \to R^{tC_2} $ be a Poincaré ring. 
    The functor of Construction \ref{cons:Esigma_alg_to_R_lin_hermitian_cat} is fully faithful. 
\end{lemma} 
\begin{proof}
    
\end{proof}

Now we observe that given a hermitian object $ (x,q) $ of $ \left(\mathcal{C}, \Qoppa_\mathcal{C}\right) $, its endomorphism algebra admits a canonical lift to a $ \mathbb{E}_\sigma $-algebra. 
\begin{construction}\label{cons:endomorphism_of_hermitian_obj}
    There is a functor $ \mathrm{End}(-)\colon \left(\Cat^h_R\right)_{\left(\Mod_R^\omega,\Qoppa_R\right)} \to \mathbb{E}_\sigma\mathrm{Alg} $ lifting the functor $ \left(\Cat^h_R\right)_{\left(\Mod_R^\omega,\Qoppa_R\right)} \to \mathbb{E}_1\mathrm{Alg}^{hC_2} $ of \cite[Proposition 3.1.16]{CDHHLMNNSI}. 
\end{construction}
\begin{theorem}
    The functors of Construction \ref{cons:Esigma_alg_to_R_lin_hermitian_cat} and \ref{cons:endomorphism_of_hermitian_obj} form an adjoint pair. 
\end{theorem}

% \section{Modules over Poincaré Ring Spectra}
% \label{subsection:modules_over_poincare_ring_spectra}


\section{The Poincaré Picard space}
\label{subsection:the_poincare_picard_group}

Recall that the Poincaré space functor $ \Pn \colon \Catp \to \CAlg(\Spaces) $ is lax symmetric monoidal with respect to tensor product of Poincaré $ \infty $-categories and smash product of $ \Einfty $-spaces \cite[Corollary 5.2.8]{CDHHLMNNSI}. In particular, we can consider invertible objects in $\Pn(A)$ for a Poincaré ring spectrum $A$.

\begin{definition}
    \label{definition:poincare_picard_space}
    Let $A$ be a Poincaré ring spectrum. We define the \emph{Picard space of $A$} to be $$\Picp(A):=\Pic(\Pn(A)).$$
\end{definition}

\begin{remark}
    \label{remark:poincare_picard_points_desc}
    Let $ \left(\Mod_R^\omega, \Qoppa_R \right)$ be a Poincaré ring spectrum, where $(M_R=R, N_R= R^{\varphi C_2}, R^{\varphi C_2}\to R^{tC_2})$ is the module with genuine involution associated to $ \Qoppa_R $. 
    Then a point in the Poincaré Picard space is the data of a pair $ (\mathcal{L}, q ) $, where $ \mathcal{L} $ is an invertible module in $ \Mod_R^\omega $ and $ q $ is a point in $ \Omega^\infty\Qoppa_R(\mathcal{L}) $. 
    By \cite[Proposition 1.3.11]{CDHHLMNNSI}, the data of $ q $ is equivalent to the data of points in the lower left and upper right corner of the square
    \begin{equation}
    \begin{tikzcd}
        \Qoppa(\mathcal{L}) \ar[r] \ar[d] & \hom_R(\mathcal{L}, R^{\varphi C_2}) \ni \ell(q) \ar[d] \\
        b(q) \in \hom_{R \otimes R}\left(\mathcal{L} \otimes \mathcal{L}, R\right)^{hC_2} \ar[r] & \hom_R(\mathcal{L}, R^{tC_2})
    \end{tikzcd}
    \end{equation} 
    and a path between their images in the lower right corner. 
    In particular, the adjoint of $ b(q) $ must define a nondegenerate hermitian form on $ \mathcal{L} $, that is, an equivalence $ \mathcal{L} \simeq \hom_{R}(\mathcal{L}, R^*) $ where $ R^* $ is considered as an $ R $-module via the action of the generator of $ C_2 $. \Lucy{add equivariance/symmetry data}

    Write $ (\mathcal{L}^\vee,q^\vee) $ is for the inverse of $ (\mathcal{L},q) $. 
    By definition of invertibility, there exists an $ R $-linear map $ \ell(q^\vee) \colon \mathcal{L}^\vee \to R^{\varphi C_2} $ so that the following diagram commutes
    \begin{equation}\label{diagram:pnpic_linear_part_condition}
    \begin{tikzcd}[column sep=huge]
        \mathcal{L} \otimes_R \mathcal{L}^\vee \ar[d,"\mathrm{ev}", "\sim"']  \ar[r,"{\ell(q) \otimes \ell(q^\vee)}"] & R^{\varphi C_2} \otimes_R R^{\varphi C_2} \ar[d,"\mathrm{multiplication}"] \\
        R \ar[r,"\mathrm{given}"] & N_R   
    \end{tikzcd}
    \end{equation} 
\end{remark}

\begin{lemma}
    \label{lemma:connectivity_of_qoppa_at_the_unit}
    Let $(R,\Qoppa)$ be a connective Poincaré ring spectrum\Viktor{define connectivity and make conditions here precise. As stated this works for R and C connective. More precisely, $conn(\Qoppa(\Sigma^n R))\leq \min(conn(\Sigma^{-2n}R),conn(\Sigma^{-n} C))$}. Then, for any integer $n$, the spectrum $\Qoppa(\Sigma^n R)$ is $(-2n)$-connective.
\end{lemma}

\begin{proof}
    This follows from the fiber sequence $$(\Sigma^{-2n} R)_{hC_2}\rightarrow \Qoppa(\Sigma^{n} R)\rightarrow \hom_R(\Sigma^n R,C)\simeq \Sigma^{-n} C.$$\Viktor{write out details}
\end{proof}

\begin{remark}
    \label{remark:poincare_picard_is_2-torsion}
    The functor $\Picp:\CAlgp\rightarrow \CAlg^{\gp} (\mathcal{S})$ preserves certain structures. Let $A$ be a Poincaré ring spectrum. Since $A$ is a module over $(\mathbb{S},\Qoppa_u)$, the space $\Picp(A)$ is something over $\Picp(\mathbb{S},\Qoppa_u)$.\Viktor{there is no truth in here yet. Work in progress. Noah had an example using Witt vectors which showed that $\pi_0$ does not need to be 2-torsion}
\end{remark}

Since the forgetful functor $\Pn(A)\to \mathrm{Mod}_A^\omega$ is symmetric monoidal we get an induced map \[ U:\Picp(A)\to \Pic(A)\] of spectra. For a point $(\mathcal{L},q)\in \pi_0(\Picp(A))$ we will refer to $\mathcal{L}:=U(\mathcal{L},q)$ as the \textit{underlying invertible module}.  Note that the $A$-module $A^*$ is (nonequivariantly) isomorphic to $A$ via the involution, and so the fact that $\mathcal{L}\simeq \mathrm{hom}_A(\mathcal{L},A^*)$ forces $\mathcal{L}$ to be $2$-torsion. In particular we get a refined map \[U:\Picp(A)\to \Pic(A)[2]\] which factors the underlying invertible module map. 

\begin{example}
    \label{example:picard_of_the_unit}
    Let $(\mathbb{S},\Qoppa_u)$ be the universal Poincaré ring spectrum from Example \ref{example:universal_poincare_ring_spectrum}. The only $2$-torsion element of $\Pic(\mathbb{S})\simeq \mathbf{Z}$ is $\mathbb{S}$. Therefore, any element in $\Picp(\mathbb{S},\Qoppa_u)$ lies above $\mathbb{S}$ under $U$. With Remark \ref{example:universal_tate_poincare_splits_at_unit}, we conclude $\pi_0(\Picp(\mathbb{S},\Qoppa_u))\simeq \pi_0(\mathbb{S}_{hC_2}\oplus \mathbb{S}^\times)^\times \simeq(\mathbf{Z}\times\mathbf{Z/2})^\times\simeq \mathbf{Z}/2\times \mathbf{Z}/2$. \todo{V:did we mod out by isomorphisms here?}
\end{example}

\begin{remark}
One might hope that the map $\Picp(A)\to \Pic(A)[2]$ is close to an equivalence. This however is quite far from being true. Let $k$ be a finite field of characteristic $2$, and let $\mathbb{S}_{W(k)}$ be the spherical Witt vectors on $k$ in the sense of \cite[Example 5.2.7]{lurie-elliptic-2}. Then by \cite[Example 3.4]{Nikolaus-Frob} we know that $\mathbb{S}_{W(k)}$ must satisfy that the map $\phi_2:\mathbb{S}_{W(k)}\to \mathbb{S}_{W(k)}^{tC_2}$ is an equivalence where the action is trivial.

Consider now the Poincar{\'e} ring $(\mathrm{Mod}_{\mathbb{S}_{W(k)}}^\omega, \Qoppa_{\mathbb{S}_{W(k)}}^u)$ where $\Qoppa^u_{\mathbb{S}_{W(k)}}$ is the Tate Poincar{\'e} structure. We have that $\pi_0(\Pic(\mathbb{S}_{W(k)}))\cong \mathbb{Z}$ and is generated by $\Sigma \mathbb{S}_{W(k)}$. To see this note that for $\mathcal{L}$ an invertible module over $\mathbb{S}_{W(k)}$, $\mathcal{L}$ must be bounded below since otherwise it would not be perfect. Then for $\pi_n(\mathcal{L})$ its bottom homotopy group, we have that $\pi_n(\mathcal{L}/2)\cong k$ since it must be an invertible $k$-module and $k$ is a field. Thus we get a map $\mathbb{S}^{n}\to \mathcal{L}$ lifting a generator of $k$, and by adjunction an $\mathbb{S}_{W(k)}$-module map $\Sigma^{n}\mathbb{S}_{W(k)}\to \mathcal{L}$ which on $\pi_n((-)/2)$ gives an isomorphism $k\cong k$. Therefore \[\mathbb{S}_{W(k)}[n]\otimes k \simeq k[n] \to k[n] \simeq \mathcal{L}\otimes k\] is an equivalence, where the equivalence $k[n]\simeq \mathcal{L}\otimes k$ follows from the fact that base change preserves invertible objects. The map $\mathbb{S}_{W(k)}[n]\to\mathcal{L}$ is then a $k$-local, and therefore an $\mathbb{F}_p$-local, equivalence. Both sides are connective and $p$-complete so it follows that the map $\mathbb{S}_{W(k)}[n]\to \mathcal{L}$ is an equivalence.\Noah{There is probably a reference for this fact, I'll look around for one.}

Thus $\pi_0(\Pic(\mathbb{S}_{W(k)}))=0$. On the other hand, we have that the unit map $\mathbb{S}_{W(k)}\to \Qoppa^u_{\mathbb{S}_{W(k)}}(\mathbb{S}_{W(k)})$ is split by the map $\Qoppa^u_{\mathbb{S}_{W(k)}}(\mathbb{S}_{W(k)})\to \mathbb{S}_{W(k)}^{\phi C_2}=\mathbb{S}_{W(k)}$. Consequently $\pi_0(\Qoppa^u_{\mathbb{S}_{W(k)}}(\mathbb{S}_{W(k)}))\cong \pi_0(\mathbb{S}_{W(k)}\oplus (\mathbb{S}_{W(k)})_{hC_2})\cong W(k)\times W(k)$. As a ring this is $W_2(W(k))$, and in order for $q\in W_2(W(k))$ to induce a Poincar{\'e} structure we must have that $q\in W_2(W(k))^\times \cong W(k)^\times \times W(k)^\times$. 

We then have that $\pi_0(\Picp(\mathbb{S}_{W(k)}))\cong W(k)^\times \times W(k)^\times/H$ where $H$ is the subgroup of Poincar{\'e} structures $q$ on $\mathbb{S}_{W(k)}$ which are identified by some automorphism $f:\mathbb{S}_{W(k)}\to \mathbb{S}_{W(k)}$. By the defining property of spherical Witt vectors there is an equivalence $\mathrm{Maps}_{\mathrm{CAlg}}(\mathbb{S}_{W(k)}, \mathbb{S}_{W(k)})\simeq \mathrm{Maps}_{Perf}(k,k)=\mathrm{Gal}(k/\mathbb{F}_2)$ and the action on $W(k)^\times \times W(k)^\times$ is given by $g\in \mathrm{Gal}(k/\mathbb{F}_2)$ acts via $W(g)\times W(g)$. Consequently \[\pi_0(\Picp(\mathbb{S}_{W(k)}))\cong (W(k)^\times \times W(k)^\times) /\mathrm{Gal}(k/\mathbb{F}_2)\] which even for $k=\mathbb{F}_2$ is not zero and in fact not even $2^\infty$-torsion. 
\end{remark}

In the usual Picard spectrum one has the relationship $\pic = B\mathbb{G}_m$, where $\mathbb{G}_m$ is the spectral algebraic group scheme sending a ring spectrum $E$ to the spectrum of $E$-linear equivalences of $E$ $\mathrm{gl}_1E:=\mathrm{Aut}_E(E)$.\footnote{Normally the automorphism space of an object is only $\mathbb{A}_\infty$, but as the unit in a symmetric monoidal category, the automorphisms of $E$inherit a canonical and in fact functorial $\mathbb{E}_\infty$ structure and this construction makes sense.} Equivalently $\mathbb{G}_m$ is the affine groupscheme given by $\mathbb{G}_m=\mathrm{Sp}\textrm{\'et}(\mathbb{S}\{x^{\pm 1}\})$, where $\mathbb{S}\{x^{\pm 1}\}$ is the free $\mathbb{E}_\infty$ ring on the $\mathbb{E}_\infty$ space $\mathbb{Z}$. This relationship between $\pic$ and $\mathbb{G}_m$ has many important applications, for example relating the higher homtopy groups of $\pic(A)$ with those of $A$. We will spend the rest of this section on establishing such an equivalence in the Poincare setting.

\begin{construction}
	The underlying $\mathbb{E}_\infty$ ring of $\gmq$ will again be $\mathbb{S}\{x^{\pm 1}\}$, but in order to promote this ring to a Poincare ring it will be helpful to write it as \[\mathbb{S}\{x^{\pm 1}, y^{\pm 1}\}\otimes_{\mathbb{S}\{z^{\pm 1}\}}\mathbb{S}\] where the map $\mathbb{S}\{z^{\pm 1}\}\to \mathbb{S}\{x^{\pm 1}, y^{\pm 1}\}$ is induced by $z\mapsto xy$, This ring naturally lifts to a Borel $C_2$-ring given by $C_2$ swaps $x$ and $y$ and does nothing to $z$. Now take $\gmq$ to be the Poincar{\'e} ring with underlying Borel $C_2$ structure as described above and geometric fixed points $(\gmq)^{\phi C_2}=\mathbb{S}$ and the map $(\gmq)^{\phi C_2}\to (\gmq)^{tC_2}$ given by the unit map. Endowing $(\gmq)^{\phi C_2}$ with the $\gmq$-module structre given by $x,y\mapsto 1$, it remains to show that the unit map $(\gmq)^{\phi C_2}\to (\gmq)^{tC_2}$ factors the Tate valued Frobenius $\gmq\to (\gmq)^{tC_2}$ in order to promote $\gmq$ to a Poincar{\'e} ring.
	
	By construction of $\gmq$ this amounts to showing that on $\pi_0$ the Tate valued Frobenius sends $x,y\mapsto 1$ in $\pi_0((\gmq)^{tC_2})$. This map sends both $x$ and $y$ to $xy\in \pi_0((\gmq)^{tC_2})$. These are equal to $1$ in $\pi_0((\gmq)^{tC_2})$ since the functor $(-)^{tC_2}$ is lax-monoidal so $(\gmq)^{tC_2}$ is a modules over $\mathbb{S}\{x^{\pm 1}, y^{\pm 1}\}^{tC_2}\otimes_{\mathbb{S}\{z\}^{tC_2}}\mathbb{S}^{tC_2}$ which has the image of $xy$ equal to $1$.
\end{construction}

\begin{theorem}
	There is a natural equivalence of \[\Omega \Picp(-)\simeq \gmq\] of functors on Poincar{\'e} rings.
\end{theorem}
\begin{proof}
	This amounts to identifying the space $\mathrm{Aut}_{\mathrm{Pn(\mathrm{Mod}_A)}}(A,u)$ functorially, where $(A,u)$ is the Poincar{\'e} object $A$ with bilinear form given by the unit map $\mathbb{S}\to \Qoppa_A(A)$. Note that any automorphism of Hermetian objects will automatically be Poincar{\'e} and so we may instead describe the automorphisms as a Hermetian object. We then have that $\mathrm{He}(\mathrm{Mod}_A)\to \mathrm{Mod}_A$ is a cocartesian fibration by definition, and classified by the functor which takes a module $M$ to the groupoid $\Omega^\infty \Qoppa_A(M)$. Thus we get that $\mathrm{Aut}_{\mathrm{He}(\mathrm{Mod}_A)}((A,u))$ is exactly the fiber of the map \[\mathrm{Aut}_{\mathrm{Mod}_A}(A)\to \Qoppa_A(A)\] or in other words an automorphism $(A,u)\to (A,u)$ is the data of an automorphism $a\in \mathrm{Aut}(A)$ together with a path $q:u\mapsto a^*u$ in $\Omega^{\infty +1}\Qoppa_A(A)$.  
	
	There is a natural transformation $\gmq(-)\to \Omega\Picp(-)$ given as follows: we get a map $\gmq((\mathrm{Mod}_A, \Qoppa_A))\to \mathrm{Aut}_A(A)$ given by forgetting the Poincar{\'e} structure everywhere, and so it is enough to see that on $\pi_0$ the automorphisms of $A$ coming from $\gmq$ preserve $u$. By using the linear and quadratic decomposition of $\Qoppa_A$, for an element $a\in \pi_0(A)^\times$ send $u$ to $u$ is must be sent to $1\in \pi_0(A^{\phi C_2})$ and must act by $1$ on $A^{hC_2}$. By the following Lemma this second condition is equivalent to $a\sigma(a)\in \pi_0(A)^\times$ being equal to $1$, but then these two conditions are exactly describing a map out of $\gmq$ as desired. 
	
	Consequently we have a comparison map $\gmq(\mathrm{Mod}_A, \Qoppa_A)\to \Omega\Picp(\mathrm{Mod}_A,\Qoppa_A)$, and the above argument in fact shows that this map is an equivalence on $\pi_0$. To finish the argument, note that the pushout description of $\gmq$ induces a pullback of mapping spaces 
	\[
	\begin{tikzcd}
		\gmq(\mathrm{Mod}_A, \Qoppa_A) \ar[d] \ar[r] & \mathrm{Maps}_{\CAlg(\mathrm{Sp}^{C_2})}(\mathbb{S}\{x^{\pm 1}, y^{\pm 1}\}, A)\simeq \mathrm{gl}_1(A)\ar[d]\\
		* \ar[r] & \mathrm{Maps}_{\CAlg(\mathrm{Sp^{C_2}})}(\mathbb{S}\{z\},A)\simeq \Omega^\infty\Qoppa_A(A) 
	\end{tikzcd}
	\] which finishes the proof.
\end{proof}

\begin{lemma}
	Let $A\in \mathrm{CAlg}(\mathrm{Sp}^{BC_2})$ and $s\in \pi_0(A)^\times$. Then $a\sigma(a)=1$ in $\pi_0(A)$ if and only if $(a\otimes a)^*$ acts by $1$ on $\pi_0(A^{hC_2})=\pi_0(\mathrm{Hom}_{A\otimes A}(A\otimes A, A)^{hC_2})$.
\end{lemma}
\begin{proof}
	The only if direction follows from the fact that the evaluation map $\mathrm{Hom}_{A\otimes A}(A\otimes A, A)\to A$ is an $A\otimes A$-module map. Now suppose that $a\sigma(a)=1$ in  $A$. Then before taking homotopy fixed points the induced map $a^*=id$ because $A$ is $\mathbb{E}_\infty$.\footnote{Or just $\mathbb{E}_2$.} 
\end{proof}

\subsection{prime factorization and the picard group of hearts}\Viktor{still in development}
We establish a Poincaré analogue of Fausk's result which describes the picard group of the derived category of a scheme $X$ in terms of connected components of $X$ and the classical picard group of $X$.

Let $X$ be a simplicial set. The set $\pi_0(X)$ is the set of connected compoents of $X$, i.e. simplicial subsets which are connected and form $X$ via a coproduct. In other words, the functor $\pi_0$ records a unique and maximal decomposition of $X$ into coproducts. To establish the the result mentioned above, we study the dual analogue of connected components in the sense of a maximal decomposition of $X$ into products. %Later on, we will generalize this to arbitrary symmetric monoidal structures in a given $\infty$-category.

\begin{definition}\label{definition:factor}
    Let $X$ be a simplicial set. We call a map of simplicial sets $X\rightarrow Y$ a factor of $X$, if there is an isomorphism $X\simeq Y_\times Z$, for some simplicial set $Z$, such that $Y \times Z\simeq X\rightarrow Y$ is a structure map of the given product.
\end{definition}

\begin{definition}\label{definition:prime_indecomposable}
    Let $X$ be a simplicial set. We say that $X$ is prime, or indecomposable, if it is nonempty and every factor of $X$ is isomorphic to  either $\Delta^0$ or $X$. We let $\operatorname{prin}(X)$ denote the set of prime factors of $X$.
\end{definition}

\begin{proposition}\label{proposition:prime_factorization}
    Let $X_\bullet$ be a simplicial set, then $X_\bullet$ is the product of its prime factors.
\end{proposition}
\begin{proof}
    \Viktor{}
\end{proof}

\begin{proposition}
    Let $X$ be a simplicial set. Then $\operatorname{prim}(X)\simeq \operatorname{prim}(X^\simeq)$.
\end{proposition}
\begin{proof}
    Let $f:X\rightarrow Y$ be a weak equivalence of simplicial sets. Then $f^\simeq: X^\simeq\rightarrow Y^\simeq$ is a weak equivalence of spaces. \Viktor{}
\end{proof}

\begin{proposition}
    Let $X$ be a simplicial set. Then $\operatorname{prim}(X)\simeq \pi_0(\operatorname{Spec}(X)),$ where $\operatorname{Spec}(X^\simeq)$ is the Balmer spectrum of $X$ with respect to the symmetric monoidal structure given by cartesian product.
\end{proposition}
\begin{proof}
    \Viktor{}
\end{proof}

\begin{remark}
    Let $R$ be a commutative ring. Then the scheme $\operatorname{Spec}(R)$ is isomorphic to the Balmer spectrum of $\operatorname{Mod}^\heartsuit_R$. When we view $R$ as a discrete simplicial set, we thus have $$\operatorname{prim}(\operatorname{Mod}_R^\heartsuit)\simeq \pi_0(\operatorname{Spec}(R)).$$ \Viktor{does this need a proof?}
\end{remark}

\begin{definition}
    Let $X$ be a prime simplicial set. A $c$-structure on $X$ is a map of simplicial sets $X\rightarrow \mathbf{Z}$ satisfying (todo). Let $Y$ be a simplicial set, then a $c$-structure on $Y$ is a product of $c$-structures on each of its prime components. We write $X_{\geq n}$ for the homotopy pullback of $\mathbf{Z}_{\geq n}$ along $c$, $X_{\leq n}$ for the homotopy pullback of $\mathbf{Z}_{\leq n}$ along $c$, and  $X^\heartsuit$ for the pullback of $X_{\leq n}$ along $X_{\geq n}\rightarrow X$. \Viktor{when X is a stable infinity category and prime, then a c-structure should be a t-structure on it}
\end{definition}

\begin{theorem}
    Let $X$ be a prime simplicial set and $c: X\rightarrow \mathbf{Z}$ a c-structure.  Then we have a fiber sequence of monoids \Viktor{what kind exactly}
>$$X^\heartsuit\rightarrow X^\simeq \rightarrow \mathbf{Z}.$$
\end{theorem}
\begin{proof}
    \Viktor{}
\end{proof}

\begin{corollary}[Fausk]
    Let $R$ be a discrete ring. Then we have a short exact sequence:
>$$0\rightarrow \operatorname{Pic}(\operatorname{Mod}(R)^\heartsuit) \rightarrow\pi_0(\operatorname{Pic}(\operatorname{Mod}(R)))\rightarrow H^0(\operatorname{Spec}(R);\mathbf{Z})\rightarrow 0.$$
\end{corollary}
\begin{proof}
    \Viktor{apply pic to the previous sequence and take $\pi_0$}
\end{proof}

\subsection{Hermitian line bundles}
\begin{definition}
    Let $ R $ be a commutative discrete ring with a $ C_2 $-action $ \sigma \colon R \to R $. 
    Write $ \sigma_*R $ for the $ R $-module with underlying abelian group $ R $ and action $ r \cdot m = \sigma(r) \cdot m $. 
    Let $ M $ be an $ R $-module. 
    Define the \emph{adjoint} of $ M $ to be the $ R $-module $ M^\dag := \hom_R \left(M, \sigma_* R\right)$. \Lucy{see 3.8-3.11 in \href{https://arxiv.org/pdf/2009.09124}{this paper}} 
    Also recall that there is a canonical $ R $-linear isomorphism $ \left(M^\dag\right)^\dag \simeq M $. 
    Note that given two $ R $-modules $ M, N $, the adjoint satisfies $ M^\dag \otimes N^\dag \simeq (M\otimes N)^\dag $. 
    Let $ I $ be a projective $ R $-module (in particular, there is a canonical identification $ (I^\dag)^\dag \simeq I $). 
    A \emph{$ \sigma $-hermitian form on $ I $} is an $ R $-linear isomorphism $ \varphi \colon I \xrightarrow{\sim} I^\dag $ so that $ \varphi^\dag = \varphi $. 
\end{definition}
\begin{observation}\label{obs:tensoring_hermitian_forms}    
    Let $ R $ be a commutative discrete ring with a $ C_2 $-action $ \sigma \colon R \to R $. 
    Given two discrete $ R $-modules $ M, N $ equipped with $ \sigma $-hermitian forms $ \varphi, \psi $, respectively, $ \varphi \otimes \psi $ defines a $ \sigma $-hermitian form on $ M \otimes_R N $. 
    Using the canonical isomorphism mentioned above, if $ \varphi $ is a $ \sigma $-hermitian form on $ M $, then $ \varphi^\dag $ induces a $ \sigma $-hermitian form on $ M^\dag $.  
    Finally, observe that $ R $ has a canonical $ \sigma $-hermitian form which is the adjoint of the map $ R \otimes R \to R $, $ r \otimes s \mapsto r \sigma(s) $. 
\end{observation}  
\begin{definition}
    Let $ R $ be a commutative discrete ring with a $ C_2 $-action $ \sigma \colon R \to R $. 
    Define the \emph{hermitian Picard group of $ R $}\Lucy{workshop the name later} to have underlying set consisting of pairs $ (I, \varphi) $ where $ I $ is an invertible $ R $-module and $ \varphi $ is a $ \sigma $-hermitian form on $ I $. 
    
    By Observation \ref{obs:tensoring_hermitian_forms}, this set inherits a group structure. 
    We write $ \mathrm{hPic}(R) $ for the group of $\sigma$-hermitian line bundles on $ \Spec R $.  
\end{definition}     
\begin{theorem}\label{theorem:Poincare_Pic_of_fixpt_Mackey_functor}
    Let $ R $ be a discrete commutative ring with a $ C_2 $-action $ \sigma \colon R \xrightarrow{\sim} R $ via ring maps. 
    Regard $ R $ as a Poincaré ring via Example \ref{ex:fixpt_Mackey_functor}. 
    Then there is a split short exact sequence of abelian groups
    \begin{equation*}
        0 \to \mathrm{hPic}(R) \to \pi_0\pnpic(\underline{R}^\sigma) \to C_{C_2}(\Spec R, \ZZ^{-}) \to 0
    \end{equation*}
    where $ R $ is endowed with the genuine symmetric Poincaré structure and $ \ZZ^{-} $ is endowed with the $ C_2 $-action given by multiplication by $ -1 $ and $ C_{C_2} $ denotes continuous functions which are moreover $ C_2 $-equivariant. 
    Moreover, forgetting the hermitian form (resp. forgetting the $ C_2 $-action) induces a commutative diagram
    \begin{equation*}
    \begin{tikzcd}
        0 \ar[r] & \mathrm{hPic}(R) \ar[r]\ar[d] & \pi_0\pnpic(R) \ar[r] \ar[d] & C_{C_2}(\Spec R, \ZZ^{-}) \ar[r] \ar[d] & 0 \\
        0 \ar[r] & \mathrm{Pic}^{\mathrm{cl}}(R) \ar[r] & \pi_0 \mathrm{Pic}\left(\perf_R\right) \ar[r] & C(\Spec R, \ZZ) \ar[r] & 0 
    \end{tikzcd}    
    \end{equation*}
    where the bottom row is that of \cite[Theorem 3.5]{MR1966659}. 
\end{theorem}
\begin{proof} 
    An object of $ \pi_0 \pnpic(R) $ is a pair $ (I, q) $ where $ I $ is an invertible $ R$-module and $ q $ is a point in $ \pi_0 \Omega^\infty \Qoppa_{R^{gs}}(I) $. 
    By the proof of \cite[Theorem 3.5]{MR1966659}, $ I $ induces a continuous map $ \Psi(I) \colon \Spec R \to \ZZ $. 
    Write $ \sigma $ for the involution on $ R $. 
    Now $ q $ in particular induces an equivalence $ q \colon I \xrightarrow{\sim} I^\dag \simeq (\sigma_*I)^\vee $. 
    For each point $ \mathfrak{p} \in \Spec R $, localizing $ q $ gives an equivalence
    \begin{equation*}
        q_{\mathfrak{p}} \colon I_{\mathfrak{p}} \xrightarrow{\sim} (\sigma_*I)^\vee_{\mathfrak{p}} \simeq \left(\sigma_*(I_{\sigma(\mathfrak{p})})\right)^\vee \,.
    \end{equation*}
    Since $ I_{\mathfrak{p}} $ is an invertible module over a local ring, \cite[Proposition 3.2]{MR1966659} implies that $ q_{\mathfrak{p}} $ induces an equivalence
    \begin{equation*}
        I_{\mathfrak{p}} \simeq R_{\mathfrak{p}}[\phi(\mathfrak{p})] \xrightarrow{\sim} \left(\sigma_*(R_{\sigma(\mathfrak{p})}[\phi(\sigma(\mathfrak{p}))])\right)^\vee \simeq (\sigma_*(R_{\sigma(\mathfrak{p})}))^\vee [-\phi(\sigma(\mathfrak{p}))] \,.
    \end{equation*}
    Since $ R $ is discrete, this implies in particular that $ \Psi(I)(\sigma(\mathfrak{p})) = -\Psi(I)(\mathfrak{p}) $, i.e. that $ \Psi(I) $ is $ C_2 $-equivariant. 
    It follows immediately from \cite[Theorem 3.5]{MR1966659} that $ \Psi $ is a homomorphism and that an element of the kernel of $ \Psi $ lifts to $ \mathrm{hPic}(R) $. 

    Now consider a $ C_2 $-equivariant map $ g \colon \Spec R \to \ZZ $. 
    As in \emph{loc. cit.}, the image of $ g $ is finite and $ C_2 $-invariant, say $ \{n_1, -n_1, \ldots, n_m, -n_m\} $ or $ \{0, n_1, -n_1, \ldots, n_m, -n_m\} $ for some $ n_i \neq 0 $. 
    As in \emph{loc. cit.}, the disjoint subsets $ U_{\pm n_i} := g^{-1}(\pm n_i) $ correspond to an orthogonal basis of idempotents $ e_{U_{\pm n_i}} $ in $ R $. 
    Since $ g $ is $ C_2 $-equivariant with respect to the sign action on $ \ZZ $, we have $ \sigma(U_{n_i}) = U_{-n_i} $. 
    Moreover, it follows from Lemma 3.4 \emph{ibid.} that $ \sigma(e_{U_{n_i}}) = e_{U_{-n_i}} $. 
    Consider the $ R $-module $ \Phi(g) := \bigoplus_{i=1}^m \left(e_{U_{n_i}}R[n_i] \oplus e_{U_{-n_i}}R[-n_i] \right) $. 
    Observe that $ \left(e_{U_{-n_i}}R[-n_i]\right)^\dag = \hom_R(e_{U_{-n_i}}R[-n_i], \sigma_* R)= \hom_R\left(\sigma_*(e_{U_{-n_i}}R), R\right)[n_i] = \hom_R\left(e_{U_{n_i}}R, R\right)[n_i] $. 
    Finally, we claim that there is a canonical $ \sigma $-hermitian form $ q_g \in \Omega^\infty \Qoppa_{R^{\mathrm{gs}}}(\Phi(g)) $ whose adjoint $ q_g^\dag \colon \Phi(g) \xrightarrow{\sim} \Phi(g)^\dag $ corresponds to the identity. 
    That $ q_g $ defines a point of $ \hom_{R^{\otimes 2}}(\Phi(g)^{\otimes 2}, R)^{hC_2} $ is evident. 
    Observe that to give a lift of $ q_g $ to $ \Qoppa_{R^{\mathrm{gs}}}(\Phi(g)) = \hom_{N^{C_2}R}\left(N^{C_2}\Phi(g), R\right) $ is equivalent to giving a commutative diagram
    \begin{equation}\label{diagram:lifting_sym_form_to_gen_sym_form}
    \begin{tikzcd}
        \Phi(g) \otimes_R R^{\varphi C_2} \ar[d] \ar[r,dashed,"{\exists ?}"] & R^{\varphi C_2} \ar[d] \\
        \left(\Phi(g)^{\otimes 2}\right)^{tC_2} \ar[r,"{q_g^{tC_2}}"] & R^{tC_2}
    \end{tikzcd} \,.
    \end{equation} 
    Let us write $ \eta \colon R \to \pi_0 R^{\varphi C_2} $ for the ring map induced by the structure map. 
    Since $ R $ is a $ C_2 $-$ \EE_\infty $-ring, $ \eta $ is invariant with respect to the given action on $ R $ and the trivial action on $ \pi_0 R^{\varphi C_2} $. 
    Consider $ e_{U_{n_i}} $ an idempotent corresponding to an element of the image of $ g $ so that $ n_i \neq 0 $. Then
    \begin{equation*}
    \begin{split}
        \eta(e_{U_{n_i}}) &= \eta(e_{U_{n_i}})^2 \qquad \text{ ring maps preserve idempotents } \\
        &= \eta(e_{U_{n_i}}) \cdot \eta(e_{U_{-n_i}}) \qquad \text{ $C_2$-invariance of }\eta \\
        &= \eta(e_{U_{n_i}}e_{U_{-n_i}}) \qquad \text{ $ \eta $ is a ring map } \\
        &= 0 \qquad \text{ orthogonality and }n_i \neq 0 \,.  
    \end{split}
    \end{equation*}
    In particular, if $ 0 $ is not in the image of $ g $, $ \Phi(g) \otimes_R R^{\varphi C_2} \simeq 0 $ and (\ref{diagram:lifting_sym_form_to_gen_sym_form}) commutes vacuously. 
    If $ 0 $ is in the image of $ g $, then $ e_{U_0}R $ is a discrete/projective $ e_{U_0}R $-module and $ q_g $ evidently defines a genuine hermitian form on $ e_{U_0}R $ (compare \cite[Remark 4.2.21]{CDHHLMNNSI}). 

    Thus, $ g \mapsto (\Phi(g), q_g) $ defines a splitting of $ \Psi $ which agrees with the splitting constructed in \cite[Theorem 3.5]{MR1966659} on underlying objects. 
\end{proof}

\subsection{Poincaré structures on schemes with involution}
Let $ X $ be a scheme with an involution $ \sigma \colon X \xrightarrow{\sim} X $. 
We want to introduce a Poincaré structure $ \Qoppa $ on $ \perf(X) $ so that the duality is given by $ E \mapsto E^\vee \otimes \sigma_*(\mathcal{O}_X) $ (contrast with \S3 of \href{https://arxiv.org/abs/2402.15136}{this paper}). 

Let $ \mathcal{C} $ be a stably symmetric monoidal $\infty $-category with an involution, i.e. an exact autoequivalence $ \sigma \colon \mathcal{C} \xrightarrow{\sim} \mathcal{C} $ and a functor $ BC_2 \to \EE_\infty \mathrm{Alg} \Catex $ sending $ * \mapsto \mathcal{C} $ and a generator of $ \mathrm{End}_{BC_2}(*) \simeq C_2 $ to $ \sigma $. 
Then $ \sigma $ induces a $ C_2 $-action on the $ \infty $-groupoid of $ \otimes $-invertible objects $ \Pic(\mathcal{C}) $. 
Let $ L \in \Pic(\mathcal{C})^{hC_2} $ be a homotopy fixed point of this action. 
In other words, $ L $ is endowed with the choice of an equivalence $ \phi \colon L \simeq \sigma(L) $, a homotopy from $ \sigma(\phi) \circ \phi $ to the identity on $ L $, and higher coherences. 

Consider a functor $ f \colon \mathcal{C}^\op \to \Spectra $ which is $ C_2 $-equivariant with respect to the $ \sigma $-action on $ \mathcal{C} $ and the trivial action on $ \Spectra $. 
In particular, the ``$ C_2 $-equivariance'' of $ f $ is additional data: for each $ x \in \mathcal{C} $, an equivalence of spectra $ c_x \colon f(x) \simeq f(\sigma(x)) $ which is natural in $ x $, a homotopy from $ c_{\sigma(x)} \circ c_x $ to $ \mathrm{id}_x $, and higher coherences. 
\begin{lemma}\label{lemma:maps_into_fixed_object_are_equivariant}
    Let $ \mathcal{C} $ be a stably symmetric monoidal $\infty $-category with an involution, and suppose given $ L \in \Pic(\mathcal{C})^{hC_2} $ a homotopy fixed point of this action. 
    Then the functor $ \hom_{\mathcal{C}}(-, L) $ promotes canonically to a $ C_2 $-equivariant functor $ \mathcal{C}^\op \to \Spectra $ in the sense of the previous paragraph. 
\end{lemma}
\begin{proof}
    Note that the Yoneda embedding $ y \colon \mathcal{C} \to \mathrm{Fun}\left(\mathcal{C}^\op, \Spectra\right) $ is equivariant with respect to the given action on $ \mathcal{C} $ and the action of $ C_2 $ on the functor category via $ F \mapsto \sigma^*F = F \circ \sigma $. \Lucy{This may be `overkill,' but this is true because $ \mathcal{C} $ can be regarded as a $ \mathcal{O}^\op_{C_2} $-parametrized $\infty$-category (with empty fiber over $ C_2/C_2 $). Then there is a parametrized Yoneda embedding. }
    Now since $ \Pic(\mathcal{C}) \subseteq \mathcal{C} $ induces $ \Pic(\mathcal{C})^{hC_2} \to \mathcal{C}^{hC_2} $, we may take the image of $ L $ under the Yoneda embedding: $ y(L) \in \mathrm{Fun}\left(\mathcal{C}^\op, \Spectra\right)^{hC_2} \simeq \mathrm{Fun}_{C_2}\left(\mathcal{C}^\op, \Spectra\right) $. 
\end{proof}
Now fix a presentably symmetric monoidal $ \infty $-category $ \mathcal{D} $, and regard it as having the trivial $ C_2 $-action. 
Recall the $ \mathrm{Fin}_* $-cartesian fibration $ \left(\Cat^{BC_2}_{\op//q^*\mathcal{D}} \right)^{\otimes} \simeq \left(\left(\Cat_{\op//\mathcal{D}}\right)^{BC_2} \right)^{\otimes}\to (\Cat^{BC_2})^\times $ from \cite[p. 13]{CHN2024}. 
Set 
\begin{equation*}
    \mathcal{W}_{\mathcal{D}}^\otimes := \EE_\infty \mathrm{Alg} \left(\Cat^{BC_2}\right)^\times \times_{(\Cat^{BC_2})^\times} \left(\Cat^{BC_2}_{\op//q^*\mathcal{D}} \right)^{\otimes}
\end{equation*}
This is a $ \mathrm{Fin}_* $-cartesian fibration classified by the lax symmetric monoidal functor
\begin{equation}\label{eq:functor_classifying_eqvt_functors}
\begin{split}
     \EE_\infty \mathrm{Alg}(\Cat)^{BC_2} \to \Cat \\
     \mathcal{C}^\otimes \mapsto \Fun_{C_2}(\mathcal{C}^\op, q^* \mathcal{D}) \,.
\end{split}
\end{equation} 
In particular, an object of the underlying $ \infty $-category of $ \mathcal{W}_{\mathcal{D}}^\otimes $ is a pair $ (\mathcal{C}, f) $ where $ \mathcal{C} $ is a symmetric monoidal $ \infty $-category with a $ C_2 $-action (via a symmetric monoidal functor) and $ f \colon \mathcal{C}^\op \to q^*\mathcal{D} $ is a $ C_2 $-equivariant functor. 
\begin{construction}\label{cons:Poincare_structure_from_equivariant_functor}
    Given a $ C_2 $-equivariant functor $ f \colon \mathcal{C}^\op \to \mathcal{D} $, we may regard the data of the $ C_2 $-equivariance of $ f $ as a commutative diagram
    \begin{equation*}
    \begin{tikzcd}
        \widetilde{\mathcal{C}^\op} \ar[r, "{\widetilde{f}}"] \ar[d] & \mathcal{D} \times BC_2 \ar[d] \\
        BC_2\ar[r,equals] & BC_2
    \end{tikzcd}
    \end{equation*}
    where the vertical maps are cocartesian fibrations and the restriction of $ \widetilde{f} $ to the fiber over the point $ * \in BC_2 $ recovers $ f $. 
    The diagram induces a map on cocartesian sections
    \begin{equation*}
        \overline{f} \colon \Fun_{BC_2}^{\mathrm{cocart}}(BC_2, \widetilde{\mathcal{C}^\op}) \to \mathcal{D}^{BC_2} \,.
    \end{equation*}
    Now if $ \mathrm{C} $ is a symmetric monoidal $ \infty $-category, we can associate to $ f $ the composite
    \begin{equation*}
        T_f \colon \mathcal{C}^\op \xrightarrow{x \mapsto x \otimes \sigma(x)} \Fun_{BC_2}^{\mathrm{cocart}}(BC_2, \widetilde{\mathcal{C}^\op}) \xrightarrow{\overline{f}} \mathcal{D}^{BC_2} \,. 
    \end{equation*}
    Finally, if $ \mathcal{D} $ admits $ BC_2 $-indexed limits, we can take homotopy fixed points of $ C_2 $-objects in which case we define the functor $ \Qoppa_f^s \colon \mathcal{C}^\op \to \mathcal{D} $ as\Lucy{I'm just using the same notation as Harpaz-Nardin-Shah here, but $(-)^s$ is maybe a little weird because it should be `hermitian,' not `symmetric.'} the composite
    \begin{equation*}
         \Qoppa_f^s \colon \mathcal{C}^\op \xrightarrow{T_f} \mathcal{D}^{BC_2} \xrightarrow{(-)^{hC_2}} \mathcal{D} \,. 
    \end{equation*}
\end{construction}
\begin{observation}\label{obs:cross_effect_twisted_poincare_structure}
    Let $ \mathcal{C} $, $ L $, be as before. 
    Then the cross effect $ B_{L} $ of $ \Qoppa^s_L$ is given by $ B_L(x,y) = \mathrm{hom}_{\mathcal{C}}(x \otimes \sigma (y), L) $. 
\end{observation}
\begin{proposition}\label{prop:multiplicativity_of_Poincare_struct_from_eqvt_functor}
    Let $ \mathcal{D} $ be a symmetric monoidal $ \infty $-category which admits $ BC_2 $-indexed limits; endow $ \mathcal{D} $ with the trivial $ C_2 $-action.   
    Then the assignment $ (\mathcal{C}, f) \mapsto ( \mathcal{C}, \Qoppa^s_{f}) $ of Construction \ref{cons:Poincare_structure_from_equivariant_functor} assembles to form a lax symmetric monoidal functor 
    \begin{equation*}
        \mathcal{W}_{\mathcal{D}}^\otimes \to \left(\Cat^{BC_2}\right)_{\op//\mathcal{D}}^\otimes 
    \end{equation*}
    sitting in a commutative diagram
    \begin{equation}\label{diagram:mult_of_Poincare_struct_from_eqvt_functor}
    \begin{tikzcd}
         \mathcal{W}_{\mathcal{D}}^\otimes \ar[r] \ar[d] & \left(\Cat^{BC_2}\right)_{\op//\mathcal{D}}^\otimes \ar[d] \\
         \EE_\infty \mathrm{Alg}(\Cat^{BC_2})^\times \ar[r,"{\mathcal{C}^\otimes \mapsto \mathcal{C}}"] & \left(\Cat^{BC_2}\right)^\times 
    \end{tikzcd}
    \end{equation}
    in which both vertical arrows are both $ \mathrm{Fin}_* $-cartesian fibrations and cocartesian fibrations of $ \infty $-operads. 
\end{proposition}

\begin{construction}\label{cons:functor_classifying_twisted_diagonal} \Lucy{This is basically Construction 3.1.4 of \cite{CHN2024} with minor edits.}
    Consider the functors $ r \colon \EE_\infty\mathrm{Alg}(\Cat^{BC_2}) \xrightarrow{\mathcal{C}^\otimes \mapsto \mathcal{C}} \Cat^{BC_2} $ and $ p \colon \EE_\infty\mathrm{Alg}(\Cat^{BC_2}) \xrightarrow{\mathrm{forget}} \Cat \xrightarrow{q^*} \Cat^{BC_2} $ where $ q \colon BC_2 \to * $. 
    We construct a symmetric monoidal natural transformation $ \tau \colon r^\times \implies p^\times $ whose component at a given category $ \mathcal{C} $ with $ C_2 $-action $ \sigma \colon \mathcal{C} \simeq \mathcal{C}$ is the $ C_2 $-equivariant functor $ \mathcal{C} \xrightarrow{x \mapsto x \otimes \sigma(x)} p(\mathcal{C}) $. 

    Let $ \mathrm{Span}\left(\mathrm{Fin}_{C_2}^{\mathrm{free}}\right) $ be the span $ \infty $-category of finite sets with free $ C_2 $-action. 
    For an $ \infty $-category with finite products $ \mathcal{E} $, there is a natural equivalence $ \Fun^\times\left(\mathrm{Span}\left(\mathrm{Fin}_{C_2}^{\mathrm{free}}\right), \mathcal{E}\right) \simeq \mathrm{CMon}(\mathcal{E})^{BC_2} $ between product-preserving functors $ \mathrm{Span}\left(\mathrm{Fin}_{C_2}^{\mathrm{free}}\right) \to \mathcal{E} $ and commutative monoids in $ \mathcal{E} $ with $ C_2 $-action. 
    Taking $ \mathcal{E} \simeq \Cat $, we may identify the functor $ r $ as restriction along the inclusion $ i \colon BC_2 \to \mathrm{Span}\left(\mathrm{Fin}_{C_2}^{\mathrm{free}}\right) $ of the maximal subgroupoid in the full subcategory on a finite $ C_2 $-set with a single orbit. 
    On the other hand, we can identify the functor $ p $ as restriction along the map $ j \colon BC_2 \to \{*\} \xrightarrow{* \mapsto C_2} \mathrm{Span}\left(\mathrm{Fin}_{C_2}^{\mathrm{free}}\right) $. 
    Now the span $ C_2 \xleftarrow{\pi_1} C_2 \times C_2 \xrightarrow{\pi_2} C_2 $ determines a morphism in $ \mathrm{Span}\left(\mathrm{Fin}_{C_2}^{\mathrm{free}}\right) $ which is equivariant with respect to the given action on the source $ C_2 $ and the \emph{trivial action} on the target $ C_2 $. 
    This morphism determines a functor $ \Delta^{1} \times BC_2 \to \mathrm{Span}\left(\mathrm{Fin}_{C_2}^{\mathrm{free}}\right) $ whose restriction to $ \{0\} \times BC_2 $ agrees with $ i $ and whose restriction to $ \{1\} \times BC_2 $ agrees with $ j $. 
    This determines a natural transformation $ i^* \implies j^* $ of functors $ \Fun \left(\mathrm{Span}\left(\mathrm{Fin}_{C_2}^{\mathrm{free}}\right), \Cat\right) \to \Fun(BC_2, \Cat) $, and precomposing with the inclusion $ \Fun^\times \left(\mathrm{Span}\left(\mathrm{Fin}_{C_2}^{\mathrm{free}}\right), \Cat\right) \subset \Fun \left(\mathrm{Span}\left(\mathrm{Fin}_{C_2}^{\mathrm{free}}\right), \Cat\right)$ gives the desired natural transformation $ \tau \colon r \implies p $. 
    Since $ r $ and $ p $ preserve products, we may lift them to symmetric monoidal functors $ r^\times, p^\times \colon \EE_\infty\mathrm{Alg}(\Cat^{BC_2})^\times \to (\Cat^{BC_2})^\times $, and $ \tau $ refines to a symmetric monoidal natural transformation $ \tau^\times \colon r^\times \implies p^\times $.
\end{construction}
\begin{proof}
    [Proof of Proposition \ref{prop:multiplicativity_of_Poincare_struct_from_eqvt_functor}] 
    \Lucy{Pretty similar to proof of \cite[Proposition 3.1.3]{CHN2024}, the main thing is the fact (stated after proof of Lemma 3.1.1 of \emph{op. cit.}) that $ \mathrm{Fin}_* $-cartesian fibrations are classified by lax symmetric monoidal functors.}  
    Horizontally composing the natural transformation of Construction \ref{cons:functor_classifying_twisted_diagonal} with the functor \ref{cons:Poincare_structure_from_equivariant_functor} and unstraightening induces a commutative diagram
    \begin{equation}
    \begin{tikzcd}
         \mathcal{W}_{\mathcal{D}}^\otimes \ar[d] \ar[r] & \left(\Cat^{BC_2}_{\op//\mathcal{D}}\right)^\otimes \ar[d] \\ 
         \left(\Cat^{BC_2}\right)_{\op//\mathcal{D}}^\otimes \ar[r,"{\mathcal{C}^\otimes \mapsto \mathcal{C}}"] & \left(\Cat^{BC_2}\right)^\times      
    \end{tikzcd}    \,,
    \end{equation}
    where we have used that $ (q^* \mathcal{D})^{hC_2} \simeq \mathcal{D}^{BC_2} $. 
\end{proof}
\begin{definition}
    Define 
    \begin{equation*}
        \mathcal{W}_{\mathrm{ex}}^\otimes \subseteq \mathcal{W}_{\Spectra}^\otimes \times_{\EE_\infty\mathrm{Alg}(\Cat)} \EE_\infty\mathrm{Alg}(\Catex)
    \end{equation*}
    to be the full sub-operad on those colors $ (\mathcal{C}, f) $ so that $ f $ is exact. 
\end{definition}
\begin{observation}
    The commutative square (\ref{diagram:mult_of_Poincare_struct_from_eqvt_functor}) restricts to a commutative square of $ \infty $-operads 
    \begin{equation}\label{diagram:from_eqvt_functor_to_Cath}
    \begin{tikzcd}
        \mathcal{W}_{\mathrm{ex}}^\otimes \ar[r,"{(\mathcal{C},f) \mapsto (\mathcal{C}, \Qoppa_f^s)}"] \ar[d] & \Cath^\otimes \ar[d] \\
        \left(\EE_\infty\mathrm{Alg}(\Catex)^{BC_2}\right)^\otimes \ar[r] & \Catex^\otimes  
    \end{tikzcd}
    \end{equation}
    where $ \Qoppa^s_f $ was defined in Construction \ref{cons:Poincare_structure_from_equivariant_functor}. 
\end{observation}
\begin{lemma}
    Both vertical maps in \ref{diagram:from_eqvt_functor_to_Cath} are cocartesian fibrations of $ \infty $-operads. 
    In particular, $ \mathcal{W}_{\mathrm{ex}}^\otimes $ is a symmetric monoidal $ \infty $-category. 
\end{lemma}
\begin{proof}
    That $ \Cath^\otimes $ is symmetric monoidal and the right-hand projection is a symmetric monoidal functor is \cite[Theorem 5.2.7]{CDHHLMNNSI}. 
    Furthermore, by Proposition \ref{prop:multiplicativity_of_Poincare_struct_from_eqvt_functor} and base change, we have a cocartesian fibration of $ \infty $-operads:
    \begin{equation*}
       \pi \colon \left(\EE_\infty\mathrm{Alg}(\Catex)^{BC_2}\right)^\otimes \times_{\left(\Cat^{BC_2}\right)^\times} \left(\Cat^{BC_2}_{\op//\mathcal{D}}\right)^\otimes \to \left(\EE_\infty\mathrm{Alg}(\Catex)^{BC_2}\right)^\otimes 
    \end{equation*}
    so that $ \mathcal{W}_{\mathrm{ex}}^\otimes $ includes as a full sub-operad on the fiber product on the left on those colors $ (\mathcal{C}, f \colon \mathcal{C}^\op \to \Spectra) $ so that $ f $ is exact. 
    It suffices to show that if $ \alpha \colon (\mathcal{C}_i, f_i \colon \mathcal{C}_i^\op \to \Spectra)_{i \in I} \to (\mathcal{D}, f \colon \mathcal{D}^\op \to \Spectra) $ is a $ \pi $-cocartesian arrow so that $(\mathcal{C}_i, f_i \colon \mathcal{C}_i^\op \to \Spectra)_{i \in I} $ is in $ \mathcal{W}_{\mathrm{ex}}^\otimes $, then $ (\mathcal{D}, f \colon \mathcal{D}^\op \to \Spectra) $ is also in $ \mathcal{W}_{\mathrm{ex}}^\otimes $. 
    This holds because the left Kan extension of the multi-exact functor $ \Pi_i  f_i \colon \Pi_i \mathcal{C}_i^\op \to \Spectra $ along $ \Pi_i \mathcal{C}_i^\op \to \bigotimes_i \mathcal{C}_i^\op \xrightarrow{\alpha} \mathcal{D}^\op $ is exact. 
\end{proof}
As in \cite[p. 15]{CHN2024}, we can identify objects of the underlying $ \infty $-category of $ \mathcal{W}_{\mathrm{ex}}^\otimes $ with pairs $ \left(\mathcal{C}^\otimes, \sigma_\mathcal{C}, L, \lambda\right) $ where $ \mathcal{C} $ is a symmetric monoidal stable $ \infty $-category with involution $ \sigma_{\mathcal{C}} $, and $ (L, \lambda) \in \mathrm{Ind}(\mathcal{C})^{hC_2} $ is a fixed point with respect to the induced action on $ \mathrm{Ind}(\mathcal{C})$. 
\begin{recollection}
    A symmetric monoidal $ \infty $-category $ \mathcal{C} $ is said to be \emph{rigid} if every object in $ \mathcal{C} $ is dualizable. 
    \Lucy{\href{https://mathoverflow.net/a/337430}{reference} for when derived categories of connective objects over more general objects are compactly generated}
\end{recollection}
\begin{proposition}
    % Let $  \left(\mathcal{C}^\otimes, \sigma_\mathcal{C}, L, \lambda\right) \in \mathcal{W}_{\mathrm{ex}} $ be as above. 
    Suppose $ \mathcal{C} $ is a rigid stably symmetric monoidal $ \infty $-category with a $ C_2 $-action $ \sigma_{\mathcal{C}} $ via symmetric monoidal functors, and let $ (L,\lambda) \in \mathrm{Ind}(\mathcal{C})^{BC_2} $. 
    Then the hermitian structure $ \Qoppa^s_{L} $ is non-degenerate if and only if $ L $ belongs to $ \mathcal{C} $, and it is furthermore Poincaré if and only if the underlying object $ L $ is tensor-invertible in $ \mathcal{C} $. 
    In addition, if $ g \colon \mathcal{C} \to \mathcal{C}' $ is a symmetric monoidal $ C_2 $-equivariant exact functor, $ (L,\lambda) \in \mathcal{C}^{BC_2} $ and $ (L',\lambda') \in (\mathcal{C}')^{BC_2} $ are tensor-invertible, and $ g(L,\lambda) \simeq (L',\lambda') $ an equivalence in $ (\mathcal{C}')^{BC_2} $, then the induced hermitian functor $ \left(\mathcal{C}, \Qoppa^s_{L}\right) \to \left(\mathcal{C}', \Qoppa^s_{L'}\right) $ is Poincaré. 
\end{proposition}
\begin{proof}
    If the bilinear part $ B_L $ is represented by $ D \colon \mathcal{C}^\op \to \mathcal{C} $, then $ D(1_{\mathcal{C}}) = \sigma_{\mathcal{C}}(L) $; in particular, $ \sigma_{\mathcal{C}}(L) \simeq L $ is an object of $ \mathcal{C} $.  
    On the other hand, if $ L $ belongs to $ \mathcal{C} $, then for $ x , y \in \mathcal{C} $, the bilinear part $ B_L $ is $ B_L(x,y) = \hom_{\mathrm{Ind}(\mathcal{C})}\left(x \otimes \sigma_{\mathcal{C}}(y), L \right) \simeq \hom_{\mathcal{C}}\left(x \otimes \sigma_{\mathcal{C}}(y), L \right) \simeq \hom_{\mathcal{C}}\left(x , L \otimes \sigma_{\mathcal{C}}(y)^\vee \right) $. 
    The natural transformation $ \mathrm{id}_{\mathcal{C}} \to D^{\op} \circ D $ is given by $ y \to L \otimes \sigma_{\mathcal{C}} \left(L \otimes \sigma_{\mathcal{C}}(y)^\vee \right)^\vee $ induced by the adjoint of the equivalence $ L \simeq \sigma_{\mathcal{C}} (L) $, which is an equivalence if and only if $ L $ is invertible. 

    Finally, the natural transformation $ g \circ D_L \Rightarrow D_{L'} \circ g $ is $ g\left(L \otimes \sigma_{\mathcal{C}}(y)^\vee\right) \simeq g(L) \otimes g\left(\sigma_{\mathcal{C}}(y)^\vee\right) \to L' \otimes \sigma_{\mathcal{C}'}(g(y))^\vee $ which is an equivalence if and only if the map $ g(L) \to L' $ is an equivalence\Lucy{how much equivariance on $ g(L) \to L' $ is necessary?}. 
\end{proof}

\subsection{Poincaré Picard groups of schemes} 
Let $ X $ be a scheme with an involution. 
\begin{definition}
    Define a category $ \mathrm{qSch}^{C_2} $ so that
    \begin{itemize}
        \item an object of $ \mathrm{qSch}^{C_2} $ consists of the data of qcqs schemes $ X $ and $ Y $, an involution $ \lambda \colon X \to X $, and a morphism $ p \colon X \to Y $ which exhibits $ Y $ as a \emph{good quotient} of the involution on $ X $ in the sense of \cite[Remark 4.20]{azumaya_involution}. 
        \item a morphism from $ (X,\lambda, Y, p) $ to $ (Z,\nu, W, q) $ consists of a $ C_2 $-equivariant morphism $ X \to Z $ and a morphism $ Y \to W $ so that the diagram
        \begin{equation*}
        \begin{tikzcd}
            X \ar[d] \ar[r] & Z \ar[d] \\
            Y \ar[r] & W
        \end{tikzcd}
        \end{equation*}
        commutes. 
    \end{itemize}
\end{definition}
\begin{remark}
    Suppose $ (X,\lambda, Y, p) $ is an object of $ \mathrm{qSch}^{C_2} $ and $ U \to Y $ is an open subscheme. 
    Then $ (X_U, \lambda|_U, U, p|_U) $ is an object of $ \mathrm{qSch}^{C_2} $. 
\end{remark}
\begin{proposition}
    Write $ U \colon \mathrm{qSch}^{C_2} \to \mathrm{qSch} $ for the functor so that $ U(X, \lambda, Y, p) = X $.  
    The category $ \mathrm{qSch}^{C_2} $ has a symmetric monoidal structure $ \boxtimes $ so that $ U $ is symmetric monoidal, where $ \mathrm{qSch} $ is endowed with the product symmetric monoidal structure. 
\end{proposition}
\begin{proof}
    If $ X $, $ Z $ are schemes with involutions $ \lambda_X $, $ \lambda_Z $, then $ \lambda_X \times \lambda_Z $ endows $ X \times Z $ with an involution. 
    It suffices to show that $ X \times Z $ admits a good quotient, as a good quotient is a categorical quotient and is therefore unique up to isomorphism. 
    By \cite[Remark 4.20]{azumaya_involution}, a good quotient exists if and only if every $ C_2 $-orbit is contained in an affine open subscheme. 
    Consider a $ C_2 $-orbit in $ X \times Z $. 
    Its image under the projection $ \pi_1 \colon X \times Z \to X $ (resp. $ \pi_2 \colon X \times Z \to Z $) is contained in an affine open subscheme $ U \subseteq X $ (resp. $ V \subseteq Z $). 
    Thus the orbit under consideration is contained in $ U \times Z $, which is affine. 
\end{proof}
\begin{construction}\label{cons:structure_sheaf_of_Green_functors}
    Assume that $ X $ has a \emph{good quotient} $ Y $ in the sense of \cite[Remark 4.20]{azumaya_involution}. 
    We write $ p \colon X \to Y $ for the quotient map.
    Let $ j \colon \Spec A  \simeq U \subseteq Y $ be an affine open subscheme of $ Y $. 
    Because $ p $ is an affine map,\Lucy{by assumption!} the fiber product $ \Spec B := \Spec A \times_{Y} X $ is an affine open of $ X $ which is invariant under the $ C_2 $-action. 
    In particular $ \Spec B $ inherits a $ C_2 $-action from $ X $ (hence so does its ring of functions $ B$).  
    Now $ A \to B $ acquires the structure of a $C_2$-Green functor $ \underline{\mathcal{O}}(U) $. 
    Regarding $ \underline{\mathcal{O}}(U) $ as a $ C_2 $-spectrum, by the isotropy separation sequence, we have an equivalence of $ A $-modules $ \underline{\mathcal{O}}(U)^{\varphi C_2} \simeq \mathrm{cofib} (\mathrm{tr} \colon B_{hC_2} \to A) $. 
\end{construction}
\begin{lemma} \label{lemma:identify_structure_sheaf_of_Green_func}
Let $ X $ be a scheme with an involution. 
Assume that $ X $ has a \emph{good quotient} $ Y $ in the sense of \cite[Remark 4.20]{azumaya_involution}, and write $ p \colon X \to Y $ for the quotient map. 
\begin{enumerate}[label=(\roman*)]
    \item \label{lem_item:structure_sheaf_of_Green_func_is_functor} The assignment of Construction \ref{cons:structure_sheaf_of_Green_functors} lifts to a contravariant functor from (the nerve of) the category of affine opens of $ Y $ to the $\infty $-category of Poincaré rings/$ C_2 $-$ \EE_\infty $-rings/Tambara functors. 
    \item \label{lem_item:structure_sheaf_of_Green_func_is_sheaf} The presheaf $ \underline{\mathcal{O}} $ of \ref{lem_item:structure_sheaf_of_Green_func_is_functor} defines a Zariski sheaf. 
    \item \label{lem_item:structured_pushforward_is_equiv} Write $ p_* \mathcal{O}_X $ for the sheaf of $ \EE_\infty $-$ \mathcal{O}_Y $-algebras (all functors are derived). 
    Then the pushforward $ p_* $ induces an equivalence $ \mathcal{D}(X) \xrightarrow{\sim} \Mod_{p_*\mathcal{O}_X} $. 
\end{enumerate}
\end{lemma}
\begin{proof}
    Part \ref{lem_item:structure_sheaf_of_Green_func_is_functor} follows from a similar argument to \cite[Theorem 5.1]{LYang_normedrings}; functoriality follows from noting that $ \tau_{\geq 0} $ is a functor.  
    Part \ref{lem_item:structure_sheaf_of_Green_func_is_sheaf} follows from Lemma \ref{lemma:limits_of_param_alg_detected_orbitwise}. 
    To prove part \ref{lem_item:structured_pushforward_is_equiv}, consider a Zariski cover $ \{j_i \colon U_i \to Y \} $ of $ Y $ by affine opens. 
    By Zariski descent, $ \displaystyle\mathcal{D}(X) \simeq \lim_{p^*(j_i) = p \times_Y j_i \colon U_i \times_Y X \to X} \Mod_{\mathcal{O}_X(U_i \times_Y X)} $ and $ \displaystyle\Mod_{p_*(\mathcal{O}_X)} \simeq \lim_{j_i \colon U_i \to Y} \Mod_{p_*\mathcal{O}_X(U_i)} $, hence the result follows. 
\end{proof}
\begin{lemma}\label{lemma:limits_of_param_alg_detected_orbitwise}
    Let $ K $ be a simplicial set, and let $ f \colon K^{\triangleleft} \to C_2 \EE_\infty\mathrm{Alg}(\Spectra^{C_2}) $ be a diagram. 
    Then $ f $ is a limit diagram if and only if $ f^e \colon K^{\triangleleft} \to \EE_\infty\mathrm{Alg}(\Spectra) $ and $ f^{C_2} \colon K^{\triangleleft} \to \EE_\infty\mathrm{Alg}(\Spectra) $ are both limit diagrams. 
\end{lemma}
\begin{proof}
    The result follows from the observation that limits in $ \EE_\infty \mathrm{Alg} \left(\Spectra^{C_2}\right) $ are computed in $ \Spectra^{C_2} $. 
\end{proof}
\begin{construction}\label{cons:C2_mod_over_sheaf_of_Green_func}
    Let $ p \colon X \to Y $ as before. 
    Consider the composites
    \begin{equation}\label{eq:functor_classifying_C2_mod_over_sheaf_of_Green_func}
    \begin{split}
        \Mod_{\underline{\mathcal{O}}} \colon & \mathrm{Op}(Y)^\op \xrightarrow{\underline{\mathcal{O}}} C_2\EE_\infty\mathrm{Alg}\left(\Spectra^{C_2}\right) \xrightarrow{\Mod_{(-)}} \Cat   \\
        \Mod_{\underline{\mathcal{O}}}^\otimes \colon & \mathrm{Op}(Y)^\op \xrightarrow{\underline{\mathcal{O}}} C_2\EE_\infty\mathrm{Alg}\left(\Spectra^{C_2}\right) \xrightarrow{\Mod_{(-)}^\otimes} C_2\otimes\Cat \,,  
    \end{split}    
    \end{equation}
    where $ C_2\otimes\Cat $ denotes the $ \infty $-category of (small)\Lucy{bleh...cardinals} $ C_2 $-symmetric monoidal $ C_2 $-$ \infty $-categories. 
    In the notation of Construction \ref{cons:structure_sheaf_of_Green_functors}, this functor sends the affine open $ \Spec A \subseteq Y $ to the category of modules in $ C_2 $-spectra over the $ C_2 $-$ \EE_\infty $-algebra which has underlying $ C_2 $-Mackey functor $ A \to B $. 
    Define\Lucy{invent better notation later} $ \Mod_{\underline{\mathcal{O}}} $, $ \Mod_{\underline{\mathcal{O}}}^\otimes $ to be the limits in $ \Cat $, $ C_2 \otimes \Cat $, resp. of the functors in (\ref{eq:functor_classifying_C2_mod_over_sheaf_of_Green_func}). 
    In particular, if we write $ s \colon \int \Mod_{\underline{\mathcal{O}}} \to \mathrm{Op}(Y)^\op $ for the cocartesian fibration obtained by taking the Grothendieck construction on (\ref{eq:functor_classifying_C2_mod_over_sheaf_of_Green_func}), an object of $ \Mod_{\underline{\mathcal{O}}} $ is a cocartesian section of $ s $. 
    In other words, it is a choice, for each affine open $ \Spec A $ of $ Y $ (same notation as before), of a module over the $ C_2 $-$ \EE_\infty $-algebra which has underlying $ C_2 $-Mackey functor $ A \to B $ which glue compatibly. 
\end{construction} 
Observe that for each $ A \to B $, there is a quadratic norm functor $ N^{C_2} \colon \Mod_{B}(\Spectra) \to \Mod_{N^{C_2}B}\left(\Spectra^{C_2}\right) $ and a quadratic relative norm functor $ N^{C_2} \colon \Mod_{B}(\Spectra) \to \Mod_{A\to B}\left(\Spectra^{C_2}\right) $. 
\begin{construction}\label{cons:globalize_relative_norm}
    Let $ X $ be a scheme with an involution, and let $ p \colon X \to Y $ exhibit $ Y $ as a good quotient of $ X $. 
    Assume that $ p $ is affine. 
    The norm functors (resp. relative norm functors) $ N^{C_2}_e $ assemble under Construction \ref{cons:structure_sheaf_of_Green_functors} to a `global' norm functor $ N^{C_2}_Y \colon \pi_{\#}\mathcal{O}_X \Mod \to N^{C_2} \pi_{\#} \mathcal{O}_X\Mod $ (resp. relative norm functor $ N^{C_2}_Y \colon \pi_{\#}\mathcal{O}_X \Mod \to \underline{\mathcal{O}}\Mod $). 
    Moreover, these functors are quadratic. 
\end{construction}
For each affine open $ j \colon \Spec A \subseteq Y $, write $ B = \Gamma \mathcal{O}_{\Spec A \times_Y X} $, consider the composite 
% Then the functor $ N^{C_2}_Y $ is the limit over all $ \Spec A \subseteq Y $ of the functors 
\begin{equation*}
     \pi_{\#}\mathcal{O}_X \Mod \xrightarrow{j^*} \Mod_B(\Spectra) \xrightarrow{N^{C_2}} \Mod_{N^{C_2}B}(\Spectra^{C_2}) \xrightarrow{ -\otimes_{N^{C_2}B} (A \to B)} \Mod_{A \to B}(\Spectra^{C_2})\,,
\end{equation*}  
where the last map is base change along the map $ N^{C_2}B \to (A\to B) $ which is a structure map for the $ C_2 $-$ \EE_\infty $-algebra structure on $ A \to B $. 
\Lucy{todo: use effective descent/limit definition for $ \underline{\mathcal{O}} $-modules.} 
Now since quadratic functors are closed under limits \cite[Theorem 6.1.1.10]{LurHA} and $ N^{C_2}_Y $ can be written as a limit of a diagram of quadratic functors, $ N^{C_2}_Y $ is also quadratic. 
\begin{definition}
    Varying $ X \to Y $, Constructions \ref{cons:C2_mod_over_sheaf_of_Green_func} and \ref{cons:globalize_relative_norm} define a functor \Lucy{want: codomain consists of $ C_2 $-stable $ C_2 $-presentable $ C_2 $-$ \infty $-categories}
    \begin{align*}
        \left(\mathrm{qSch}^{C_2}\right)^{\op} &\to C_2 \otimes \Cat  \\
        \left(X, \lambda, Y, p \right) &\mapsto \underline{\Mod}_{\underline{\mathcal{O}}}\left(\underline{\Spectra}^{C_2}\right)
    \end{align*}
\end{definition} 
\begin{definition}
    Suppose $ \mathcal{C} $ is a $ C_2 $-stable $ C_2 $-symmetric monoidal $ C_2 $-$ \infty $-category. 
    Define a functor
    \begin{align*}
       \mathrm{eInv} \colon  C_2 \otimes \Cat^{\mathrm{ex}} &\to \Spaces \\
        \left(\mathcal{C}, \otimes\right) & \mapsto \left(\mathrm{C}^{C_2}\right)^{\simeq} \times_{(\mathrm{C}^e)^{\simeq,hC_2}} \Pic (\mathrm{C}^e)^{\simeq,hC_2} \,.
    \end{align*}
    In other words, $ \mathrm{eInv} $ sends a $ C_2 $-symmetric monoidal $ C_2 $-$ \infty $-category to the full subgroupoid of $ \mathcal{C}^{C_2} $ on those objects $ L $ so that $ L^e $ is an invertible object in $ \mathcal{C}^e $. 

    Write $ \widetilde{\mathrm{eInv}} $ for the Grothendieck construction on $ \mathrm{eInv} $. 
\end{definition}
There is a functor
\begin{equation}\label{eq:C2_cat_to_Poincare_cat}
\begin{split}
    \widetilde{\mathrm{eInv}} &\to \EE_\infty \mathrm{Alg}\left(\mathrm{Cat}^h \right) \\
    (\mathcal{C}, L) &\mapsto (\mathcal{C}^e, \mathcal{C}^e \xrightarrow{N_{\mathcal{C}}} \mathcal{C}^{C_2} \xrightarrow{\hom_{\mathcal{C}^{C_2}}(-,L)} \Spectra)
\end{split}
\end{equation}
\Lucy{Pretty sure distributive norm functors are 2-excisive (results \href{https://arxiv.org/pdf/2010.09097}{here} should generalize readily)...if not, can define the problem away.}
\begin{lemma}
    The functor of (\ref{eq:C2_cat_to_Poincare_cat}) lifts to a functor $ \widetilde{\mathrm{eInv}}  \to \Catp $. 
\end{lemma}
\Lucy{Example: Special case where $ X $ has trivial $ C_2 $ action and $ X = Y $.}
\begin{lemma}\label{lemma:line_bundle_as_C2_sheaf}
    Let $ X $ be a scheme with involution $ \sigma \colon X \xrightarrow{\sim} X $ equipped with a good quotient $ \pi \colon X \to Y $. 
    Let $ L $ be a line bundle on $ Y $. 
    Then the canonical map 
    \begin{equation}\label{eq:line_bundle_as_C2_sheaf}
        L \to \pi_{\#} \pi^* L 
    \end{equation} 
    promotes (\ref{eq:line_bundle_as_C2_sheaf}) to a sheaf of $ \underline{\mathcal{O}} $-modules on $ Y $. 
    We will write $ \underline{L} $ for (\ref{eq:line_bundle_as_C2_sheaf}). 
\end{lemma}
\begin{proof}
    Follows from naturality of the unit and the canonical identification $ \pi^* \mathcal{O}_Y $ with $ \mathcal{O}_X $. 
\end{proof}
\begin{definition}\label{defn:quadratic_functor_from_good_quotient}
    Let $ X $ be a scheme with involution $ \sigma \colon X \xrightarrow{\sim} X $ equipped with a good quotient $ \pi \colon X \to Y $. 
    Let $ L $ be a line bundle on $ Y $. 
    Define $ \Qoppa_{\sigma, L} $ to be the functor
    \begin{equation*}
        \perf_X^\op \xrightarrow{\pi_{\#}} \pi_{\#}\mathcal{O}_X\Mod^{\omega,\op} \xrightarrow{N^{C_2}} N^{C_2}\pi_{\#}\mathcal{O}_X\Mod\left(\Spectra^{C_2}\right)^\op \xrightarrow{\hom_{N^{C_2}\pi_{\#}\mathcal{O}_X}(-,\underline{L})} \Spectra \,,
    \end{equation*}
    where $ \underline{L} $ is a $ \underline{\mathcal{O}} $-module by Lemma \ref{lemma:line_bundle_as_C2_sheaf} and $ \underline{\mathcal{O}} $ is a $ N^{C_2} \pi_{\#}\mathcal{O}_X $-algebra by Lemma \ref{lemma:identify_structure_sheaf_of_Green_func}. 
    By Construction \ref{cons:globalize_relative_norm} and the fact that the composite of an exact (1-excisive) functor and an $ m$-excisive functor is $m$-excisive (see \cite[\S2.2]{Kpoly}), $ \Qoppa_{\sigma,L} $ is quadratic. 
    % \Lucy{want to show that the assignment $ (\sigma:X \to X, \pi: X \to Y, L) \mapsto (\perf_X, \Qoppa_{\sigma,L}) $ defines a functor from some category of schemes with involution (+line bundle) to the $ \infty $-category of Poincaré $ \infty $-categories.}
\end{definition}
\begin{example} 
    Suppose $ L = \mathcal{O}_Y $. 
    Then we drop $ L $ from notation and the quadratic functor $ \Qoppa_{\sigma} $ of Definition \ref{defn:quadratic_functor_from_good_quotient} takes the form
    \begin{equation*}
        \perf_X^\op \xrightarrow{\pi_{\#}} \pi_{\#}\mathcal{O}_X\Mod^{\omega,\op} \xrightarrow{N^{C_2}} N^{C_2}\pi_{\#}\mathcal{O}_X\Mod\left(\Spectra^{C_2}\right)^\op \xrightarrow{\hom_{N^{C_2}\pi_{\#}\mathcal{O}_X}(-,\underline{\mathcal{O}})} \Spectra \,.
    \end{equation*}    
\end{example}
% \begin{lemma}\label{lemma:fully_faithful_pushforward}
%     Let $\pi \colon X \to Y $ be a finite étale map. 
%     Then the canonical functor $ \Mod_{\mathcal{O}_X} \to \Mod_{\pi_{\#}\mathcal{O}_X} $ is fully faithful. 
%     \Lucy{asked JH Nov 22nd: James says that this should definitely be true, and moreover in greater generality (for instance, if $ \pi$ is faithfully flat.); try, for instance, Barr--Beck.}
% \end{lemma}
% \begin{proof}
%     Affine-locally on $ Y $, I think this is an \emph{equivalence}. 
%     Conclude by descent. 
% \end{proof}
\begin{lemma}\label{lemma:bilinear_parts_agree}
    Let $ X $ be a scheme with involution $ \sigma \colon X \xrightarrow{\sim} X $, and let $ Y $ be a good quotient of $ X $. 
    % Assume that the quotient map $ q \colon X \to Y $ is étale. \Lucy{Weakest possible assumption? \href{https://mathoverflow.net/questions/169052/are-quotients-of-affine-schemes-by-finite-groups-faithfully-flat}{relevant MO post}.}
    Let $ L $ be a line bundle on $ Y $, and let $ \Qoppa_{\sigma,L} $ be the quadratic functor on $ \perf_X $ of Definition \ref{defn:quadratic_functor_from_good_quotient}. 
    Then the bilinear part of $ \Qoppa_{\sigma,L} $ agrees with that of Observation \ref{obs:cross_effect_twisted_poincare_structure}. 
    In particular, $ \left(\perf_X, \Qoppa_{\sigma,L} \right) $ is a Poincaré $ \infty $-category. 
\end{lemma}
\begin{proof}
    By definition of the bilinear part of a quadratic functor, it suffices to show that there is an equivalence $ \hom_{\pi_{\#}\mathcal{O}_X\Mod}\left(\pi_{\#}E \otimes_{\pi_{\#}\mathcal{O}_X} \pi_{\#}E, \pi_{\#}\mathcal{O}_X\right) \simeq \hom_{\mathcal{O}_X\Mod}\left(E \otimes_{\mathcal{O}_X} \sigma^*E, \mathcal{O}_X\right) $ for any perfect complex $ E $ on $ X $. 
    This follows from Lemma \ref{lemma:identify_structure_sheaf_of_Green_func}\ref{lem_item:structured_pushforward_is_equiv}. %\ref{lemma:fully_faithful_pushforward}.  
\end{proof}
\begin{remark}
    Compare the description of the space of bilinear forms in Lemma \ref{lemma:bilinear_parts_agree} with the description of a $ \delta $-hermitian form $ H $ in \cite[p. 216]{MR1162189}. 
\end{remark}

\section{The Poincaré Brauer Group}
\label{subsection:the_poincare_brauer_group}
Let $A$ be a Poincaré ring spectrum. 
By Remark \ref{remark:poincare_ring_spectra_to_modules_with_Poincare_structure}, $ \Mod_A^\omega $ promotes to a commutative algebra object in the $\infty$-category of Poincaré $\infty$-categories $\Catp$ , and we may thus consider modules over it. 
In this section, we will use modules over Poincaré ring spectra to define derived analogues of the involutive Brauer group for Poincaré ring spectra.

Recall that a Poincaré $\infty$-category is called idempotent complete if the underlying stable $\infty$-category is idempotent complete. The full subcategory of $\Catp$ spanned by idempotent complete Poincaré $\infty$-categories is denoted by $\Catpidem$ \cite[Definition 1.3.2]{CDHHLMNNSII}.

\begin{definition}
    \label{definition:poincare_brauer_space}
    Let $A$ be a Poincaré ring spectrum. We define the \emph{Poincaré Brauer space of $A$} as $$\Brp(A):=\Pic(\Mod_A(\Catpidem)).$$
    The assignment $ A \mapsto \Brp(A) $ defines a functor
    \begin{equation*}
        \Brp \colon \CAlgp \to \CAlg^{\gp}(\Spaces)
    \end{equation*}
    valued in grouplike $ \Einfty $-spaces. 
\end{definition}

\begin{remark}
    The symmetric monoidal forgetful functor $ \Mod_A(\Catpidem) \to \Mod_A(\Cat^{\mathrm{ex}}_\infty) $ induces a map $ \Brp(A) \to \Br(A) $ of grouplike $ \Einfty $-spaces, where $ \Br(A) $ is the Brauer space $ \mathrm{br}_{\mathrm{alg}}(A) $ of \cite[1154-1155]{MR3190610}. 
\end{remark}
\begin{proposition}
    Let $A$ be a Poincaré ring spectrum. Then we have a canonical equivalence $$\Omega \Brp(A) \simeq \Picp(A).$$ \Lucy{What else do we need to do to show that we have an equivalence of \emph{functors}?}
\end{proposition}
\begin{proof} 
    Since $ \Omega\Brp(R) $ is given by the space of automorphisms of any object in $ \Brp(R) $, it suffices to determine the space of autoequivalences of $ \left(\Mod_R^\omega, \Qoppa_R \right) $. 
    An autoequivalence is the data of a pair $ (f, \eta) $ where $ f \colon \Mod_R^\omega \xrightarrow{\simeq} \Mod_R^\omega $ is an exact $ R $-linear autoequivalence and $ \eta \colon \Qoppa_R \xrightarrow{\sim} \Qoppa_R \circ f^{\mathrm{op}} $ is a natural equivalence. 
    Since $ \Catp_R\to \Catex_R $ is symmetric monoidal, $ f $ is of the form $ - \otimes_R \mathcal{L} $ where $ \mathcal{L} $ is an invertible $ R $-module. 
    Since taking bilinear and linear parts is functorial/by \cite[Proposition 1.3.11]{CDHHLMNNSI}, $ \eta $ is equivalently the data of a pair of equivalences
    \begin{align*}
        b(\eta) &\colon \hom_{R \otimes R}((-\otimes \mathcal{L}) \otimes (-\otimes \mathcal{L}), R)^{hC_2} \simeq \hom_{R \otimes R}(-\otimes -, R)^{hC_2} \\
        \ell(\eta) &\colon \hom_R( -\otimes \mathcal{L}, R^{\varphi C_2}) \simeq \hom_R( -, R^{\varphi C_2})
    \end{align*} 
    plus a path between their images in $ \hom_R(\mathcal{L}, R^{tC_2}) $. 
    The transformation $ b(\eta) $ is equivalent to the data of an $ R $-bilinear equivalence $ R \simeq \mathcal{L}^\vee \otimes \mathcal{L}^\vee $ \Lucy{maybe one of these should be conjugate dual here?}, and the transformation $ \ell(\eta) $ is equivalent to the data of an $ R^{\varphi C_2} $-linear\Lucy{is the $R^{\varphi C_2}$-linearity of this $\simeq$ correct?} equivalence $ \ell (\eta)\colon R^{\varphi C_2}\otimes_R \mathcal{L}^\vee \xrightarrow{\sim} R^{\varphi C_2} $. 

    Now consider the composites 
    \begin{align*}
        R \otimes_R \mathcal{L}^\vee \xrightarrow{\mathrm{unit}\,\otimes \mathrm{id}} R^{\varphi C_2} \otimes \mathcal{L}^\vee \xrightarrow{ \ell(\eta)} R^{\varphi C_2} \\
        R \otimes_R \mathcal{L} \xrightarrow{\mathrm{unit}\,\otimes \mathrm{id}} R^{\varphi C_2} \otimes \mathcal{L} \xrightarrow{ \ell(\eta)^{-1} \otimes \mathrm{id}_{\mathcal{L}}} R^{\varphi C_2} \,.
    \end{align*}
    These correspond to the $ \ell(q^\vee), \ell(q) $ of Remark \ref{rmk:poincare_picard_points_desc}, respectively. 
    In particular, the condition that $ \ell(q^\vee), \ell(q) $ make the diagram (\ref{diagram:pnpic_linear_part_condition}) commute is equivalent to the condition that $ \ell(\eta) $ is an equivalence by an adjunction argument. \Lucy{under construction--not sure what to say about the $(-)^{tC_2} $ part yet. }
\end{proof}

\begin{proposition}\label{prop:relative_poincare_cats_basic_properties}
    Let $ (\Mod_R^\omega, \Qoppa_R) $ be a Poincaré ring spectrum. 
    \begin{enumerate}[label=(\arabic*)]
        \item \label{propitem:Rlin_Poincare_cats_is_symm_mon} The $ \infty $-category $ \Mod_{\left(\Mod_R^\omega, \Qoppa_R \right)}(\Catpidem) $ admits all small limits and colimits, and it inherits a canonical symmetric monoidal structure, and for every morphism $ \left(R, R^{\varphi C_2} \to R^{tC_2}\right) \to (S, S^{\varphi C_2} \to S^{tC_2}) $, the functor $ \Mod_{\left(\Mod_R^\omega, \Qoppa_R \right)}(\Catpidem) \to \Mod_{\left(\Mod_S^\omega, \Qoppa_S \right)}(\Catpidem) $ is a symmetric monoidal left adjoint. 
        \item \label{propitem:classify_R_lin_hermitian_struct} Let $ A $ be an $ \EE_1 $-$ R $-algebra in spectra, and regard the category of compact right $ A $-modules $ \Mod_A^\omega $ as left-tensored over $ \Mod_R^\omega $ in the canonical way. 
        Then the pullback
        \begin{equation}
        \begin{tikzcd}
            & \Mod_{(\Mod_R^\omega, \Qoppa_R)}\left(\Cath\right) \ar[d] \\
            \{\Mod_A^\omega\} \ar[r] & \Catex_R
        \end{tikzcd}
        \end{equation}
        is canonically equivalent to $ \Mod_{N_R A \otimes_{N_R R} R^L }\left(\Spectra^{C_2}\right) $ where $ R^L $ is the $ \EE_\infty $-$ N_R R $-algebra with $ (R^L)^e \simeq R $ and $ (R^L)^{\varphi C_2}  \simeq C $. 

        A $ N_R A \otimes_{N_R R} R^L $-module classifies a $ (\Mod_R^\omega, \Qoppa_R) $-module in Poincaré $ \infty $-categories if its underlying $ R $-module is invertible in the sense of \cite[Definition 3.1.4]{CDHHLMNNSI}. 
        \item \label{propitem:Rlin_Poincare_cats_tensor_mod_gen_inv} Let $ A,B $ be $ R $-algebras with associated ($R$-linear) modules with genuine involution $ (M_A, N_A, N_A \to M_A^{tC_2}) $ and $ (M_B, N_B, N_B \to M_B^{tC_2}) $, respectively so that (under item \ref{propitem:classify_R_lin_hermitian_struct}) $ \left(\Mod_A^\omega, \Qoppa_A \right) $ and $ \left(\Mod_B^\omega, \Qoppa_B \right) $ are objects of $ \Mod_{\left(\Mod_R^\omega, \Qoppa_R \right)}(\Catpidem) $. 
        Then the symmetric monoidal structure of item \ref{propitem:Rlin_Poincare_cats_is_symm_mon} is so that the underlying $ R$-linear $ \infty $-category with perfect duality $ \left(\Mod_A^\omega, \Qoppa_A \right) \otimes_{\left(\Mod_R^\omega, \Qoppa_R \right)} \left(\Mod_B^\omega, \Qoppa_B \right) $ is $ \Mod_A^\omega \otimes_{\Mod_R^\omega} \Mod_B^\omega \simeq \Mod_{A \otimes_R B}^\omega $, and the associated module with genuine involution is given by $ M_A \otimes_R M_B $, $ N_A \otimes_{R^{\varphi C_2}} N_B $, and the structure map is $ N_A \otimes_{R^{\varphi C_2}} N_B \to M_A^{tC_2} \otimes_{R^{tC_2}} M_B^{tC_2} \to (M_A \otimes_R M_B)^{tC_2} $ where the latter map arises canonically from lax monoidality of the Tate construction. 
        % Commented out Aug 14th--just leaving this here just in case: \Lucy{What if it should be $ N_A \otimes_{R^{\varphi C_2}} N_B \to M_A^{tC_2} \otimes_{R^{tC_2}} M_B^{tC_2} \to (M_A \otimes_R M_B)^{tC_2} $ instead?} 

        \item \label{prop_item:R_linear_poincare_cats_maps} Let $ \left(\mathcal{C}, \Qoppa_{\mathcal{C}}\right),  \left(\mathcal{D}, \Qoppa_{\mathcal{D}}\right) $ be objects of $ \Mod_{\left(\Mod_R^\omega, \Qoppa_R \right)}(\Cath) $. 
        Then the forgetful functor induces $ \hom_{\Cath_R} \left(\left(\mathcal{C}, \Qoppa_{\mathcal{C}}\right),  \left(\mathcal{D}, \Qoppa_{\mathcal{D}}\right)\right) \to \hom_{\Catex_R}\left(\mathcal{C}, \mathcal{D}\right) $ on mapping spaces so that the fiber over an $ R $-linear functor $ F \colon \mathcal{C} \to \mathcal{D} $ is the mapping space $ \mathrm{map}_{\Qoppa_R}\left(F_! \Qoppa_\mathcal{C}, \Qoppa_{\mathcal{D}}\right) \simeq \mathrm{map}_{\Qoppa_R}(\Qoppa_{\mathcal{C}}, \Qoppa_{\mathcal{D}} \circ F^\op) $, where the mapping space is taken in $ \Fun_{\Qoppa_R}^q(\mathcal{D}^{\op},\Spectra) $ and $ \Fun_{\Qoppa_R}^q(\mathcal{C}^{\op},\Spectra) $, respectively.\footnote{The proof of \ref{propitem:classify_R_lin_hermitian_struct} in particular shows that $ \Fun^q(\mathcal{C}^{\op},\Spectra) $ is left-tensored over $ \Fun^q(\Mod_R^{\omega,\op},\Spectra) $ in the sense of \cite[Definition 4.2.1.19]{LurHA}, so this makese sense.} 

        \item The symmetric monoidal forgetful functor $ \theta \colon \Mod_{\left(\Mod_R^\omega, \Qoppa_R \right)}(\Cath) \to \Mod_{\Mod_R^\omega}(\Catex) $ is a (co)cartesian fibration. \Lucy{what is it classified by?}
    \end{enumerate}
\end{proposition}
\begin{remark}
    A special case of part \ref{propitem:classify_R_lin_hermitian_struct} is \cite[Example 5.4.13]{CDHHLMNNSI}.
\end{remark}
\begin{proof}
    \begin{enumerate}[label=(\arabic*)]
        \item The first part of the statement follows from \cite[\S6.1]{CDHHLMNNSI} and \cite[\S4.2.3]{LurHA}. 
        \item Let $ \mathcal{LM}^\otimes $ denote the $ \infty $-operad of \cite[Definition 4.2.1.7]{LurHA}. 
        Our strategy of proof will be similar to that of \cite[\S5.3]{CDHHLMNNSI}: First, we show that an $ \mathcal{LM}^\otimes $-algebra object in $ \Cath $ is equivalent to an $ \mathcal{LM}^\otimes $-algebra object in an operad of functor categories. 
        Then, we use a (suitably coherent version of) the classification of hermitian structures on module categories as categories of modules over the Hill--Hopkins--Ravenel norm \cite[Theorem 3.3.1]{CDHHLMNNSI} to conclude. 
        Recall that the action of $ \Mod_R^{\omega} $ on $ \Mod_A^\omega $ is given by a functor $ \mathcal{LM}^\otimes \to \Cat_\infty^\times $, and define $ \Fun_{\Mod_R^{\omega,\op}}(\Mod_A^{\omega,\op},\Spectra)^\otimes $ via the following pullback square of $ \infty $-operads: 
        \begin{equation}
        \begin{tikzcd}
            \Fun_{\Mod_R^{\omega,\op}}(\Mod_A^{\omega,\op},\Spectra)^\otimes \ar[r,"{p}"]\ar[d] & \mathcal{LM}^\otimes \ar[d,"{\Mod_R^\omega,\Mod_A^\omega}"] \ar[d] \\
            \left(\Cat_\infty\right)_{\op/-/\Spectra}^\otimes \ar[r] & \Cat_\infty^\times 
        \end{tikzcd}\,.
        \end{equation}
        Informally, an object $ F \in \Fun_{\Mod_R^{\omega,\op}}(\Mod_A^{\omega,\op},\Spectra)^\otimes_{\mathfrak{a}} $ is a functor $ F \colon \Mod_R^{\omega,\op} \to \Spectra $ and an object $ G $ over the fiber of $ \mathfrak{m} $ is a functor $ G \colon \Mod_A^{\omega,\op} \to \Spectra $. 
        The $p $-cocartesian edge over the canonical map $ (\mathfrak{a},\mathfrak{m}) \to \mathfrak{m} $ in $ \mathcal{LM}^\otimes $ sends $ (F,G) $ to the lower arrow in the diagram
        \begin{equation*}
        \begin{tikzcd}[column sep=4.5em]
            \Mod_R^{\omega,\op}\times \Mod_A^{\omega,\op} \ar[r,"{F\times G}"] \ar[d,"{-\otimes_R -}"'] & \Spectra \times \Spectra \ar[d,"{\otimes_{\Spectra}}"] \\
            \Mod_A^{\omega,\op} \ar[r,"{F \otimes G := \mathrm{LKE}_{\otimes_R}( \otimes_{\Spectra} \circ (F \times G))}"] & \Spectra
        \end{tikzcd}\,.
        \end{equation*}
        Now define $ \Fun_{\Mod_R^{\omega,\op}}^q(\Mod_A^{\omega,\op},\Spectra)^\otimes $ to consist of the full subcategory of $ \Fun_{\Mod_R^{\omega,\op}}(\Mod_A^{\omega,\op},\Spectra)^\otimes $ consisting of those tuples of functors which are all quadratic. 
        The inclusion $ \Fun_{\Mod_R^{\omega,\op}}^q(\Mod_A^{\omega,\op},\Spectra)^\otimes \to \Fun_{\Mod_R^{\omega,\op}}(\Mod_A^{\omega,\op},\Spectra)^\otimes $ exhibits the former as an $ \infty $-operad, and moreover the localization is compatible with the $ \mathcal{LM}^\otimes $-monoidal structure in the sense of \cite[Definition 2.2.1.6]{LurHA}. 
        We can extend the previous diagram to 
        \begin{equation}\label{diagram:module_structure_on_quadratic_functors}
        \begin{tikzcd}
            \Fun_{\Mod_R^{\omega,\op}}^q(\Mod_A^{\omega,\op},\Spectra)^\otimes \ar[r]\ar[d] & \Fun_{\Mod_R^{\omega,\op}}(\Mod_A^{\omega,\op},\Spectra)^\otimes \ar[r,"{p}"]\ar[d] & \mathcal{LM}^\otimes \ar[d,"{\Mod_R^\omega,\Mod_A^\omega}"] \ar[d] \\
           \Cath^\otimes \ar[r]& \left(\Cat_\infty\right)_{\op/-/\Spectra}^\otimes \ar[r] & \Cat_\infty^\otimes 
        \end{tikzcd}\,.
        \end{equation}
        Modifying \cite[Construction 5.3.15 \& Lemma 5.3.15]{CDHHLMNNSI} slightly (note that Corollary 5.1.4 did not assume the tensor factors to be equivalent), we obtain an analogous commutative diagram of $ \infty $-operads
        \begin{equation}\label{diagram:module_structure_on_pb_functors}
        \begin{tikzcd}
            \Fun_{\Mod_R^{\omega,\op}}^p(\Mod_A^{\omega,\op},\Spectra)^\otimes \ar[d] \ar[r] & \Fun_{\Mod_R^{\omega,\op}}^q(\Mod_A^{\omega,\op},\Spectra)^\otimes \ar[r]\ar[d] & \Fun_{\Mod_R^{\omega,\op}}(\Mod_A^{\omega,\op},\Spectra)^\otimes \ar[r,"{p}"]\ar[d] & \mathcal{LM}^\otimes \ar[d,"{\Mod_R^\omega,\Mod_A^\omega}"] \ar[d] \\
           \Catp^\otimes \ar[r] &\Cath^\otimes \ar[r]& \left(\Cat_\infty\right)_{\op/-/\Spectra}^\otimes \ar[r] & \Cat_\infty^\otimes 
        \end{tikzcd}
        \end{equation}
        in which all squares are pullbacks. 
        Now suppose $ A $ is given a module with genuine involution $ (M_A, N_A,N_A \to M_A^{tC_2}) $ and call the associated Poincaré $\infty$-category $ \overline{\Mod}_A $.  
        Then to lift $ \overline{\Mod}_A $ to a module over $ \left(\Mod_R^\omega,\Qoppa_R\right) $ compatibly with the $ \Mod_R^\omega $-module structure on $ \Mod_A^\omega $ is to give a map of $ \infty $-operads $ \mathcal{LM}^\otimes \to \Cath^\otimes $ so that the restriction along the canonical inclusion $ \mathrm{Assoc}^\otimes \to \mathcal{LM}^\otimes $ gives the algebra object $ \left(\Mod_R^\omega,\Qoppa_R\right) $ and postcomposing with the canonical projection to $ \Catex^\times $ recovers the given $ \Mod_R^\omega $-module structure on $ \Mod_A^\omega $. 
        By the pullback square (\ref{diagram:module_structure_on_quadratic_functors}), this is equivalent to giving an object of $ \mathrm{Alg}_{\mathcal{LM}/\mathcal{LM}}\left(\Fun_{\Mod_R^{\omega,\op}}^q(\Mod_A^{\omega,\op},\Spectra)^\otimes\right) $. 
        Now let us identify the bilinear functor $ \Mod_R^\omega \times \Mod_A^\omega \xrightarrow{ - \otimes_R -} \Mod_A^\omega $ with the exact functor $ \Mod_R^\omega \otimes \Mod_A^\omega \simeq \Mod_{R \otimes A}^\omega \to \Mod_A^\omega $ which is induction along the action map $ R \otimes A \to A $. 
        Using \cite[Corollary 3.4.1]{CDHHLMNNSI} and unravelling definitions gives the claim for $ R$-linear hermitian structures. 
        The proof for $ R $-linear Poincaré structures considers (\ref{diagram:module_structure_on_pb_functors}) instead but otherwise proceeds in an identical fashion.  
        \item By \cite[Theorem 4.4.2.8]{LurHA}, the relative tensor product $ \left(\Mod_A^\omega, \Qoppa_A \right) \otimes_{\left(\Mod_R^\omega, \Qoppa_R \right)} \left(\Mod_B^\omega, \Qoppa_B \right) $ is computed as the geometric realization of the bar construction 
        \begin{align*}
            p \colon \Delta^\op & \to \Cath \\
            [n] &\mapsto \left(\Mod_A^\omega, \Qoppa_A \right) \otimes \left(\Mod_R^\omega, \Qoppa_R \right)^{\otimes n} \otimes \left(\Mod_B^\omega, \Qoppa_B \right)
        \end{align*} 
        Write $ f \colon \Cath \to \Catex $ for the forgetful functor. 
        Then $ f \circ p $ has a colimit with value $ \Mod_A^\omega \otimes_{\Mod_R^\omega} \Mod_B^\omega \simeq \Mod_{A \otimes_R B}^\omega $. 
        Writing $ g \colon \Catex \to \{*\} $, by Example 4.3.1.3 of \cite{HTT} we see that $ f \circ p $ is a $ g $-colimit. 
        By Proposition 4.3.1.5(2) and Example 4.3.1.3 of \cite{HTT}, $ p $ admits a colimit in $ \Cath $ if and only if it admits an $ f $-colimit. 
        Now recall that $ f $ is a cocartesian fibration with pushforward given by left Kan extension \cite[Corollary 1.4.2]{CDHHLMNNSI}. 
        We show that $ f $ satisfies the conditions of \cite[Corollary 4.3.1.11]{HTT}. 
        \begin{itemize}
            \item Condition (1) follows from Theorem 6.1.1.10 of \cite{LurHA} applied to $ \Spectra^\op $ (see the end of \cite[Construction 1.1.26]{CDHHLMNNSI}). 
            \item Condition (2) follows from \cite[Corollary 1.4.2]{CDHHLMNNSI}, the adjoint functor theorem, and presentability of $ \Fun^q(\mathcal{C}) $, which is discussed in the proof of \cite[Lemma 5.3.3]{CDHHLMNNSI} (also see \cite[Remark 6.1.1.11]{LurHA}). 
        \end{itemize}
        Thus the preceding discussion shows that there exists a map of simplicial sets $ p' $ making the diagram commute
        \begin{equation*}
        \begin{tikzcd}
            \Delta^\op \ar[d] \ar[r,"p"] & \Cath \ar[d,"f"] \\
            \left(\Delta^\op\right)^{\triangleright} \ar[r]\ar[ru,"{p'}"] & \Catex 
        \end{tikzcd}\,.
        \end{equation*}
        Since $ \{0\} \to \Delta^1 $ is left anodyne, by \cite[Corollary 2.1.2.7]{HTT} the inclusions
        \begin{align*}
            \{0\} \times \Delta^\op & \to \Delta^1 \times \Delta^\op \\
            \iota \colon \left(\{0\} \times (\Delta^\op)^\triangleright\right) \sqcup_{\{0\} \times \Delta^\op}\left(\Delta^1 \times \Delta^\op \right) & \to \Delta^1 \times \left(\Delta^\op\right)^\triangleright 
        \end{align*}
        are left anodyne. 
        The former implies that there exists a map $ p'' $ making the diagram 
        \begin{equation*}
        \begin{tikzcd}
            \{0\} \times \Delta^\op \ar[r,"p"]\ar[d] &\Cath \ar[d,"f"] \\
            \Delta^1 \times \Delta^\op \ar[r] \ar[ru,"{p''}"] & \Catex
        \end{tikzcd}
        \end{equation*}
        commute. 
        The maps $ p' $ and $ p'' $ assemble to give a map $ p''' := p' \sqcup_p p'' $ making the diagram 
        \begin{equation*}
        \begin{tikzcd}[arrows={crossing over},row sep=large]
             \{0\} \times \Delta^\op \ar[rr,"p"]\ar[d] &  &\Cath \ar[d,"f"] \\
            \left(\{0\} \times (\Delta^\op)^\triangleright\right) \sqcup_{\{0\} \times \Delta^\op}\left(\Delta^1 \times \Delta^\op \right) \ar[r,"\iota"] \ar[rru,"{p'''}", near start] & \Delta^1 \times \left(\Delta^\op\right)^\triangleright \ar[r] \ar[ru,"{\overline{p}}"', bend right=10] & \Catex
        \end{tikzcd}
        \end{equation*}
        commute, and likewise $ \overline{p} $ exists making the diagram commute since $ \iota $ is left anodyne. 
        Now we show that $ \overline{p} $ satisfies the conditions of \cite[Proposition 4.3.1.9]{HTT}. 
        By (the opposite/dual/cocartesian version of) \cite[Remark 3.1.1.10]{HTT} and Proposition 3.1.1.5(2'') \emph{ibid.} and the fact that $ f $ is a cocartesian fibration, we can choose $ \overline{p} $ so that for all $ k \in (\Delta^\op)^\triangleright $, $ \overline{p}|_{\Delta^1 \times \{k\}} $ is $f$-cocartesian. 
        Furthermore, since we can choose $ \Delta^\op, \,\left(\Delta^\op\right)^\triangleright $ to have the markings $ (-)^\flat $ in \cite[Remark 3.1.1.10]{HTT}, $ f \circ \overline{p}|_{\Delta^1 \times \{\infty\}} $ is a degenerate edge in $ \Catex $. 

        Now \cite[Proposition 4.3.1.9]{HTT} implies that $ \overline{p}_0 $ is an $ f $-colimit diagram if and only if $ \overline{p}_1 $ is an $ f $-colimit diagram. 
        Now notice that $ \overline{p}|_{\{1\} \times \left(\Delta^\op\right)^{\triangleright}} $ has image contained in the fiber of $ f $ over $ \Mod_{A \otimes_R B}^\omega $. 
        By \cite[Proposition 4.3.1.10]{HTT}, it suffices to show that $ \overline{p}_1 $ is a colimit diagram in $ \Fun^q\left(\Mod_{A \otimes_R B}^\omega\right) $. 
        Write $ \overline{M}_A \in \Mod_{N^{C_2}A} $ and $ \overline{M}_B \in \Mod_{N^{C_2}B} $ for the corresponding modules (see introduction to \S3.3 of \cite{CDHHLMNNSI}). 
        Unraveling definitions and using \cite[Theorem 3.3.1 \& Corollary 3.4.1 \& Lemma 5.4.6]{CDHHLMNNSI}, it follows that the diagram $ \overline{p}_1|_{\{1\} \times \Delta^\op} $ is the bar construction 
        \begin{align*}
            [n] & \mapsto \overline{M}_A \otimes_{N^{C_2}R} R^{\otimes_{N^{C_2}R} n} \otimes_{N^{C_2} R} \overline{M}_B \,.
        \end{align*}
        This proves the result. 
        %\Lucy{Point is: $ \iota $ is \emph{marked anodyne} and . Then compute the colimit in $ \Fun^q \left(\Mod_{A\otimes_R B}^\omega \right) $.}

        \item  Let $ \left(\mathcal{C}, \Qoppa_{\mathcal{C}}\right) $ be an object of $ \Mod_{\left(\Mod_R^\omega, \Qoppa_R \right)}(\Cath) $ and let $ F \colon \mathcal{C} = \theta \left(\mathcal{C}, \Qoppa_{\mathcal{C}}\right)\to \mathcal{D} $ be an $ R $-linear functor. 
        Now define $ \Qoppa_{\mathcal{D}} \colon \mathcal{D}^\op \to \Spectra $ to be the left Kan extension of $ \Qoppa_{\mathcal{C}} $ along $ F^\op $. 
        Now $ \left(\mathcal{D}, \Qoppa_{\mathcal{D}}\right) \in \Cath $ and there is a canonical map $ (f, \eta )\colon \left(\mathcal{C}, \Qoppa_{\mathcal{C}}\right) \to \left(\mathcal{D}, \Qoppa_{\mathcal{D}}\right)$. 
        Now $ F $ is classified by a functor $ \Delta^1 \times \mathcal{LM}^\otimes \to \Catex^\otimes $, and we may form the pullback
        \begin{equation}
        \begin{tikzcd}
            \mathcal{N} \ar[r] \ar[d] & \Delta^1 \times \mathcal{LM}^\otimes \ar[d]  \\
            \Cath^\otimes \ar[r,"{p}"] & \Catex^\otimes
        \end{tikzcd}\,.
        \end{equation}
        Since $ p $ is a cocartesian fibration \cite[Theorem 5.2.7]{CDHHLMNNSI}, $ \mathcal{N} \to \Delta^1 \times \mathcal{LM}^\otimes $ is a cocartesian fibration, and the nontrivial morphism in $ \Delta^1 $ classifies a map $ F_! \colon \Fun_{\Mod_R^{\omega,\op}}^q(\mathcal{C}^{\op},\Spectra)^\otimes \to \Fun_{\Mod_R^{\omega,\op}}^q(\mathcal{D}^{\op},\Spectra)^\otimes $ of $ \infty $-operads over $ \mathcal{LM}^\otimes $. 
        Passing to algebra objects, we obtain the desired result on mapping spaces. 

        \item By \cite[Proposition 2.4.2.8]{HTT}, it suffices to show that $ \theta $ is a locally (co)cartesian fibration, and that locally (co)cartesian edges are closed under composition. 
        We give the proof that $ \theta $ is a cocartesian fibration; the proof that $ \theta $ is a cartesian fibration is formally dual and will be left to the reader. 

        Let $ \left(\mathcal{C}, \Qoppa_{\mathcal{C}}\right) $ be an object of $ \Mod_{\left(\Mod_R^\omega, \Qoppa_R \right)}(\Cath) $ and let $ F \colon \mathcal{C} = \theta \left(\mathcal{C}, \Qoppa_{\mathcal{C}}\right)\to \mathcal{D} $ be an $ R $-linear functor. 
        Now define $ \Qoppa_{\mathcal{D}} \colon \mathcal{D}^\op \to \Spectra $ to be the left Kan extension of $ \Qoppa_{\mathcal{C}} $ along $ F^\op $. 
        By the proof of \ref{prop_item:R_linear_poincare_cats_maps}, we see that the image of $ \Qoppa_{\mathcal{C}} $ under $ F_! $ is a lift of $ \left(\mathcal{D}, \Qoppa_{\mathcal{D}}\right) $ to an object of $ \Mod_{\left(\Mod_R^\omega, \Qoppa_R \right)}(\Cath) $ and $ (f, \eta) $ to a morphism in $ \Mod_{\left(\Mod_R^\omega, \Qoppa_R \right)}(\Cath) $. 

        Now by Lemma 2.4.4.1 and the locally cocartesian version of Proposition 2.4.1.10 of \cite{HTT}, we must show that for all choices $ \Qoppa_{\mathcal{D}}' $ of an $ R $-linear Hermitian structure on $ \mathcal{D} $, precomposition with $ F_! $ induces a pullback square
        \begin{equation}
        \begin{tikzcd}
            \hom_{\Cath_R}\left(\left(\mathcal{D}, \Qoppa_{\mathcal{D}}\right), \left(\mathcal{D}, \Qoppa_{\mathcal{D}}'\right)\right) \ar[d] \ar[r] & \hom_{\Cath_R}\left(\left(\mathcal{C}, \Qoppa_{\mathcal{C}}\right), \left(\mathcal{D}, \Qoppa_{\mathcal{D}}'\right)\right) \ar[d] \\
            \hom_{\Catex_R}\left(\mathcal{D}, \mathcal{D}\right) \ar[r] & \hom_{\Catex_R}\left(\mathcal{C}, \mathcal{D} \right)
        \end{tikzcd}\,.
        \end{equation}
        By \ref{prop_item:R_linear_poincare_cats_maps}, $ F_! $ induces equivalences on the fibers of the vertical maps, hence $ (f, \eta) $ is locally $ \theta $-cocartesian. 
        The locally $ \theta $-cocartesian maps are manifestly closed under composition, hence we are done.        \qedhere
    \end{enumerate}
\end{proof}
\begin{corollary}\label{cor:poincare_fourier_mukai}
    % Will want to use this to show that some functor: ``algebras'' to Poincaré infinity categories is fully faithful; see proof of Proposition 3.1 in Antieau--Gepner. 
    Let $ R $ be a Poincaré ring, and let $ A, B $ be $ \mathbb{E}_1 $-$ R $-algebras with genuine involution. 
    Then there is an equivalence $ \hom_{\Catpidem_R}\left(\left(\Mod_A^\omega, \Qoppa_A\right), \left(\Mod_B^\omega, \Qoppa_B\right)\right) \simeq \left(\BiMod_{A \otimes_R B^\op}\right)_{A^{\varphi C_2} \otimes_R B^{\varphi C_2} /-} $.   
\end{corollary}
\begin{proof}
    \Lucy{todo--probably need to fix the statement with \emph{duals} when the proof is written}
\end{proof}

As in the Picard group case, the forgetful functor $\theta$ induces a map of spectra $\Brp(A)\to \Br(A)$ which will again factor through the $2$-trosion on $\pi_0$. As a consequence of Proposition~\ref{propitem:classify_R_lin_hermitian_struct} we can identify the fiber of this map.
\begin{corollary}
Let $(\mathrm{Mod}_A^\omega, \Qoppa_A)$ be a Poincar{\'e} ring with underlying genuine $C_2$ spectrum $A^L$ as in Proposition~\ref{prop:relative_poincare_cats_basic_properties}\ref{propitem:classify_R_lin_hermitian_struct}. Then the fiber of the map \[\Brp(A)\to \Br(A)\] can be naturally identified with $\Pic(\mathrm{Mod}_{A^L}(\mathrm{Sp}^{C_2}))$.
\end{corollary}
\begin{proof}
Since $\theta:\mathrm{Mod}_{(\Mod_A^\omega, \Qoppa_A)}(\Catpidem)\to \mathrm{Mod}_{\mathrm{Mod}_A}(\Catex)$ is (co)cartesian, symmetric monoidal, and conservative, it follows that the induced functor on the groupoid core of invertible objects is a Kan fibration by \cite[\href{https://kerodon.net/tag/01EZ}{Proposition 01EZ}]{kerodon}. Consequently we need only identify the fiber instead of the homotopy fiber, which follows from the identification of the fiber in Proposition~\ref{prop:relative_poincare_cats_basic_properties}\ref{propitem:classify_R_lin_hermitian_struct}. \Noah{Needs a lot more detail I know, but I think the skeleton is here.}
\end{proof}
Write $ \mathrm{Pn} $ for the composite $ \Catp_R \xrightarrow{U} \Catp \xrightarrow{\mathrm{Pn}} \Spaces $. 
\begin{proposition}
    Let $ (R, R^{\varphi C_2} \to R^{tC_2}) $ be a Poincaré ring. 
    Then $ \left(\Mod_R^\omega, \Qoppa_R \right) $ corepresents the functor $ \mathrm{Pn}\colon \Catp_R \to \Spaces $.
\end{proposition}
\begin{proof}
    Recall that Proposition \ref{prop:relative_poincare_cats_basic_properties}.\ref{propitem:Rlin_Poincare_cats_is_symm_mon} furnishes an adjoint pair $ \Catp_R \rlarrows \Catp $ of functors. 
    Write $ \overline{\mathcal{C}} = (\mathcal{C},\Qoppa_{\mathcal{C}}) \in \Catpidem_R $. 
    Then
    \begin{equation*}
        \mathrm{Pn}(\mathcal{C}) = \hom_{\Catp}\left((\Spectra^f,\Qoppa^u), U(\overline{\mathcal{C}})\right) \simeq \hom_{\Catp_R}\left(\left(\Mod_R^\omega,\Qoppa_R\right)\otimes(\Spectra^f,\Qoppa^u), \overline{\mathcal{C}}\right) \,,
    \end{equation*}
    where the first equivalence is \cite[Proposition 4.1.3]{CDHHLMNNSI}. 
\end{proof}

\subsection{Azumaya algebras with genuine involution}
Let $ R $ be an $ \EE_\infty $-ring spectrum. 
\begin{recollection}\label{rec:Azumaya_alg_wo_involution} 
    Recall \cites{MR2927172,MR3190610} that an $ \EE_1 $-$ R $-algebra is said to be \emph{Azumaya} if it is a compact generator of $ \Mod_R $ and if the natural $ R $-algebra map giving the bimodule structure on $ A $
    \begin{equation*}
        A \otimes_R A^{\mathrm{op}} \to \mathrm{End}_R(A)
    \end{equation*}
    is an equivalence of $ R $-algebras. 
\end{recollection}

\begin{definition}
    Let $ (R, R \to R^{\varphi C_2} \to R^{tC_2}) $ be a Poincaré ring spectrum. 
    An \emph{Azumaya algebra with genuine type 1 (anti-)involution} over $ R $ is the data of 
    \begin{enumerate}[label=(\alph*)]
        \item \label{defnitem:Azumaya_alg_gi_the_involution} An $ \EE_1 $-$ R $-algebra $ A $ equipped with an anti-involution $ \tau \colon A \to A^\op $ so that $ A $ is an Azumaya $ R $-algebra in the sense of Recollection \ref{rec:Azumaya_alg_wo_involution} 
        \item A left $ A \otimes_{R} R^{\varphi C_2} $-module $ A^{\varphi C_2} $ 
        \item \label{defn_item:Azumaya_gen_inv_geom_fixpt} An equivalence of $ A \otimes_R A^\op $-modules\Lucy{compatibility with Tate?}
        \begin{equation*}
            \hom_R(A, R^{\varphi C_2}) \simeq A^{\varphi C_2} \otimes_{R^{\varphi C_2}} \hom_R(A^{\varphi C_2}, R^{\varphi C_2})
        \end{equation*}
        \item An $ A $-linear map $ A^{\varphi C_2} \to A^{tC_2} $. 
        Here we regard $ A^{tC_2} $, which is canonically a $ (A \otimes A)^{tC_2} $-module, as an $ A $-module via the Tate-valued diagonal $ A \to (A \otimes A)^{tC_2} $. 
    \end{enumerate}
    Similarly, an \emph{Azumaya algebra with genuine type 2 (anti-)involution} over $ R $ is obtained by replacing $ \tau $ in item \ref{defnitem:Azumaya_alg_gi_the_involution} with $ \tau \colon A \to \sigma_R^* A^\op $, where $ \sigma_R \colon R \xrightarrow{\sim} R $ is the given involution on $ R $. 
    \Lucy{define the category!}
\end{definition}
\begin{remark}\label{rmk:azumaya_geninv_gives_module_geninv}
    If $ A $ is an Azumaya algebra with genuine involution over $ R $, then in particular $ M_A = A $, $ N_A = A^{\varphi C_2} $ is a module with genuine involution over $ A $ in the sense of \cite[Definition 3.2.3]{CDHHLMNNSI}. 
\end{remark}
With ordinary Azumaya algebras, the prototypical Azumaya algebra with anti-involution arises from endomorphism rings of perfect modules. 
Choosing a (nondegenerate symmetric bilinear) form on a perfect module $ P $ endows its endomorphism algebra with additional structure. 
\begin{example}
    Let $ (R, R \to R^{\varphi C_2} \to R^{tC_2} ) $ be a Poincaré ring, and let $ (P,q ) \in \mathrm{Pn}\left(\Mod_R^\omega, \Qoppa_R \right) $. 

    Then $ A:= \mathrm{End}_R(P) $ admits a canonical lift to an Azumaya algebra with genuine involution over $ R $ with $ A^{\varphi C_2}:= \hom_R(P, R^{\varphi C_2}) $. 
    \Lucy{to-do: explain! Also $ A^{\varphi C_2}$ could also be $ P \otimes_R R^{\varphi C_2}$?} 

    By \cite[Proposition 3.1.16]{CDHHLMNNSI}, $ A $ inherits a canonical anti-involution. \Lucy{todo: an $ R $-linear enhancement of Proposition 3.1.16?}
    Since $ P $ is compact, it is dualizable with respect to the symmetric monoidal structure on $ \Mod_R^\omega $ \cite[Theorem III.7.9]{elmendorf2007rings}. 
    Since $ \otimes_R R^{\varphi C_2} $ is symmetric monoidal, in particular it takes $ P $ to a dualizable object--call it $ \overline{P} $. 
    Now there is a canonical choice of equivalence \ref{defn_item:Azumaya_gen_inv_geom_fixpt} since both sides are canonically equivalent to $ \overline{P} \otimes_{R^{\varphi C_2}} \overline{P}^\vee $. 
\end{example}

\begin{proposition}
    Let $ R $ be the Eilenberg--Mac Lane spectrum associated to a discrete ring, and suppose $ R $ has a given $ C_2 $-action. 
    Let $ A $ be a classical Azumaya algebra over $ R $ with an involution of type 2. 
    Regard $ R $ as a Poincaré ring spectrum with the genuine symmetric Poincaré structure of Example \ref{example:genuine_symmetric_poincare_structure}. 

    Then there is a canonical Azumaya algebra with genuine involution over $ R^{\mathrm{gs}} $ so that $ A^{\varphi C_2} := \tau_{\geq 0} A^{tC_2} $.
\end{proposition}
\begin{proof}
    By \cite[Examples 3.1.9 \& 3.2.9]{CDHHLMNNSI}, it suffices to exhibit $ N $ and an $ A $-linear map $ N \to A^{tC_2} $. 
    \Lucy{todo}
\end{proof}

\begin{proposition}\label{prop:mod_over_azumaya_geninv_is_invertible}
    Let $ (R, R \to R^{\varphi C_2} \to R^{tC_2} ) $ be a Poincaré ring, and let $ (A, A^{\varphi C_2} \to A^{tC_2}) $ be an Azumaya algebra with genuine involution over $ R $. 
    Then $ \left(\Mod_A^\omega, \Qoppa_A \right) $ is an invertible object in $ \Mod_{\left(\Mod_R^\omega, \Qoppa_R\right)}\left(\Catpidem\right) $. 
\end{proposition}
\begin{remark}
    Contrast Proposition \ref{prop:mod_over_azumaya_geninv_is_invertible} with \cite[Theorem 3.15]{MR3190610}, where it is shown that an $ R $-linear stable $ \infty $-category is invertible \emph{if and only if} it is equivalent to modules over an Azumaya $ R $-algebra. 
    The difference lies in the fact that not every $ R $-linear (Morita anti-)equivalence $ \Mod_A^\omega \simeq \Mod_{A^\op}^\omega $ is induced by a map of $ \EE_1 $-rings $ A \to A^\op $. 
    \Lucy{this is not quite the same (in a literal sense), but I think similar in spirit to the ``counterexamples'' paper by First--Williams}
\end{remark}
\begin{proof}
    First, by \cite[Example 3.2.9]{CDHHLMNNSI}, we see that $ \left(\Mod_A^\omega, \Qoppa_A \right) $ is indeed an $ R$-linear Poincaré $ \infty $-category (and not merely hermitian). 
    To show that the associated Poincaré $ \infty $-category is invertible, we must identify a dual $ \left(\Mod_A^\omega, \Qoppa_A \right)^\vee $ and exhibit an equivalence $ \left(\Mod_A^\omega, \Qoppa_A \right) \otimes \left(\Mod_A^\omega, \Qoppa_A \right)^\vee \simeq \left(\Mod_R^\omega, \Qoppa_R \right) $. 
    Since $ \Catpidem_R \to \Catex_R $ is symmetric monoidal, we see that the underlying $ R $-linear $ \infty $-category associated to the dual must be $ \Mod_{A^\op}^\omega $. 
    Moreover, the canonical evaluation map $ \mathrm{ev} \colon \Mod_A^\omega \otimes \Mod_{A^\op}^\omega \xrightarrow{\simeq} \Mod_R^\omega $ sends $ A \otimes A^\op $ to $ A $. 
    Endow $ \Mod_{A^\op}^\omega $ with a Poincaré structure corresponding to the module with genuine involution $ M_{A^\op}:= A^\op $, $ N_{A^\op} := \hom_R(A^{\varphi C_2}, R^{\varphi C_2}) $. 
    It remains to exhibit a natural equivalence 
    \begin{equation}\label{eq:invertible_quadratic_compatibility}
        \eta \colon \left(\Qoppa_A \otimes \Qoppa_{A^\op}\right) \xrightarrow{\simeq} \mathrm{ev}^* \Qoppa_R 
    \end{equation} 
    of [quadratic] functors $ \Mod_A^\omega \otimes \Mod_{A^\op}^\omega \to \Spectra $. 
    By \cite[Theorem 3.3.1]{CDHHLMNNSI}, it suffices to exhibit equivalences on the bilinear and linear parts of (\ref{eq:invertible_quadratic_compatibility}) which glue compatibly. 
    By Proposition \ref{prop:relative_poincare_cats_basic_properties}\ref{propitem:Rlin_Poincare_cats_tensor_mod_gen_inv}, on linear parts, it suffices to exhibit an $ A \otimes_R A^\op $-linear equivalence 
    \begin{equation*}
        \hom_R(A, R^{\varphi C_2}) \simeq N_A \otimes_{R^{\varphi C_2}} N_{A^\op}
    \end{equation*}
    and on bilinear parts, it suffices to exhibit an $ (A \otimes_R A^\op)^{\otimes_R 2} $-linear equivalence 
    \begin{equation*}
        \hom_{R \otimes R}(A \otimes_R A, R) \simeq M_A \otimes_R M_{A^\op} \,.
    \end{equation*}
    \Lucy{modify defn of Azumaya alg with genuine involution to include an $ A \otimes A^\op $-linear equivalence $ A \simeq A^\vee $? Or simply define $ M_A := A^\vee $?}
\end{proof}

\section{Poincar{\'e} schemes}
\begin{defn}
Let $\aps$ be the $(\infty,1)$-category defined by the pullback \[
\begin{tikzcd}
\aps \arrow[rr]\arrow[d] & & \operatorname{Fun}(\Delta^2, \calg(\Sp))\arrow[d,"d_1^*"]\\
\calg(\Sp^{BC_2})\arrow[rr,"U(-)\to (-)^{tC_2}"] & & \operatorname{Fun}(\Delta^1, \calg(\Sp))
\end{tikzcd}
\] where $U:\Sp^{BC_2}\to \Sp$ is the functor which forgets the $C_2$-action.
\end{defn}

We record here a few structural results about this category.

\begin{thm}
The following statements about $\aps$ hold:
\begin{enumerate}
\item The category $\aps$ is a cocomplete and symmetric monoidal infinite category;
\item the pullback diagram above is homotopy Cartesian;
\item the functor $\aps\to \calg(\Sp^{BC_2})$ is symmetric monoidal and (co)continuous;
\item the functor $\aps\to \calg(\Sp)^{\Delta^2}$ is lax symmetric monoidal;
\item and the functor $\aps\to \calg(\Sp)^{\Delta^2}\xrightarrow{ev_{[1]}} \calg(\Sp)$ is symmetric monoidal.
\end{enumerate}
\end{thm}
\begin{proof}
    For (2) it is enough to show that $d_1^*$ is a cartesian fibration which follows from \cite[Corollary 2.4.6.5]{HTT}. 

    For (3), let $p:K\to \aps$ be a map of simplicial sets, $K$ a small simplicial set. Suppose the $K^\vartriangleright\to \aps$ be an extension such that $K^\vartriangleright\to \aps\to \calg(\Sp^{BC_2})$ is a colimit diagram. By \cite[Proposition 2.4.3.2]{HTT} the diagram \[\begin{tikzcd}
        \aps_{p/}\arrow[r]\arrow[d] &\calg(\Sp)^{\Delta^2}_{p/-}\arrow[d]\\
        \calg(\Sp^{BC_2})_{p/}\arrow[r] & \calg(\Sp)^{\Delta^1}_{p/-}
    \end{tikzcd}\] is again homotopy cartesian. Then 
    \begin{align*}
        \hom_{\aps}(p(\infty), -)&\simeq \hom_{\calg(\Sp^{BC_2})}(p(\infty), -)\times_{\hom_{\calg(\Sp)^{\Delta^1}}(p(\infty),-)}\hom_{\calg(\Sp)^{\Delta^2}}(p(\infty))\\
        &\simeq 
    \end{align*}
\end{proof}

We will denote elements of $\aps$ by $\underline{A}=(A,s:A^{\Phi C_2}\to A^{tC_2})$. Here $s:A^{\Phi C_2}\to A^{tC_2}$ is the image of $\underline{A}$ under the top horizontal map above. The use of the notation $A^{\Phi C_2}$ is justified by the following.

\begin{lem}
Let $\aps\to \calg(\Sp)$ be the composition of the functors \[\aps\to \operatorname{Fun}(\Delta^2,\calg(\Sp))\xrightarrow{ev_{[1]}}\calg(\Sp).\] Then this functor factors as a composition $\aps\to \calg(\Sp^{C_2})\xrightarrow{(-)^{\Phi C_2}}\calg(\Sp)$. 
\end{lem}
\begin{proof}
The commutativity of the diagram
\[
\begin{tikzcd}
 & & \operatorname{Fun}(\Delta^2, \calg(\Sp)) \arrow[rd, "d_0^*"] \arrow[dd, "d_1^*"] & \\
 & & & \operatorname{Fun}(\Delta^1,\calg(\Sp)) \arrow[dd,"ev_{[1]}"]\\
\calg(\Sp^{BC_2}) \arrow[rr, "U(-)\to (-)^{tC_2}"] \arrow[rrd, "id"] & & \operatorname{Fun}(\Delta^1, \calg(\Sp)) \arrow[rd, "ev_{[1]}"] & \\
  & & \calg(\Sp^{BC_2}) \arrow[r, "(-)^{tc_2}"] & \calg(\Sp)
\end{tikzcd}
\] induces a functor on the pullback infinity categories $\aps\to \calg(\Sp^{C_2})$ which makes the corresponding cube commute. The functor $ev_{[1]}:\operatorname{Fun}(\Delta^2, \calg(\Sp))\to \calg(\Sp)$ factors through $d_0^*$ and so $\aps\to \operatorname{Fun}(\Delta^2, \calg(\Sp))\to \calg(\Sp)$ is equivalent to the composition \[\aps\to \calg(\Sp^{C_2})\to \operatorname{Fun}(\Delta^1,\calg(\Sp))\to \calg(\Sp)\] and the composition of the last two maps is the geometric fixed point functor as desired.
\end{proof}

The following Lemma gives the justification of the name Poincar{\'e} scheme.

\begin{lem}
    There is a symmetric monoidal functor \[\perfpn:\aps\to \mathrm{Cat}_{\infty}^{\mathrm{Pn}}\] to the category of Poincar{\'e} infinity categories which has essential image the subcategory spanned by objects $(\perf(R),\Qoppa)$ which are $\mathbb{E}_\infty$-algebras.
\end{lem}

\begin{defn}
     A map $f:\underline{A}\to \underline{B}\in \aps$ is faithfully flat if the underlying map $f:A\to B$ is faithfully flat and the map $f^{\Phi C_2}:A^{\Phi C_2}\to B^{\Phi C_2}$ is also faithfully flat.
\end{defn}

\begin{lem}
    The fpqc covers on $\aps$ form a Grothendieck site. 
\end{lem}

%\begin{defn}
%    The infinity category of Poincar{\'e} schemes, denoted $\psch$, is the infinity category \[\psch:= \operatorname{Ind}(\aps^{op})\]
%\end{defn}

\printbibliography
\end{document}
