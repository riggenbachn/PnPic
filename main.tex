\documentclass{article}
\usepackage[utf8]{inputenc}
%Packages Used------------------------------------------
% the following is to get qoppa and Qoppa
\DeclareFontFamily{T1}{cbgreek}{}
\DeclareFontShape{T1}{cbgreek}{m}{n}{<-6>  grmn0500 <6-7> grmn0600 <7-8> grmn0700 <8-9> grmn0800 <9-10> grmn0900 <10-12> grmn1000 <12-17> grmn1200 <17-> grmn1728}{}
\DeclareSymbolFont{quadratics}{T1}{cbgreek}{m}{n}
\DeclareMathSymbol{\qoppa}{\mathord}{quadratics}{19}
\DeclareMathSymbol{\Qoppa}{\mathord}{quadratics}{21}

\usepackage{amsmath, amssymb, amsthm}
\usepackage{longtable}
%\usepackage{amsfonts}
%\usepackage{mathtools}
%\usepackage{wasysym}
%\usepackage{MnSymbol}
%\usepackage{thmtools}
%\usepackage{stmaryrd}
\usepackage[letterpaper,margin=1in]{geometry}   
%\usepackage{slashed}
%\usepackage[english]{babel}				
%\usepackage[pdfencoding=auto, psdextra, draft=false]{hyperref}
\usepackage{bookmark}
\usepackage{url}		
\usepackage{lmodern}			
\usepackage[T1]{fontenc}
%\usepackage{xspace}		
%\usepackage{fancyhdr}
\usepackage{enumerate}
\usepackage{enumitem}
%\usepackage{mathrsfs}
\usepackage{graphicx}
\usepackage{soul,color}
\usepackage{tikz-cd}
\usepackage[maxbibnames=99,style=alphabetic]{biblatex}
\usepackage{csquotes}
\usepackage{chngcntr}
\usepackage[bbgreekl]{mathbbol}
\counterwithin{equation}{section}
\addbibresource{biblio.bib}
\usepackage{todonotes}

%hyperlink setup
\definecolor{darkred}{RGB}{128,0,0}
\definecolor{darkgreen}{RGB}{0,128,0}
\definecolor{darkblue}{RGB}{0,0,128}

\hypersetup{linktocpage,
	pdfborder = {0 0 0},
	colorlinks,
	citecolor=darkgreen,
	filecolor=darkred,
	linkcolor=darkblue,
	urlcolor=cyan!50!black!90}

%Greek and Latin black board bold-----------------------
\DeclareSymbolFontAlphabet{\mathbb}{AMSb}
\DeclareSymbolFontAlphabet{\mathbbl}{bbold}

%shortcut commands-------------------------------------------
\DeclareMathOperator{\Br}{Br} % Brauer functor
\DeclareMathOperator{\Brp}{Br^p} % Poincare Brauer functor
\DeclareMathOperator{\CAlg}{CAlg} % Commutative Algebra objects
\DeclareMathOperator{\CAlgp}{CAlg^p} % Poincare ring spectra
\DeclareMathOperator{\Cat}{Cat} % Categories
\DeclareMathOperator{\Catex}{\Cat_\infty^{ex}} % stable categories with exact functors
\DeclareMathOperator{\Cath}{Cat^h_\infty} % Hermitian Categories
\DeclareMathOperator{\Catp}{Cat^p_\infty} % Poincare Categories
\DeclareMathOperator{\Catpidem}{Cat^p_{\infty, idem}} % idempotent complete Poincare Categories
\DeclareMathOperator{\Einfty}{\mathbf{E}_\infty} % E-infinity 
\DeclareMathOperator{\ex}{ex} % exact 
\DeclareMathOperator{\Fun}{Fun} % Functors
\DeclareMathOperator{\gp}{gp} % grouplike
\DeclareMathOperator{\id}{id} % identity
\DeclareMathOperator{\idem}{idem} % idempotent
\DeclareMathOperator{\Mod}{Mod} % Modules
\DeclareMathOperator{\Pic}{Pic} % Picard functor
\DeclareMathOperator{\Picp}{Pic^p} % Poincare Picard functor
\DeclareMathOperator{\Pn}{Pn} % Poincare space functor
\DeclareMathOperator{\Spectra}{Sp} % Spectra
\DeclareMathOperator{\Spaces}{\mathcal{S}} % Spaces
\DeclareMathOperator{\Mon}{Mon} % monoids

\newcommand{\pf}{{\bf Proof. \ }}
\renewcommand{\epsilon}{\varepsilon}
\renewcommand{\rho}{\varrho}
\renewcommand{\phi}{\varphi}
\newcommand{\NN}{\ensuremath{\mathbb{N}}\xspace}
\newcommand{\ZZ}{\ensuremath{\mathbb{Z}}\xspace}
\newcommand{\QQ}{\ensuremath{\mathbb{Q}}\xspace}
\newcommand{\RR}{\ensuremath{\mathbb{R}}\xspace}
\newcommand{\CC}{\ensuremath{\mathbb{C}}\xspace}
\newcommand{\FF}{\ensuremath{\mathbb{F}}\xspace}
\newcommand{\EE}{\mathbb{E}}
\newcommand{\TT}{\ensuremath{\mathbb{T}}\xspace}
\newcommand{\RP}{\ensuremath{\mathbb{RP}}\xspace}
\newcommand{\DD}{\ensuremath{\mathbbl{\Delta}}\xspace}
\newcommand{\op}{\mathrm{op}} % opposite functor
\newcommand{\tc}{\ensuremath{\mathrm{TC}}}
\newcommand{\thh}{\ensuremath{\mathrm{THH}}}
\newcommand{\tp}{\ensuremath{\mathrm{TP}}}
\newcommand{\tr}{\ensuremath{\mathrm{TR}}}
\newcommand{\pnpic}{\ensuremath{\mathrm{PnPic}}}
\newcommand{\pnbr}{\ensuremath{\mathrm{PnBr}}}
\newcommand{\pic}{\ensuremath{\mathrm{Pic}}}
\newcommand{\br}{\ensuremath{\mathrm{Br}}}
\DeclareMathOperator*{\colim}{\ensuremath{\operatorname{colim}}}
\newcommand{\aps}{\mathrm{APS}}
\newcommand{\psch}{\mathrm{PSch}}
\newcommand{\perf}{\mathrm{Perf}}
\newcommand{\perfpn}{\mathrm{Perf}^{\mathrm{Pn}}}

%Theorem Environments ----------------------------------------------------------------
\newtheorem{theorem}[equation]{Theorem}
\newtheorem{proposition}[equation]{Proposition}
\newtheorem{lemma}[equation]{Lemma}
\newtheorem{corollary}[equation]{Corollary}

\theoremstyle{definition}
\newtheorem{definition}[equation]{Definition}
\newtheorem{construction}[equation]{Construction}
\newtheorem{remark}[equation]{Remark}
\newtheorem{observation}[equation]{Observation}
\newtheorem{notation}[equation]{Notation}
\newtheorem{example}[equation]{Example}
\newtheorem{recollection}[equation]{Recollection}

\newcommand{\Viktor}[1]{\todo{V: #1}}
\newcommand{\Noah}[1]{\todo[color=red]{N: #1}}
\newcommand{\Lucy}[1]{\todo[color=cyan]{\linespread{1}\footnotesize L: #1}}

\title{Poincar{\'e} Schemes}
\author{Ben Antieau, Viktor Burghardt, Noah Riggenbach, Lucy Yang}
\date{}
\addbibresource{biblio.bib}

\begin{document}

\maketitle
\begin{abstract}
    We do stuff \Noah{Change this}
\end{abstract}
\tableofcontents

\section{Introduction}

\begin{theorem}
Let $\underline{A}$ be an affine Poincar{\'e} scheme with underlying $\mathbb{E}_\infty$-ring spectrum with involution $A$. Then the natural maps \[\pi_i(\pnpic(\underline{A}))\to \pi_i(\pic(A))\] \Noah{I think there is some interaction with the homotopy fixed points, or maybe even the genuine fixed points}are surjective on $2$-torsion.
\end{theorem}

\begin{theorem}
    Let $A$ be an $\mathbb{E}_\infty$ ring with involution, and let $\underline{NA}$ be the associated Tate affine Poincar{\'e} scheme. Let $\br_\nu(A)$ be the Brauer group of Azumaya algebras over $A$ with involution. \Noah{I think we need to define this for ring spectra. For $A$ discrete this is done in \cite{azumaya_involution}.} Then the natural map \[\pnbr(\underline{NA})\to \br_\nu(A)\] is an equivalence\Noah{probably of $\mathbb{E}_\infty$ dodads}
\end{theorem}

\begin{theorem}
    The functors $\pnpic,\pnbr:\mathrm{APS}\to \Spectra$ are fppf sheaves.
\end{theorem}

\begin{theorem}
    There is a Poincar{\'e} group scheme $\mathbb{G}_m^\Qoppa$ such that \[B\mathbb{G}_m^\Qoppa\simeq \pnpic\] as fppf stacks.
\end{theorem}

\subsection{Conventions}
\label{subsection:conventions}
    \begin{longtable}{lll}
        $\Brp$ & Poincaré Brauer space\\
        $\CAlg$ & $\infty$-categoriy of $\Einfty$-ring spectra\\
        $\CAlg(\Spaces)$ & $\infty$-categoriy of $\Einfty$-spaces\\
        $\CAlg^{\gp}(\Spaces)$ & $\infty$-categoriy of grouplike $\Einfty$-spaces\\
        $\CAlgp$ & $\infty$-categoriy of Poincaré ring spectra\\
        $\Catex$ & $\infty$-category of small stable $\infty$-categories and exact functors\\
        $\Catp$ & $\infty$-category of Poincaré $\infty$-categories\\
        $\Catpidem$ & $\infty$-category of idempotent complete Poincaré $\infty$-categories\\
        $\Picp$ & Poincaré Picard space\\
        $\Spaces$ & $\infty$-category of spaces\\
        $\Spectra$ & $\infty$-category of spectra
    \end{longtable}

\section{Poincaré Structures on Compact Modules}
\label{section:poincare_structures_on_compact_modules}
We will use this section to recall notions and results about Poincaré $\infty$-categories which we require in the sections to follow. This section can safely be skipped by anyone who posses extensive knowledge of Poincaré $\infty$-categories, as found in \cite{CDHHLMNNSI}.

\begin{notation}
    \label{notation:omission_of_e_infty}
    Let $R$ be an $\mathbf{E}_\infty$-ring spectrum. We will drop $\mathbf{E}_\infty$ from our notation and simply call $R$ a \emph{ring spectrum}. Moreover, we will denote the $\infty$-category $\CAlg(\Spectra)$ of commutative algebra objects in the $\infty$-category of spectra $\Spectra$ by $\CAlg$. 
\end{notation}

Let $R$ be a ring spectrum and let $\Mod_R$ be the $\infty$-category of modules over $R$. We will study Poincaré structures on the $\infty$-category $\Mod_R^\omega$ of compact modules over $R$. 

\Viktor{-characterization in terms of modules with genuine involution, -characterization of symmetric monoidal structures, -Pn}

\section{Poincaré Ring Spectra}
\label{section:poincare_ring_spectra}
In this section we will define the ring theoretic building blocks of Poincaré schemes and the corresponding category they live in. Affine Poincaré Schemes will then be the dual objects, similar to how affine schemes are dual to commutative rings.

\begin{definition}
    \label{definition:poincare_ring_spectrum}
    Let $R$ be a ring spectrum. A \emph{Poincaré structure} on $R$ is a symmetric monoidal Poincaré $\infty$-category $\qoppa: (\Mod_R^\omega)^{\op}\rightarrow \Spectra $. We call such a symmetric monoidal Poincaré $\infty$-category a \emph{Poincaré ring spectrum}. We will denote the full subcategory of $\CAlg(\Catp)$ spanned by Poincaré ring spectra by $\CAlgp$ and call it the \emph{$\infty$-category of Poincaré ring spectra}.
\end{definition}

\begin{remark}
    \label{remark:notational_difference_to_nine-authored_papers}
    Poincaré ring spectra, as defined in Definition \ref{definition:poincare_ring_spectrum}, were studied in \Viktor{cite 9 authored paper}. Note that we chose a different notation. In \Viktor{cite 9 authored paper} Poincaré ring spectra are being referred to as $\mathbf{E}_\infty$-\emph{ring spectra with genuine involution}.
\end{remark}

\begin{remark}
    \label{remark:poincare_ring_spectra_as_nr-algebras}
    Let $R$ be a ring spectrum. By \Viktor{cite 9-authors} there is a natural equivalence between symmetric monoidal Poincaré structures on $\Mod_R^\omega$ and algebra objects over the genuine $C_2$-spectrum $NR$ \Viktor{reference}. In particular, a Poincaré structure on $R$ can be identified with the following data:
    \begin{itemize}
        \item A $C_2$-action on $R$ via maps of ring spectra, i.e. a functor $\lambda: BC_2\rightarrow \CAlg$.
        \item An $R$-algebra $R\rightarrow C$.
        \item An $R$-algebra map $C\rightarrow R^{tC_2}$. 
    \end{itemize}
        Here $R^{tC_2}$ is the Tate construction with respect to the above action. Since the Tate construction is lax symmetric monoidal, $R^{tC_2}$ is naturally an $R$-algebra via the Tate-valued norm. A ring spectrum equipped with a Poincaré structure will be called a \emph{Poincaré ring spectrum}.
\end{remark}

\begin{remark}
    \label{remark:poincare_structures_are_factorizations}
    By Remark \ref{remark:poincare_ring_spectra_as_nr-algebras}, a Poincaré structure on a ring spectrum $R$ with a $C_2$-action via maps of ring spectra is a factorization $R\rightarrow C \rightarrow R^{tC_2}$ in $\CAlg$ of the natural map $R\rightarrow R^{tC_2}$.
\end{remark}

\begin{remark}
    \label{remark:poincare_ring_spectra_as_algebra_objects}
    Let $\mathcal{M}$ be the full subcategory of $\Catp$ spanned by Poincaré $\infty$-categories with underlying $\infty$-category $\Mod^\omega_R$ for some ring spectrum $R$. Then the symmetric monoidal structure of $\Catp_\infty$ restricts to a symmetric monoidal structure on $\mathcal{M}$ by Example \ref{example:universal_poincare_ring_spectrum} and \Viktor{cite 9-authors I.5.1.5 and I.5.1.6}. Then we have $\CAlgp\simeq \CAlg(\mathcal{M})$. In particular, the symmetric monoidal structure of $\CAlg(Catp)$ restricts to a symmetric monoidal structure on $\CAlgp$.
\end{remark}

\begin{notation}
    \label{notation:spectrum_with_trivial_action}
    Let $R$ be a ring spectrum. We will denote by $\underline{R}$ the spectrum $R$ with trivial action. More precisely, $\underline{R}:BC_2\rightarrow \Spectra $ is the constant functor. \Lucy{This is commonly used for constant Mackey functors--could be ambiguous}
\end{notation}

\begin{example}
    \label{example:classification_of_poincare_structures_when_tate_vanishes}
    Let $R$ be a ring spectrum. If $2\in \pi_0(R)$ is invertible, we have $\underline{R}^{tC_2}\simeq 0$\Viktor{explain/reference}. A Poincaré structure on $R$ with the trivial action is then given by an $R$-algebra $R\rightarrow C$.
\end{example}

\begin{example}
    \label{example:tate_poincare_structure}
    Let $R$ be a ring spectrum equipped with a $C_2$-action via maps of ring spectra. The Tate-valued norm endows $ R^{tC_2} $ with a natural $R$-algebra structure, which induces a Poincaré structure on $R$ given by the factorization $R\xrightarrow{\id} R\rightarrow R^{tC_2}$. 
    We will call this Poincaré structure the \emph{Tate Poincaré structure on $R$}.
\end{example}

\begin{example}
    \label{example:universal_poincare_ring_spectrum}
    The sphere spectrum $\mathbb{S}$ together with the Tate Poincaré structure will be called the \emph{universal Poincaré ring spectrum}. \Viktor{expain why/translate universality statement to poincare ring spectra}
\end{example}

\begin{example}
    \label{example:symmetric_poincare_structure}
    Let $R$ be a ring spectrum equipped with a $C_2$-action via maps of ring spectra. The identity map $\id: R^{tC_2}\rightarrow R^{tC_2}$ induces a Poincaré structure on $R$
    given by the factorization $R\rightarrow R^{tC_2}\xrightarrow{id} R^{tC_2}$. We will call this Poincaré structure the \emph{symmetric Poincaré structure on $R$}.
\end{example}

\begin{example}
    \label{example:genuine_symmetric_poincare_structure}
    Let $R$ be a connective ring spectrum equipped with a $C_2$-action via maps of ring spectra. The connective cover $\tau_{\geq 0}(R^{tC_2})\rightarrow R^{tC_2}$ of $R^{tC_2}$ induces a Poincaré structure on $R$ given by the factorization $R\rightarrow \tau_{\geq 0}(R^{tC_2})\rightarrow R^{tC_2}$. We will call this Poincaré structure the \emph{genuine symmetric Poincaré structure on $R$}.
\end{example}

\Viktor{copy more examples from notes}

\begin{definition}
    \label{definition:category_of_poincare_ring_spectra}
    Let $A$ and $R$ be Poincaré ring spectra. A \emph{map of Poincaré ring spectra} between $A$ and $R$ is a map of ring spectra $f:A\rightarrow R$ compatible with the corresponding Poincaré structures via the following additional data: 
    \begin{itemize}
        \item \Viktor{this should become a remark and go below the definition of calgp}
    \end{itemize}
\end{definition}

\section{Modules over Poincaré Ring Spectra}
\label{subsection:modules_over_poincare_ring_spectra}

Let $A$ be a Poincaré ring spectrum. Then $A$ is a commutative algebra object in the $\infty$-category of Poincaré $\infty$-categories $\Catp$\Viktor{ref}. We may thus consider modules over it. In this section we will use modules over Poincaré ring spectra to define analogues of the Brauer and Picard groups for Poincaré ring spectra.\\



\subsection{The Poincaré Picard Group}
\label{subsection:the_poincare_picard_group}

Recall that the Poincaré space functor $ \Pn \colon \Catp \to \CAlg(\Spaces) $ is lax symmetric monoidal with respect to tensor product of Poincaré $ \infty $-categories and smash product of $ \EE_\infty $-spaces \cite[Corollary 5.2.8]{CDHHLMNNSI}. In particular, we can consider invertible objects in $\Pn(A)$ for a Poincaré ring spectrum $A$.

\begin{definition}
    \label{definition:poincare_picard_space}
    Let $A$ be a Poincaré ring spectrum. We define the \emph{Picard space of $A$} to be $$\Picp(A):=\Pic(\Pn(A)).$$
\end{definition}

\begin{remark}
    \label{remark:poincare_picard_points_desc}
    Let $ \left(\Mod_R^\omega, \Qoppa_R \right)$ be a Poincaré ring spectrum, where $(M_R=R, N_R= R^{\varphi C_2}, R^{\varphi C_2}\to R^{tC_2})$ is the module with genuine involution associated to $ \Qoppa_R $. 
    Then a point in the Poincaré Picard space is the data of a pair $ (\mathcal{L}, q ) $, where $ \mathcal{L} $ is an invertible module in $ \Mod_R^\omega $ and $ q $ is a point in $ \Omega^\infty\Qoppa_R(\mathcal{L}) $. 
    By \cite[Proposition 1.3.11]{CDHHLMNNSI}, the data of $ q $ is equivalent to the data of points in the lower left and upper right corner of the square
    \begin{equation}
    \begin{tikzcd}
        \Qoppa(\mathcal{L}) \ar[r] \ar[d] & \hom_R(\mathcal{L}, R^{\varphi C_2}) \ni \ell(q) \ar[d] \\
        b(q) \in \hom_{R \otimes R}\left(\mathcal{L} \otimes \mathcal{L}, R\right)^{hC_2} \ar[r] & \hom_R(\mathcal{L}, R^{tC_2})
    \end{tikzcd}
    \end{equation} 
    and a path between their images in the lower right corner. 
    In particular, the adjoint of $ b(q) $ must define a nondegenerate hermitian form on $ \mathcal{L} $, that is, an equivalence $ \mathcal{L} \simeq \hom_{R}(\mathcal{L}, R^*) $ where $ R^* $ is considered as an $ R $-module via the action of the generator of $ C_2 $. \Lucy{add equivariance/symmetry data}

    Write $ (\mathcal{L}^\vee,q^\vee) $ is for the inverse of $ (\mathcal{L},q) $. 
    By definition of invertibility, there exists an $ R $-linear map $ \ell(q^\vee) \colon \mathcal{L}^\vee \to R^{\varphi C_2} $ so that the following diagram commutes
    \begin{equation}\label{diagram:pnpic_linear_part_condition}
    \begin{tikzcd}[column sep=huge]
        \mathcal{L} \otimes_R \mathcal{L}^\vee \ar[d,"\mathrm{ev}", "\sim"']  \ar[r,"{\ell(q) \otimes \ell(q^\vee)}"] & R^{\varphi C_2} \otimes_R R^{\varphi C_2} \ar[d,"\mathrm{multiplication}"] \\
        R \ar[r,"\mathrm{given}"] & N_R   
    \end{tikzcd}
    \end{equation} 
\end{remark}

\subsection{The Poincaré Brauer Group}
\label{subsection:the_poincare_brauer_group}

Recall that a Poincaré $\infty$-category is called idempotent complete if the underlying stable $\infty$-category is idempotent complete. The full subcategory of $\Catp$ spanned by idempotent complete Poincaré $\infty$-categories is denoted by $\Catpidem$ \cite[Definition 1.3.2]{CDHHLMNNSII}.

\begin{definition}
    \label{definition:poincare_brauer_space}
    Let $A$ be a Poincaré ring spectrum. We define the \emph{Poincaré Brauer space of $A$} as $$\Brp(A):=\Pic(\Mod_A(\Catpidem)).$$
    The assignment $ A \mapsto \Brp(A) $ defines a functor
    \begin{equation*}
        \Brp \colon \CAlgp \to \CAlg^{\gp}(\Spaces)
    \end{equation*}
    valued in grouplike $ \Einfty $-spaces. 
\end{definition}

\begin{remark}
    The symmetric monoidal forgetful functor $ \Mod_A(\Catpidem) \to \Mod_A(\Cat^{\mathrm{ex}}_\infty) $ induces a map $ \Brp(A) \to \Br(A) $ of grouplike $ \Einfty $-spaces, where $ \Br(A) $ is the Brauer space $ \mathrm{br}_{\mathrm{alg}}(A) $ of \cite[1154-1155]{MR3190610}. 
\end{remark}
\begin{proposition}
    Let $A$ be a Poincaré ring spectrum. Then we have a canonical equivalence $$\Omega \Brp(A) \simeq \Picp(A).$$ \Lucy{What else do we need to do to show that we have an equivalence of \emph{functors}?}
\end{proposition}
\begin{proof} 
    \Viktor{todo} 
    Since $ \Omega\Brp(R) $ is given by the space of automorphisms of any object in $ \Brp(R) $, it suffices to determine the space of autoequivalences of $ \left(\Mod_R^\omega, \Qoppa_R \right) $. 
    An autoequivalence is the data of a pair $ (f, \eta) $ where $ f \colon \Mod_R^\omega \xrightarrow{\simeq} \Mod_R^\omega $ is an exact $ R $-linear autoequivalence and $ \eta \colon \Qoppa_R \xrightarrow{\sim} \Qoppa_R \circ f^{\mathrm{op}} $ is a natural equivalence. 
    Since $ \Catp_R\to \Catex_R $ is symmetric monoidal, $ f $ is of the form $ - \otimes_R \mathcal{L} $ where $ \mathcal{L} $ is an invertible $ R $-module. 
    Since taking bilinear and linear parts is functorial/by \cite[Proposition 1.3.11]{CDHHLMNNSI}, $ \eta $ is equivalently the data of a pair of equivalences
    \begin{align*}
        b(\eta) &\colon \hom_{R \otimes R}((-\otimes \mathcal{L}) \otimes (-\otimes \mathcal{L}), R)^{hC_2} \simeq \hom_{R \otimes R}(-\otimes -, R)^{hC_2} \\
        \ell(\eta) &\colon \hom_R( -\otimes \mathcal{L}, R^{\varphi C_2}) \simeq \hom_R( -, R^{\varphi C_2})
    \end{align*} 
    plus a path between their images in $ \hom_R(\mathcal{L}, R^{tC_2}) $. 
    The transformation $ b(\eta) $ is equivalent to the data of an $ R $-bilinear equivalence $ R \simeq \mathcal{L}^\vee \otimes \mathcal{L}^\vee $ \Lucy{maybe one of these should be conjugate dual here?}, and the transformation $ \ell(\eta) $ is equivalent to the data of an $ R^{\varphi C_2} $-linear\Lucy{is the $R^{\varphi C_2}$-linearity of this $\simeq$ correct?} equivalence $ \ell (\eta)\colon R^{\varphi C_2}\otimes_R \mathcal{L}^\vee \xrightarrow{\sim} R^{\varphi C_2} $. 

    Now consider the composites 
    \begin{align*}
        R \otimes_R \mathcal{L}^\vee \xrightarrow{\mathrm{unit}\,\otimes \mathrm{id}} R^{\varphi C_2} \otimes \mathcal{L}^\vee \xrightarrow{ \ell(\eta)} R^{\varphi C_2} \\
        R \otimes_R \mathcal{L} \xrightarrow{\mathrm{unit}\,\otimes \mathrm{id}} R^{\varphi C_2} \otimes \mathcal{L} \xrightarrow{ \ell(\eta)^{-1} \otimes \mathrm{id}_{\mathcal{L}}} R^{\varphi C_2} \,.
    \end{align*}
    These correspond to the $ \ell(q^\vee), \ell(q) $ of Remark \ref{rmk:poincare_picard_points_desc}, respectively. 
    In particular, the condition that $ \ell(q^\vee), \ell(q) $ make the diagram (\ref{diagram:pnpic_linear_part_condition}) commute is equivalent to the condition that $ \ell(\eta) $ is an equivalence by an adjunction argument. \Lucy{under construction--not sure what to say about the $(-)^{tC_2} $ part yet. }
\end{proof}

\begin{proposition}
    Let $ (\Mod_R^\omega, \Qoppa_R) $ be a Poincaré ring spectrum. 
    \begin{enumerate}
        \item \label{propitem:Rlin_Poincare_cats_is_symm_mon} The $ \infty $-category $ \Mod_{\left(\Mod_R^\omega, \Qoppa_R \right)}(\Catpidem) $ inherits a canonical symmetric monoidal structure, and for every morphism $ \left(R, R^{\varphi C_2} \to R^{tC_2}\right) \to (S, S^{\varphi C_2} \to S^{tC_2}) $, the functor $ \Mod_{\left(\Mod_R^\omega, \Qoppa_R \right)}(\Catpidem) \to \Mod_{\left(\Mod_S^\omega, \Qoppa_S \right)}(\Catpidem) $ is symmetric monoidal. 
        \item \label{propitem:classify_R_lin_hermitian_struct} Let $ A $ be an $ \EE_1 $-$ R $-algebra in spectra, and regard the category of compact right $ A $-modules $ \Mod_A^\omega $ as left-tensored over $ \Mod_R^\omega $ in the canonical way. 
        Then the pullback
        \begin{equation}
        \begin{tikzcd}
            & \Mod_{(\Mod_R^\omega, \Qoppa_R)}\left(\Cath\right) \ar[d] \\
            \{\Mod_A^\omega\} \ar[r] & \Catex_R
        \end{tikzcd}
        \end{equation}
        is canonically equivalent to $ \Mod_{N_R A \otimes_{N_R R} R^L }\left(\Spectra^{C_2}\right) $ where $ R^L $ is the $ \EE_\infty $-$ N_R R $-algebra with $ (R^L)^e \simeq R $ and $ (R^L)^{\varphi C_2}  \simeq C $. 

        A $ N_R A \otimes_{N_R R} R^L $-module classifies a $ (\Mod_R^\omega, \Qoppa_R) $-module in Poincaré $ \infty $-categories if its underlying $ R $-module is invertible in the sense of \cite[Definition 3.1.4]{CDHHLMNNSI}. 
        \item Let $ A,B $ be $ R $-algebras with associated ($R$-linear) modules with genuine involution $ (M_A, N_A, N_A \to M_A^{tC_2}) $ and $ (M_B, N_B, N_B \to M_B^{tC_2}) $, respectively so that (under item \ref{propitem:classify_R_lin_hermitian_struct}) $ \left(\Mod_A^\omega, \Qoppa_A \right) $ and $ \left(\Mod_B^\omega, \Qoppa_B \right) $ are objects of $ \Mod_{\left(\Mod_R^\omega, \Qoppa_R \right)}(\Catpidem) $. 
        Then the symmetric monoidal structure of item \ref{propitem:Rlin_Poincare_cats_is_symm_mon} is so that the underlying $ R$-linear $ \infty $-category with perfect duality $ \left(\Mod_A^\omega, \Qoppa_A \right) \otimes_{\left(\Mod_R^\omega, \Qoppa_R \right)} \left(\Mod_B^\omega, \Qoppa_B \right) $ is $ \Mod_A^\omega \otimes_{\Mod_R^\omega} \Mod_B^\omega \simeq \Mod_{A \otimes_R B}^\omega $, and the associated module with genuine involution is given by $ M_A \otimes_R M_B $, $ N_A \otimes_{R^{\varphi C_2}} N_B $, and the structure map is $ N_A \otimes_{R^{\varphi C_2}} N_B \to M_A^{tC_2} \otimes_{R^{tC_2}} M_B^{tC_2} \to (M_A \otimes_R M_B)^{tC_2} $ where the latter map arises canonically from lax monoidality of the Tate construction. 
        % Commented out Aug 14th--just leaving this here just in case: \Lucy{What if it should be $ N_A \otimes_{R^{\varphi C_2}} N_B \to M_A^{tC_2} \otimes_{R^{tC_2}} M_B^{tC_2} \to (M_A \otimes_R M_B)^{tC_2} $ instead?} 
    \end{enumerate}
\end{proposition}
\begin{remark}
    A special case of part \ref{propitem:classify_R_lin_hermitian_struct} is \cite[Example 5.4.13]{CDHHLMNNSI}.
\end{remark}
\begin{proof}
    \begin{enumerate}
        \item 
        \item Let $ \mathcal{LM}^\otimes $ denote the $ \infty $-operad of \cite[Definition 4.2.1.7]{LurHA}. 
        Our strategy of proof will be similar to that of \cite[\S5.3]{CDHHLMNNSI}: First, we show that an $ \mathcal{LM}^\otimes $-algebra object in $ \Cath $ is equivalent to an $ \mathcal{LM}^\otimes $-algebra object in an operad of functor categories. 
        Then, we use a (suitably coherent version of) the classification of hermitian structures on module categories as categories of modules over the Hill--Hopkins--Ravenel norm \cite[Theorem 3.3.1]{CDHHLMNNSI} to conclude. 
        Recall that the action of $ \Mod_R^{\omega} $ on $ \Mod_A^\omega $ is given by a functor $ \mathcal{LM}^\otimes \to \Cat_\infty^\times $, and define $ \Fun_{\Mod_R^{\omega,\op}}(\Mod_A^{\omega,\op},\Spectra)^\otimes $ via the following pullback square of $ \infty $-operads: 
        \begin{equation}
        \begin{tikzcd}
            \Fun_{\Mod_R^{\omega,\op}}(\Mod_A^{\omega,\op},\Spectra)^\otimes \ar[r,"{p}"]\ar[d] & \mathcal{LM}^\otimes \ar[d,"{\Mod_R^\omega,\Mod_A^\omega}"] \ar[d] \\
            \left(\Cat_\infty\right)_{\op/-/\Spectra}^\otimes \ar[r] & \Cat_\infty^\times 
        \end{tikzcd}\,.
        \end{equation}
        Informally, an object $ F \in \Fun_{\Mod_R^{\omega,\op}}(\Mod_A^{\omega,\op},\Spectra)^\otimes_{\mathfrak{a}} $ is a functor $ F \colon \Mod_R^{\omega,\op} \to \Spectra $ and an object $ G $ over the fiber of $ \mathfrak{m} $ is a functor $ G \colon \Mod_A^{\omega,\op} \to \Spectra $. 
        The $p $-cartesian edge over the canonical map $ (\mathfrak{a},\mathfrak{m}) \to \mathfrak{m} $ in $ \mathcal{LM}^\otimes $ sends $ (F,G) $ to the lower arrow in the diagram
        \begin{equation*}
        \begin{tikzcd}[column sep=4.5em]
            \Mod_R^{\omega,\op}\times \Mod_A^{\omega,\op} \ar[r,"{F\times G}"] \ar[d,"{-\otimes_R -}"'] & \Spectra \times \Spectra \ar[d,"{\otimes_{\Spectra}}"] \\
            \Mod_A^{\omega,\op} \ar[r,"{F \otimes G := \mathrm{LKE}_{\otimes_R}( \otimes_{\Spectra} \circ (F \times G))}"] & \Spectra
        \end{tikzcd}\,.
        \end{equation*}
        Now define $ \Fun_{\Mod_R^{\omega,\op}}^q(\Mod_A^{\omega,\op},\Spectra)^\otimes $ to consist of the full subcategory of $ \Fun_{\Mod_R^{\omega,\op}}(\Mod_A^{\omega,\op},\Spectra)^\otimes $ consisting of those tuples of functors which are all quadratic. 
        The inclusion $ \Fun_{\Mod_R^{\omega,\op}}^q(\Mod_A^{\omega,\op},\Spectra)^\otimes \to \Fun_{\Mod_R^{\omega,\op}}(\Mod_A^{\omega,\op},\Spectra)^\otimes $ exhibits the former as an $ \infty $-operad, and moreover the localization is compatible with the $ \mathcal{LM}^\otimes $-monoidal structure in the sense of \cite[Definition 2.2.1.6]{LurHA}. 
        We can extend the previous diagram to 
        \begin{equation}\label{diagram:module_structure_on_quadratic_functors}
        \begin{tikzcd}
            \Fun_{\Mod_R^{\omega,\op}}^q(\Mod_A^{\omega,\op},\Spectra)^\otimes \ar[r]\ar[d] & \Fun_{\Mod_R^{\omega,\op}}(\Mod_A^{\omega,\op},\Spectra)^\otimes \ar[r,"{p}"]\ar[d] & \mathcal{LM}^\otimes \ar[d,"{\Mod_R^\omega,\Mod_A^\omega}"] \ar[d] \\
           \Cath^\otimes \ar[r]& \left(\Cat_\infty\right)_{\op/-/\Spectra}^\otimes \ar[r] & \Cat_\infty^\otimes 
        \end{tikzcd}\,.
        \end{equation}
        Modifying \cite[Construction 5.3.15 \& Lemma 5.3.15]{CDHHLMNNSI} slightly (note that Corollary 5.1.4 did not assume the tensor factors to be equivalent), we obtain an analogous commutative diagram of $ \infty $-operads
        \begin{equation}\label{diagram:module_structure_on_pb_functors}
        \begin{tikzcd}
            \Fun_{\Mod_R^{\omega,\op}}^p(\Mod_A^{\omega,\op},\Spectra)^\otimes \ar[d] \ar[r] & \Fun_{\Mod_R^{\omega,\op}}^q(\Mod_A^{\omega,\op},\Spectra)^\otimes \ar[r]\ar[d] & \Fun_{\Mod_R^{\omega,\op}}(\Mod_A^{\omega,\op},\Spectra)^\otimes \ar[r,"{p}"]\ar[d] & \mathcal{LM}^\otimes \ar[d,"{\Mod_R^\omega,\Mod_A^\omega}"] \ar[d] \\
           \Catp^\otimes \ar[r] &\Cath^\otimes \ar[r]& \left(\Cat_\infty\right)_{\op/-/\Spectra}^\otimes \ar[r] & \Cat_\infty^\otimes 
        \end{tikzcd}
        \end{equation}
        in which all squares are pullbacks. 
        Now suppose $ A $ is given a module with genuine involution $ (M_A, N_A,N_A \to M_A^{tC_2}) $ and call the associated Poincaré $\infty$-category $ \overline{\Mod}_A $.  
        Then to lift $ \overline{\Mod}_A $ to a module over $ \left(\Mod_R^\omega,\Qoppa_R\right) $ compatibly with the $ \Mod_R^\omega $-module structure on $ \Mod_A^\omega $ is to give a map of $ \infty $-operads $ \mathcal{LM}^\otimes \to \Cath^\otimes $ so that the restriction along the canonical inclusion $ \mathrm{Assoc}^\otimes \to \mathcal{LM}^\otimes $ gives the algebra object $ \left(\Mod_R^\omega,\Qoppa_R\right) $ and postcomposing with the canonical projection to $ \Catex^\times $ recovers the given $ \Mod_R^\omega $-module structure on $ \Mod_A^\omega $. 
        By the pullback square (\ref{diagram:module_structure_on_quadratic_functors}), this is equivalent to giving an object of $ \mathrm{Alg}_{\mathcal{LM}/\mathcal{LM}}\left(\Fun_{\Mod_R^{\omega,\op}}^q(\Mod_A^{\omega,\op},\Spectra)^\otimes\right) $. 
        Now let us identify the bilinear functor $ \Mod_R^\omega \times \Mod_A^\omega \xrightarrow{ - \otimes_R -} \Mod_A^\omega $ with the exact functor $ \Mod_R^\omega \otimes \Mod_A^\omega \simeq \Mod_{R \otimes A}^\omega \to \Mod_A^\omega $ which is induction along the action map $ R \otimes A \to A $. 
        Using \cite[Corollary 3.4.1]{CDHHLMNNSI} and unravelling definitions gives the claim for $ R$-linear hermitian structures. 
        The proof for $ R $-linear Poincaré structures considers (\ref{diagram:module_structure_on_pb_functors}) instead but otherwise proceeds in an identical fashion.  
        \item By \cite[Theorem 4.4.2.8]{LurHA}, the relative tensor product $ \left(\Mod_A^\omega, \Qoppa_A \right) \otimes_{\left(\Mod_R^\omega, \Qoppa_R \right)} \left(\Mod_B^\omega, \Qoppa_B \right) $ is computed as the geometric realization of the bar construction 
        \begin{align*}
            p \colon \Delta^\op & \to \Cath \\
            [n] &\mapsto \left(\Mod_A^\omega, \Qoppa_A \right) \otimes \left(\Mod_R^\omega, \Qoppa_R \right)^{\otimes n} \otimes \left(\Mod_B^\omega, \Qoppa_B \right)
        \end{align*} 
        Write $ f \colon \Cath \to \Catex $ for the forgetful functor. 
        Then $ f \circ p $ has a colimit with value $ \Mod_A^\omega \otimes_{\Mod_R^\omega} \Mod_B^\omega \simeq \Mod_{A \otimes_R B}^\omega $. 
        Writing $ g \colon \Catex \to \{*\} $, by Example 4.3.1.3 of \cite{HTT} we see that $ f \circ p $ is a $ g $-colimit. 
        By Proposition 4.3.1.5(2) and Example 4.3.1.3 of \cite{HTT}, $ p $ admits a colimit in $ \Cath $ if and only if it admits an $ f $-colimit. 
        Now recall that $ f $ is a cocartesian fibration with pushforward given by left Kan extension \cite[Corollary 1.4.2]{CDHHLMNNSI}. 
        We show that $ f $ satisfies the conditions of \cite[Corollary 4.3.1.11]{HTT}. 
        \begin{itemize}
            \item Condition (1) follows from Theorem 6.1.1.10 of \cite{LurHA} applied to $ \Spectra^\op $ (see the end of \cite[Construction 1.1.26]{CDHHLMNNSI}). 
            \item Condition (2) follows from \cite[Corollary 1.4.2]{CDHHLMNNSI}, the adjoint functor theorem, and presentability of $ \Fun^q(\mathcal{C}) $, which is discussed in the proof of \cite[Lemma 5.3.3]{CDHHLMNNSI} (also see \cite[Remark 6.1.1.11]{LurHA}). 
        \end{itemize}
        Thus the preceding discussion shows that there exists a map of simplicial sets $ p' $ making the diagram commute
        \begin{equation*}
        \begin{tikzcd}
            \Delta^\op \ar[d] \ar[r,"p"] & \Cath \ar[d,"f"] \\
            \left(\Delta^\op\right)^{\triangleright} \ar[r]\ar[ru,"{p'}"] & \Catex 
        \end{tikzcd}\,.
        \end{equation*}
        Since $ \{0\} \to \Delta^1 $ is left anodyne, by \cite[Corollary 2.1.2.7]{HTT} the inclusions
        \begin{align*}
            \{0\} \times \Delta^\op & \to \Delta^1 \times \Delta^\op \\
            \iota \colon \left(\{0\} \times (\Delta^\op)^\triangleright\right) \sqcup_{\{0\} \times \Delta^\op}\left(\Delta^1 \times \Delta^\op \right) & \to \Delta^1 \times \left(\Delta^\op\right)^\triangleright 
        \end{align*}
        are left anodyne. 
        The former implies that there exists a map $ p'' $ making the diagram 
        \begin{equation*}
        \begin{tikzcd}
            \{0\} \times \Delta^\op \ar[r,"p"]\ar[d] &\Cath \ar[d,"f"] \\
            \Delta^1 \times \Delta^\op \ar[r] \ar[ru,"{p''}"] & \Catex
        \end{tikzcd}
        \end{equation*}
        commute. 
        The maps $ p' $ and $ p'' $ assemble to give a map $ p''' $ making the diagram 
        \begin{equation*}
        \begin{tikzcd}[arrows={crossing over},row sep=large]
            & \{0\} \times \Delta^\op \ar[r,"p"]\ar[ld, bend right=15] \ar[d] &\Cath \ar[d,"f"] \\
            \left(\{0\} \times (\Delta^\op)^\triangleright\right) \sqcup_{\{0\} \times \Delta^\op}\left(\Delta^1 \times \Delta^\op \right) \ar[r] \ar[rru,"{p'''}", near start] & \Delta^1 \times \left(\Delta^\op\right)^\triangleright \ar[r] \ar[ru,"{\overline{p}}"', bend right=10] & \Catex
        \end{tikzcd}
        \end{equation*}
        commute, and likewise $ \overline{p} $ exists making the diagram commute since $ \iota $ is left anodyne. 
        Now we show that $ \overline{p} $ satisfies the conditions of \cite[Proposition 4.3.1.9]{HTT}. 
        By (the opposite/dual/cocartesian version of) \cite[Remark 3.1.1.10]{HTT} and Proposition 3.1.1.5(2'') \emph{ibid.} and the fact that $ f $ is a cocartesian fibration, we can choose $ \overline{p} $ so that for all $ k \in (\Delta^\op)^\triangleright $, $ \overline{p}|_{\Delta^1 \times \{k\}} $ is $f$-cocartesian. 
        Furthermore, since we can choose $ \Delta^\op, \,\left(\Delta^\op\right)^\triangleright $ to have the markings $ (-)^\flat $ in \cite[Remark 3.1.1.10]{HTT}, $ f \circ \overline{p}|_{\Delta^1 \times \{\infty\}} $ is a degenerate edge in $ \Catex $. 

        Now \cite[Proposition 4.3.1.9]{HTT} implies that $ \overline{p}_0 $ is an $ f $-colimit diagram if and only if $ \overline{p}_1 $ is an $ f $-colimit diagram. 
        Now notice that $ \overline{p}|_{\{1\} \times \left(\Delta^\op\right)^{\triangleright}} $ has image contained in the fiber of $ f $ over $ \Mod_{A \otimes_R B}^\omega $. 
        By \cite[Proposition 4.3.1.10]{HTT}, it suffices to show that $ \overline{p}_1 $ is a colimit diagram in $ \Fun^q\left(\Mod_{A \otimes_R B}^\omega\right) $. 
        Write $ \overline{M}_A \in \Mod_{N^{C_2}A} $ and $ \overline{M}_B \in \Mod_{N^{C_2}B} $ for the corresponding modules (see introduction to \S3.3 of \cite{CDHHLMNNSI}). 
        Unraveling definitions and using \cite[Theorem 3.3.1 \& Corollary 3.4.1 \& Lemma 5.4.6]{CDHHLMNNSI}, it follows that the diagram $ \overline{p}_1|_{\{1\} \times \Delta^\op} $ is the bar construction 
        \begin{align*}
            [n] & \mapsto \overline{M}_A \otimes_{N^{C_2}R} R^{\otimes_{N^{C_2}R} n} \otimes_{N^{C_2} R} \overline{M}_B \,.
        \end{align*}
        This proves the result. \qedhere
        %\Lucy{Point is: $ \iota $ is \emph{marked anodyne} and . Then compute the colimit in $ \Fun^q \left(\Mod_{A\otimes_R B}^\omega \right) $.}
    \end{enumerate}
\end{proof}

\subsection{Azumaya algebras with genuine involution}
Let $ R $ be an $ \EE_\infty $-ring spectrum. 
\begin{recollection}\label{rec:Azumaya_alg_wo_involution} 
    Recall \cites{MR2927172,MR3190610} that an $ \EE_1 $-$ R $-algebra is said to be \emph{Azumaya} if it is a compact generator of $ \Mod_R $ and if the natural $ R $-algebra map giving the bimodule structure on $ A $
    \begin{equation*}
        A \otimes_R A^{\mathrm{op}} \to \mathrm{End}_R(A)
    \end{equation*}
    is an equivalence of $ R $-algebras. 
\end{recollection}

\begin{definition}
    \Lucy{For context, compare Examples 3.1.9 and 3.2.9 of the first 9-author paper.}
    Let $ (R, R \to R^{\varphi C_2} \to R^{tC_2}) $ be a Poincaré ring spectrum. 
    An \emph{Azumaya algebra with genuine involution}\Lucy{terminology: ``involution'' or ``anti-involution''?} over $ R $ is the data of 
    \begin{enumerate}[label=(\alph*)]
        \item An $ \EE_1 $-$ R $-algebra $ A $ equipped with an anti-involution $ \tau \colon A \to A^\op $ so that $ A $ is an Azumaya $ R $-algebra in the sense of Recollection \ref{rec:Azumaya_alg_wo_involution} 
        \item A left $ A \otimes_{R} R^{\varphi C_2} $-module $ A^{\varphi C_2} $ 
        \item \label{defn_item:Azumaya_gen_inv_geom_fixpt} An equivalence of $ A \otimes_R A^\op $-modules\Lucy{compatibility with Tate?}
        \begin{equation*}
            \hom_R(A, R^{\varphi C_2}) \simeq A^{\varphi C_2} \otimes_{R^{\varphi C_2}} \hom_R(A^{\varphi C_2}, R^{\varphi C_2})
        \end{equation*}
        \item An $ A $-linear map $ A^{\varphi C_2} \to A^{tC_2} $ where $ A^{tC_2} $ is regarded as an $ A $-module via the Tate-valued Frobenius.\Lucy{Tate-valued norm?}
    \end{enumerate}
    \Lucy{define the category!}
\end{definition}
\begin{remark}\label{rmk:azumaya_geninv_gives_module_geninv}
    If $ A $ is an Azumaya algebra with genuine involution over $ R $, then in particular $ M_A = A $, $ N_A = A^{\varphi C_2} $ is a module with genuine involution over $ A $ in the sense of \cite[Definition 3.2.3]{CDHHLMNNSI}. 
\end{remark}
With ordinary Azumaya algebras, the prototypical Azumaya algebra with anti-involution arises from endomorphism rings of perfect modules. 
Choosing a (nondegenerate symmetric bilinear) form on a perfect module $ P $ endows its endomorphism algebra with additional structure. 
\begin{example}
    Let $ (R, R \to R^{\varphi C_2} \to R^{tC_2} ) $ be a Poincaré ring, and let $ (P,q ) \in \mathrm{Pn}\left(\Mod_R^\omega, \Qoppa_R \right) $. 

    Then $ A:= \mathrm{End}_R(P) $ admits a canonical lift to an Azumaya algebra with genuine involution over $ R $ with $ A^{\varphi C_2}:= \hom_R(P, R^{\varphi C_2}) $. 
    \Lucy{to-do: explain! Also $ A^{\varphi C_2}$ could also be $ P \otimes_R R^{\varphi C_2}$?} 

    By \cite[Proposition 3.1.16]{CDHHLMNNSI}, $ A $ inherits a canonical anti-involution. \Lucy{todo: Make some noises about an $ R $-linear enhancement of Proposition 3.1.16?}
    Since $ P $ is compact, it is dualizable with respect to the symmetric monoidal structure on $ \Mod_R^\omega $ \cite[Theorem III.7.9]{elmendorf2007rings}. 
    Since $ \otimes_R R^{\varphi C_2} $ is symmetric monoidal, in particular it takes $ P $ to a dualizable object--call it $ \overline{P} $. 
    Now there is a canonical choice of equivalence \ref{defn_item:Azumaya_gen_inv_geom_fixpt} since both sides are canonically equivalent to $ \overline{P} \otimes_{R^{\varphi C_2}} \overline{P}^\vee $. 
\end{example}

\begin{proposition}
    Let $ R $ be the Eilenberg--Mac Lane spectrum associated to a discrete ring, and suppose $ R $ has a given $ C_2 $-action. 
    Let $ A $ be a classical Azumaya algebra over $ R $ with an involution of type 2. 
    Regard $ R $ as a Poincaré ring spectrum with the genuine symmetric Poincaré structure of Example \ref{example:genuine_symmetric_poincare_structure}. 

    Then there is a canonical Azumaya algebra with genuine involution over $ R^{\mathrm{gs}} $ so that $ A^{\varphi C_2} := \tau_{\geq 0} A^{tC_2} $.
\end{proposition}
\begin{proof}
    \Lucy{todo}
\end{proof}

\begin{proposition}\label{prop:mod_over_azumaya_geninv_is_invertible}
    Let $ (R, R \to R^{\varphi C_2} \to R^{tC_2} ) $ be a Poincaré ring, and let $ (A, A^{\varphi C_2} \to A^{tC_2}) $ be an Azumaya algebra with genuine involution over $ R $. 
    Then $ \left(\Mod_A^\omega, \Qoppa_A \right) $ is an invertible object in $ \Mod_{\left(\Mod_R^\omega, \Qoppa_R\right)}\left(\Catpidem\right) $. 
\end{proposition}
\begin{remark}
    Contrast Proposition \ref{prop:mod_over_azumaya_geninv_is_invertible} with \cite[Theorem 3.15]{MR3190610}, where it is shown that an $ R $-linear stable $ \infty $-category is invertible \emph{if and only if} it is equivalent to modules over an Azumaya $ R $-algebra. 
    The difference lies in the fact that not every $ R $-linear (Morita anti-)equivalence $ \Mod_A^\omega \simeq \Mod_{A^\op}^\omega $ is induced by a map of $ \EE_1 $-rings $ A \to A^\op $. 
    \Lucy{this is not quite the same (in a literal sense), but I think similar in spirit to the ``counterexamples'' paper by First--Williams}
\end{remark}
\begin{proof}
    First, by \cite[Example 3.2.9]{CDHHLMNNSI}, we see that $ \left(\Mod_A^\omega, \Qoppa_A \right) $ is indeed an $ R$-linear Poincaré $ \infty $-category (and not merely hermitian). 
    To show that the associated Poincaré $ \infty $-category is invertible, we must identify a dual $ \left(\Mod_A^\omega, \Qoppa_A \right)^\vee $ and exhibit an equivalence $ \left(\Mod_A^\omega, \Qoppa_A \right) \otimes \left(\Mod_A^\omega, \Qoppa_A \right)^\vee \simeq \left(\Mod_R^\omega, \Qoppa_R \right) $. 
    Since $ \Catpidem_R \to \Catex_R $ is symmetric monoidal, we see that the underlying $ R $-linear $ \infty $-category associated to the dual must be $ \Mod_{A^\op}^\omega $. 
    Moreover, the canonical evaluation map $ \mathrm{ev} \colon \Mod_A^\omega \otimes \Mod_{A^\op}^\omega \xrightarrow{\simeq} \Mod_R^\omega $ sends $ A \otimes A^\op $ to $ A $. 
    Endow $ \Mod_{A^\op} $ with a Poincaré structure corresponding to the module with genuine involution $ M_{A^\op}:= A^\op $, $ N_{A^\op} := \hom_R(A^{\varphi C_2}, R^{\varphi C_2}) $. 
    It remains to exhibit a natural equivalence 
    \begin{equation}\label{eq:invertible_quadratic_compatibility}
        \eta \colon \left(\Qoppa_A \otimes \Qoppa_{A^\op}\right) \xrightarrow{\simeq} \mathrm{ev}^* \Qoppa_R 
    \end{equation} 
    of [quadratic] functors $ \Mod_A^\omega \otimes \Mod_{A^\op}^\omega \to \Spectra $. 
    By \cite[Theorem 3.3.1]{CDHHLMNNSI}, it suffices to exhibit equivalences on the bilinear and linear parts of (\ref{eq:invertible_quadratic_compatibility}) which glue compatibly. 
    \Lucy{finish}
\end{proof}

\section{Poincar{\'e} schemes}
\begin{defn}
Let $\aps$ be the $(\infty,1)$-category defined by the pullback \[
\begin{tikzcd}
\aps \arrow[rr]\arrow[d] & & \operatorname{Fun}(\Delta^2, \calg(\Sp))\arrow[d,"d_1^*"]\\
\calg(\Sp^{BC_2})\arrow[rr,"U(-)\to (-)^{tC_2}"] & & \operatorname{Fun}(\Delta^1, \calg(\Sp))
\end{tikzcd}
\] where $U:\Sp^{BC_2}\to \Sp$ is the functor which forgets the $C_2$-action.
\end{defn}

We record here a few structural results about this category.

\begin{thm}
The following statements about $\aps$ hold:
\begin{enumerate}
\item The category $\aps$ is a cocomplete and symmetric monoidal infinite category;
\item the pullback diagram above is homotopy Cartesian;
\item the functor $\aps\to \calg(\Sp^{BC_2})$ is symmetric monoidal and (co)continuous;
\item the functor $\aps\to \calg(\Sp)^{\Delta^2}$ is lax symmetric monoidal;
\item and the functor $\aps\to \calg(\Sp)^{\Delta^2}\xrightarrow{ev_{[1]}} \calg(\Sp)$ is symmetric monoidal.
\end{enumerate}
\end{thm}
\begin{proof}
    For (2) it is enough to show that $d_1^*$ is a cartesian fibration which follows from \cite[Corollary 2.4.6.5]{HTT}. 

    For (3), let $p:K\to \aps$ be a map of simplicial sets, $K$ a small simplicial set. Suppose the $K^\vartriangleright\to \aps$ be an extension such that $K^\vartriangleright\to \aps\to \calg(\Sp^{BC_2})$ is a colimit diagram. By \cite[Proposition 2.4.3.2]{HTT} the diagram \[\begin{tikzcd}
        \aps_{p/}\arrow[r]\arrow[d] &\calg(\Sp)^{\Delta^2}_{p/-}\arrow[d]\\
        \calg(\Sp^{BC_2})_{p/}\arrow[r] & \calg(\Sp)^{\Delta^1}_{p/-}
    \end{tikzcd}\] is again homotopy cartesian. Then 
    \begin{align*}
        \hom_{\aps}(p(\infty), -)&\simeq \hom_{\calg(\Sp^{BC_2})}(p(\infty), -)\times_{\hom_{\calg(\Sp)^{\Delta^1}}(p(\infty),-)}\hom_{\calg(\Sp)^{\Delta^2}}(p(\infty))\\
        &\simeq 
    \end{align*}
\end{proof}

We will denote elements of $\aps$ by $\underline{A}=(A,s:A^{\Phi C_2}\to A^{tC_2})$. Here $s:A^{\Phi C_2}\to A^{tC_2}$ is the image of $\underline{A}$ under the top horizontal map above. The use of the notation $A^{\Phi C_2}$ is justified by the following.

\begin{lem}
Let $\aps\to \calg(\Sp)$ be the composition of the functors \[\aps\to \operatorname{Fun}(\Delta^2,\calg(\Sp))\xrightarrow{ev_{[1]}}\calg(\Sp).\] Then this functor factors as a composition $\aps\to \calg(\Sp^{C_2})\xrightarrow{(-)^{\Phi C_2}}\calg(\Sp)$. 
\end{lem}
\begin{proof}
The commutativity of the diagram
\[
\begin{tikzcd}
 & & \operatorname{Fun}(\Delta^2, \calg(\Sp)) \arrow[rd, "d_0^*"] \arrow[dd, "d_1^*"] & \\
 & & & \operatorname{Fun}(\Delta^1,\calg(\Sp)) \arrow[dd,"ev_{[1]}"]\\
\calg(\Sp^{BC_2}) \arrow[rr, "U(-)\to (-)^{tC_2}"] \arrow[rrd, "id"] & & \operatorname{Fun}(\Delta^1, \calg(\Sp)) \arrow[rd, "ev_{[1]}"] & \\
  & & \calg(\Sp^{BC_2}) \arrow[r, "(-)^{tc_2}"] & \calg(\Sp)
\end{tikzcd}
\] induces a functor on the pullback infinity categories $\aps\to \calg(\Sp^{C_2})$ which makes the corresponding cube commute. The functor $ev_{[1]}:\operatorname{Fun}(\Delta^2, \calg(\Sp))\to \calg(\Sp)$ factors through $d_0^*$ and so $\aps\to \operatorname{Fun}(\Delta^2, \calg(\Sp))\to \calg(\Sp)$ is equivalent to the composition \[\aps\to \calg(\Sp^{C_2})\to \operatorname{Fun}(\Delta^1,\calg(\Sp))\to \calg(\Sp)\] and the composition of the last two maps is the geometric fixed point functor as desired.
\end{proof}

The following Lemma gives the justification of the name Poincar{\'e} scheme.

\begin{lem}
    There is a symmetric monoidal functor \[\perfpn:\aps\to \mathrm{Cat}_{\infty}^{\mathrm{Pn}}\] to the category of Poincar{\'e} infinity categories which has essential image the subcategory spanned by objects $(\perf(R),\Qoppa)$ which are $\mathbb{E}_\infty$-algebras.
\end{lem}

\begin{defn}
     A map $f:\underline{A}\to \underline{B}\in \aps$ is faithfully flat if the underlying map $f:A\to B$ is faithfully flat and the map $f^{\Phi C_2}:A^{\Phi C_2}\to B^{\Phi C_2}$ is also faithfully flat.
\end{defn}

\begin{lem}
    The fpqc covers on $\aps$ form a Grothendieck site. 
\end{lem}

%\begin{defn}
%    The infinity category of Poincar{\'e} schemes, denoted $\psch$, is the infinity category \[\psch:= \operatorname{Ind}(\aps^{op})\]
%\end{defn}

\printbibliography
\end{document}