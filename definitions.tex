\begin{definition}
    Let $\CAlgp$ be the $(\infty,1)$-category defined by the pullback 
    \begin{equation}\label{diagram:poincare_ring_defn}    
    \begin{tikzcd}
    \CAlgp \arrow[rr]\arrow[d] & & \operatorname{Fun}(\Delta^2, \CAlg(\Spectra))\arrow[d,"d_1^*"]\\
    \CAlg(\Spectra^{BC_2})\arrow[rr,"U(-)\to (-)^{tC_2}"] & & \operatorname{Fun}(\Delta^1, \CAlg(\Spectra))
    \end{tikzcd}
    \end{equation}
    where $U:\Spectra^{BC_2}\to \Spectra$ is the functor which forgets the $C_2$-action and the lower horizontal arrow is the Tate-valued norm (see Definition 3.8 and Lemma 3.10 of \cite{LYang_normedrings}). 
\end{definition}
\Noah{I keep trying to make this work but the technical details are actively killing me. Better, I think, to use the following definition instead.}

\begin{definition}
Define the category of affine Hermitian schemes, denoted $\mathrm{AHS}$, to be the infinity category given by the Grothendieck construction applied to the functor \[\CAlg(\Spectra)^{op}\to \mathrm{Cat}_\infty\] given by sending a ring $R$ to the category $\CAlg(\mathrm{Mod}_{NR})$ of $\mathbb{E}_\infty$ algebras in modules with genuine involutions over $R$. Then define the category of affine Poincare schemes, denoted by $\aps$, to be the full subcategory of $\mathrm{AHS}$ spanned by the pairs $(R, M)$ where $M\in \CAlg(\mathrm{Mod}_{NR})$ is invertible. 
\end{definition}

We record here a few structural results about this category.

\begin{theorem}
The following statements about $\CAlgp$ hold:
\begin{enumerate}
\item \label{thmitem:defining_diagram_homotopy_pullback} the pullback diagram (\ref{diagram:poincare_ring_defn}) is homotopy Cartesian; % i.e. the pullback in simplicial sets computes the pullback in $ \infty $-categories.
\item \label{thmitem:poincare_ring_has_colimits} The category $\CAlgp$ has all small colimits;
\item \label{thmitem:poincare_ring_to_ring_preserves_colims} the functor $ \CAlgp \to \CAlg(\Spectra^{BC_2})$ preserves all small colimits;
\item \label{thmitem:poincare_ring_has_limits} The category $\CAlgp$ has all small limits;
\item \label{thmitem:poincare_ring_to_ring_preserves_lims} the functor $ \CAlgp \to \CAlg(\Spectra^{BC_2})$ preserves all small limits;
\item the functor $ \CAlgp\to \CAlg(\Spectra)^{\Delta^2}$ is lax symmetric monoidal;
\item and the functor $ \CAlgp\to \CAlg(\Spectra)^{\Delta^2}\xrightarrow{ev_{[1]}} \CAlg(\Spectra)$ is symmetric monoidal.
\end{enumerate}
\end{theorem}
\begin{proof}
    To prove (\ref{thmitem:defining_diagram_homotopy_pullback}), it is enough to show that $d_1^*$ is an categorical fibration. 
    By \cite[Corollary 2.4.6.5]{HTT}, it suffices to show that $ d_1^* $ is an inner fibration and that, for any equivalence $ p \colon X \to X' $ in $ \Fun(\Delta^1, \CAlg(\Spectra)) $ and any object $ Y \in \Fun(\Delta^2, \CAlg(\Spectra)) $ so that $ d_1^*(Y) = X $, there is an equivalence $ Y \to Y' $ lifting $ p $. 
    It is clear that $ d_1^* $ satisfies the latter condition. 
    That $ d_1^*$ is an inner fibration follows from \cite[Corollary 2.3.2.5]{HTT}. 
    In fact, $ d_1^* $ is a cocartesian and cartesian fibration; this follows from the existence of colimits and limits, resp., in $ \CAlg(\Spectra) $.
    % In fact $d_1^*$ is a left fibration by \cite[Corollary 2.1.2.9]{HTT} (taking $p=id$, $i=d_1:\Delta^1\to \Delta^2$)\Lucy{Is this reference correct? The conclusion asserts that some map of simplicial sets is a \emph{categorical} fibration. The following argument is `sketchy'--depending on how precise we want to be about quasicategories, we may want to argue with left/right anodyne maps instead.} \Noah{To prove 2 we only need that $d_1^*$ is a categorical fibration, although I think it is a cartesian fibration. I think we should be careful about infinity categories, and I believe the sketch you wrote down. I also think that this must be known already thoug, and will hunt down a correct reference.}
    % There is a (pseudo-)functor
    % \begin{align*}
    %     F \colon \Fun(\Delta^1, \CAlg(\Spectra)) &\to \Cat_\infty \\
    %     (\varphi \colon A \to B) &\mapsto (\left(\CAlg(\Spectra)_{A/-/B}\right)_{/\varphi}) 
    % \end{align*}
    % which sends a square
    % \begin{equation}\label{diagram:morphism_of_arrows}
    % \begin{tikzcd}
    %     A \ar[r,"\varphi"] \ar[d] & B \ar[d] \\
    %     C \ar[r,"\psi"] & D 
    % \end{tikzcd}
    % \end{equation}
    % regarded as a morphism from $ \varphi $ to $ \psi $, to the functor
    % \begin{equation}
    % \begin{split}
    %     \left(\CAlg(\Spectra)_{A/-/B}\right)_{/\varphi} &\to \left(\CAlg(\Spectra)_{C/-/D}\right)_{/\psi} \\
    %     (A \to R \to B) & \mapsto C \simeq A \otimes_A C \xrightarrow{\varphi \otimes \mathrm{id}_C} B \otimes_A C \to D
    % \end{split}
    % \end{equation}
    % where $ B \otimes_A C \to D $ is the canonical map induced by the commuting square (\ref{diagram:morphism_of_arrows}). 
    % The functor $ F $ classifies the cocartesian fibration $ d_1^* $. 

    To prove (\ref{thmitem:poincare_ring_has_colimits}), let $p:K\to \CAlgp$ be a map of simplicial sets, where $K$ is a small simplicial set. 
    Write $ f' $ for the functor $\CAlgp \to \CAlg(\Spectra^{BC_2})$ and $ g'\colon \CAlgp \to  \CAlg(\Spectra)^{\Delta^2} $ and $ f \colon \CAlg(\Spectra)^{\Delta^2} \to  \CAlg(\Spectra)^{\Delta^1} $ and $ g \colon \CAlg(\Spectra^{BC_2}) \to  \CAlg(\Spectra)^{\Delta^1} $. 
    Choose an extension $ \overline{f'p} \colon K^\vartriangleright\to \CAlg(\Spectra^{BC_2}) $ be an extension of $ f'\circ p $ which is a colimit diagram. 
    By \cite[Proposition 4.3.1.5(2)]{HTT}, it suffices to exhibit a lift $ K^\vartriangleright \to \CAlgp $ which is an $ f' $-colimit diagram. 
    By \cite[Prposition 4.3.1.5(4)]{HTT} and (\ref{thmitem:defining_diagram_homotopy_pullback}), it suffices to show that there exists an extension $ \overline{g'p} \colon K^\vartriangleright \to \CAlg(\Spectra)^{\Delta^2} $ of $ g \circ \overline{f'p} $ which is an $ f $-colimit. 
    \Lucy{Show that $ f $ satisfies the hypotheses of \cite[Corollary 4.3.1.11]{HTT}, then use that corollary to conclude.} 

    Part (\ref{thmitem:poincare_ring_to_ring_preserves_colims}) follows from (\ref{thmitem:poincare_ring_has_colimits}). 
    % For (\ref{thmitem:poincare_ring_to_ring_preserves_colims}), let $p:K\to \CAlgp$ be a map of simplicial sets, $K$ a small simplicial set. 
    % Suppose the $K^\vartriangleright\to \CAlgp$ be an extension such that $K^\vartriangleright\to \CAlgp \xrightarrow{f'} \CAlg(\Spectra^{BC_2})$ is a colimit diagram. 
    % By \cite[Proposition 2.4.3.2]{HTT} the diagram \[\begin{tikzcd}
    %     \CAlgp_{p/}\arrow[r]\arrow[d] &\CAlg(\Spectra)^{\Delta^2}_{p/-}\arrow[d]\\
    %     \CAlg(\Spectra^{BC_2})_{p/}\arrow[r] & \CAlg(\Spectra)^{\Delta^1}_{p/-}
    % \end{tikzcd}\] is again homotopy cartesian. Then 
    % \begin{align*}
    %     \hom_{\CAlgp}(p(\infty), -)&\simeq \hom_{\CAlg(\Spectra^{BC_2})}(p(\infty), -)\times_{\hom_{\CAlg(\Spectra)^{\Delta^1}}(p(\infty),-)}\hom_{\CAlg(\Spectra)^{\Delta^2}}(p(\infty))\\
    %     &\simeq 
    % \end{align*}

    The proofs of parts (\ref{thmitem:poincare_ring_has_limits}) and (\ref{thmitem:poincare_ring_to_ring_preserves_lims}) are analogous to those of (\ref{thmitem:poincare_ring_has_colimits}) and (\ref{thmitem:poincare_ring_to_ring_preserves_colims}) and have been omitted.  
\end{proof}

We will denote objects of $\CAlgp$ by $\underline{A}=(A,s:A^{\Phi C_2}\to A^{tC_2})$. 
Here $ s:A^{\Phi C_2}\to A^{tC_2}$ is the image of $\underline{A}$ under the top horizontal map above. 
The use of the notation $A^{\Phi C_2}$ is justified by Lemma \ref{lemma:Poincare_ring_geom_fixpt}. 
\begin{remark}\label{rmk:Poincare_ring_has_underlying_C2_spectrum_alg}
    Recall that there is a symmetric monoidal récollement $ \Spectra^{C_2} \simeq \Spectra^{BC_2} \times_{(-)^{tC_2},\Spectra, \mathrm{ev}_1} \Spectra^{\Delta^1} $ (cf. \cite[Theorem 6.24]{MNN}). 
    There is a commutative diagram of $ \infty $-categories: 
    \[
    \begin{tikzcd}
     & & \operatorname{Fun}(\Delta^2, \CAlg(\Spectra)) \arrow[rd, "d_0^*"] \arrow[dd, "d_1^*"] & \\
     & & & \operatorname{Fun}(\Delta^1,\CAlg(\Spectra)) \arrow[dd,"ev_{[1]}"]\\
    \CAlg(\Spectra^{BC_2}) \arrow[rr, "U(-)\to (-)^{tC_2}"] \arrow[rrd, "id"] & & \operatorname{Fun}(\Delta^1, \CAlg(\Spectra)) \arrow[rd, "ev_{[1]}"] & \\
      & & \CAlg(\Spectra^{BC_2}) \arrow[r, "(-)^{tC_2}"] & \CAlg(\Spectra)
    \end{tikzcd}\,.
    \] 
    The diagram induces a functor from the pullback of the upper left cospan to the pullback of the lower right cospan $ U \colon \CAlgp\to \CAlg(\Spectra^{C_2}) $. % which makes the corresponding cube commute. 
\end{remark}
\begin{lemma}\label{lemma:Poincare_ring_geom_fixpt}
    % Let $\CAlgp\to \CAlg(\Spectra)$ be the composition of the functors \[\CAlgp\to \operatorname{Fun}(\Delta^2,\CAlg(\Spectra))\xrightarrow{\mathrm{ev}_{1}}\CAlg(\Spectra).\] 
    % Then this functor factors as a composition $ \CAlgp \xrightarrow{U} \CAlg(\Spectra^{C_2})\xrightarrow{(-)^{\Phi C_2}}\CAlg(\Spectra)$ where $ U $ is the functor of Remark \ref{rmk:Poincare_ring_has_underlying_C2_spectrum_alg}. 
    There is a commutative diagram
    \begin{equation*}
    \begin{tikzcd}
        \CAlgp \ar[r] \ar[d,"U"] & \operatorname{Fun}(\Delta^2,\CAlg(\Spectra)) \ar[d,"{\mathrm{ev}_1}"] \\
        \CAlg(\Spectra^{C_2}) \ar[r,"{(-)^{\Phi C_2}}"] & \CAlg(\Spectra)
    \end{tikzcd}
    \end{equation*}
    where $ U $ is the functor of Remark \ref{rmk:Poincare_ring_has_underlying_C2_spectrum_alg}. \Lucy{add reference to definition of $ \CAlgp $?}
\end{lemma}
\begin{proof}
    Follows from the récollement of $ \Spectra^{C_2} $ and Remark \ref{rmk:Poincare_ring_has_underlying_C2_spectrum_alg}. 
% The functor $ev_{[1]}:\operatorname{Fun}(\Delta^2, \CAlg(\Spectra))\to \CAlg(\Spectra)$ factors through $d_0^*$ and so $\CAlgp\to \operatorname{Fun}(\Delta^2, \CAlg(\Spectra))\to \CAlg(\Spectra)$ is equivalent to the composition \[\CAlgp\to \CAlg(\Spectra^{C_2})\to \operatorname{Fun}(\Delta^1,\CAlg(\Spectra))\to \CAlg(\Spectra)\] and the composition of the last two maps is the geometric fixed point functor as desired.
\end{proof}

The following Lemma gives the justification of the name Poincar{\'e} scheme.
\begin{construction}\label{con:functor_APS_to_Pn_cat}
    We shall construct a functor \[\perfpn:\aps\to \mathrm{Cat}_{\infty}^{\mathrm{Pn}}\] to the category of Poincar{\'e} infinity categories. 

    \Lucy{For symmetric monoidal structure--maybe want to swap out $ \Mod_{NA}$ for $ \CAlg_{NA} $? }
    Recall that $ \Cath_\infty \to \left( \Catex_\infty \right)^{op} $ is a cocartesian fibration \cite[\S1.4.]{CDHHLMNNSI} 
    We will first construct a map of cocartesian fibrations
    \begin{equation}
    \begin{tikzcd}
        \aps \ar[r,dotted] \ar[d] & \Cath_\infty \ar[d] \\
        \CAlg\left(\Spectra^{BC_2}\right)\ar[r] & \left( \Catex_\infty \right)^{op}
    \end{tikzcd}\,,
    \end{equation}
    then show that the dotted arrow factors through the subcategory $ \Cat^p_\infty \subseteq \Cat^h_\infty $. 
    To construct a map of cartesian fibrations, it suffices to exhibit a natural transformation of classifying functors. 
    Unraveling the definitions, by Theorem 3.2.13 of \cite{CDHHLMNNSI} we must exhibit for each $ A \in \CAlg(\Spectra)^{BC_2} $, a functor
    \begin{equation}
        \left(\CAlg(\Spectra)_{A/-/A^{tC_2}}\right)_{/\varphi} \to \Mod_{N^{C_2}(A^e)}\left(\Spectra^{C_2}\right)
    \end{equation}
    (where $ \varphi \colon A \to A^{tC_2} $ is the Tate-valued Frobenius and $ N^{C_2} $ is the Hill--Hopkins--Ravenel norm) which is natural in $ A $. 

    That the resulting functor factors through the subcategory $ \Cat^p_\infty $ follows from Proposition 3.1.3 and Lemma 3.3.3 of \emph{loc. cit.}
\end{construction}
\begin{lemma}
    The functor of Construction \ref{con:functor_APS_to_Pn_cat} is symmetric monoidal and has essential image the subcategory spanned by objects $(\perf(R),\Qoppa)$ which are $\mathbb{E}_\infty$-algebras.
\end{lemma}

\begin{definition}
     A map $f:\underline{A}\to \underline{B}\in \aps$ is faithfully flat if the underlying map $f:A\to B$ is faithfully flat and the map $f^{\Phi C_2}:A^{\Phi C_2}\to B^{\Phi C_2}$ is also faithfully flat.
\end{definition}

\begin{lemma}
    The fpqc covers on $\aps$ form a Grothendieck site. 
\end{lemma}

%\begin{definition}
%    The infinity category of Poincar{\'e} schemes, denoted $\psch$, is the infinity category \[\psch:= \operatorname{Ind}(\aps^{op})\]
%\end{definition}
